\section*{Introduzione}

\subsection*{Luogo e data dell'incontro}
\begin{itemize}
	\item \textbf{luogo:} videoconferenza su \glock{Meet};
	\item \textbf{data:} 2020-12-09;
	\item \textbf{ora di inizio:} 15:00;
	\item \textbf{ora di fine:} 16:30.
\end{itemize}

\subsection*{Ordine del giorno}
	\begin{enumerate}
	\item sintesi dei progetti e valutazioni
	\item analisi generale capitolati
	\item decisione capitolato con votazioni
	\item discussione sui template
	\item discussione sul glossario
	\item definizione versioni
	\item decisione esperto latex
	\item varie	ed eventuali
\end{enumerate}

\subsection*{Presenze}
\begin{itemize}
	\item \textbf{totale presenti:} 7 su 7;
	\item \textbf{presenti: }
	\begin{itemize}
		\item Badan Antonio;
		\item Bertoldo Damiano;
		\item Budai Matteo;
		\item De Grandi Samuele;
		\item Piacere Ivan;
		\item Privitera Sara;
		\item Spigolon Daniele;
	\end{itemize}
\end{itemize}

\section*{Svolgimento}
\subsection*{Sintesi dei progetti e valutazioni}
			Si è discusso riguardo lo studio di fattibilità analizzata dai vari componenti, esponendo pro e contro per i vari capitolati analizzati (Zextras, Red Babel, SanMarco Informatica e Zero12).

\subsection*{Analisi generale capitolati}
			Si è fatta un'analisi generale dei vari capitolati, in base ai seminari, studi di fattibilità e alle risposte ricevute dai proponenti alle domande poste via mail, che ha portato all'inserimento del capitolato proposto da Zucchetti invece di Synclab nei tre capitolati preferiti dal gruppo. 


\subsection*{Decisione finale del capitolato}
			Dopo una discussione sui tre capitolati preferiti dal gruppo(Zero12, Zucchetti, Imola Informatica), si è passati a una votazione tra questi. La votazione ha portato alla conferma del capitolato proposto da Imola Informatica come scelta primaria del gruppo che successivamente è stata comunicata ai restanti gruppi come scelta effettiva.

\subsection*{Discussione sui template}
			Si è discusso sui template caricati e il loro utilizzo come ad esempio i verbali in \glock{\LaTeX}.


\subsection*{Discussione sul glossario}
      		Si è iniziato a discutere su come organizzare il glossario e a chi incaricare la stesura delle varie parole.


\subsection*{Definizione versioni}
			Si è discusso su come organizzare le varie versioni dei documenti.
			Al momento si procede con versione x.y.z dove x sta per l'approvazione, y per la verifica e z per la stesura.
			Molto probabilmente in futuro questa procedura verrà cambiata.


\subsection*{Decisione esperto latex}
			Si è discusso sul fatto che potrebbe essere utile avere un esperto per determinati campi ed è stato incaricato Bertoldo Damiano come esperto di \glock{\LaTeX}.


\subsection*{Varie ed eventuali}
			Si è discusso sul fatto che i report dei seminari sono più per uso interno e di supporto allo studio di fattibilità e non come documento da condividere.
			Si è iniziato a studiare cosa inserire nelle norme di progetto e a come suddividere i vari campi all'interno del documento.
			Si è deciso che bisognerà stabilire un limite di tempo massimo per le riunioni e in supporto a questo potrebbe essere incaricata una persona che fa da moderatore.


