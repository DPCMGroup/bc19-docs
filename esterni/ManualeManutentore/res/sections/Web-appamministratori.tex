\section{Web-app amministratori}
\subsection{Introduzione}
La web app costituisce l'interfaccia tramite la quale l'amministratore può interagire col sistema.
Le funzionalità offerte sono:
\begin{itemize}
	\item login e logout;
	\item visualizzazione delle stanze e delle postazioni con i relativi stati;
	\item aggiunta, rimozione e modifica di stanze e postazioni;
	\item impostazione di postazioni come guaste;
	\item impostazione di stanze come inaccessibili;
	\item visualizzazione delle credenziali degli utenti;
	\item aggiunta, rimozione e modifica di credenziali;
	\item visualizzazione e scaricamento report sulle occupazioni e sulle igienizzazioni;
	\item visualizzazione di notifiche riguardanti il salvataggio dei dati sulla blockchain.
\end{itemize}

\subsection{Requisiti e installazione}
Il codice sorgente della web-app è pubblicato su GitHub all'indirizzo \url{https://github.com/DPCMGroup/bc19-webapp}.
Lo si può scaricare compresso in formato zip direttamente dal sito o, se si ha già installato il programma git, eseguendo il comando
\begin{verbatim}
	git clone https://github.com/DPCMGroup/bc19-webapp
\end{verbatim}

\subsubsection{Linguaggi}
\paragraph{Typescript}
Superset del linguaggio Javascript.

\subsubsection{Tecnologie}
\paragraph{Node.js}
Programma focalizzato sull'esecuzione di codice javascript al di fuori del browser.
Per l'installazione fare riferimento alla pagina \url{https://nodejs.org/en/download/} 
\paragraph{npm}
Acronimo di Node Package Manager, permette di ottenere le librerie necessarie allo sviluppo.
Si ottiene insieme a Node.js tramite l'installazione di quest'ultimo.
\paragraph{Angular}
Framework per applicazioni web.
Per l'installazione fare riferimento alla pagina \url{https://angular.io/guide/setup-local#install-the-angular-cli}
\paragraph{Bootstrap}
Framework per la creazione di pagine web.
Per l'integrazione di bootstrap in angular fare riferimento alla pagina \url{https://www.npmjs.com/package/@ng-bootstrap/ng-bootstrap#installation}

\subsubsection{Test}

\subsection{Architettura}
L'architettura della web-app segue il modello a componenti imposto da Angular, che, a sua volta, si basa sul pattern Model-View-ViewModel, descritto dall'immagine sottostante.
\begin{figure}[H]
	\centering
	\includegraphics[width=15cm]{res/images/mvvm.jpg}
	\caption{Model-View-ViewModel}
	\label{fig:Model-View-ViewModel}
\end{figure}
Secondo questo modello la parte visiva del sito viene suddivisa in parti, e ognuna di queste parti viene controllata da una diversa classe.
La parte visiva viene chiamata view, mentre la parte di controllo viene chiamata view model.

\subsection{Diagrammi dei package}
\subsection{Diagrammi delle classi}
\subsubsection{Login}
\subsubsection{Gestione stanze e postazioni}
\begin{figure}[H]
	\centering
	\includegraphics[width=18cm]{res/images/webapp-visualAddEditStanzePostazioni-diagrammaClassi.png}
	\caption{Diagramma delle classi per la gestione delle stanze e delle postazioni}
	\label{fig:DiagrammaClassiStanzePostazioni}
\end{figure}
\subsubsection{Gestione credenziali}
\subsubsection{Sezione report}
\subsubsection{Sezione notifiche}
\subsection{Diagrammi di sequenza}
\subsubsection{Login}
\subsubsection{Gestione stanze e postazioni}
\begin{figure}[H]
	\centering
	\includegraphics[width=18cm]{res/images/webapp-visualStanzePostazioni-diagrammaSequenza.png}
	\caption{Diagramma di sequenza per la visualizzazione delle stanze e delle postazioni}
	\label{fig:DiagrammaSequenzaStanzePostazioni1}
\end{figure}
\begin{figure}[H]
	\centering
	\includegraphics[width=18cm]{res/images/webapp-addEditStanzePostazioni-diagrammaSequenza.png}
	\caption{Diagramma di sequenza per l'aggiunta e la modifica di una stanza}
	\label{fig:DiagrammaSequenzaStanzePostazioni2}
\end{figure}
\subsubsection{Gestione credenziali}
\subsubsection{Sezione report}
\subsubsection{Sezione notifiche}



