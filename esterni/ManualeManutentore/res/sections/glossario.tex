\section{Glossario}

 %\addtocontents{toc}{\setcounter{tocdepth}{5}}

\subsection*{A}
\subsubsection*{Android} Sistema operativo per dispositivi mobili.
\subsubsection*{Angular} Framework per lo sviluppo e il design di applicazioni single-page.
\subsubsection*{API} Acronimo per Application Programming Interface (API). È un'interfaccia con ruolo di intermediazione fra software diversi.
\subsubsection*{Applicazione} Programma per calcolatore predisposto per eseguire una categoria di funzioni specifiche.
\subsubsection*{Architettura software} L’architettura software ci dice come un sistema è organizzato, ossia quali sono i ruoli delle componenti del sistema e quali interazioni esistono fra di esse. Essa definisce, inoltre, le interfacce necessarie all’interazione tra componenti o con l’ambiente. Il tutto non viene fatto in modo estemporaneo, ma seguendo dei paradigmi di composizione precisi.
\subsubsection*{Asincrono} Dispositivo che opera senza un riferimento temporale di sincronizzazione rispetto a un altro dispositivo. \\
Protocollo asincrono: modalità di trasmissione dati che non dipende dal compiersi di altri processi.

\subsection*{B}
\subsubsection*{Back-end} Back-end è un termine largamente utilizzato per caratterizzare le interfacce che hanno come destinatario un programma. Una applicazione back-end è un programma con il quale l'utente interagisce indirettamente, in generale attraverso l'utilizzo di una applicazione front-end.
\subsubsection*{Blockchain} Una blockchain è una struttura dati condivisa e immutabile. È definita come un registro
digitale le cui voci sono raggruppate in blocchi, concatenati in ordine cronologico, e la cui integrità è garantita dall'uso della crittografia.
\subsection*{C}
\subsubsection*{CSS} CSS (Cascading Style Sheets), è un linguaggio usato per definire la formattazione di documenti HTML, XHTML e XML, ad esempio i siti web e le relative pagine.
\subsection*{D}
\subsubsection*{Docker} Il software Docker è una tecnologia di containerizzazione che consente la creazione e l’utilizzo dei container Linux.
\subsection*{E}
\subsubsection*{Ethereum} Blockchain per la creazione e pubblicazione di \glock{smart contract}.
\subsection*{F}
\subsubsection*{Framework} È una infrastruttura intesa come "struttura o complesso di elementi che costituiscono la base di sostegno o comunque la parte sottostante di altre strutture". Si intende la piattaforma che funge da strato intermedio tra un sistema operativo e il software che lo utilizza.
\subsection*{G}
\subsubsection*{GitHub} GitHub è un servizio di hosting per progetti software.
\subsubsection*{GitHub Actions} GitHub Actions è uno strumento fornito da GitHub che permette l’automazione di compiti di varia natura.
\subsection*{H}
\subsubsection*{HTML} È un linguaggio di markup per la strutturazione delle pagine web.
\subsection*{I}
\subsubsection*{Interfaccia} Servizi offerti da una entità a un'altra entità.
\subsection*{J}
\subsubsection*{Javascript} JavaScript è un linguaggio di scripting orientato agli oggetti e agli eventi, comunemente utilizzato nella programmazione Web.
\subsection*{K}
\subsubsection*{Kotlin} Kotlin è un linguaggio di programmazione versatile, multi-paradigma e open-source creato da JetBrains.
\subsection*{L}
\subsubsection*{Linux} È una famiglia di sistemi operativi open-source di tipo Unix-like.
\subsection*{M}
\subsubsection*{MacOS} Sistema operativo Unix-like dei personal computer di Apple Inc..
\subsubsection*{MVC} Acronimo di Model-View-Controller (MVC, talvolta tradotto in italiano con la dicitura modello-vista-controllo). È un “pattern architetturale”  soprattutto utilizzato per la programmazione orientata agli oggetti, in grado di separare la rappresentazione interna dei dati da ciò che viene presentato e accettato dall'utente.
\subsection*{N}
\subsubsection*{NFC} Acronimo di Near Field Communication (Comunicazione a Corto Raggio). Indica una tecnologia di trasmissione di dati senza fili a distanze tipicamente inferiori ai 10 cm.
\subsubsection*{Node.js} Ambiente open-source per l'esecuzione di codice Javascript a run-time, al di fuori dei browser. Ciò permette l'esecuzione di codice Javascript server-side.
\subsection*{P}
\subsubsection*{Postazione} Spazio fisico identificato da un tag RFID univoco dove l’utilizzatore appoggia il cellulare mentre sta svolgendo il suo lavoro. Ciascuna postazione di lavoro è inserita in una stanza
dell'organizzazione (laboratorio, ufficio, biblioteca, etc\dots).
\subsubsection*{Python} Python è un linguaggio di programmazione di più "alto livello" rispetto alla maggior parte degli altri linguaggi, orientato a oggetti, adatto, tra gli altri usi, a sviluppare applicazioni distribuite, scripting, computazione numerica e system testing.
\subsection*{R}
\subsubsection*{Repository} Spazio di archiviazione remoto. È utilizzato per la condivisione di software e documenti.
\subsubsection*{RFID} Radio-frequency identification, in telecomunicazioni ed elettronica, si intende una tecnologia per l'identificazione e/o memorizzazione automatica di informazioni inerenti a oggetti, animali o persone basata sulla capacità di memorizzazione di dati da parte di particolari etichette elettroniche, chiamate tag, e sulla capacità di queste di rispondere all'interrogazione a distanza da parte di appositi apparati fissi o portatili, chiamati reader.
\subsection*{S}
\subsubsection*{Sanificazione} Viene inteso l'atto di pulizia delle postazioni o stanze.
\subsubsection*{Server} Calcolatore che svolge funzioni di servizio per tutti i calcolatori collegati oppure programma, generalmente sempre attivo, che esegue determinate funzioni quando queste sono richieste da altri programmi.
\subsubsection*{SQL} È un linguaggio standardizzato per database basati sul modello relazionale.
\subsection*{T}
\subsubsection*{Tag NFC} Sono dei transponder RFID, ovvero dei minuscoli chip collegati a un'antenna. Il chip ha un codice univoco e una parte di memoria riscrivibile. L'antenna permette al chip di interagire con un lettore NFC, come uno smartphone NFC.
\subsubsection*{Test} Esperimento variamente espletato allo scopo di saggiare, mediante determinate reazioni, l'entità o la consistenza di un'attitudine o di una capacità individuale.
\subsection*{V}
\subsubsection*{Versionamento} È la gestione di versioni multiple di un insieme di informazioni: gli strumenti software per il controllo versione sono ritenuti molto spesso necessari per la maggior parte dei progetti di sviluppo software o documentali gestiti da un team collaborativo di sviluppo o redazione.
\subsection*{W}
\subsubsection*{Windows} Sistema operativo creato da Microsoft.
\subsubsection*{Workflow} È l’automazione totale o parziale di un processo aziendale, in cui documenti, informazioni o compiti passano da un partecipante a un altro per svolgere attività, secondo un insieme di regole definite.

%\addtocontents{toc}{\setcounter{tocdepth}{4}}
