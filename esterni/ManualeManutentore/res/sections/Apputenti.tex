\section{App utenti}
\subsection{Introduzione}
Questa parte del documento è orientata agli utenti che utilizzano l'applicazione Android.

\subsubsection{Scopo del prodotto}
L'applicazione Android è sviluppata per due diverse tipologie di utente ovvero il dipendente e l'addetto alle pulizie.

In generale offre all'utente le seguenti funzionalità:
\begin{itemize}
	\item \textbf{Login:} L'utente ha la possibilità di autenticarsi inserendo il proprio username e password; \\
	\item \textbf{Logout:} L'utente ha la possibilità di deautenticarsi premendo sull'elemento della lista "Logout" del menù principale in alto a destra. \\
\end{itemize}

Per quanto riguarda il dipendente, l'applicazione offre i seguenti servizi:
\begin{itemize}
	\item \textbf{Scansione:} Il dipendente ha la possibilità di scansionare una postazione per poter visualizzare lo stato di essa e altre informazioni come le prenotazioni associate ad essa; \\
	\item \textbf{Occupazione:} Il dipendente dopo aver scansionato una postazione può occuparla se questa è prenotata da lui o è libera e igienizzata; \\
	\item \textbf{Igienizzare:} Il dipendente dopo aver scansionato una postazione la può igienizzare se questa risulta non igienizzata; \\
	\item \textbf{Lista prenotazioni:} Il dipendente può visualizzare le prenotazioni effettuate premendo sull'elemento della lista "Visualizza prenotazioni" del menù principale in alto a destra; \\
	\item \textbf{Disdire prenotazione:} Il dipendente può disdire una prenotazione dopo che è entrato nella pagina in cui visualizza tutte le prenotazioni effettuate; \\
	\item \textbf{Guida:} Il dipendente può visualizzare la guida premendo sull'elemento della lista "Guida" del menù principale in alto a destra; \\
	\item \textbf{Prenota postazione:} Il dipendente può prenotare una postazione premendo sull'elemento della lista "Prenota postazione" del menù principale in alto a destra.
	Dopo aver premuto dovrà inserire la data, l'ora di inizio, l'ora di fine e la stanza obbligatoriamente e in modo facoltativo anche il nome del collega.
	Una volta premuto sul bottone "Cerca", se è stato inserito il nome del collega, visualizzerà tutte le sue prenotazioni effettuate nella stanza, nel range orario e nella data inseriti e scorrendo sotto visualizzerà tutte le postazioni di quella stanza con il loro stato e potrà decidere quale prenotare se disponibile. \\	
\end{itemize}

Per l'addetto alle pulizie l'applicazione offre le seguenti funzionalità:
\begin{itemize}
\item \textbf{Visualizzare stanze da igienizzare:} \\
\item \textbf{Visualizzare postazioni da igienizzare:} \\
\item \textbf{Marcare stanze come igienizzate:} \\
\item \textbf{Marcare postazioni come igienizzate:} \\
\end{itemize}



\subsection{Requisiti e installazione}

\subsubsection{Requisiti}
Per poter sviluppare sul proprio PC l'applicazione del sistema Stalker sono necessari i software e gli strumenti indicati in questa pagina. I software da installare saranno divisi in base al loro scopo.

Per scaricare il codice sorgente dell'applicazione, è sufficiente andare nella pagina di GitHub che lo ospita, che si trova \href{https://github.com/DPCMGroup/bc19-mobile}{qui}, cliccare su Clone or download e successivamente premere su Download ZIP.

Un'alternativa più efficace a questo procedimento è scaricare il progetto tramite Git. Se non si dispone di Git è possibile scaricarlo seguendo quanto indicato in Source Code Management. Per scaricare il progetto in questo modo, invocare il seguente comando tramite un terminale o prompt dei comandi nel sistema in uso:\\
\textbf{https://github.com/DPCMGroup/bc19-mobile.git}

\subsubsection{Prerequisiti hardware e software}
Le tecnologie utilizzate per sviluppare l'applicazione Android richiedono parecchie risorse nel loro uso contemporaneo. Si consiglia quindi di avere un computer con processore almeno quad-core e una memoria RAM di almeno 8 GB.

\subsubsection{Ambiente di sviluppo}

\paragraph{Android Studio}
L'applicazione è stata sviluppata utilizzando l'ambiente di sviluppo Android Studio, attualmente alla versione 4.1.3.
\\
\\
\textbf{Installazione di Android Studio su Windows}
\\
È possibile scaricare Android Studio su Windows visitando il sito ufficiale riportato in seguito, che inoltre fornisce un'ottima documentazione per lo sviluppo di applicazioni. Il link per la pagina da cui si può scaricare Android Studio si trova cliccando \href{https://developer.android.com/studio}{qui}, andando alla sezione "Android Studio downloads".
Per eseguire l'installazione, bisognerà seguire la guida riportata nella sezione Windows cliccando nel seguente link \href{https://developer.android.com/studio/install}{qui}.
\\
\\
\textbf{Installazione di Android Studio su MacOS}
\\
La guida per scaricare Android Studio per MacOS è identica a quella per Windows.
Per eseguire l'installazione, invece, bisognerà seguire la guida riportata nella sezione Mac cliccando nel seguente link \href{https://developer.android.com/studio/install}{qui}.
\\
\\
\textbf{Installazione di Android Studio su Ubuntu (e derivate, e altri derivati di Debian)}
\\
La guida per scaricare Android Studio per Linux è identica a quella per Windows e MacOS.
Per eseguire l'installazione, invece, bisognerà seguire la guida riportata nella sezione Linux cliccando nel seguente link \href{https://developer.android.com/studio/install}{qui}.
\\
\subsubsection{Linguaggi utilizzati}

\paragraph{Kotlin}
\textbf{Installazione di Kotlin su Windows}
\\
...
\\
\\
\textbf{Installazione di Kotlin su MacOS}
\\
...
\\
\\
\textbf{Installazione di Kotlin su Ubuntu (e derivate, e altri derivati di Debian)}
\\
...

\paragraph{XML}
La configurazione dell'applicazione e alcune sue dipendenze sono gestite tramite un file denominato AndroidManifest.xml. Inoltre, lo sviluppo di applicazioni Android richiede una cartella di progetto denominata res che contiene file XML per gestire risorse come layout, immagini, menu, stringhe ed altro ancora. Quindi è richiesta una buona conoscenza del linguaggio XML.

\subsubsection{Librerie utilizzate}

\subsubsection{Source code management}
Per poter effettuare il versionamento del codice sorgente è richiesto di utilizzare Git. Per poterlo installare è necessario recarsi a \href{https://git-scm.com/downloads}{questa pagina}.
Non è strettamente necessario, ma è consigliato per integrare le proprie modifiche nel repository.

\subsubsection{Build automation}
La build automation (ovvero la gestione del processo di build) è affidata a Gradle, integrato e utilizzato in Android Studio. I file di build sono due: uno per tutto il progetto ed uno per il solo modulo app.
Tramite Gradle il progetto dell'applicazione viene compilato, testato ed eseguito attraverso l'IDE Android Studio.


\subsection{Estendibilità}

\subsubsection{Creazione di un metodo}
Tramite l'utilizzo dell'architettura Model View Presenter è facile da implementare dei nuovi metodi separandoli con la business logic e la vista ed in seguito creare i rispettivi collegamenti utilizzando il presenter.


\subsection{Architettura}



\subsubsection{Model}

\subsubsection{View}

\subsubsection{Presenter}

\subsubsection{Contract}


\subsection{Diagramma dei package}

\subsection{Diagramma delle classi}

\subsection{Diagramma di sequenza}
