\section{App utenti}
\subsection{Introduzione}
Questa parte del documento è orientata agli utenti che utilizzano l'applicazione Android.

\subsubsection{Scopo del prodotto}
L'applicazione Android è sviluppata per due diverse tipologie di utente ovvero il dipendente e l'addetto alle pulizie.

In generale offre all'utente le seguenti funzionalità:
\begin{itemize}
	\item \textbf{Login:} L'utente ha la possibilità di autenticarsi inserendo il proprio username e password; \\
	\item \textbf{Logout:} L'utente ha la possibilità di deautenticarsi premendo sull'elemento della lista "Logout" del menù principale in alto a destra. \\
\end{itemize}

Per quanto riguarda il dipendente, l'applicazione offre i seguenti servizi:
\begin{itemize}
	\item \textbf{Scansione:} Il dipendente ha la possibilità di scansionare una postazione per poter visualizzare lo stato di essa e altre informazioni come le prenotazioni associate ad essa; \\
	\item \textbf{Occupazione:} Il dipendente dopo aver scansionato una postazione può occuparla se questa è prenotata da lui o è libera e igienizzata; \\
	\item \textbf{Igienizzare:} Il dipendente dopo aver scansionato una postazione la può igienizzare se questa risulta non igienizzata; \\
	\item \textbf{Lista prenotazioni:} Il dipendente può visualizzare le prenotazioni effettuate premendo sull'elemento della lista "Visualizza prenotazioni" del menù principale in alto a destra; \\
	\item \textbf{Disdire prenotazione:} Il dipendente può disdire una prenotazione dopo che è entrato nella pagina in cui visualizza tutte le prenotazioni effettuate; \\
	\item \textbf{Guida:} Il dipendente può visualizzare la guida premendo sull'elemento della lista "Guida" del menù principale in alto a destra; \\
	\item \textbf{Prenota postazione:} Il dipendente può prenotare una postazione premendo sull'elemento della lista "Prenota postazione" del menù principale in alto a destra.
	Dopo aver premuto dovrà inserire la data, l'ora di inizio, l'ora di fine e la stanza obbligatoriamente e in modo facoltativo anche il nome del collega.
	Una volta premuto sul bottone "Cerca", se è stato inserito il nome del collega, visualizzerà tutte le sue prenotazioni effettuate nella stanza, nel range orario e nella data inseriti e scorrendo sotto visualizzerà tutte le postazioni di quella stanza con il loro stato e potrà decidere quale prenotare se disponibile. \\	
\end{itemize}

Per l'addetto alle pulizie l'applicazione offre le seguenti funzionalità:
\begin{itemize}
\item \textbf{Visualizzare stanze da igienizzare:} \\
\item \textbf{Visualizzare postazioni da igienizzare:} \\
\item \textbf{Marcare stanze come igienizzate:} \\
\item \textbf{Marcare postazioni come igienizzate:} \\
\end{itemize}



\subsection{Requisiti e installazione}


\subsubsection{Requisiti}

\subsubsection{Prerequisiti hardware e software}

\subsubsection{Ambiente di sviluppo}

\paragraph{Android Studio}

\subsubsection{Linguaggi utilizzati}

\paragraph{Kotlin}

\paragraph{XML}

\subsubsection{Librerie utilizzate}

\subsubsection{Source code management}

\subsubsection{Build automation}


\subsection{Estendibilità}

\subsection{Architettura}



\subsubsection{Model}

\subsubsection{View}

\subsubsection{Presenter}

\subsubsection{Contract}


\subsection{Diagramma dei package}

\subsection{Diagramma delle classi}

\subsection{Diagramma di sequenza}
