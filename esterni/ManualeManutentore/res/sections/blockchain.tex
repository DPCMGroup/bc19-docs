\section{Blockchain}

\subsection{Introduzione}
Nel nostro sistema la blockchain viene utilizzata per certificare i dati di occupazione e igienizzazione delle postazioni. La tecnologia che abbiamo utilizzato per la nostra blockchain è Ethereum.

\subsection{Requisiti e installazione}

\subsubsection{Tecnologie}
\paragraph{Geth}
Il funzionamento di una rete Ethereum si basa sull'esecuzione di uno o più nodi. Ogni nodo è rappresentato da una Ethereum Virtual Machine (EVM), che manteniene una copia del registro delle transazioni e che si può utilizzare per effettuare transazioni con gli altri nodi. Nel nostro caso la rete è costituita da un solo nodo. Esistono varie implementazioni della EVM e noi abbiamo adottato Geth. \newline
Per l'installazione fare riferimento alla pagina \url{https://geth.ethereum.org/docs/install-and-build/installing-geth}. \newline
La versione da noi utilizzata è la 1.9.25.

\subsubsection{Esecuzione}
\paragraph{Modifica credenziali}
I file di configurazione di Geth forniti presentano un account di gestione della blockchain predefinito. Si consiglia fortemente di modificarne la password prima di usarlo.
Per ottenere l'indirizzo dell'account da modificare aprire una shell nella cartella della blockchain ed eseguirvi il comando \newline
\texttt{geth account list -{}-datadir .} \newline
All'inizio della riga sarà mostrata la stringa \newline
\texttt{Account \#0 \{...\}} \newline con al posto dei puntini un codice di 40 caratteri. Quel codice è l'indirizzo dell'account. \newline
Per modificare la password dell'account eseguire il comando \newline
\texttt{geth account update indirizzo\_account -{}-datadir .} \newline
Verrà chiesto di sbloccare l'account con la password "1234". Successivamente si dovrà inserire due volte la nuova password. \newline
Dopo aver modificato la password scrivere quella nuova nel file \texttt{password.sec}, al posto della precedente.

\paragraph{Esecuzione blockchain}
Per rendere la blockchain operativa eseguire da riga di comando la seguente stringa:\newline
\texttt{geth --networkid 4224 --mine --minerthreads 2 --datadir "." --nodiscover --rpc --rpcport "8545" --port "30303" --rpccorsdomain "*" --allow-insecure-unlock --nat "any" --rpcapi "eth, web3, personal, net, miner" --unlock 0 --password ./password.sec --ipcpath "~/.ethereum/geth.ipc"} \newline
La prima inizializzazione della blockchain può richiedere diversi minuti per essere completata. Una volta disponibile la blockchain sarà accessibile all'indirizzo \url{http://localhost:8545}. Essa continuerà a minare blocchi, anche se non vengono effettuate transazioni, fino a quando la shell non verrà chiusa o non si eseguirà il comando per fermare il minaggio. Per effettuare questa azione eseguire il comando \newline
\texttt{geth attach url\_della\_blockchain -{}-exec "miner.stop()"} \newline
Per far ripartire il minaggio eseguire \newline
\texttt{geth attach url\_della\_blockchain -{}-exec "miner.start()"} \newline
I dati generati dalla blockchain vengono salvati nella cartella \texttt{geth}.
Una documentazione più approfondita si può trovare alla pagina \url{https://geth.ethereum.org/docs/}.


