\section{Database}
\subsection{Introduzione}
Questa sezione tratterà la gestione del database dell'applicazione BlockCovid. Nel sistema implementato da DPCM2077, il database è direttamente collegato al backend.

\subsubsection{Scopo}
Il database permette al backend di memorizzare i dati che servono per il corretto funzionamento dell'applicazione. Essendo il database implementato nel backend, questa sezione è dedicata a chi gestisce il database e sviluppa il backend.

\subsection{Requisiti e installazione}
Per poter aggiungere funzionalità al database del sistema BlocKCovid, sono necessari gli strumenti indicati in questa sezione. Per semplicità è consigliato l'utilizzo di docker per l'esecuzione del servizio.

\subsubsection{Prerequisiti hardware e software}
È consigliato l'utilizzo di un server con risorse sufficienti per il funzionamento del database. Questo dipende molto dalla quantità dei dati e per questo è consigliato avere risorse hardware equivalenti ad un processore quad-core e 8 GB di RAM e sufficiente spazio su disco.
\\\\
Il database di riferimento è MariaDB 10.5.9.

\subsubsection{Ottenimento script}
Per scaricare gli script necessari alla creazione e popolamento database è necessario \textbf{Git}. Se non si dispone di Git è possibile scaricarlo seguendo quanto indicato nella sezione \textbf{§4.2.4}. Per scaricare il progetto, invocare il seguente comando da terminale: \textit{git clone https://github.com/DPCMGroup/bc19-db.git}.

\subsubsection{Linguaggi utilizzati}
\paragraph{SQL}
Il linguaggio utilizzato per la creazione e popolamento delle tabelle è stato SQL.

\subsubsection{Docker container}
Docker viene utilizzato per l'esecuzione del sevizio in produzione.
\paragraph{Installazione di Docker su Windows}
È possibile installare Docker su Windows visitando il suo sito ufficiale, alla \href{https://hub.docker.com/editions/community/docker-ce-desktop-windows}{seguente pagina}. La guida all'installazione e al primo utilizzo è presente nello stesso link in cui si scarica l'eseguibile per l'installazione.
\paragraph{Installazione di Docker su MacOS}
L'installazione per MacOS è identica a quella per Windows, ma la pagina a cui scaricarlo si trova a \href{https://hub.docker.com/editions/community/docker-ce-desktop-mac}{questo link}.
\paragraph{Installazione Docker Linux}
È possibile installare Docker su Ubuntu seguendo le guide presenti sul sito ufficiale, disponibili a \href{https://docs.docker.com/engine/install/ubuntu/}{questo indirizzo}.

\subsubsection{Esecuzione}
Attualmente sono presenti due modi per eseguire il database:
\begin{itemize}
	\item Tramite servizio;
	\item Tramite l'utilizzo di docker (consigliato).
\end{itemize}
\paragraph{Servizio}
Per eseguire il database è necessario averlo installato. Se non lo si ha, si può scaricare e seguire la guida presente a \href{https://mariadb.org/download/}{questo link}. \\
Ad installazione completata, il database verrà avviato in automatico ad ogni accensione.
\paragraph{Docker}
Per eseguire il database tramite docker è sufficiente seguire la seguente \href{https://hub.docker.com/\_/mariadb}{guida}.