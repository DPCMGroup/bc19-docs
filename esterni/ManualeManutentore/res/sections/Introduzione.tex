\section{Introduzione}

\subsection{Scopo documento}
Lo scopo di questo documento consiste nel fornire una guida all'installazione e alla manutenzione del sistema BlockCovid. Il contenuto è suddiviso in sezioni dedicate ai singoli componenti del prodotto:
\begin{itemize}
	\item App utenti: l'\glock{applicazione} mobile dedicata ai dipendenti e agli addetti alle pulizie;
	\item Web app: l'\glock{interfaccia} web per gli amministratori;
	\item \glock{Backend}: il \glock{server} che espone la \glock{API};
	\item Database;
	\item \glock{Blockchain};
\end{itemize}
Per ogni sezione sono indicati i requisiti e i passi per l'installazione. Per le componenti più complesse viene fornita una descrizione dell'\glock{architettura} e una guida all'estensione.

\subsection{Glossario}
All’interno del documento sono presenti termini che assumono significati diversi a seconda del contesto. Per evitare ambiguità, è stata posta alla fine del documento una sezione di nome Glossario che conterrà tali termini con il loro significato specifico. Per segnalare che un termine del testo è presente all’interno del Glossario, verrà aggiunta una G pedice posta a fianco del termine ambiguo.
