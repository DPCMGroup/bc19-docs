\section*{Introduzione}
\subsection*{Luogo e data dell'incontro}
\begin{itemize}
	\item \textbf{luogo:} videoconferenza su Meet;
	\item \textbf{data:} 2021-03-19;
	\item \textbf{ora di inizio:} 14:00;
	\item \textbf{ora di fine:} 14:30.
\end{itemize}
\subsection*{Presenze}
\subsubsection*{Interni}
\begin{itemize}
	\item \textbf{presenti: }
	\begin{itemize}
		\item Badan Antonio;
		\item Bertoldo Damiano;
		\item Budai Matteo;
		\item De Grandi Samuele;
		\item Piacere Ivan;
		\item Privitera Sara;
		\item Spigolon Daniele;
	\end{itemize}
\end{itemize}
\subsubsection*{Esterni}
\begin{itemize}
	\item \textbf{presenti: }
	\begin{itemize}
		\item Patera Lorenzo, referente di Imola Informatica;
		\item Nervegna Fabrizio, esperto di blockchain (Imola Informatica);
		\item Pagliara Emanuele, esperto di blockchain (Imola Informatica).
	\end{itemize}
\end{itemize}
\section{Svolgimento}
Viene fornito un resoconto della discussione tenuta con il proponente Imola Informatica in data 2021-03-19.

\subsection*{Presentazione PoC a Imola Informatica}
Il Proof of Concept realizzato per potersi candidare alla RP è stato presentato al proponente, il quale ha dato riscontro molto positivo sul lavoro svolto.

\subsection*{Discussione su Blockchain}
Si è concordato sul fatto di mettere sulla blockchain unicamente l'hash del report giornaliero evitando di implementare smart contract per vari motivi, tra cui:
\begin{enumerate}
	\item \textbf{privacy}: il nodo in futuro potrebbe essere pubblico e quindi non si vuole mettere in rete dati sensibili;
	\item \textbf{costi}: le transazioni hanno un costo e se ne vuole limitare il numero.
\end{enumerate}
Ne consegue, quindi, una semplificazione dell'implementazione della parte relativa allo sviluppo della blockchain.





