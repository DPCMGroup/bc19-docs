\section*{Tracciamento delle Decisioni}
Si noterà che alcuni dei temi trattati non hanno una decisione corrispondente. In alcuni casi, come per esempio l'ultimo punto, ciò è dovuto al fatto che il tema trattato non implica delle azioni da eseguire. In altri casi invece, come nella modifica della prenotazione, l'opinione del proponente ci permette di scegliere o meno di introdurre una funzionalità, e noi scegliamo di non introdurla.

\begin{center}
	\rowcolors{2}{lightest-grayest}{white}
	\begin{longtable}{|c|p{13cm}|}
	\hline
	\rowcolor{lighter-grayer}
	\textbf{Codice} & \textbf{Descrizione} \\
	\hline
	\endfirsthead

	\hline
	VE\_2021-02-22\_1 & Aggiungere come requisito opzionale nell'AdR un report dei dipendenti che hanno condiviso la stanza con un certo dipendente \\
	VE\_2021-02-22\_2 & Introdurre la seguente funzionalità nell'AdR: quando un dipendente occupa una postazione, essa viene automaticamente prenotata per tutto il giorno  \\
	VE\_2021-02-22\_3 & Introdurre come requisito nell'AdR il rilevamento dell'abbandono della postazione dopo 30 minuti di assenza del telefono dal tag \\
	VE\_2021-02-22\_4 & Introdurre come requisito nell'AdR una notifica 30 minuti prima dell'inizio di una prenotazione \\
	VE\_2021-02-22\_5 & Introdurre come requisito nell'AdR una notifica 5 minuti prima della fine della prenotazione di cui si sta usufruendo \\
	VE\_2021-02-22\_6 & Nell'AdR specificare che le prenotazioni avranno granularità di 1 ora \\
	VE\_2021-02-22\_7 & Nell'AdR aggiungere ai vari report delle occupazioni e delle igienizzazioni i seguenti dati:
	\begin{itemize}
		\item l'informazione leggibile;
		\item l'hash dell'informazione;
		\item l'indirizzo della transazioni nella quale risiede l'hash.
	\end{itemize} \\
	VE\_2021-02-22\_8 & Introdurre come requisito il supporto dell'applicazione mobile a partire dai sistemi operativi seguenti:\newline
	Android: 6.0 Marshmallow\newline
	iOS: iOS9 \\							
	\hline

	\end{longtable}
\end{center}