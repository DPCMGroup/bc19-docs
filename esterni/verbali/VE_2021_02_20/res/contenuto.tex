\section*{Introduzione}
\subsection*{Dettagli e-mail}
\begin{itemize}
	\item \textbf{Autore dell'e-mail per DPCM 2077:} Ivan Piacere;
	\item \textbf{Autore dell'e-mail per Imola Informatica:} Lorenzo Patera;
	\item \textbf{periodo:} 2021-02-20 - 2021-03-01.
\end{itemize}

\section{Svolgimento}
Viene fornito un resoconto della discussione incentrato sui chiarimenti dati dal proponente e suddiviso per tema trattato.


\subsection*{È necessario criptare o hashare le informazioni salvate su blockchain?}
Dobbiamo assumere di trovarci sempre su un sistema blockchain pubblico.
Sulla blockchain va salvato l'hash della nostra informazione, e, visto che da esso non si può ottenere l'informazione che l'ha generato, non è un problema che sia pubblico.

\subsection*{Può essere utile fornire un report dei vicini che un dipendente ha avuto ad ogni occupazione, magari dando la possibilità di scegliere quanto vicini (1, 2 o più postazioni di distanza, volendo anche tutti quelli che hanno condiviso la stanza con lui)?}
Possiamo pensare di implementarlo se avanza tempo. Lo dobbiamo considerare come requisito desiderabile/opzionale.

\subsection*{Gestione interferenza tra più prenotazioni/occupazioni}
Quando un utente occupa una postazione c'è il rischio che qualcun altro prenoti la stessa postazione a partire da un periodo di poco successivo. Per esempio, se alle 15:00 l'utente U1 sta occupando la postazione P, l'utente U2 può prenotarla a partire anche dalle 15:30. In questo caso U1 dovrà, dopo 30 minuti, liberare la postazione.
Per evitare questo problema avremmo potuto porre un intervallo di tempo, per esempio 1 ora, successivo all'inizio dell'occupazione, entro il quale nessun altro potesse prenotare la postazione.
Un'altra soluzione sarebbe quella di porre un lasso di tempo minimo tra il momento in cui prenoto e per cui prenoto. Per esempio, se il lasso fosse di 24 ore e ora fossero le 15:00, io potrei prenotare postazioni solo a partire dalle 15:00 di domani.
Abbiamo però pensato ad una soluzione più generale, che ti chiediamo di valutare: consideriamo le occupazioni come prenotazioni.
Quindi quando un utente intende occupare una postazione igienizzata, gli viene chiesto per quanto tempo la vuole occupare. Per la durata specificata la postazione risulterà prenotata dall'utente, e quindi non sarà disponibile per altre prenotazioni. Per esempio: l'utente U occupa la postazione P a partire dalle 15:00. Gli viene chiesto quanto a lungo la prenoterà e lui indica 3 ore. Allora la postazione P risulterà prenotata per 3 ore, o fino a quando U non la libererà, se dovesse farlo prima delle 3 ore.
Per rendere il processo di occupazione semplice ma anche efficace abbiamo pensato alla seguente funzionalità: \newline
Per cominciare un'occupazione l'utente deve scansionare il tag di una postazione. Se la postazione è igienizzata, l'applicazione chiede all'utente per quanto tempo vuole occuparla. A questo punto l'utente può indicare la durata oppure può semplicemente lasciare il telefono sul tag.
In questo secondo caso, dopo un intervallo di tempo, la prenotazione viene effettuata automaticamente impostando la durata a un certo valore.
Inoltre, quando l'utente sceglie quanto a lungo prenotare la postazione, abbiamo pensato che lo può fare in più modi:
\begin{enumerate}
\item indicando l'ora di fine della prenotazione (voglio occupare fino alle 12:30);
\item indicando la durata della prenotazione (voglio occupare per 3 ore);
\item devono essere presenti entrambe le modalità.
\end{enumerate}
\indent Il proponente ha detto che la proposta è valida, ma è sufficiente un sistema meno complicato: l'utente può prenotare una postazione per lo stesso giorno solo se questa risulta libera in quel momento. Bisogna infatti considerare che non è inusuale che in azienda una persona prenoti la postazione per 3 ore durante la mattina e poi si protragga in azienda per più tempo.

\subsection*{Come gestire l'abbandono della postazione da parte dell'utente}
La postazione va considerata abbandonata dopo un intervallo di 30 minuti da quando il telefono è stato sollevato dal tag. Questo intervallo deve poter essere modificato facilmente a livello di codice.

\subsection*{Notifica pre inizio prenotazione}
Sarebbe una cosa gradita. Nel caso si facesse, 30 minuti sarebbe un buon anticipo. \newline
Questo lasso di tempo deve comunque essere modificabile dall'amministratore.

\subsection*{Notifica pre fine prenotazione}
Sarebbe una cosa buona. 5 minuti andrebbero bene come intervallo. La durata dovrebbe essere una cosa modificabile dall'amministratore.

\subsection*{Deve essere possibile modificare una prenotazione? }
Per ogni postazione si potrebbero cambiare:
\begin{enumerate}
	\item dataora di inizio;
	\item dataora di fine;
	\item postazione prenotata.
\end{enumerate}	
Nel caso io stia usufruendo di una prenotazione, non potrei modificare la dataora di inizio né la postazione, ma potrei modificare la dataora di fine.\newline
Per il proponente la funzionalità sarebbe gradita.

\subsection*{Granularità orari prenotazioni}
Vanno bene granularità al livello delle ore o delle mezze giornate.
Un dipendente solitamente arriva, occupa ed è difficile che vada via per il resto della giornata, o almeno per la mezza giornata.

\subsection*{Come fornire dati certificati su occupazioni e igienizzazioni}
I report delle occupazioni e igienizzazioni devono fornire anche:
\begin{itemize}
	\item l'informazione leggibile;
	\item l'hash dell'informazione;
	\item l'indirizzo della transazione nella quale risiede l'hash.
\end{itemize}

\subsection*{Versioni di Android e iOS da supportare}
Android: 6.0 Marshmallow \newline
iOS: iOS9


\subsection*{Chiarimento sulla frase "scansione dei codici nel tempo sufficiente a certificare la presenza della persona in postazione" del documento di presentazione del capitolato}
Dobbiamo trovare, attraverso test, il migliore bilanciamento tra frequenza del rilevamento e consumo della batteria. Si devono ottenere frequenza e consumo ragionevoli. Sicuramente la frequenza di scansione deve essere inferiore all'ora e la batteria del telefono deve durare più di un'ora.


\subsection*{Disponibilità server aziendali}
Durante lo scambio di domande e risposte col proponente sono state fornite, ad ognuno di noi, le credenziali personali e le istruzioni per accedere a un server privato dell'azione tramite VPN.




