\section*{Introduzione}
\subsection*{Luogo e data dell'incontro}
\begin{itemize}
	\item \textbf{luogo:} videoconferenza su Meet;
	\item \textbf{data:} 2021-06-01;
	\item \textbf{ora di inizio:} 16:00;
	\item \textbf{ora di fine:} 16:30.
\end{itemize}
\subsection*{Presenze}
\subsubsection*{Interni}
\begin{itemize}
	\item \textbf{presenti: }
	\begin{itemize}
		\item Badan Antonio;
		\item Bertoldo Damiano;
		\item Budai Matteo;
		\item De Grandi Samuele;
		\item Piacere Ivan;
		\item Privitera Sara;
		\item Spigolon Daniele;
	\end{itemize}
\end{itemize}
\subsubsection*{Esterni}
\begin{itemize}
	\item \textbf{presenti: }
	\begin{itemize}
		\item Patera Lorenzo, referente di Imola Informatica.
	\end{itemize}
\end{itemize}
\section{Svolgimento}
Viene fornito un resoconto della discussione tenuta con il proponente Imola Informatica in data 2021-06-01.

\subsection*{Presentazione demo a Imola Informatica in vista del collaudo}
Il gruppo, in vista dell'imminente collaudo, ha pensato di fare vedere al proponente una demo dell'applicazione web e mobile.
Il proponente ha fornito un giudizio positivo sul lavoro svolto, dicendo che il prodotto soddisfa tutti i requisiti funzionali necessari. 
Nel caso rimanesse del tempo residuo prima del collaudo, verranno implementati i seguenti requisiti non necessari per l'applicazione mobile: scansione automatica tag e notifiche push.
Per quanto riguarda i report di igienizzazione / occupazione dell'applicazione web, ci è stato confermato che non è necessario scaricare un file in formato PDF.
Un consiglio è stato di migliorare leggermente la parte grafica per una migliore leggibilità delle informazioni, non mostrando solo etichette di testo.






