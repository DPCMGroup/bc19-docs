\section*{Introduzione}
\subsection*{Dettagli e-mail}
\begin{itemize}
	\item \textbf{Autore dell'e-mail per DPCM 2077:} Ivan Piacere;
	\item \textbf{Autore dell'e-mail per Imola Informatica:} Lorenzo Patera;
	\item \textbf{periodo:} 2021-02-20 - 2021-03-01;
\end{itemize}

\section*{Svolgimento}



\subsection*{Tema}
1. Dobbiamo fare in modo che le informazioni caricate sulla blockchain siano criptate, o anche solo hashate, in modo che se in futuro si dovesse usare una rete pubblica invece di una privata, le nostre informazioni non potrebbero essere lette da altri?

\subsubsection{Risposta}
1. Assumete di trovarvi sempre su un sistema blockchain pubblico. Quello che ci interessa è avere un hash (riassunto one-way) dei dati salvato a cadenza "regolare" (decidete voi, anche singola transazione) sulla blockchain in modo da avere una evidenza immutabile che quel dato è stato prodotto in un determinato giorno-orario. Il controllo è demandato a "terzi", ovvero, vi dovete occupare di conservare il documento che genera l'hash, l'hash (giusto per un controllo di coerenza, essendo generabile a partire dal documento) e un link alla transazione blockchain che lo contiene, così da consentire la ricerca e l'evidenza che quel particolare hash è presente in quella transazione. Sulla blockchain pubblica non è un problema che ci sia l'hash in chiaro visto che non consente di tornare al documento originale.
\subsection*{Tema}
2. Può essere utile fornire un report dei vicini che un dipendente ha avuto ad ogni occupazione, magari dando la possibilità di scegliere quanto vicini (1, 2 o più postazioni di distanza, volendo anche tutti quelli che hanno condiviso la stanza con lui)?
\subsubsection{Risposta}
2. Non è nei requisiti. Sicuramente può essere utile, ma lasciatelo come requisito desiderabile/opzionale. Se vi avanza tempo può essere una bella espansione.

\subsection*{Tema}
3. Quando un utente occupa una postazione c'è il rischio che qualcun altro prenoti la stessa postazione a partire da un periodo di poco successivo. Per esempio, se alle 15:00 l'utente U1 sta occupando la postazione P, l'utente U2 può prenotarla a partire anche dalle 15:30. In questo caso U1 dovrà, dopo 30 minuti, liberare la postazione.
Per evitare questo problema avremmo potuto porre un intervallo di tempo, per esempio 1 ora, successivo all'inizio dell'occupazione, entro il quale nessun altro potesse prenotare la postazione.
Un altra soluzione sarebbe quella di porre un lasso di tempo minimo tra il momento in cui prenoto e per cui prenoto. Per esempio, se il lasso fosse di 24 ore e ora fossero le 15:00, io potrei prenotare postazioni solo a partire dalle 15:00 di domani.
Abbiamo però pensato ad una soluzione più generale, che ti chiediamo di valutare: consideriamo le occupazioni come prenotazioni.
Quindi quando un utente intende occupare una postazione igienizzata, gli viene chiesto per quanto tempo la vuole occupare. Per la durata specificata la postazione risulterà prenotata dall'utente, e quindi non sarà disponibile per altre prenotazioni. Per esempio: l'utente U occupa la postazione P a partire dalle 15:00. Gli viene chiesto quanto a lungo la prenoterà e lui indica 3 ore. Allora la postazione P risulterà prenotata per 3 ore, o fino a quando U non la libererà, se dovesse farlo prima delle 3 ore.
Per rendere il processo di occupazione semplice ma anche efficace abbiamo pensato ad altre funzionalità:
1- Per cominciare un'occupazione l'utente deve scansionare il tag di una postazione. Se la postazione è igienizzata, l'applicazione chiede all'utente per quanto tempo vuole occuparla. A questo punto l'utente può indicare la durata oppure può semplicemente lasciare il telefono sul tag.
In questo secondo caso, dopo un (1) intervallo di tempo , la prenotazione viene effettuata automaticamente impostando la (2) durata a un certo valore. Potresti indicarci dei valori di riferimento per l'intervallo(1) prima della prenotazione automatica e per la sua durata(2)?
Inoltre, quando l'utente sceglie quanto a lungo prenotare la postazione, abbiamo pensato che lo può fare in più modi:
a- fino a quando (voglio occupare fino alle 12:30)
b- per quanto tempo (voglio occupare per 3 ore)
c- come preferisce, devono essere presenti entrambe le modalità
Qual è la possibilità migliore?
2- Se durante l'occupazione l'utente solleva il telefono dal tag gli viene chiesto se se ne sta andando. Se risponde sì la sua occupazione viene chiusa. Se invece dice di no, ci sono due possibilità:
a- la prenotazione viene mantenuta attiva al massimo fino alla sue fine naturale, indipendentemente che l'utente torni o meno
b- la prenotazione viene mantenuta attiva per un certo intervallo di tempo, ma se l'utente non riporta il telefono sul tag entro questo intervallo l'occupazione viene chiusa.
Quale delle due possibilità è preferibile?
Un altro modo di gestire l'abbandono prematuro della postazione consisterebbe nel non occuparsene affatto, e quindi non chiedere all'utente se se ne sta andando, ma mantenere sempre la prenotazione attiva fino alla sua fine naturale. Questo sistema potrebbe portare ad un utilizzo poco efficiente delle postazioni ma te lo proponiamo comunque, nel caso questo tipo di efficienza non sia di importanza primaria.
\subsubsection{Risposta}
3. Capisco le perplessità con il sistema di occupazione/prenotazione postazioni. Direi di farlo più facile di come lo state pensando: l'utente può prenotare una postazione per lo stesso giorno solo se questa risulta libera in quel momento. Non è inusuale che in azienda una persona prenoti la postazione per 3 ore durante la mattina e poi si protragga in azienda per più tempo. Attualmente il nostro sistema di prenotazione "basic" prevede di prenotare la presenza per l'intera giornata. Tutti i ragionamenti che mi avete descritto sono sensati, se volete "complicare" la proposta potete sviluppare quello che ritenete più opportuno. Riguardo alle "conferme" di checkout postazione, direi di gestirle tramite timeout (personalizzabili). Se un utente ha il cell scarico non vogliamo che resti per sempre sulla postazione, così come "gli sciattoni" devono risultare assenti dopo un lasso di tempo.

\subsection*{Tema}
4. Oltre alle questioni sopra riportate ce ne sono alcune riguardanti le prenotazioni in generale:
1- L'utente deve ricevere una notifica quando l'inizio di una sua prenotazione si sta avvicinando? quanto tempo prima?
\subsubsection{Risposta}
4.1 - Gradita, non nel capitolato. In generale 30 minuti direi che mi sembra un buon lasso di tempo. Ovviamente fate in modo che sia facilmente modificabile :)

\subsection*{Tema}
2- Quando la fine della prenotazione si sta avvicinando, l'utente deve esserne avvisato? Quanto tempo prima della fine questo deve succedere?
\subsubsection{Risposta}
4.2 - Vedi 3, il sistema di prenotazione lo farei più semplice di come lo state pensando. In generale in una buona applicazione l'utente dovrebbe sempre essere avvisato.\\

\subsection*{Tema}
3- Abbiamo pensato di implementare anche la possibilità di modificare le prenotazioni. Per ogni postazione quindi si potrebbero cambiare:
a- dataora di inizio
b- dataora di fine
c- postazione prenotata
Nel caso io stia usufruendo di una prenotazione, non potrò modificare la dataora di inizio né la postazione, ma potrò modificare la dataora di fine.
Questa funzionalità può andare bene?
\subsubsection{Risposta}
4.3 - Come prima, funzionalità gradita.\\

\subsection*{Tema}
4- Con quale granularità vanno gestiti gli orari delle prenotazioni? Abbiamo pensato di gestirle con granularità al livello dei minuti. Per esempio possono prenotare dalle 15:34 alle 18:22. Può andare bene?
\subsubsection{Risposta}
4.4 - Granularità molto più larghe vanno benissimo. Anche di ore o sulla mezza giornata. Non stiamo facendo uno scheduler dei processi, tenete presente che un dipendente solitamente arriva, occupa ed è difficile che vada via per il resto della giornata, o almeno per la mezza giornata.


\subsection*{Tema}
5. Nel caso qualcuno voglia avere un certificato dei dati di sanificazione e occupazione, come si deve procedere? Sono sufficienti i report di cui si parla nel capitolato? ovvero le seguenti 2 tabelle:
- utilizzo (utente - postazione - dataora inizio utilizzo - dataora fine utilizzo - durata utilizzo)
- pulizia (utente che ha effettuato la pulizia - ruolo utente(dipendente/addetto) - postazione pulita - dataora esecuzione pulizia)
Lo chiediamo perché questo non fornisce alcuna indicazione che i dati siano salvati su una blockchain. E' forse necessario fornire gli hash dei blocchi della blockchain?
\subsubsection{Risposta}
5. In parte vi ho risposto in 1. Nei report serve sapere dove trovare la transazione e il documento originale che genera l'hash.

\subsection*{Tema}
6. Quali versioni di Android e iOS dobbiamo supportare?
\subsubsection{Risposta}
6. Ragionevoli per cui un device vecchio di 4-5 anni possa funzionare.


\subsection*{Tema}
7. Nel capitolato si parla di "scansione dei codici nel tempo sufficiente a certificare la presenza della persona in postazione". Cosa si intende per tempo sufficiente? E' un limite superiore entro il quale non possiamo assolutamente salire? Se sì, a quanto corrisponde?
\subsubsection{Risposta}
7. Qui dovete testare qual è il migliore il trade-off per bilanciare il conoscere con esattezza la posizione di un dipendente (monitoring continuo) e la durata della batteria (non monitoro mai). Si intende che dovete controllare a intervalli larghi abbastanza da non spegnere il cellulare in 1 ora e stretti abbastanza da sapere che il dipendente è andato via, ma con un margine non di ore.


\subsection*{Tema}
Inoltre avremmo un'altra richiesta, indipendente dal prodotto:
A breve (circa una settimana), dovremo far provare il primo prototipo ai professori e per farlo dovremo ospitare la parte backend su un server pubblico. Sul documento di presentazione del capitolato avete indicato che ci potete fornire dei server per l'installazione dei prodotti. Potresti darci più informazioni al riguardo, e magari, se sono già disponibili, dirci come accedervi?
\subsubsection{Risposta}
Possiamo fornirvi un server. Abbiamo bisogno di nome e cognome e di un indirizzo email di ogni componente del gruppo. Vi creeremo un utenza sulla nostra VPN che vi consentirà di accedere al server.




