\section{Casi d'uso}

\subsection{Attori dei casi d'uso}

\subsubsection{Attori primari}
\begin{figure}[H]
		\centering
		\includegraphics[width=10cm]{res/images/utentigenerali.png}
		\caption{Gerarchia degli attori principali}
		\label{fig:Gerarchia attori principali}
	\end{figure}

\textbf{Utente generico}\\
Si riferisce ad un utente generico che accede alla piattaforma.\\
\\
\textbf{Utente non autenticato}\\
Si riferisce ad un utente generico che non ha ancora effettuato l’autenticazione alla piattaforma.\\
\\
\textbf{Utente autenticato}\\
Si riferisce ad un utente generico che si è autenticato nel sistema con la procedura di login.\\
\\
\textbf{Amministartore}\\
Si riferisce ad un utente che si è autenticato nel sistema con il ruolo di amministratore.\\
\\
\textbf{Dipendente}\\
Si riferisce ad un utente che si è autenticato nel sistema con il ruolo di dipendente.\\
\\
\textbf{Addetto pulizie}\\
Si riferisce ad un utente che si è autenticato nel sistema con il ruolo di addetto alle pulizie.\\

\subsubsection{Attori secondari}
\textbf{Tag RFID}\\
Si riferisce ai tag presenti sulle postazioni. Ogni tag è assegnato ad una postazione e permette di identificarla digitalmente.\\
\\
\textbf{Ethereum}\\
Si riferisce al servizio che memorizza su blockchain gli eventi di occupazione e igienizzazione delle postazioni.\\

\subsection{Elenco dei casi d'uso}
In questa sezione sono riportati tutti i casi d'uso individuati, divisi per attore principale. Quando ritenuto utile essi sono accompagnati da un grafico.
\\
\subsubsection{ UC1 - Guida introduttiva}
\begin{itemize}
           	\item\textbf{Attori Primari:} utente generico.
           	\item\textbf{Descrizione:} l'utente riceve una guida riguardo il login e il funzionamento.
           	\item\textbf{Scenario principale:} l’utente accede alla pagina introduttiva e visualizza la guida.
           	\item\textbf{Precondizione:} il sistema è raggiungibile e funzionante, l’utente accede alla pagina iniziale del sito della piattaforma.
           	\item\textbf{Postcondizione:} il sistema fornisce all’utente, attraverso la lettura della guida, tutte le istruzioni necessarie ad effettuare il login.
\end{itemize}

\subsubsection{ UC2 - Login}
\begin{itemize}
           	\item\textbf{Attori Primari:} utente non autenticato.
           	\item\textbf{Descrizione:} l’utente tenta di autenticarsi all'applicativo.
           	\item\textbf{Scenario principale:} l’utente non è ancora autenticato ed esegue il login.
           	\item\textbf{Precondizione:} l’utente non si è autenticato nell'applicativo. 
           	\item\textbf{Postcondizione:} l’utente si è autenticato con successo, ed è stato identificato dal sistema
           	nel ruolo di amministratore, dipendente o addetto alle pulizie. A seconda della tipologia di utente vengono rese
           	disponibili diverse funzionalità.
\end{itemize}
\subsubsection{ UC3 - Logout}
\begin{itemize}
	\item\textbf{Attori Primari:} 
	\item\textbf{Descrizione:} 
	\item\textbf{Scenario principale:} 
	\item\textbf{Precondizione:} 
	\item\textbf{Postcondizione:}
\end{itemize}
\subsubsection{ UC3.2 - Visualizzazione messaggio di errore relativo a chiave non registrata}
\begin{itemize}
	\item\textbf{Attori Primari:} utente non autenticato.
	\item\textbf{Descrizione:} l'utente visualizza un messaggio di errore dovuto al fatto che ha tentato il login senza essersi registrato in precedenza.
	\item\textbf{Scenario principale:} l’utente tenta di eseguire la procedura di login all'applicativo senza essere registrato.
	\item\textbf{Precondizione:} l'utente tenta di autenticarsi nell'applicativo.
	\item\textbf{Postcondizione:} viene visualizzato un messaggio di errore per informare l'utente del fatto che è necessario registrarsi
	all'applicativo prima di poter effettuare la procedura di login.
\end{itemize}
\subsubsection{ UC3.3 - Visualizzazione schermata relativa a utente non abilitato}
\begin{itemize}
           	\item\textbf{Attori Primari:} utente non autenticato.
           	\item\textbf{Descrizione:} l'utente tenta di autenticarsi nell'applicativo, tuttavia a causa della disabilitazione del suo account, il login viene interrotto, e
           	l'utente visualizza il messaggio di errore che illustra la causa della disabilitazione dell'account.
           	\item\textbf{Scenario principale:} l’utente non autenticato con account disabilitato tenta di autenticarsi. 
           	La procedura di autenticazione viene bloccata a causa dello stato dell'account.
           	\item\textbf{Precondizione:} un utente non autenticato, registrato all'applicativo e con account disabilitato tenta di effettuare il login automatico. 
           	\item\textbf{Postcondizione:} viene visualizzato un messaggio di errore per informare l'utente del fatto che il 
           	proprio account è stato disabilitato dall'amministratore. Se quest'ultimo, durante la procedura di disabilitazione, ha inserito un messaggio contenente la causa di tale azione, allora tale messaggio viene visualizzato.
\end{itemize}

\subsubsection{ UC4 - Gestione impostazioni}
\begin{itemize}
           	\item\textbf{Attori Primari:} 
           	\item\textbf{Descrizione:} 
           	\item\textbf{Scenario principale:} 
           	\item\textbf{Precondizione:} 
           	\item\textbf{Postcondizione:}
\end{itemize}

\subsubsection{ UC5 -  Gestione stanze e postazioni}
\begin{itemize}
           	\item\textbf{Attori Primari:} 
           	\item\textbf{Descrizione:} 
           	\item\textbf{Scenario principale:} 
           	\item\textbf{Precondizione:} 
           	\item\textbf{Postcondizione:}
\end{itemize}

\subsubsection{ UC6 - Gestione credenziali dipendenti e addetti}
\begin{itemize}
           	\item\textbf{Attori Primari:} 
           	\item\textbf{Descrizione:} 
           	\item\textbf{Scenario principale:} 
           	\item\textbf{Precondizione:} 
           	\item\textbf{Postcondizione:}
\end{itemize}

\subsubsection{ UC7 - Esplorazione postazioni occupate da specifico utente}
\begin{itemize}
           	\item\textbf{Attori Primari:} 
           	\item\textbf{Descrizione:} 
           	\item\textbf{Scenario principale:} 
           	\item\textbf{Precondizione:} 
           	\item\textbf{Postcondizione:}
\end{itemize}

\subsubsection{ UC8 - Report sanificazioni postazione}
\begin{itemize}
           	\item\textbf{Attori Primari:} 
           	\item\textbf{Descrizione:} 
           	\item\textbf{Scenario principale:} 
           	\item\textbf{Precondizione:} 
           	\item\textbf{Postcondizione:}
\end{itemize}

\subsubsection{ UC9 - Visualizzazione stato postazione}
\begin{figure}[H]
	\centering
	\includegraphics[width=15cm]{res/images/UC9.png}
	\caption{Visualizzazione stato postazione}
	\label{fig:Visualizzazione stato postazione}
\end{figure}
\begin{itemize}
           	\item\textbf{Attori Primari:} dipendente.
           	\item\textbf{Attori Secondari:} tagRFID.
           	\item\textbf{Descrizione:} l’utente scansiona il tag RFID presente alla postazione per ricevere indicazioni sullo stato della stessa.
           	\item\textbf{Scenario principale:} l’utente appoggia lo smartphone sul tag RFID e visualizza nell'applicazione mobile lo stato della postazione.
           	\item\textbf{Precondizione:} l’utente scansiona con il proprio smarthphone il tag RFID della postazione desiderata e visualizza il suo stato.
           	\item\textbf{Postcondizione:} l’utente conosce lo stato della postazione.
           	\item\textbf{Estensioni:}
           	\begin{itemize}
           		\item[$-$]  Visualizzazione errore: lo smartphone è troppo distante dal tag RFID per effettuare correttamente la scansione (UC 9.2).
           	\end{itemize}
\end{itemize}
\subsubsection{ UC9.1 - Scansione tag RFID}
\begin{itemize}
	\item\textbf{Attori Primari:} dipendente.
	\item\textbf{Attori Secondari:} tagRFID.
	\item\textbf{Descrizione:} l’utente scansiona il tag RFID presente alla postazione per ricevere indicazioni sullo stato della stessa.
	\item\textbf{Scenario principale:} l’utente appoggia lo smartphone sul tag RFID.
	\item\textbf{Precondizione:} l’utente si è autenticato nell'applicazione mobile e accede alla sezione per la scansione del sensore.
	\item\textbf{Postcondizione:} l’utente ha effettuato correttamente la scansione.
\end{itemize}
\subsubsection{ UC9.2 - Visualizzazione errore scansione: tag RFID troppo distante}
\begin{itemize}
	\item\textbf{Attori Primari:} dipendente.
	\item\textbf{Descrizione:} l’utente visualizza un messaggio di errore in quanto tenta di effettuare una scansione del tag RFID, ma appoggia il telefono non abbastanza vicino al sensore.
	\item\textbf{Scenario principale:} viene visualizzato un messaggio di errore che segnala che lo smartphone non è stato appoggiato in un punto abbastanza vicino
	al tag RFID.
	\item\textbf{Precondizione:} l’utente ha tentato di effettuare una scansione del tag RFID.
	\item\textbf{Postcondizione:} viene visualizzato un messaggio di errore specifico.
\end{itemize}
\subsubsection{ UC9.3 - Visualizzazione stato}
\begin{itemize}
	\item\textbf{Attori Primari:} dipendente.
	\item\textbf{Descrizione:} l’utente visualizza uno dei possibili stati della postazione:
	\begin{itemize}
	\item[$-$]libera e igienizzata;
	\item[$-$]libera e non igienizzata;
	\item[$-$]occupata;
	\item[$-$]non accessibile.
    \end{itemize}
	\item\textbf{Scenario principale:} l’utente visualizza nell'applicazione mobile lo stato della postazione.
	\item\textbf{Precondizione:} l’utente sta navigando nella sezione per la visualizzazione dello stato della postazione.
	\item\textbf{Postcondizione:} l’utente conosce lo stato della postazione.
\end{itemize}

\subsubsection{ UC10 - Segnalazione presenza}
\begin{figure}[H]
	\centering
	\includegraphics[width=15cm]{res/images/UC10.png}
	\caption{Segnalazione presenza}
	\label{fig:Segnalazione presenza}
\end{figure}
\begin{itemize}
           	\item\textbf{Attori Primari:} dipendente.
           	\item\textbf{Attori Secondari:} tagRFID, Ethereum.
           	\item\textbf{Descrizione:} l’utente segnala in tempo reale la propria presenza alla postazione, appoggiando il cellulare sul tag RFID. Inoltre, registra il tempo di inizio e di fine di questa azione grazie all'utilizzo di Ethereum.
           	\item\textbf{Scenario principale:} l’utente appoggia lo smartphone sul tag RFID e segnala in questo modo la propria presenza.
           	Il tempo di inizio e di fine di questa azione vengono memorizzati grazie all'utilizzo di Ethereum.
           	\item\textbf{Precondizione:} l’utente si è autenticato nell'applicazione mobile e occupa una postazione.
           	\item\textbf{Postcondizione:} l’utente segnala in tempo reale la presenza alla postazione e registra questa azione su Ethereum.
           	\item\textbf{Estensioni:} 
           	\begin{itemize}
           		\item[$-$] Visualizzazione errore: se sposto lo smartphone dal tag RFID (ad esempio per una telefonata), 
           		dopo un timeout di qualche minuto(ad esempio 10 minuti), viene visualizzato un messaggio di errore (UC 10.2).
           		\item[$-$] Visualizzazione errore: se l'utente si dimentica di premere il bottone della fine dell'occupazione della postazione,
           		viene visualizzato un messaggio di errore (UC 10.4).
           	\end{itemize}
\end{itemize}
\subsubsection{ UC10.1 - Presenza in tempo reale}
\begin{itemize}
	\item\textbf{Attori Primari:} dipendente.
	\item\textbf{Attori Secondari:} tagRFID.
	\item\textbf{Descrizione:} l’utente segnala in tempo reale la presenza alla postazione, appoggiando il cellulare sul tagRFID.
	\item\textbf{Scenario principale:} l’utente appoggia lo smartphone sul tagRFID e segnala in questo modo la propria presenza in tempo reale.
	\item\textbf{Precondizione:} l’utente occupa una postazione e ha attivitato nel proprio smarthphone la possibilità di effettuare scansioni con tagRFID.
	\item\textbf{Postcondizione:} l’utente segnala in tempo reale la presenza alla postazione.
\end{itemize}
\subsubsection{ UC10.2 - Visualizzazione errore spostamento smartphone}
\begin{itemize}
	\item\textbf{Attori Primari:} dipendente.
	\item\textbf{Descrizione:} l’utente visualizza un messaggio di errore in quanto sposta lo smartphone dal tag RFID.
	\item\textbf{Scenario principale:} viene visualizzato un messaggio di errore che consiglia di riposizionare lo smarthpone correttamente sul tag RFID.
	\item\textbf{Precondizione:} l’utente sposta il proprio smarthphone dal tag RFID.
	\item\textbf{Postcondizione:} viene visualizzato un messaggio di errore specifico.
\end{itemize}
\subsubsection{ UC10.3 - Registrazione presenza}
\begin{itemize}
	\item\textbf{Attori Primari:} dipendente.
	\item\textbf{Attori Secondari:} Ethereum.
	\item\textbf{Descrizione:} l’utente registra l'inizio e la fine della propria occupazione della postazione.
	\item\textbf{Scenario principale:} l’utente sta navigando nella sezione della segnalazione della presenza e preme i bottoni di inizio e fine occupazione della postazione,
	nei corrispondenti momenti. Questa azione viene registrata grazie a Ethereum.
	\item\textbf{Precondizione:} l’utente occupa una postazione e preme i bottoni di inizio e fine dell'occupazione della postazione.
	\item\textbf{Postcondizione:} l'azione della precondizione viene registrata.
\end{itemize}
\subsubsection{ UC10.4 - Visualizzazione errore fine occupazione}
\begin{itemize}
	\item\textbf{Attori Primari:} dipendente.
	\item\textbf{Descrizione:} l’utente visualizza un messaggio di errore in quanto si dimentica di segnalare la fine dell'occupazione della postazione.
	\item\textbf{Scenario principale:} dopo un periodo di tempo abbastanza lungo, viene visualizzato un messaggio di errore in cui si notifica
	all'utente la fine della sua occupazione della postazione.
	\item\textbf{Precondizione:} l’utente non preme il bottone per la fine dell'occupazione della postazione.
	\item\textbf{Postcondizione:} viene visualizzato un messaggio di errore specifico.
\end{itemize}
\subsubsection{ UC11 - Pulizia autonoma}
\begin{figure}[H]
	\centering
	\includegraphics[width=15cm]{res/images/UC11.png}
	\caption{Pulizia autonoma}
	\label{fig:Pulizia autonoma}
\end{figure}
\begin{itemize}
	\item\textbf{Attori Primari:} dipendente.
	\item\textbf{Attori Secondari:} Ethereum.
	\item\textbf{Descrizione:} l’utente pulisce in autonomia la postazione e lo segnala sull'applicazione mobile. Inoltre, 
	registra di aver effettuato questa azione, grazie all'utilizzo di Ethereum.	\item\textbf{Scenario principale:} l’utente dopo aver igienizzato la postazione con il kit aziendale, modifica lo stato della postazione in libera e igienizzata. Inoltre, 
	registra di aver effettuato questa azione, grazie all'utilizzo di Ethereum.
	\item\textbf{Precondizione:} l’utente è autenticato nell'applicazione mobile, ha igienizzato la propria postazione e sta navigando nella sezione per la pulizia 
	autonoma della postazione.
	\item\textbf{Postcondizione:} l’utente modifica lo stato della postazione in libera e igienizzata e registra l'avvenuta pulizia autonoma della postazione.
\end{itemize}
\subsubsection{ UC11.1 - Modifica stato postazione }
\begin{itemize}
	\item\textbf{Attori Primari:} dipendente.
	\item\textbf{Descrizione:} l’utente dopo aver igienizzato autonomamente la postazione con il kit aziendale, modifica lo stato della stessa.
	\item\textbf{Scenario principale:} l’utente modifica lo stato della postazione in libera e igienizzata.
	\item\textbf{Precondizione:} l’utente sta navigando all'interno della sezione pulizia autonoma delle postazioni.
	\item\textbf{Postcondizione:} l’utente modifica lo stato della postazione.
\end{itemize}

\subsubsection{ UC11.2 - Registrazione pulizia autonoma }
\begin{itemize}
           	\item\textbf{Attori Primari:} dipendente.
           	\item\textbf{Attori Secondari:} Ethereum.
           	\item\textbf{Descrizione:} l’utente segnala all'applicazione mobile di aver igienizzato autonomamente la postazione con il kit aziendale.
           	\item\textbf{Scenario principale:} l’utente igienizza la postazione con il kit aziendale e registra questa azione grazie all'utilizzo di Ethereum.
           	\item\textbf{Precondizione:} l’utente è autenticato nell'applicazione mobile, ha pulito la postazione e sta navigando nella sezione di 
           	registrazione della pulizia autonoma.
           	\item\textbf{Postcondizione:} l’utente registra l'avvenuta igienizzazione autonoma della postazione.
\end{itemize}
\subsubsection{ UC12 - Gestione prenotazione }
\begin{figure}[H]
	\centering
	\includegraphics[width=15cm]{res/images/UC12.png}
	\caption{Gestione prenotazione}
	\label{fig:Gestione prenotazione}
\end{figure}
\begin{itemize}
	\item\textbf{Attori Primari:} dipendente.
	\item\textbf{Descrizione:} l’utente vuole prenotare una postazione di una stanza in un determinato giorno e orario.
	\item\textbf{Scenario principale:} l’utente ha visualizzato o gestito le prenotazioni di una stanza all’interno del sistema.
	\item\textbf{Precondizione:} l’utente si è autenticato nell'applicazione mobile e sta navigando nella sezione di gestione delle prenotazioni.
	\item\textbf{Postcondizione:} l’utente ha visualizzato o gestito le prenotazioni all’interno del sistema.
\end{itemize}
\subsubsection{ UC12.1 - Cerca postazione}
\begin{itemize}
	\item\textbf{Attori Primari:} dipendente.
	\item\textbf{Descrizione:} l’utente vuole prenotare una postazione di una stanza in un determinato giorno e orario.
	\item\textbf{Scenario principale:} 
	\begin{itemize}
		\item[$-$] L’utente inserisce la data (UC 10.1.1);
		\item[$-$] L’utente inserisce l'orario (UC 10.1.3);
		\item[$-$] L’utente inserisce l'identificativo della stanza (UC 10.1.5);
	\end{itemize}
	\item\textbf{Precondizione:} l’utente sta navigando nella sezione dell'applicazione dedicata alla ricerca di una postazione.
	\item\textbf{Postcondizione:} l’utente sta navigando nella sezione dell'applicazione dedicata alla ricerca di una postazione.
	\item\textbf{Estensioni:} l’utente inserisce i campi necessari alla ricerca della postazione desiderata.
	\begin{itemize}
		\item[$-$] L’utente inserisce una data non valida (UC 10.1.2);
		\item[$-$] L’utente inserisce un orario non valida (UC 10.1.4);
		\item[$-$] L’utente inserisce una stanza non valida (UC 10.1.6).
	\end{itemize}
\end{itemize}
\subsubsection{ UC12.1.1 - Inserimento data }
\begin{itemize}
	\item\textbf{Attori Primari:} dipendente.
	\item\textbf{Descrizione:} l’utente sta cercando una postazione disponibile e deve compilare il campo della data.
	\item\textbf{Scenario principale:} l’utente ha compilato il campo data.
	\item\textbf{Precondizione:} l’utente sta compilando i campi richiesti per effettuare la ricerca.
	\item\textbf{Postcondizione:} l’utente ha compilato il campo richiesto.
\end{itemize}
\subsubsection{ UC12.1.2 - Visualizzazione errore: data non valida  }
\begin{itemize}
	\item\textbf{Attori Primari:} dipendente.
	\item\textbf{Descrizione:} l’utente ha compilato un campo per la data, ma è in un formato non corretto.
	\item\textbf{Scenario principale:} 
	\begin{itemize}
		\item[$-$] L’utente ha inserito il campo per la data;
		\item[$-$] Il sistema elabora la richiesta;
		\item[$-$] Viene visualizzato un errore che segnala che la data non è valida.
	\end{itemize}
	\item\textbf{Precondizione:} l’utente ha compilato il campo per la data.
	\item\textbf{Postcondizione:} l’utente visualizza un messaggio di errore specifico e non porta a termine l’azione.
\end{itemize}
\subsubsection{ UC12.1.3 - Inserimento ora }
\begin{itemize}
	\item\textbf{Attori Primari:} dipendente.
	\item\textbf{Descrizione:} l’utente sta cercando una postazione disponibile e deve compilare il campo orario.
	\item\textbf{Scenario principale:} l’utente ha compilato il campo orario.
	\item\textbf{Precondizione:} l’utente sta compilando i campi richiesti per effettuae la ricerca.
	\item\textbf{Postcondizione:} l’utente ha compilato il campo richiesto.
\end{itemize}
\subsubsection{ UC12.1.4 - Visualizzazione errore: orario non valido }
\begin{itemize}
	\item\textbf{Attori Primari:} dipendente.
	\item\textbf{Descrizione:} l’utente ha compilato un campo per l'orario, ma è in un formato non corretto.
	\item\textbf{Scenario principale:} 
	\begin{itemize}
		\item[$-$] L’utente ha inserito il campo per l'orario;
		\item[$-$] Il sistema elabora la richiesta;
		\item[$-$] Viene visualizzato un errore che segnala che l'orario non è valido.
	\end{itemize}
	\item\textbf{Precondizione:} l’utente ha compilato il campo per l'orario.
	\item\textbf{Postcondizione:} l’utente visualizza un messaggio di errore specifico e non porta a termine l’azione.
\end{itemize}
\subsubsection{ UC12.1.5 - Inserimento stanza }
\begin{itemize}
	\item\textbf{Attori Primari:} dipendente.
	\item\textbf{Descrizione:} l’utente sta cercando una postazione disponibile e deve compilare il campo identificativo della stanza.
	\item\textbf{Scenario principale:} l’utente ha compilato il campo stanza.
	\item\textbf{Precondizione:} l’utente sta compilando i campi richiesti per effettuare la ricerca.
	\item\textbf{Postcondizione:} l’utente ha compilato il campo richiesto.
\end{itemize}
\subsubsection{ UC12.1.6 - Visualizzazione errore: stanza non valida }
\begin{itemize}
	\item\textbf{Attori Primari:} dipendente.
\item\textbf{Descrizione:} l’utente ha compilato un campo per l'identificativo della stanza, ma è in un formato non corretto.
\item\textbf{Scenario principale:} 
\begin{itemize}
	\item[$-$] L’utente ha inserito il campo per la stanza;
	\item[$-$] Il sistema elabora la richiesta;
	\item[$-$] Viene visualizzato un errore che segnala che l'identificativo della stanza non è valido.
\end{itemize}
\item\textbf{Precondizione:} l’utente ha compilato il campo per la stanza.
\item\textbf{Postcondizione:} l’utente visualizza un messaggio di errore specifico e non porta a termine l’azione.
\end{itemize}
\subsubsection{ UC12.2 - Visualizzazione postazioni  }
\begin{itemize}
	\item\textbf{Attori Primari:} dipendente.
	\item\textbf{Descrizione:} l’utente visualizza le postazioni con colori diversi in base allo stato. L'identificativo di una postazione è costituito da una lettera seguita da un numero (ad esempio: D10). 
	\item\textbf{Scenario principale:} viene visualizzato lo schema delle postazioni rappresentato con colori diversi in base allo stato in cui si trovano.
	\item\textbf{Precondizione:} l’utente ha inserito i campi data, orario e stanza correttamente e premuto il bottone di ricerca, nella sezione 
	medesima.
	\item\textbf{Postcondizione:} l’utente visualizza le postazioni nella stanza.
	\item\textbf{Estensioni:}
	\begin{itemize}
		\item[$-$] Visualizzazione errore: tutte le postazioni nella stanza selezionata sono occupate o guaste.(UC 12.3);
	\end{itemize}
\end{itemize}
\subsubsection{ UC12.3 - Visualizzazione errore: stanze occupate }
\begin{itemize}
	\item\textbf{Attori Primari:} dipendente.
	\item\textbf{Descrizione:} l’utente visualizza un messaggio di errore in quanto non ci sono postazioni prenotabili.
	\item\textbf{Scenario principale:} 
	\begin{itemize}
		\item[$-$] L’utente ha inserito i campi data, orario, stanza;
		\item[$-$] Il sistema elabora la richiesta;
		\item[$-$] Viene visualizzato un messaggio di errore che consiglia di selezionare un'altra fascia oraria.
	\end{itemize}
	\item\textbf{Precondizione:} l’utente ha compilato i campi data, orario, stanza oppure ha selezionato la sezione prima postazione 
	disponibile (UC 12.4).
	\item\textbf{Postcondizione:} l’utente visualizza un messaggio di errore specifico e non porta a termine l’azione.
\end{itemize}
\subsubsection{ UC12.4(opzionale) - Prima postazione libera }
\begin{itemize}
	\item\textbf{Attori Primari:} dipendente.
	\item\textbf{Descrizione:} l’utente riceve l'identificativo di una postazione igienizzata e libera in una determinata stanza, 
	in modo automatico e in base ad un criterio di distanziamento dalle altre postazioni.
	\item\textbf{Scenario principale:} viene fornito l'identificativo di una postazione libera e igienizzata, oppure di una libera e non igienizzata (si veda UC 11).
	\item\textbf{Precondizione:} l’utente preme il bottone nella sezione della prima postazione disponibile.
	\item\textbf{Postcondizione:} l’utente riceve l'identificativo di una postazione.
	\item\textbf{Estensioni:}
	\begin{itemize}
		\item[$-$] tutte le postazioni nella stanza selezionata sono occupate o guaste.(UC 12.3).
	\end{itemize}
\end{itemize}
\subsubsection{ UC12.5 - Selezione postazione }
\begin{itemize}
	\item\textbf{Attori Primari:} dipendente.
	\item\textbf{Descrizione:} l’utente dopo aver visualizzato le postazioni con colori diversi in base allo stato, ne seleziona una. E' possibile scegliere una postazione libera e igienizzata o non igienizzata.
	\item\textbf{Scenario principale:} viene selezionata una postazione.
	\item\textbf{Precondizione:} l’utente ha visualizzato correttamente le postazioni di una stanza.
	\item\textbf{Postcondizione:} l’utente seleziona una postazione.
\end{itemize}
\subsubsection{ UC12.6 - Prenotazione postazione }
\begin{itemize}
	\item\textbf{Attori Primari:} dipendente.
	\item\textbf{Descrizione:} l’utente dopo aver selezionato la postazione, effettua la prenotazione. 
	\item\textbf{Scenario principale:} l’utente sta navigando nella sezione per la prenotazione di una postazione
	e preme il bottone per effettuare questa azione.
	\item\textbf{Precondizione:} l’utente ha selezionato correttamente la postazione di una stanza.
	\item\textbf{Postcondizione:} l'utente prenota la postazione.
	\item\textbf{Estensioni:}
	\begin{itemize}
		\item[$-$] Nel caso in cui si scelga di prenotare una postazione libera e sporca si veda il caso d'uso "Pulizia autonoma"(UC 11),
		per poi tornare allo scenario principale.
	\end{itemize}
\end{itemize}
\subsubsection{ UC13 - Elenco stanze e postazioni da igienizzare}
\begin{figure}[H]
		\centering
		\includegraphics[width=15cm]{res/images/UC13.png}
		\caption{Elenco stanze e postazioni da igienizzare}
		\label{fig:Elenco stanze e postazioni da igienizzare}
	\end{figure}
\begin{itemize}
           	\item\textbf{Attori Primari:} addetto pulizie.
           	\item\textbf{Descrizione:} l'utente riceve un elenco delle postazioni[UC13.2] e delle stanze[UC13.1] che necessitano di igienizzazione.
           	\item\textbf{Scenario principale:} l'utente si trova all'interno del sistema e verifica tutte le postazioni che sono state utilizzate almeno da un dipendente che saranno quindi da igienizzare oppure può decidere di igienizzare l’intera stanza.
           	\item\textbf{Precondizione:} l'utente va nella sezione dedicata nel sistema e deve premere un bottone.
           	\item\textbf{Postcondizione:} l'utente verifica quali postazioni sono state occupate da almeno un dipendente così da ottenere un elenco delle stanze e postazioni da igienizzare.
\end{itemize}

\subsubsection{UC13.1 - Elenco stanze da igienizzare}
\begin{itemize}
           	\item\textbf{Attori Primari:} addetto pulizie.
           	\item\textbf{Descrizione:} l'utente riceve un elenco delle stanza che necessitano di igienizzazione.
           	\item\textbf{Scenario principale:} l'utente si trova all'interno del sistema e verifica tutte le stanze che sono state utilizzate almeno da un dipendente 				che saranno quindi da igienizzare.
           	\item\textbf{Precondizione:} l'utente va nella sezione dedicata nel sistema e deve premere un bottone.
           	\item\textbf{Postcondizione:} l'utente verifica quali stanze sono state occupate da almeno un dipendente così da ottenere un elenco delle stanze da 				igienizzare.
\end{itemize}
\subsubsection{UC13.2 - Elenco postazioni da igienizzare}
\begin{itemize}
           	\item\textbf{Attori Primari:} addetto pulizie.
           	\item\textbf{Descrizione:} l'utente riceve un elenco delle postazioni che necessitano di igienizzazione.
           	\item\textbf{Scenario principale:} l'utente si trova all'interno del sistema e verifica tutte le postazioni che sono state utilizzate almeno da un dipendente 	che saranno quindi da igienizzare.
           	\item\textbf{Precondizione:} l'utente va nella sezione dedicata nel sistema e deve premere un bottone.
           	\item\textbf{Postcondizione:} l'utente verifica quali postazioni sono state occupate da almeno un dipendente così da ottenere un elenco delle postazioni da igienizzare.
\end{itemize}

\subsubsection{ UC14 - Marcatura stanza e postazione come igienizzata}
\begin{figure}[H]
		\centering
		\includegraphics[width=18cm]{res/images/UC14.png}
		\caption{Marcatura stanza e postazione come igienizzata}
		\label{fig:Marcatura stanza e postazione come igienizzata}
	\end{figure}
\begin{itemize}
           	\item\textbf{Attori Primari:} Addetto pulizie.
		\item\textbf{Attori Secondario:} Ethereum.
           	\item\textbf{Descrizione:} dopo l'igienizzazione l'utente marca le stanze o le postazioni che ha pulito come igienizzate.
           	\item\textbf{Scenario principale:} l'utente igienizza la stanza[UC14.1] o la postazione[UC14.2] e in seguito accede al sistema e la marca come igienizzata attraverso l'utilizzo di Ethereum.
           	\item\textbf{Precondizione:} l'utente ottiene l'elenco delle stanze e delle postazioni da igienizzare.
           	\item\textbf{Postcondizione:} l'utente, dopo l'igienizzazione, marca le stanze e le postazioni come igienizzate.
\end{itemize}
\subsubsection{UC14.1 - Marcatura stanza come igienizzata}
\begin{itemize}
           	\item\textbf{Attori Primari:} Addetto pulizie.
		\item\textbf{Attori Secondario:} Ethereum.
           	\item\textbf{Descrizione:} dopo l'igienizzazione l'utente marca la stanza come igienizzata.
           	\item\textbf{Scenario principale:} l'utente igienizza la stanza e in seguito accede al sistema e la marca come igienizzata attraverso l'utilizzo di Ethereum.
           	\item\textbf{Precondizione:} l'utente ottiene l'elenco delle stanze da igienizzare.
           	\item\textbf{Postcondizione:} l'utente, dopo l'igienizzazione, marca le stanze come igienizzate.
\end{itemize}
\subsubsection{UC14.2 - Marcatura postazione come igienizzata}
\begin{itemize}
           	\item\textbf{Attori Primari:} Addetto pulizie.
		\item\textbf{Attori Secondario:} Ethereum.
           	\item\textbf{Descrizione:} dopo l'igienizzazione l'utente marca la postazione come igienizzata.
           	\item\textbf{Scenario principale:} l'utente igienizza la postazione e in seguito accede al sistema e la marca come igienizzata attraverso l'utilizzo di ethereum.
           	\item\textbf{Precondizione:} l'utente ottiene l'elenco delle postazioni da igienizzare.
           	\item\textbf{Postcondizione:} l'utente, dopo l'igienizzazione, marca le postazioni come igienizzate.
\end{itemize}
