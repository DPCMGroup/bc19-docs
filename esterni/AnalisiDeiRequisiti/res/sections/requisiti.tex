\section{Requisiti}
Per la classificazione dei requisiti si fa riferimento alla sezione Verifica\footnote{Sezione §3.4} del documento \dext{Norme di Progetto v. 1.0.0}.
Per ogni requisito delle tabelle sottostanti è stata decisa la seguente struttura: 
\begin{itemize}
\item\textbf{Requisito:} R[Priorità][Tipologia][Identificativo];
\item\textbf{Descrizione:} descrizione breve ma completa del requisito, meno ambigua possibile;
\item\textbf{Fonti:} ogni requisito può derivare dalle seguenti fonti:
	\begin{itemize}
		\item Capitolato: si tratta di un requisito individuato dalla lettura del capitolato;
		\item Interna: si tratta di un requisito individuato nella fase di analisi;
		\item Caso d'uso: si tratta di un requisito estrapolato dai casi d'uso individuati;
		\item Verbale: si tratta di un requisito individuato nel verbale in seguito ai chiarimenti con il proponente.
	\end{itemize}
\end{itemize}
I verbali esterni che costituiscono fonte di requisiti sono stati approvati da Lorenzo Patera, referente di Imola Informatica.


\subsection{Requisiti funzionali}
\begin{center}
	\rowcolors{2}{lightest-grayest}{white}
	\begin{longtable}{|c|p{10cm}|p{4cm}|}
		\hline
		\rowcolor{lighter-grayer}
		\textbf{Requisito} & \textbf{Descrizione} & \textbf{Fonti}  \\
		\hline
		\endhead
		
		% ----- Modificare da qui -----
		 R2F1 & L'utente può leggere una breve guida per il login e il funzionamento & Interna - UC1 \\
		\hline
		R1F2	&	L'utente non autenticato deve potersi autenticare nel webserver & Capitolato - UC2	\\
		\hline
		R1F2.1	&	L'utente non autenticato deve ricevere un messaggio di errore nel caso abbia sbagliato a inserire le credenziali nel web server& Interna - UC2.1	\\
		\hline
		R1F2.2	&	L'utente non autenticato che stia provando ad accedere al web server deve poter ricevere un messaggio di errore nel caso il suo account sia disabilitato& Interna - UC2.2	\\
		\hline
		R1F3	&	L'utente non autenticato deve potersi autenticare nell'applicazione mobile & Capitolato - UC3	\\
		\hline
		R1F3.1	&	L'utente non autenticato deve ricevere un messaggio di errore nel caso abbia sbagliato a inserire le credenziali nell'applicazione mobile& Interna - UC3.1	\\
		\hline
		R1F3.2	&	L'utente non autenticato che stia provando ad accedere al web server deve poter ricevere un messaggio di errore nel caso il suo account sia disabilitato& Interna - UC3.2	\\
		\hline
		R1F4	&	L'utente autenticato come amministratore deve potersi disautenticare dal web server& Interna - UC4	\\
		\hline
		R1F5	&	L'utente autenticato come dipendente deve potersi disautenticare dall'applicazione mobile& Interna - UC5	\\
		\hline
		R1F6	&	L'utente autenticato come addetto alle pulizie deve potersi disautenticare dall'applicazione mobile& Interna - UC6	\\
		\hline
		R1F7&L'amministratore deve poter visualizzare le stanze e le postazioni salvate nel sistema& Capitolato - UC7	\\
		\hline
		R1F7.1&L'amministratore deve poter visualizzare le stanze e le postazioni salvate nel sistema, in modo schematico& Capitolato - UC7	\\
		\hline
		R1F7.2&L'amministratore deve poter visualizzare le stanze e le postazioni salvate nel sistema colorate in base al loro stato attuale. Gli stati possibili sono:
		\begin{itemize}
			\item libera e igienizzata
			\item libera e non igienizzata
			\item occupata
			\item prenotata e igienizzata
			\item prenotata e non igienizzata
			\item inaccessibile
		\end{itemize}& Capitolato - UC7	\\
		\hline
		R1F7.3&Per ogni stanza deve essere indicato il numero di occupanti attuali	& Capitolato - UC7	\\
		\hline
		R1F8&L'amministratore deve poter visualizzare un calendario delle prenotazioni delle postazioni	& Capitolato - UC8	\\
		\hline
		R1F9&L'amministratore deve poter aggiungere una stanza al sistema e assegnarle un nome	& Capitolato - UC9	\\
		\hline
		R1F9&	La creazione di una nuova stanza va salvata in Ethereum&Verbale esterno 2021-01-01 - UC9 	\\
		\hline
		R1F9.1 & Se l'amministratore tenta di assegnare a una stanza un nome giù in utilizzo per un'altra stanza deve essergli impedito e deve essere avvisato dell'errore & Interna - UC9.1 \\
		R1F10&L'amministratore deve poter eliminare una stanza dal sistema	& Capitolato - UC10	\\
		\hline
		R1F10.1&L'eliminazione di una stanza va salvata in Ethereum	& Verbale esterno 2021-01-01 - UC10	\\
		\hline
		R1F11&L'amministratore deve poter modificare il nome di una stanza	& Capitolato - UC11	\\
		\hline
		R1F12&L'amministratore deve poter impostare una stanza come inaccessibile per alcuni giorni, a partire anche da un giorno diverso da quello odierno	& Capitolato - UC12	\\
		\hline
		R1F13&L'amministratore deve poter aggiungere una postazione in una stanza, specificando:
		\begin{itemize}
			\item codice della postazione
			\item codice del tag RFID che la identifica
			\item posizione all'interno di una stanza
		\end{itemize} & Capitolato - UC13	\\
		\hline
		R1F13.1&L'aggiunta di una postazione va salvata in Ethereum	&Verbale esterno 2021-01-01 - UC13 	\\
		\hline
		R1F14&L'amministratore deve poter eliminare una postazione	& Capitolato - UC14	\\
		\hline
		R1F14.1	&L'eliminazione di una postazione va salvata in Ethereum	& Verbale esterno 2021-01-01 - UC14	\\
		\hline
		R1F16&L'amministratore deve poter modificare i dati e la posizione di una postazione	& Capitolato - UC16	\\
		\hline
		R1F16.1&	La modifica di una postazione va salvata in Ethereum& Verbale esterno 2021-01-01 - UC16	\\
		\hline
		R1F16.2&	Se si tenta di creare o modificare una postazione assegnandole un codice già in utilizzo, l'azione deve fallire e deve comparire un messaggio d'errore& Verbale esterno 2020-12-18 - UC16 	\\
		\hline
		R1F16.3&Se si tenta di creare o modificare una postazione assegnandole un codice del tag già in utilizzo, l'azione deve fallire e deve comparire un messaggio d'errore	&Verbale esterno 2020-12-18 - UC16 	\\
		\hline
		
		R1F17&L'amministratore deve poter gestire le credenziali degli utenti	& Capitolato - UC17		\\
		\hline
		R1F17.1&L'amministratore deve poter visualizzare una lista delle credenziali di tutti gli utente del sistema	& Capitolato - UC17.1	\\
		\hline
		R1F17.2&L'amministratore deve poter creare nuove credenziali per accedere al sistema. Le informazioni contenute nel profilo di un utente sono: Nome, Cognome, Nome utente, Password e Email. Tra queste le informazioni necessarie per l'accesso sono password e nome utente	& Capitolato - Verbale esterno 2021-01-01 - UC17.2	\\
		\hline
		R1F17.3&L'amministratore può modificare le credenziali degli utenti del sistema	&Capitolato - UC17.3 	\\
		\hline
		R1F17.4&L'amministratore deve poter eliminare credenziali degli utenti del sistema	& Capitolato - UC17.4	\\
		\hline
		R1F17.5&Il sistema deve poter memorizzare più credenziali per ogni categoria di utenti: amministratori, dipendenti e addetti	& Capitolato - Verbale esterno 2021-01-01	\\
		\hline
		R1F18&L'amministratore deve poter esplorare la storia delle occupazioni delle postazioni	& Capitolato - UC18 - Verbale esterno 2021-01-04	\\
		\hline
		R1F18.1&L'amministratore deve poter visualizzare in forma tabellare le occupazioni di postazioni da parte di un utente specifico, con date e orari di inizio e fine delle occupazioni e codici delle postazioni.	&Capitolato - UC18.1 - Verbale esterno 2021-01-04	\\
		\hline
		R1F18.1.1&L'amministratore deve poter effettuare una ricerca tra le postazioni occupate da un utente, specificando il periodo all'interno del quale l'utente occupava la postazione. & 	Capitolato - UC18.1.1 - Verbale esterno 2021-01-04\\
		\hline
		R2F18.1.2&L'amministratore deve poter visualizzare in forma tabellare le occupazioni di postazioni da parte di un utente specifico, con date e orari di inizio e fine delle occupazioni,  codici delle postazioni e ore trascorse alla postazione. & 	Capitolato - UC18.1.2 - Verbale esterno 2021-01-04\\
		\hline
		R1F18.2&L'amministratore deve poter visualizzare in forma tabellare le occupazioni di una postazione specifica, con date e orari di inizio e fine delle occupazioni e nomi e cognomi dei dipendenti.	&Capitolato - UC18.2 - Verbale esterno 2021-01-04	\\
		\hline
		R1F18.2.1&L'amministratore deve poter effettuare una ricerca tra le occupazioni di una specifica postazione indicando il periodo all'interno del quale l'utente occupava la postazione	& 	Capitolato - UC18.2.1 - Verbale esterno 2021-01-04\\
		\hline
		R1F18.3 & L'amministratore deve poter scaricare il report delle occupazioni che sta visualizzando in un formato leggibile & Capitolato - UC18.3 - Verbale esterno 2021-01-04\\
		\hline
		R2F18.3.1 & L'amministratore deve poter scaricare il report delle occupazioni che sta visualizzando nel formato PDF & Capitolato - UC18.3.1 - Verbale esterno 2021-01-04\\
		\hline
		R2F19&L'amministratore deve poter ottenere una tabella delle sanificazioni di tutte le postazioni, con data e ora in cui sono avvenute, nome e cognome di chi le ha eseguite e il suo ruolo.	&Capitolato - UC19 - Verbale esterno 2021-01-04	\\
		\hline
		R2F19.1&L'amministratore deve poter scaricare il report delle sanificazioni che sta visualizzando in un formato leggibile.	&Capitolato - UC19.1 - Verbale esterno 2021-01-04	\\
		\hline
		R2F19.1.1&L'amministratore deve poter scaricare il report delle sanificazioni che sta visualizzando nel formato PDF.	&Capitolato - UC19.1.1 - Verbale esterno 2021-01-04	\\
		\hline
		R1F20 & Il dipendente deve poter effettuare una scansione del tag RFID presente su una postazione & Capitolato - UC20 \\
		\hline
		R1F20.1 & Se il dipendente tenta di effettuare una scansione del tag RFID ma il telefono si trova ad una distanza troppo elevata dal tag, viene visualizzato un errore & Interna - UC20.1 \\
		\hline
		R1F21&Il dipendente deve poter visualizzare lo stato di una postazione, tramite tag RFID.	&Capitolato - UC21 	\\
		\hline
		R1F22&Il dipendente deve poter segnalare la propria presenza su una postazione in tempo reale, tramite tag RFID.	&Capitolato - UC22 	\\
		\hline
		R1F22.1&Il dipendente deve poter registrare i momenti di inizio e di fine della sua occupazione di una postazione, tramite Ethereum.	&Interna - UC20.3 	\\
		\hline	
		R1F22&Il dipendente deve poter effettuare una pulizia autonoma di una postazione, tramite il kit aziendale.	&Capitolato - UC21	\\
		\hline		
		R1F22.1&Il dipendente, dopo aver effettuato una igienizzazione, deve poter marcare la postazione come libera e igienizzata.	&Interna - UC21.1	\\
		\hline	
		R1F22.2&Il dipendente deve poter registrare l'avvenuta igienizzazione autonoma della postazione, tramite Ethereum.	&Capitolato - UC22.2	\\
		\hline	
		R1F23&Il dipendente deve poter ricercare una postazione in base alla data, all'orario e all'identificativo della stanza.	&Interna - UC23.1	\\
		\hline
		R1F23.1&Il dipendente deve poter visualizzare le postazioni di una stanza.	&Capitolato - UC23.2	\\
		\hline
		R1F23.2&Il dipendente deve poter visualizzare l'identificativo della prima postazione igienizzata e libera in una stanza.	&Capitolato - UC23.4	\\
		\hline
		R1F23.3&Il dipendente deve poter selezionare una postazione di una stanza.	&Interna - UC23.5	\\
		\hline
		R1F23.4&Il dipendente deve poter prenotare una postazione selezionata.	&Capitolato - UC23.6	\\
		\hline
		R1F24&Gli utenti che hanno prenotato una postazione devono essere avvisati se la postazione stessa o la stanza in cui si trovano diventa non disponibile	& Verbale esterno 2020-12-18	\\
		\hline
		
		R1F25&L'addetto alle pulizie riceve l'elenco delle stanze e delle postazioni che necessitano di igienizzazione	& Interna - UC25	\\
		\hline
		R1F25.1&	L'addetto alle pulizie riceve l'elenco delle stanze che necessitano di igienizzazione& 	Capitolato - UC25.1\\
		\hline
		R1F25.2&L'addetto alle pulizie riceve l'elenco delle postazioni che necessitano di igienizzazione	& Interna - UC25.2	\\
		\hline
		R1F26&L'addetto alle pulizie deve registrare sulla blockchain, con l'utilizzo di Ethereum, quando igienizza una stanza o una postazione	& Interna - UC26	\\
		\hline
		R1F26.1&L'addetto alle pulizie deve poter marcare, nella sezione apposita, una stanza come igienizzata	& Capitolato - UC26.1	\\
		\hline
		R1F26.2&L'addetto alle pulizie deve poter marcare, nella sezione apposita, una postazione come igienizzata	&Interna - UC26.2 	\\
		\hline
		R1F26.3&L'addetto alle pulizie non può pulire postazioni o stanze occupate	& Verbale esterno 2021-01-01	\\
		\hline
		R1F27 &Deve essere realizzato e consegnato un Docker file contenente la componente applicativa. & Capitolato\\
		\hline
						
	\end{longtable}
\end{center}


\subsection{Requisiti di qualità}
\begin{center}
	\rowcolors{2}{lightest-grayest}{white}
	\begin{longtable}{|c|p{10cm}|p{4cm}|}
		\hline
		\rowcolor{lighter-grayer}
		\textbf{Requisito} & \textbf{Descrizione} & \textbf{Fonti}  \\
		\hline
		\endhead
		
		% ----- Modificare da qui -----
		 R1Q1 &La progettazione e la codifica devono rispettare le norme e le metriche definite nel documento \dext{Piano di qualifica v1.0.0} & Interna\\
		\hline	
		R1Q2 &Deve essere prodotto un manuale sviluppatore & Capitolato\\
		\hline
		R1Q3 &Le componenti applicative devono essere correlate da test unitari e d’integrazione. & Capitolato\\
		\hline
		R1Q4 &Il sistema deve essere testato nella sua interezza tramite test end-to-end. & Capitolato\\
		\hline
		R1Q5 &La copertura dei test deve essere maggiore o uguale all'ottanta per cento e correlata di report. & Capitolato\\
		\hline
		R1Q6 &Devono essere riportati dei report sui test effettuati relativamente all’ottimizzazione dell’applicazione rispetto al consumo della batteria dei cellulari. & Capitolato\\
		\hline
		R2Q7&	Deve essere fornita un'analisi del numero di utenti da supportare e del servizio cloud più adatto per adempiervi, analizzando prezzo, stabilità del servizio ed assistenza (supponendo di disporre di massimo 2 CPU e 1Gi2 per istanza del server).& Capitolato	\\
		\hline
		R1Q8&Deve essere fornita una documentazione delle scelte implementative e progettuali accompagnate dalle loro motivazioni	& Capitolato	\\
		\hline
		R1Q9	&Deve essere fornita una documentazione dei problemi che rimangono aperti con eventuali proposte di soluzioni	& Capitolato	\\
		\hline
		R1Q10&Devono essere scritti dei test che abbiamo una copertura di almeno l'80 per cento del codice e deve essere fornito il report della loro esecuzione	& 	Capitolato \\
		\hline
		
		R1Q11&Devono essere eseguiti dei test che mettano in relazione la precisione del sistema e il consumo della batteria dei cellulari	& Capitolato - Verbale esterno 2020-12-18	\\
		\hline
		R1Q12&L’utilizzo del lettore RFID riduce in modo rilevante l’autonomia dei cellulari, l’applicazione è da sviluppare in maniera tale da bilanciare nel miglior modo possibile batteria e scansioni. È richiesto un resoconto delle scelte fatte e dei test effettuati per garantire il miglior rapporto raggiunto.	& Capitolato	\\
		\hline
		R1Q13& Deve essere prodotto un manuale utente	& Capitolato	\\
		\hline
	\end{longtable}
\end{center}
\subsection{Requisiti di vincolo}
\begin{center}
	\rowcolors{2}{lightest-grayest}{white}
	\begin{longtable}{|c|p{10cm}|p{4cm}|}
		\hline
		\rowcolor{lighter-grayer}
		\textbf{Requisito} & \textbf{Descrizione} & \textbf{Fonti}  \\
		\hline
		\endhead
		
		% ----- Modificare da qui -----
		 R1V1 &Avere il server che esponga, in aggiunta a eventuali altri protocolli per l’interazione con il servizio specifico, delle API Rest, o in altrernativa gRPC, attraverso le quali sia possibile utilizzare l’applicativo & Capitolato \\
		\hline
R1V2&Scansione dei codici nel tempo sufficiente a certificare la presenza  della persona in postazione.	& Capitolato	\\
		\hline

R2V3&Le comunicazioni tra app e server devono essere cifrate	& Capitolato	\\
	
		\hline
R1V4&Deve essere sviluppata un'applicazione per Android o \glock{iOS}	& Capitolato	\\
		\hline


	\end{longtable}
\end{center}

\subsection{Requisiti prestazionali}
Non sono stati individuati requisiti prestazionali. Uno dei motivi sta nel fatto che la rete Ethereum, su cui si appoggerà il nostro sistema, ha tempi di esecuzione delle azioni dipendenti dal carico di utenti.
%\begin{center}
%	\rowcolors{2}{lightest-grayest}{white}
%	\begin{longtable}{|c|c|c|}
%		\hline
%		\rowcolor{lighter-grayer}
%		\textbf{Requisito} & \textbf{Descrizione} & \textbf{Fonti}  \\
%		\hline
%		\endfirsthead
%		
%		% ----- Modificare da qui -----
%		 R & & \\
%		\hline
%		
%	\end{longtable}
%\end{center}

\subsection{Tracciamento}
\subsubsection{Fonte - Requisiti}
\begin{center}
	\rowcolors{2}{lightest-grayest}{white}
	\begin{longtable}{|p{44mm}|p{22mm}|}
		\hline
		\rowcolor{lighter-grayer}
		\textbf{Fonte} &  \textbf{Requisiti}  \\
		\hline
		\endhead
		
		% ----- Modificare da qui -----
		




Interna &
R2F1 \newline
R1F2.1 \newline
R1F2.2 \newline
R1F3.1 \newline
R1F3.2 \newline
R1F4 \newline
R1F5 \newline
R1F6 \newline
R1F9.1 \newline
R1F20.1 \newline
R1F22.1 \newline
R1F22.1 \newline
R1F23 \newline
R1F23.3 \newline
R1F25 \newline
R1F25.2 \newline
R1F26 \newline
R1F26.2 \newline
R1Q1 \newline
\\
\hline
Capitolato &
R1F2 \newline
R1F3 \newline
R1F7 \newline
R1F7.1 \newline
R1F7.2 \newline
R1F7.3 \newline
R1F8 \newline
R1F9 \newline
R1F10 \newline
R1F11 \newline
R1F12 \newline
R1F13 \newline
R1F14 \newline
R1F16 \newline
R1F17 \newline
R1F17.1 \newline
R1F17.2 \newline
R1F17.3 \newline
R1F17.4 \newline
R1F17.5 \newline
R1F18 \newline
R1F18.1 \newline
R1F18.1.1 \newline
R2F18.1.2 \newline
R1F18.2 \newline
R1F18.2.1 \newline
R1F18.3 \newline
R2F18.3.1 \newline
R2F19 \newline
R2F19.1 \newline
R2F19.1.1 \newline
R1F20 \newline
R1F21 \newline
R1F22 \newline
R1F22 \newline
R1F22.2 \newline
R1F23.1 \newline
R1F23.2 \newline
R1F23.4 \newline
R1F25.1 \newline
R1F26.1 \newline
R1F27 \newline
R1Q2 \newline
R1Q3 \newline
R1Q4 \newline
R1Q5 \newline
R1Q6 \newline
R2Q7 
\\
\hline
Capitolato &
R1Q8 \newline
R1Q9 \newline
R1Q10 \newline
R1Q11 \newline
R1Q12 \newline
R1Q13 \newline
R1V1 \newline
R1V2 \newline
R2V3 \newline
R1V4 \newline
\\
\hline
Verbale esterno 2020-12-18 &
R1F16.2 \newline
R1F16.3 \newline
R1F24 \newline
R1Q11 \newline
\\
\hline
Verbale esterno 2021-01-01 &
R1F9 \newline
R1F10.1 \newline
R1F13.1 \newline
R1F14.1 \newline
R1F16.1 \newline
R1F17.2 \newline
R1F17.5 \newline
R1F26.3 \newline
\\
\hline
Verbale esterno 2021-01-04 &
R1F18 \newline
R1F18.1 \newline
R1F18.1.1 \newline
R2F18.1.2 \newline
R1F18.2 \newline
R1F18.2.1 \newline
R1F18.3 \newline
R2F18.3.1 \newline
R2F19 \newline
R2F19.1 \newline
R2F19.1.1 \newline
\\
\hline
UC1 &
R2F1 \newline
\\
\hline
UC2 &
R1F2 \newline
\\
\hline
UC2.1 &
R1F2.1 \newline
\\
\hline
UC2.2 &
R1F2.2 \newline
\\
\hline
UC3 &
R1F3 \newline
\\
\hline
UC3.1 &
R1F3.1 \newline
\\
\hline
UC3.2 &
R1F3.2 \newline
\\
\hline
UC4 &
R1F4 \newline
\\
\hline
UC5 &
R1F5 \newline
\\
\hline
UC6 &
R1F6 \newline
\\
\hline
UC7 &
R1F7 \newline
R1F7.1 \newline
R1F7.2 \newline
R1F7.3 \newline
\\
\hline
UC8 &
R1F8 \newline
\\
\hline
UC9 &
R1F9 \newline
R1F9 \newline
\\
\hline
UC9.1 &
R1F9.1 \newline
\\
\hline
UC10 &
R1F10 \newline
R1F10.1 \newline
\\
\hline
UC11 &
R1F11 \newline
\\
\hline
UC12 &
R1F12 \newline
R1F13 \newline
\\
\hline
UC13 &
R1F13.1 \newline
\\
\hline
UC14 &
R1F14 \newline
R1F14.1 \newline
\\
\hline
UC16 &
R1F16 \newline
R1F16.1 \newline
R1F16.2 \newline
R1F16.3 \newline
\\
\hline
UC17 &
R1F17 \newline
\\
\hline
UC17.1 &
R1F17.1 \newline
\\
\hline
UC17.2 &
R1F17.2 \newline
\\
\hline
UC17.3 &
R1F17.3 \newline
\\
\hline
UC17.4 &
R1F17.4 \newline
\\
\hline
UC18 &
R1F18 \newline
\\
\hline
UC18.1 &
R1F18.1 \newline
\\
\hline
UC18.1.1 &
R1F18.1.1 \newline
\\
\hline
UC18.1.2 &
R2F18.1.2 \newline
\\
\hline
UC18.2 &
R1F18.2 \newline
\\
\hline
UC18.2.1 &
R1F18.2.1 \newline
\\
\hline
UC18.3 &
R1F18.3 \newline
\\
\hline
UC18.3.1 &
R2F18.3.1 \newline
\\
\hline
UC19 &
R2F19 \newline
\\
\hline
UC19.1 &
R2F19.1 \newline
\\
\hline
UC19.1.1 &
R2F19.1.1 \newline
\\
\hline
UC20 &
R1F20 \newline
\\
\hline
UC20.1 &
R1F20.1 \newline
\\
\hline
UC21 &
R1F21 \newline
R1F22 \newline
\\
\hline
UC22 &
R1F22 \newline
\\
\hline
UC20.3 &
R1F22.1 \newline
\\
\hline
UC21.1 &
R1F22.1 \newline
\\
\hline
UC22.2 &
R1F22.2 \newline
\\
\hline
UC23.1 &
R1F23 \newline
\\
\hline
UC23.2 &
R1F23.1 \newline
\\
\hline
UC23.4 &
R1F23.2 \newline
\\
\hline
UC23.5 &
R1F23.3 \newline
\\
\hline
UC23.6 &
R1F23.4 \newline
\\
\hline
UC25 &
R1F25 \newline
\\
\hline
UC25.1 &
R1F25.1 \newline
\\
\hline
UC25.2 &
R1F25.2 \newline
\\
\hline
UC26 &
R1F26 \newline
\\
\hline
UC26.1 &
R1F26.1 \newline
\\
\hline
UC26.2 &
R1F26.2 \newline
\\
\hline

	\end{longtable}
\end{center}

\subsubsection{Requisito - Fonti}
\begin{center}
	\rowcolors{2}{lightest-grayest}{white}
	\begin{longtable}{|p{22mm}|p{22mm}|}
		\hline
		\rowcolor{lighter-grayer}
		\textbf{Requisito} &  \textbf{Fonti}  \\
		\hline
		\endhead
		
		% ----- Modificare da qui -----




R2F1 &
Interna \newline
UC1 \newline
\\
\hline
R1F2 &
Capitolato \newline
UC2 \newline
\\
\hline
R1F2.1 &
Interna \newline
UC2.1 \newline
\\
\hline
R1F2.2 &
Interna \newline
UC2.2 \newline
\\
\hline
R1F3 &
Capitolato \newline
UC3 \newline
\\
\hline
R1F3.1 &
Interna \newline
UC3.1 \newline
\\
\hline
R1F3.2 &
Interna \newline
UC3.2 \newline
\\
\hline
R1F4 &
Interna \newline
UC4 \newline
\\
\hline
R1F5 &
Interna \newline
UC5 \newline
\\
\hline
R1F6 &
Interna \newline
UC6 \newline
\\
\hline
R1F7 &
Capitolato \newline
UC7 \newline
\\
\hline
R1F7.1 &
Capitolato \newline
UC7 \newline
\\
\hline
R1F7.2 &
Capitolato \newline
UC7 \newline
\\
\hline
R1F7.3 &
Capitolato \newline
UC7 \newline
\\
\hline
R1F8 &
Capitolato \newline
UC8 \newline
\\
\hline
R1F9 &
Capitolato \newline
UC9 \newline
\\
\hline
R1F9 &
Verbale esterno 2021-01-01 \newline
UC9 \newline
\\
\hline
R1F9.1 &
Interna \newline
UC9.1 \newline
\\
\hline
R1F10 &
Capitolato \newline
UC10 \newline
\\
\hline
R1F10.1 &
Verbale esterno 2021-01-01 \newline
UC10 \newline
\\
\hline
R1F11 &
Capitolato \newline
UC11 \newline
\\
\hline
R1F12 &
Capitolato \newline
UC12 \newline
\\
\hline
R1F13 &
Capitolato \newline
UC12 \newline
\\
\hline
R1F13.1 &
Verbale esterno 2021-01-01 \newline
UC13 \newline
\\
\hline
R1F14 &
Capitolato \newline
UC14 \newline
\\
\hline
R1F14.1 &
Verbale esterno 2021-01-01 \newline
UC14 \newline
\\
\hline
R1F16 &
Capitolato \newline
UC16 \newline
\\
\hline
R1F16.1 &
Verbale esterno 2021-01-01 \newline
UC16 \newline
\\
\hline
R1F16.2 &
Verbale esterno 2020-12-18 \newline
UC16 \newline
\\
\hline
R1F16.3 &
Verbale esterno 2020-12-18 \newline
UC16 \newline
\\
\hline
R1F17 &
Capitolato \newline
UC17 \newline
\\
\hline
R1F17.1 &
Capitolato \newline
UC17.1 \newline
\\
\hline
R1F17.2 &
Capitolato \newline
Verbale esterno 2021-01-01 \newline
UC17.2 \newline
\\
\hline
R1F17.3 &
Capitolato \newline
UC17.3 \newline
\\
\hline
R1F17.4 &
Capitolato \newline
UC17.4 \newline
\\
\hline
R1F17.5 &
Capitolato \newline
Verbale esterno 2021-01-01 \newline
\\
\hline
R1F18 &
Capitolato \newline
UC18 \newline
Verbale esterno 2021-01-04 \newline
\\
\hline
R1F18.1 &
Capitolato \newline
UC18.1 \newline
Verbale esterno 2021-01-04 \newline
\\
\hline
R1F18.1.1 &
Capitolato \newline
UC18.1.1 \newline
Verbale esterno 2021-01-04 \newline
\\
\hline
R2F18.1.2 &
Capitolato \newline
UC18.1.2 \newline
Verbale esterno 2021-01-04 \newline
\\
\hline
R1F18.2 &
Capitolato \newline
UC18.2 \newline
Verbale esterno 2021-01-04 \newline
\\
\hline
R1F18.2.1 &
Capitolato \newline
UC18.2.1 \newline
Verbale esterno 2021-01-04 \newline
\\
\hline
R1F18.3 &
Capitolato \newline
UC18.3 \newline
Verbale esterno 2021-01-04 \newline
\\
\hline
R2F18.3.1 &
Capitolato \newline
UC18.3.1 \newline
Verbale esterno 2021-01-04 \newline
\\
\hline
R2F19 &
Capitolato \newline
UC19 \newline
Verbale esterno 2021-01-04 \newline
\\
\hline
R2F19.1 &
Capitolato \newline
UC19.1 \newline
Verbale esterno 2021-01-04 \newline
\\
\hline
R2F19.1.1 &
Capitolato \newline
UC19.1.1 \newline
Verbale esterno 2021-01-04 \newline
\\
\hline
R1F20 &
Capitolato \newline
UC20 \newline
\\
\hline
R1F20.1 &
Interna \newline
UC20.1 \newline
\\
\hline
R1F21 &
Capitolato \newline
UC21 \newline
\\
\hline
R1F22 &
Capitolato \newline
UC22 \newline
\\
\hline
R1F22.1 &
Interna \newline
UC20.3 \newline
\\
\hline
R1F22 &
Capitolato \newline
UC21 \newline
\\
\hline
R1F22.1 &
Interna \newline
UC21.1 \newline
\\
\hline
R1F22.2 &
Capitolato \newline
UC22.2 \newline
\\
\hline
R1F23 &
Interna \newline
UC23.1 \newline
\\
\hline
R1F23.1 &
Capitolato \newline
UC23.2 \newline
\\
\hline
R1F23.2 &
Capitolato \newline
UC23.4 \newline
\\
\hline
R1F23.3 &
Interna \newline
UC23.5 \newline
\\
\hline
R1F23.4 &
Capitolato \newline
UC23.6 \newline
\\
\hline
R1F24 &
Verbale esterno 2020-12-18 \newline
\\
\hline
R1F25 &
Interna \newline
UC25 \newline
\\
\hline
R1F25.1 &
Capitolato \newline
UC25.1 \newline
\\
\hline
R1F25.2 &
Interna \newline
UC25.2 \newline
\\
\hline
R1F26 &
Interna \newline
UC26 \newline
\\
\hline
R1F26.1 &
Capitolato \newline
UC26.1 \newline
\\
\hline
R1F26.2 &
Interna \newline
UC26.2 \newline
\\
\hline
R1F26.3 &
Verbale esterno 2021-01-01 \newline
\\
\hline
R1F27 &
Capitolato \newline
\\
\hline
R1Q1 &
Interna \newline
\\
\hline
R1Q2 &
Capitolato \newline
\\
\hline
R1Q3 &
Capitolato \newline
\\
\hline
R1Q4 &
Capitolato \newline
\\
\hline
R1Q5 &
Capitolato \newline
\\
\hline
R1Q6 &
Capitolato \newline
\\
\hline
R2Q7 &
Capitolato \newline
\\
\hline
R1Q8 &
Capitolato \newline
\\
\hline
R1Q9 &
Capitolato \newline
\\
\hline
R1Q10 &
Capitolato \newline
\\
\hline
R1Q11 &
Capitolato \newline
Verbale esterno 2020-12-18 \newline
\\
\hline
R1Q12 &
Capitolato \newline
\\
\hline
R1Q13 &
Capitolato \newline
\\
\hline
R1V1 &
Capitolato \newline
\\
\hline
R1V2 &
Capitolato \newline
\\
\hline
R2V3 &
Capitolato \newline
\\
\hline
R1V4 &
Capitolato \newline
\\
\hline
	
	\end{longtable}
\end{center}

\subsection{Considerazioni}
Durante l'avanzamento del progetto i requisiti individuati potranno essere modificati e ampliati, e potranno esserne aggiunti di nuovi.