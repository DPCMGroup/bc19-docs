\section{Requisiti}
Per ogni requisito delle tabelle sottostanti è stata decisa la seguente struttura: 
\begin{itemize}
\item\textbf{Requisito:} R[Importanza][Tipologia][Codice];\\
Il significato delle seguenti voci è:
	\begin{itemize}
	\item\textbf{Importanza:} ogni requisito può assumere il seguente valore:
		\begin{itemize}
		\item 1: requisito obbligatorio, irrinunciabile;
		\item 2: requisito desiderabile, non strettamente necessario ma che porta valore aggiunto riconoscibile;
		\item 3: requisito opzionale, relativamente utile, contrattabile più in avanti nel progetto;
		\end{itemize}
	\item\textbf{Tipologia:} ogni requisito può assumere il seguente valore:
		\begin{itemize}
		\item F: funzionale;
		\item P: prestazionale;
		\item Q: qualitativo;
		\item V: vincolo;
		\end{itemize}
	\item\textbf{Codice:} è un identificatore univoco del requisito;
	\end{itemize}
\item\textbf{Descrzione:} descrizione breve ma completa del requisito, meno ambigua possibile;
\item\textbf{Fonti:} ogni requisito può derivare dalle seguenti fonti:
	\begin{itemize}
		\item Capitolato: si tratta di un requisito individuato dallla lettura del capitolato;
		\item Interno: si tratta di un requisito individuato nella fase di analisi;
		\item Caso d'uso: si tratta di un requisito estrapolato dai casi d'uso individuati;
		\item Verbale: si tratta di un requisito individuato nel verbale in seguito ai chiarimenti con il proponente;
	\end{itemize}
\end{itemize}


\subsection{Requisiti funzionali}
\begin{center}
	\rowcolors{2}{lightest-grayest}{white}
	\begin{longtable}{|c|c|c|}
		\hline
		\rowcolor{lighter-grayer}
		\textbf{Requisito} & \textbf{Descrizione} & \textbf{Fonti}  \\
		\hline
		\endfirsthead
		
		% ----- Modificare da qui -----
		 R & & \\
		\hline
		
	\end{longtable}
\end{center}
\subsection{Requisiti di qualità}
\begin{center}
	\rowcolors{2}{lightest-grayest}{white}
	\begin{longtable}{|c|c|c|}
		\hline
		\rowcolor{lighter-grayer}
		\textbf{Requisito} & \textbf{Descrizione} & \textbf{Fonti}  \\
		\hline
		\endfirsthead
		
		% ----- Modificare da qui -----
		 R & & \\
		\hline
		
	\end{longtable}
\end{center}
\subsection{Requisiti di vincolo}
\begin{center}
	\rowcolors{2}{lightest-grayest}{white}
	\begin{longtable}{|c|c|c|}
		\hline
		\rowcolor{lighter-grayer}
		\textbf{Requisito} & \textbf{Descrizione} & \textbf{Fonti}  \\
		\hline
		\endfirsthead
		
		% ----- Modificare da qui -----
		 R & & \\
		\hline
		
	\end{longtable}
\end{center}
\subsection{Requisiti prestazionali}
Non sono stati individuati requisiti prestazionali. Uno dei motivi sta nel fatto che la rete Ethereum, su cui si appoggerà il nostro sistema, ha tempi di esecuzione delle azioni dipendenti dal carico di utenti.
%\begin{center}
%	\rowcolors{2}{lightest-grayest}{white}
%	\begin{longtable}{|c|c|c|}
%		\hline
%		\rowcolor{lighter-grayer}
%		\textbf{Requisito} & \textbf{Descrizione} & \textbf{Fonti}  \\
%		\hline
%		\endfirsthead
%		
%		% ----- Modificare da qui -----
%		 R & & \\
%		\hline
%		
%	\end{longtable}
%\end{center}

\subsection{Tracciamento}
\begin{center}
	\rowcolors{2}{lightest-grayest}{white}
	\begin{longtable}{|c|c|}
		\hline
		\rowcolor{lighter-grayer}
		\textbf{Requisito} &  \textbf{Fonti}  \\
		\hline
		\endfirsthead
		
		% ----- Modificare da qui -----
		 R & \\
		\hline
		
	\end{longtable}
\end{center}
\subsection{Considerazioni}
Durante l'avanzamento del progetto i requisiti individuati potranno essere modificati e ampliati, e potranno esserne aggiunti di nuovi.