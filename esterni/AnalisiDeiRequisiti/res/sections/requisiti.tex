\section{Requisiti}
Per la classificazione dei requisiti si fa riferimento alla sezione Verifica\footnote{Sezione §3.4} del documento \dext{Norme di Progetto v. 1.0.0}.
Per ogni requisito delle tabelle sottostanti è stata decisa la seguente struttura: 
\begin{itemize}
\item\textbf{Requisito:} R[Priorità][Tipologia][Identificativo];
\item\textbf{Descrizione:} descrizione breve ma completa del requisito, meno ambigua possibile;
\item\textbf{Fonti:} ogni requisito può derivare dalle seguenti fonti:
	\begin{itemize}
		\item Capitolato: si tratta di un requisito individuato dalla lettura del capitolato;
		\item Interno: si tratta di un requisito individuato nella fase di analisi;
		\item Caso d'uso: si tratta di un requisito estrapolato dai casi d'uso individuati;
		\item Verbale: si tratta di un requisito individuato nel verbale in seguito ai chiarimenti con il proponente.
	\end{itemize}
\end{itemize}


\subsection{Requisiti funzionali}
\begin{center}
	\rowcolors{2}{lightest-grayest}{white}
	\begin{longtable}{|c|p{10cm}|p{4cm}|}
		\hline
		\rowcolor{lighter-grayer}
		\textbf{Requisito} & \textbf{Descrizione} & \textbf{Fonti}  \\
		\hline
		\endhead
		
		% ----- Modificare da qui -----
		 R2F1 & L'utente può leggere una breve guida per il login e il funzionamento & Interno - UC1 \\
		\hline
		R1F2	&	L'utente non autenticato deve potersi autenticare dal sistema& Interno - UC2	\\
		\hline
		R1F2.1	&	L'utente non autenticato deve ricevere un messaggio di errore nel caso abbia sbagliato a inserire le credenziali& Interno - UC2.1	\\
		\hline
		R1F2.2	&	L'utente non autenticato deve poter ricevere un messaggio di errore nel caso il suo account venga disabilitato& Interno - UC2.2	\\
		\hline
		R1F3	&	L'utente autenticato deve potersi disautenticare dal sistema& Interno - UC3	\\
		\hline
			R1F4&L'amministratore deve poter gestire le stanze e le postazioni salvate nel sistema	& Capitolato - UC4	\\
					\hline
			R1F4.1&L'amministratore deve poter visualizzare le stanze e le postazioni salvate nel sistema in modo schematico	& Capitolato - UC4.1	\\
					\hline
			R1F4.2&	L'amministratore deve poter vede le postazioni colorate in base al loro stato attuale.& Capitolato - UC4.1	\\
					\hline
			R1F4.3&	Gli stati delle postazioni devono essere i seguenti: libera e igienizzata, libera e non igienizzata, occupata, prenotata e igienizzata, prenotata e non igienizzata e inaccessibile& 	Capitolato - UC4.1\\
					\hline
			R1F4.4&Per ogni stanza deve essere indicato il numero di occupanti attuali	& Capitolato - UC4.1	\\
					\hline
			R1F4.5&L'amministratore deve poter visualizzare un calendario delle prenotazioni delle postazioni	& Capitolato - UC4.1.1	\\
					\hline
				R1F4.6&L'amministratore deve poter aggiungere una stanza al sistema e assegnarle un nome	& Capitolato - UC4.2	\\
						\hline
				R1F4.7&	La creazione di una nuova stanza va salvata in Ethereum&Verbale esterno 2021-01-01 - UC4.2 	\\
						\hline
				R1F4.8&L'amministratore deve poter eliminare una stanza dal sistema	& Capitolato - UC4.3	\\
						\hline
				R1F4.9&L'eliminazione di una stanza va salvata in Ethereum	& Verbale esterno 2021-01-01 - UC4.3	\\
						\hline
				R1F4.10&L'amministratore deve poter modificare il nome di una stanza	& Capitolato - UC4.4	\\
						\hline
			R1F4.11&L'amministratore deve poter impostare una stanza come inaccessibile per alcuni giorni, a partire anche da un giorno diverso da quello odierno	& Capitolato - UC4.4.1	\\
					\hline
			R1F4.12&Gli utenti che hanno prenotato una postazione devono essere avvisati se la postazione stessa o la stanza in cui si trovano diventa non disponibile	& Verbale esterno 2020-12-18 - UC4.6.1	\\
					\hline
R1F4.13&Se si tenta di creare una stanza o modificarla con un nome attualmente assegnato a un'altra, si riceve un messaggio d'errore e la stanza non viene creata o modificata	& Interno - UC4.2.1	\\
						\hline
			R1F4.14&L'amministratore deve poter aggiungere una postazione in una stanza, specificando: codice della postazione, codice del tag RFID che la identifica e posizione all'interno di una stanza	& Capitolato - UC4.5	\\
					\hline
			R1F4.15&L'aggiunta di una postazione va salvata in Ethereum	&Verbale esterno 2021-01-01 - UC4.5 	\\
					\hline
R1F4.16&L'amministratore deve poter eliminare una postazione	& Capitolato - UC4.6	\\
						\hline
		R1F4.17	&L'eliminazione di una postazione va salvata in Ethereum	& Verbale esterno 2021-01-01 - UC4.6	\\
					\hline
			R1F4.18&L'amministratore deve poter modificare i dati e la posizione di una postazione	& Capitolato - UC4.7	\\
					\hline
R1F4.19&	La modifica di una postazione va salvata in Ethereum& Verbale esterno 2021-01-01 - UC4.7	\\
					\hline
R1F4.20&	Se si tenta di creare o modificare una postazione assegnandole un codice già in utilizzo, l'azione deve fallire e deve comparire un messaggio d'errore& Verbale esterno 2020-12-18 - UC4.5.1 	\\
						\hline
			R1F4.21&Se si tenta di creare o modificare una postazione assegnandole un codice del tag già in utilizzo, l'azione deve fallire e deve comparire un messaggio d'errore	&Verbale esterno 2020-12-18 - UC4.5.2 	\\
					\hline
			R1F5&L'amministratore deve poter gestire le credenziali degli utenti	& Capitolato - UC5		\\
					\hline
R1F5.1&L'amministratore deve poter visualizzare una lista delle credenziali di tutti gli utente del sistema	& Capitolato - UC5.1	\\
					\hline
R1F5.2&L'amministratore deve poter creare nuove credenziali per accedere al sistema. Le informazioni contenute nel profilo di un utente sono: Nome, Cognome, Nome utente, Password e Email. Tra queste le informazioni necessarie per l'accesso sono password e nome utente	& Capitolato - Verbale esterno 2021-01-01 - UC5.2	\\
						\hline
			R1F5.3&L'amministratore può modificare le credenziali degli utenti del sistema	&Capitolato - UC5.3 	\\
					\hline
			R1F5.4&L'amministratore deve poter eliminare credenziali degli utenti del sistema	& Capitolato - UC5.4	\\
					\hline
R1F5.5&Il sistema deve poter memorizzare più credenziali per ogni categoria di utenti: amministratori, dipendenti e addetti	& Capitolato - Verbale esterno 2021-01-01	\\
					\hline
R1F6&L'amministratore deve poter esplorare la storia delle occupazioni delle postazioni	& Capitolato - UC6 - Verbale esterno 2021-01-04	\\
						\hline
			R1F6.1&L'amministratore deve poter visualizzare in forma tabellare le occupazioni di postazioni da parte di un utente specifico, con date e orari di inizio e fine delle occupazioni e codici delle postazioni.	&Capitolato - UC6.1 - Verbale esterno 2021-01-04	\\
					\hline
			R1F6.1.1&L'amministratore deve poter effettuare una ricerca tra le postazioni occupate da un utente, specificando il periodo all'interno del quale l'utente occupava la postazione. & 	Capitolato - UC6.1.1 - Verbale esterno 2021-01-04\\
					\hline
			R2F6.1.2&L'amministratore deve poter visualizzare in forma tabellare le occupazioni di postazioni da parte di un utente specifico, con date e orari di inizio e fine delle occupazioni,  codici delle postazioni e ore trascorse alla postazione. & 	Capitolato - UC6.1.2 - Verbale esterno 2021-01-04\\
			\hline
			R1F6.2&L'amministratore deve poter visualizzare in forma tabellare le occupazioni di una postazione specifica, con date e orari di inizio e fine delle occupazioni e nomi e cognomi dei dipendenti.	&Capitolato - UC6.2 - Verbale esterno 2021-01-04	\\
			\hline
			R1F6.2.1&L'amministratore deve poter effettuare una ricerca tra le occupazioni di una specifica postazione indicando il periodo all'interno del quale l'utente occupava la postazione	& 	Capitolato - UC6.2.1 - Verbale esterno 2021-01-04\\
			\hline
			R1F6.3 & L'amministratore deve poter scaricare il report delle occupazioni che sta visualizzando in un formato leggibile & Capitolato - UC6.3 - Verbale esterno 2021-01-04\\
			\hline
			R2F6.3.1 & L'amministratore deve poter scaricare il report delle occupazioni che sta visualizzando nel formato PDF & Capitolato - UC6.3.1 - Verbale esterno 2021-01-04\\
			\hline
R2F7&L'amministratore deve poter ottenere una tabella delle sanificazioni di tutte le postazioni, con data e ora in cui sono avvenute, nome e cognome di chi le ha eseguite e il suo ruolo.	&Capitolato - UC7 - Verbale esterno 2021-01-04	\\
					
					\hline
R2F7.1&L'amministratore deve poter scaricare il report delle sanificazioni che sta visualizzando in un formato leggibile.	&Capitolato - UC7.1 Verbale esterno 2021-01-04	\\

\hline
R2F7.1.1&L'amministratore deve poter scaricare il report delle sanificazioni che sta visualizzando nel formato PDF.	&Capitolato - UC7.1.1 - Verbale esterno 2021-01-04	\\

\hline
					R1F8&Il dipendente deve poter visualizzare lo stato di una postazione, tramite tag RFID.	&Capitolato - UC8 	\\
					\hline
			R1F9&Il dipendente deve poter segnalare la propria presenza su una postazione in tempo reale, tramite tag RFID.	&Capitolato - UC9 	\\
		\hline
		R1F9.1&Il dipendente deve poter registrare i momenti di inizio e di fine della sua occupazione di una postazione, tramite Ethereum.	&Interna - UC9.3 	\\
		\hline	
		R1F10&Il dipendente deve poter effettuare una pulizia autonoma di una postazione, tramite il kit aziendale.	&Capitolato - UC10	\\
		\hline		
		R1F10.1&Il dipendente, dopo aver effettuato una igienizzazione, deve poter marcare la postazione come libera e igienizzata.	&Interna - UC10.1	\\
		\hline	
		R1F10.2&Il dipendente deve poter registrare l'avvenuta igienizzazione autonoma della postazione, tramite Ethereum.	&Capitolato - UC10.2	\\
		\hline	
		R1F11&Il dipendente deve poter ricercare una postazione in base alla data, all'orario e all'identificativo della stanza.	&Interna - UC11.1	\\
		\hline
		R1F11.1&Il dipendente deve poter visualizzare le postazioni di una stanza.	&Capitolato - UC11.2	\\
		\hline
		R1F11.2&Il dipendente deve poter visualizzare l'identificativo della prima postazione igienizzata e libera in una stanza.	&Capitolato - UC11.4	\\
		\hline
		R1F11.3&Il dipendente deve poter selezionare una postazione di una stanza.	&Interna - UC11.5	\\
		\hline
		R1F11.4&Il dipendente deve poter prenotare una postazione selezionata.	&Capitolato - UC11.6	\\
		\hline
R1F12&L'addetto alle pulizie riceve l'elenco delle stanze e delle postazioni che necessitano di igienizzazione	& Interno - UC12	\\
						\hline
			R1F12.1&	L'addetto alle pulizie riceve l'elenco delle stanze che necessitano di igienizzazione& 	Capitolato - UC12.1\\
					\hline
			R1F12.2&L'addetto alle pulizie riceve l'elenco delle postazioni che necessitano di igienizzazione	& Interno - UC12.2	\\
					\hline
R1F13&L'addetto alle pulizie deve registrare sulla blockchain, con l'utilizzo di Ethereum, quando igienizza una stanza o una postazione	& Interno - UC13	\\
					\hline
R1F13.1&L'addetto alle pulizie deve poter marcare, nella sezione apposita, una stanza come igienizzata	& Capitolato - UC13.1	\\
						\hline
			R1F13.2&L'addetto alle pulizie deve poter marcare, nella sezione apposita, una postazione come igienizzata	&Interno - UC13.2 	\\
					\hline
			R1F13.3&L'addetto alle pulizie non può pulire postazioni o stanze occupate	& Verbale esterno 2021-01-01	\\
					\hline
						
	\end{longtable}
\end{center}
\subsection{Requisiti di qualità}
I requisiti R1Q3 e R1Q4 definiti nella tabella sottostante sono stati riportati anche tra i requisiti di vincolo in quanto imposti dal proponente.
\begin{center}
	\rowcolors{2}{lightest-grayest}{white}
	\begin{longtable}{|c|p{10cm}|p{4cm}|}
		\hline
		\rowcolor{lighter-grayer}
		\textbf{Requisito} & \textbf{Descrizione} & \textbf{Fonti}  \\
		\hline
		\endhead
		
		% ----- Modificare da qui -----
		 R1Q1 &La progettazione e la codifica devono rispettare le norme e le metriche definite nel documento \dext{Piano di qualifica v1.0.0} & Interno\\
		\hline	
		R1Q2 &Deve essere realizzato e consegnato un Docker file
		contenente la componente applicativa. & Capitolato\\
		\hline
		R1Q3 &Le componenti applicative devono essere correlate da test unitari e d’integrazione. & Capitolato\\
		\hline
		R1Q4 &Il sistema deve essere testato nella sua interezza tramite test end-to-end. & Capitolato\\
		\hline
		R1Q5 &La copertura dei test deve essere maggiore o uguale all'ottanta per cento e correlata di report. & Capitolato\\
		\hline
		R1Q6 &Devono essere riportati dei report sui test effettuati relativamente all’ottimizzazione dell’applicazione rispetto al consumo della
		batteria dei cellulari. & Capitolato\\
		\hline
		
	\end{longtable}
\end{center}
\subsection{Requisiti di vincolo}
\begin{center}
	\rowcolors{2}{lightest-grayest}{white}
	\begin{longtable}{|c|p{10cm}|p{4cm}|}
		\hline
		\rowcolor{lighter-grayer}
		\textbf{Requisito} & \textbf{Descrizione} & \textbf{Fonti}  \\
		\hline
		\endhead
		
		% ----- Modificare da qui -----
		 R1V1 &Avere il server che esponga, in aggiunta a eventuali altri protocolli per l’interazione con il servizio specifico, delle \glock{API} Rest attraverso le quali sia possibile utilizzare l'applicativo &Capitolato \\
		\hline
		R3V1.1&Deve essere possibile utilizzare gRPC come soluzione alternativa al Rest	& Capitolato	\\
		\hline
R1V2&Scansione dei codici nel tempo sufficiente a certificare la presenza  della persona in postazione.	& Capitolato	\\
		\hline
R1V2.1&L’utilizzo del lettore RFID riduce in modo rilevante l’autonomia dei cellulari, l’applicazione è da sviluppare in maniera tale da bilanciare nel miglior modo possibile batteria e scansioni. È richiesto un resoconto delle scelte fatte e dei test effettuati per garantire il miglior rapporto raggiunto.	& Capitolato	\\
		\hline
		R1V3&Avere le componenti applicative correlate da test unitari e d’integrazione.	& Capitolato	\\
		\hline
R1V4&Viene richiesto che il sistema sia testato nella sua interezza tramite test end-to-end	& Capitolato	\\
		\hline
R2V5&Le comunicazioni tra app e server devono essere cifrate	& Capitolato	\\
		\hline
		R2V6&	Deve essere fornita un'analisi del numero di utenti da supportare e del servizio cloud più adatto per adempiervi, analizzando prezzo, stabilità del servizio ed assistenza (supponendo di disporre di massimo 2 CPU e 1Gi2 per istanza del server).& Capitolato	\\
		\hline
R1V7&Deve essere sviluppata un'applicazione per Android o \glock{iOS}	& Capitolato	\\
		\hline
R1V8&Devono essere scritti dei test che abbiamo una copertura di almeno l'80 per cento del codice e deve essere fornito il report della loro esecuzione	& 	Capitolato \\
		\hline

R1V9&Devono essere eseguiti dei test che mettano in relazione la precisione del sistema e il consumo della batteria dei cellulari	& Capitolato - Verbale esterno 2020-12-18	\\
		\hline
R1V10&Deve essere fornita una documentazione delle scelte implementative e progettuali accompagnate dalle loro motivazioni	& Capitolato	\\
		\hline
	R1V11	&Deve essere fornita una documentazione dei problemi che rimangono aperti con eventuali proposte di soluzioni	& Capitolato	\\
		\hline
	\end{longtable}
\end{center}

\subsection{Requisiti prestazionali}
Non sono stati individuati requisiti prestazionali. Uno dei motivi sta nel fatto che la rete Ethereum, su cui si appoggerà il nostro sistema, ha tempi di esecuzione delle azioni dipendenti dal carico di utenti.
%\begin{center}
%	\rowcolors{2}{lightest-grayest}{white}
%	\begin{longtable}{|c|c|c|}
%		\hline
%		\rowcolor{lighter-grayer}
%		\textbf{Requisito} & \textbf{Descrizione} & \textbf{Fonti}  \\
%		\hline
%		\endfirsthead
%		
%		% ----- Modificare da qui -----
%		 R & & \\
%		\hline
%		
%	\end{longtable}
%\end{center}

\subsection{Tracciamento}
\subsubsection{Fonte - Requisiti}
\begin{center}
	\rowcolors{2}{lightest-grayest}{white}
	\begin{longtable}{|p{44mm}|p{22mm}|}
		\hline
		\rowcolor{lighter-grayer}
		\textbf{Fonte} &  \textbf{Requisiti}  \\
		\hline
		\endhead
		
		% ----- Modificare da qui -----
		
		
		 Interno & 
		 R2F1 \newline
		 	R1F2 \newline
		 	R1F2.1 \newline
		 	R1F2.2 \newline
			R1F3 \newline 	 
			R1F4.13 \newline
			R1F9.1 \newline
			R1F10.1 \newline
			R1F11 \newline
			R1F11.3 \newline
			R1F12 \newline
			R1F12.2 \newline
			R1F13 \newline
			R1F13.2 \newline
			R1Q1
	    \\
		\hline
		Verbale esterno 2020-12-18 & 
		
			R1F4.12 \newline
			R1F4.20 \newline
			R1F4.21 \newline
			R1V9
	\\
	\hline
	Verbale esterno 2021-01-01 & 
		R1F4.7 \newline
		R1F4.9 \newline
		R1F4.15 \newline
		R1F4.17 \newline
		R1F4.19 \newline
		R1F5.2 \newline
		R1F5.5 \newline
		R1F13.3
\\
\hline
Verbale esterno 2021-01-04 & 

	R1F6 \newline
	R1F6.1 \newline
	R1F6.1.1 \newline
	R1F6.1.2 \newline
	R1F6.2 \newline
	R1F6.2.1 \newline
	R1F6.3 \newline
	R2F6.3.1 \newline
	R2F7 \newline
	R2F7.1 \newline
	R2F7.1.1
\\
\hline
Capitolato & 

	R1F4  \newline
	R1F4.1\newline
	R1F4.2\newline
	R1F4.3\newline
	R1F4.4\newline
	R1F4.5\newline
	R1F4.6\newline
	R1F4.8\newline
	R1F4.10\newline
	R1F4.11\newline
	R1F4.14\newline
	R1F4.16\newline
	R1F4.18\newline
	R1F5\newline
	R1F5.1\newline
	R1F5.2\newline
	R1F5.3\newline
	R1F5.4\newline
	R1F5.5\newline
	R1F6\newline
	R1F6.1\newline
	R1F6.1.1\newline
	R2F6.1.2\newline
	R1F6.2\newline
	R1F6.2.1\newline
	R1F6.3\newline
	R2F6.3.1\newline
	R2F7\newline
	R2F7.1\newline
	R2F7.1.1\newline
	R1F8  \newline
	R1F9
	\\
	\hline
	Capitolato &
	R1F10  \newline
	R1F10.2\newline
	R1F11.1\newline
	R3F11.2\newline
	R1F11.4\newline
	R1F12.1\newline
	R1F13.1	\newline
	R1Q2  \newline
	R1Q3  \newline
	R1Q4   \newline	
	R1Q5  \newline
	R1Q6  \newline
	R1V1 \newline
	R3V1.1\newline
	R1V2 \newline
	R1V2.1\newline
	R1V3  \newline
	R1V4  \newline
	R2V5  \newline
	R2V6  \newline
	R1V7  \newline
	R1V8  \newline
	R1V9  \newline
	R1V10  \newline
	R1V11
\\
\hline
UC1 & R2F1  \\
\hline
UC2 & R1F2  \\
\hline
UC2.1 & R1F2.1  \\
\hline
UC2.2 & R1F2.2  \\
\hline
UC3 & R1F3  \\
\hline
UC4& R1F4 \\
\hline
UC4.1& 
	R1F4.1  \newline
	R1F4.2 \newline
	R1F4.3 \newline
	R1F4.4 
 \\
\hline
UC4.1.1& R1F4.5 \\
\hline
UC4.2.1& R1F4.13\\
\hline
UC4.2& 
	R1F4.6  \newline
	R1F4.7 
   \\
\hline
UC4.3& 
	R1F4.8  \newline
	R1F4.9 
   \\
\hline
UC4.4& R1F4.10\\
\hline
UC4.4.1& R1F4.11 \\
\hline
UC4.6.1& R1F4.12 \\
\hline

UC4.5&  
R1F4.14  \newline
R1F4.15  	
  \\ \hline
UC4.5.1 & R1F4.20  \\
\hline
UC4.5.2 & R1F4.21 \\
\hline

UC4.6&  
	R1F4.16  \newline
	R1F4.17  	
  \\
\hline
UC4.7& 
	R1F4.18  \newline
	R1F4.19	 
  \\
\hline
UC5& R1F5 \\
\hline
UC5.1& R1F5.1 \\
\hline
UC5.2& R1F5.2  \\
\hline
UC5.3& R1F5.3\\
\hline
UC5.4& R1F5.4 \\
\hline
UC6& R1F6 \\
\hline
UC6.1& R1F6.1 \\
\hline
UC6.1.1& R1F6.1.1 \\
\hline
UC6.1.2& R2F6.1.2 \\
\hline
UC6.2& R1F6.2 \\
\hline
UC6.2.1& R1F6.2.1 \\
\hline
UC6.3& R1F63 \\
\hline
UC6.3.1& R2F6.3.1 \\
\hline
UC7& R2F7\\
\hline
UC7.1& R2F7.1\\
\hline
UC7.1& R2F7.1.1\\
\hline
UC8& R1F8\\
\hline
UC9& R1F9\\
\hline
UC9.3& R1F9.1\\
\hline
UC10& R1F10\\
\hline
UC10.1& R1F10.1\\
\hline
UC10.2& R1F10.2\\
\hline
UC11.1& R1F11\\
\hline
UC11.2& R1F11.1\\
\hline
UC11.4& R3F11.2\\
\hline
UC11.5& R1F11.3\\
\hline
UC11.6& R1F11.4\\
\hline
UC12& R1F12\\
\hline
UC12.1& R1F12.1 \\
\hline
UC12.2& R1F12.2 \\
\hline
UC13&R1F13 \\
\hline
UC13.1& R1F13.1\\
\hline
UC13.2&R1F13.2 \\
\hline	
	\end{longtable}
\end{center}

\subsubsection{Requisito - Fonti}
\begin{center}
	\rowcolors{2}{lightest-grayest}{white}
	\begin{longtable}{|p{22mm}|p{22mm}|}
		\hline
		\rowcolor{lighter-grayer}
		\textbf{Requisito} &  \textbf{Fonti}  \\
		\hline
		\endhead
		
		% ----- Modificare da qui -----
		R2F1 & 
			Interno \newline
			UC1
		\\
	\hline
		R1F2 & 
		Interno \newline
		UC2
	\\
	\hline
		R1F2.1 & 
		Interno \newline
		UC2.1
	\\
	\hline
		R1F2.2 & 
		Interno \newline
		UC2.2
	\\
	\hline
		R1F3 & 
		Interno \newline
		UC3
	\\
	\hline
R1F4	& 
	Capitolato \newline	
	UC4
	\\
	\hline
R1F4.1	& 
	Capitolato \newline
	UC4.1	
	\\

	\hline
R1F4.2	& 
	Capitolato \newline
	UC4.1	
	\\
	\hline
R1F4.3	& 
	Capitolato \newline
	UC4.1	
	\\

	\hline
R1F4.4	& 
	Capitolato \newline
	UC4.1	
	\\
	\hline
R1F4.5	& 
	Capitolato \newline
	UC4.1.1	
	\\
	\hline
R1F4.6	& 
	Capitolato \newline
	UC4.2	
	\\
	\hline
R1F4.7	& 
	Verbale esterno 2021-01-01 \newline
	UC4.2	
	\\
	\hline
R1F4.8	& 
	Capitolato \newline
	UC4.3	
	\\
	\hline
R1F4.9	& 
	Verbale esterno 2021-01-01 \newline
	UC4.3	
	\\
	\hline
R1F4.10	& 
	Capitolato \newline	
	UC4.4
	\\
	\hline
R1F4.11	& 
	Capitolato \newline
	UC4.4.1	
	\\

	\hline
R1F4.12	& 
	Verbale esterno 2020-12-18 \newline	
	UC4.6.1
	\\
	\hline
R1F4.13	& 
	Interno \newline
	UC4.2.1	
	\\

	\hline
R1F4.14	& 
	Capitolato \newline
	UC4.5	
	\\
	\hline
R1F4.15	& 
	Verbale esterno 2021-01-01 \newline
	UC4.5	
	\\

	\hline
R1F4.16	& 
	Capitolato \newline
	UC4.6	
	\\
	\hline
R1F4.17	& 
		Verbale esterno 2021-01-01 \newline
		UC4.6
		\\
	\hline
R1F4.18	& 
	Capitolato \newline	
	UC4.7
	\\
	\hline
R1F4.19	& 
	Verbale esterno 2021-01-01 \newline
	UC4.7	
	\\

	\hline
R1F4.20	& 
	Verbale esterno 2020-12-18 \newline
	UC4.5.1
	\\
	\hline
R1F4.21	& 
	Verbale esterno 2020-12-18 \newline	
	UC4.5.2
	\\
	\hline
R1F5	& 
	Capitolato \newline	
	UC5
	\\
	\hline
R1F5.1	& 
	Capitolato \newline	
	UC5.1
	\\
	\hline
R1F5.2	& 
	Capitolato \newline	
	Verbale esterno 2021-01-01 \newline
	UC5.2
	\\
	\hline
R1F5.3	& 
	Capitolato \newline
	UC5.3	
	\\

	\hline
R1F5.4	& 
	Capitolato \newline
	UC5.4	
	\\
	\hline
R1F5.5	& 
	Capitolato \newline	
	Verbale esterno 2021-01-01
	\\
	\hline
R1F6	& 
		Capitolato \newline
		UC6 \newline
		Verbale esterno 2021-01-04\\
	\hline
R1F6.1	&
		Capitolato \newline
		UC6.1 \newline
		Verbale esterno 2021-01-04
	\\
\hline
R1F6.1.1	& 
	Capitolato \newline
	UC6.1.1 \newline
	Verbale esterno 2021-01-04
\\
\hline
R2F6.1.2	& 
	Capitolato \newline
	UC6.1.2 \newline
	Verbale esterno 2021-01-04
\\

	\hline
R1F6.2	& 
		Capitolato \newline
		UC6.2 \newline
		Verbale esterno 2021-01-04
	\\
	\hline
R1F6.2.1	& 
	Capitolato \newline
	UC6.2.1 \newline
	Verbale esterno 2021-01-04
\\
\hline
R1F6.3	&
	Capitolato \newline
	UC6.3 \newline
	Verbale esterno 2021-01-04
\\
\hline
R2F6.3.1	& 
	Capitolato \newline
	UC6.3.1 \newline
	Verbale esterno 2021-01-04
\\
	\hline
R2F7	& 
		Capitolato \newline
		UC7 \newline
		Verbale esterno 2021-01-04
	\\
	\hline
R2F7.1	&
	Capitolato \newline
	UC7.1 \newline
	Verbale esterno 2021-01-04
\\
\hline
R2F7.1.1	& 
	Capitolato \newline
	UC7.1.1 \newline
	Verbale esterno 2021-01-04
\\

	\hline
R1F8	& 
	Capitolato \newline
	UC8
\\
\hline	
R1F9	& 
	Capitolato \newline
	UC9
\\
\hline	
R1F9.1	& 
	Interna \newline
	UC9.3
\\
\hline
R1F10	& 
	Capitolato \newline
	UC10
\\
\hline
R1F10.1	& 
	Interna \newline
	UC10.1
\\
\hline	
R1F10.2	& 
	Capitolato \newline
	UC10.2
\\
\hline
R1F11	& 
	Interna \newline
	UC11.1
\\
\hline
R1F11.1	& 
	Capitolato \newline
	UC11.2
\\
\hline
R1F11.2	& 
	Capitolato \newline
	UC11.4
\\
\hline
R1F11.3	& 
	Interna \newline
	UC11.5
\\
\hline
R1F11.4	& 
	Capitolato \newline
	UC11.6
\\
\hline
R1F12	& 
		Interno \newline
		UC12
\\
	\hline
R1F12.1	& 
		Capitolato \newline
		UC12.1 \\
	\hline
R1F12.2	& 
		Interno \newline
		UC12.2
\\
	\hline
R1F13	& 
		Interno \newline
		UC13
\\
	\hline
R1F13.1		& 
	Capitolato \newline
	UC13.1	
	\\
	\hline
R1F13.2	& 
	Interno \newline
	UC13.2	
	\\
	
	\hline
R1F13.3	& Verbale esterno 2021-01-01 \\
	\hline
R1Q1	& Interno\\
	\hline
R1Q2	& Capitolato\\
\hline
R1Q3	& Capitolato\\
\hline
R1Q4	& Capitolato\\
\hline
R1Q5	& Capitolato\\
\hline
R1Q6	& Capitolato\\
\hline
R1V1	& Capitolato\\
	\hline
R3V1.1	& Capitolato\\
	\hline
R1V2	& Capitolato\\
	\hline
R1V2.1	& Capitolato\\
	\hline
R1V3	& Capitolato\\
	\hline
R1V4	& Capitolato\\
	\hline
R2V5	& Capitolato\\
	\hline
R2V6	& Capitolato\\
	\hline
R1V7	& Capitolato\\
	\hline
R1V8	& Capitolato\\
	\hline
R1V9	& 
	Capitolato \newline	
	Verbale esterno 2020-12-18
	\\
	\hline
R1V10	& Capitolato\\
	\hline
R1V11	& Capitolato\\
	\hline
	
	\end{longtable}
\end{center}

\subsection{Considerazioni}
Durante l'avanzamento del progetto i requisiti individuati potranno essere modificati e ampliati, e potranno esserne aggiunti di nuovi.