\section{Requisiti}
Per ogni requisito delle tabelle sottostanti è stata decisa la seguente struttura: 
\begin{itemize}
\item\textbf{Requisito:} R[Importanza][Tipologia][Codice];\\
Il significato delle seguenti voci è:
	\begin{itemize}
	\item\textbf{Importanza:} ogni requisito può assumere il seguente valore:
		\begin{itemize}
		\item 1: requisito obbligatorio, irrinunciabile;
		\item 2: requisito desiderabile, non strettamente necessario ma che porta valore aggiunto riconoscibile;
		\item 3: requisito opzionale, relativamente utile, contrattabile più in avanti nel progetto;
		\end{itemize}
	\item\textbf{Tipologia:} ogni requisito può assumere il seguente valore:
		\begin{itemize}
		\item F: funzionale;
		\item P: prestazionale;
		\item Q: qualitativo;
		\item V: vincolo;
		\end{itemize}
	\item\textbf{Codice:} è un identificatore univoco del requisito;
	\end{itemize}
\item\textbf{Descrzione:} descrizione breve ma completa del requisito, meno ambigua possibile;
\item\textbf{Fonti:} ogni requisito può derivare dalle seguenti fonti:
	\begin{itemize}
		\item Capitolato: si tratta di un requisito individuato dallla lettura del capitolato;
		\item Interno: si tratta di un requisito individuato nella fase di analisi;
		\item Caso d'uso: si tratta di un requisito estrapolato dai casi d'uso individuati;
		\item Verbale: si tratta di un requisito individuato nel verbale in seguito ai chiarimenti con il proponente;
	\end{itemize}
\end{itemize}


\subsection{Requisiti funzionali}
\begin{center}
	\rowcolors{2}{lightest-grayest}{white}
	\begin{longtable}{|c|p{10cm}|p{4cm}|}
		\hline
		\rowcolor{lighter-grayer}
		\textbf{Requisito} & \textbf{Descrizione} & \textbf{Fonti}  \\
		\hline
		\endfirsthead
		
		% ----- Modificare da qui -----
		 R2F1 & L'utente può leggere una breve guida per il login e ill funzionamento & Interno - UC1 \\
		\hline
		R1F3	&	L'utente autenticato deve potersi disautenticare dal sistema& Interno - UC3	\\
		\hline
		R2F4	&L'amminitratore deve poter gestire le impostazioni del sistema	& Capitolato - UC4	\\
					\hline
		R2F4.1	&	L'amministratore deve poter visualizzare la durata dei periodi che servono per accertare l'occupazione e la liberazione di una postazione da parte di un utente& Interno - UC4.1	\\
					\hline
			R2F4.2&L'amministratore deve poter modificare la durata dei periodi che servono per accertare l'occupazione e la liberazione di una postazione da parte di un utente	&Interno - UC4.2 	\\
					\hline
			R1F5&L'amministratore deve poter gestire le stanze e le postazioni salvate nel sistema	& Capitolato - UC5	\\
					\hline
			R1F5.1&L'amministratore deve poter visualizzare le stanze e le postazioni salvate nel sistema in modo schematico	& Capitolato - UC5.1	\\
					\hline
			R1F5.2&	L'amministratore deve poter vede le postazioni colorate in base al loro stato attuale.& Capitolato - UC5.1	\\
					\hline
			R1F5.3&	Gli stati delle postazioni devono essere i seguenti: libera e igienizzata, libera e non igienizzata, occupata, prenotata e igienizzata, prenotata e non igienizzata e inaccessibile& 	Capitolato - UC5.1\\
					\hline
			R1F5.4&Per ogni stanza deve essere indicato il numero di occupanti attuali	& Capitolato - UC5.1	\\
					\hline
			R1F5.5&L'amministratore deve poter visualizzare un calendario delle prenotazioni delle postazioni	& Capitolato - UC5.1.1	\\
					\hline
				R1F5.6&L'amministratore deve poter aggiungere una stanza al sistema e assegnarle un nome	& Capitolato - UC5.2	\\
						\hline
				R1F5.7&	La creazione di una nuova stanza va salvata in Ethereum&Verbale 2021-01-04 - UC5.2 	\\
						\hline
				R1F5.8&L'amministratore deve poter eliminare una stanza dal sistema	& Capitolato - UC5.3	\\
						\hline
				R1F5.9&L'eliminazione di una stanza va salvata in Ethereum	& Verbale 2021-01-04 - UC5.3	\\
						\hline
				R1F5.10&L'amministratore deve poter modificare il nome di una stanza	& Capitolato - UC5.4	\\
						\hline
			R1F5.11&L'amministratore deve poter impostare una stanza come inaccessibile per alcuni giorni, a partire anche da un giorno diverso da quello odierno	& Capitolato - UC5.4.1	\\
					\hline
			R1F5.12&Gli utenti che hanno prenotato una postazione devono essere avvisati se la postazione stessa o la stanza in cui si trova diventa non disponibile	& Verbale esterno 2020-12-18 - UC5.6.1	\\
					\hline
R1F5.13&Se si tenta di creare una stanza o modificarla con un nome attualmente assegnato a un'altra, si riceve un messaggio d'errore e la stanza non viene creata o modificata	& Interno - UC5.2.1	\\
						\hline
			R1F5.14&L'amministratore deve poter aggiungere una postazione in una stanza, specificando: codice della postazione, codice del tag RFID che la identifica e posizione all'interno di una stanza	& Capitolato - UC5.5	\\
					\hline
			R1F5.15&L'aggiunta di una postazioni va salvata in Ethereum	&Verbale 2021-01-04 - UC5.5 	\\
					\hline
R1F5.16&L'amministratore deve poter eliminare una postazione	& Capitolato - UC5.6	\\
						\hline
		R1F5.17	&L'eliminazione di una postazione va salvata in Ethereum	& Verbale 2021-01-04 - UC5.6	\\
					\hline
			R1F5.18&L'amministratore deve poter modificare i dati e la posizione di una postazione	& Capitolato - UC5.7	\\
					\hline
R1F5.19&	La modifica di una postazione va salvata in Ethereum& Verbale 2021-01-04 - UC5.7	\\
					\hline
R1F5.20&	Se si tenta di creare o modificare una postazione assegnandole un codice già in utilizzo l'azione deve fallire e deve comparire un messaggio d'errore& Verbale 2020-12-18 - UC5.5.1 	\\
						\hline
			R1F5.21&Se si tenta di creare o modificare una postazione assegnandole un codice del tag già in utilizzo l'azione deve fallire e deve comparire un messaggio d'errore	&Verbale 2020-12-18 - UC5.5.2 	\\
					\hline
			R1F6&L'amministratore deve poter gestire le credenziali degli utenti	& Capitolato - UC6		\\
					\hline
R1F6.1&L'amministratore deve poter visualizzare una lista delle credenziali di tutti gli utente del sistema	& Capitolato - UC6.1	\\
					\hline
R1F6.2&L'amministratore deve poter creare nuove credenziali per accedere al sistema. Le informazioni contenute nel profilo di un utente sono: Nome, Cognome, Nome utente, Password e E-mail. Tra queste le informazioni necessarie per l'accesso sono password ed e-mail	& Capitolato - Verbale 2021-01-04 - UC6.2	\\
						\hline
			R1F6.3&L'amministratore può modificare le credenziali degli utenti del sistema	&Capitolato - UC6.3 	\\
					\hline
			R1F6.4&L'amministratore deve poter eliminare credenziali di utenti del sistema	& Capitolato - UC6.4	\\
					\hline
R1F6.5&Il sistema deve poter memorizzare più credenziali per ogni categoria di utenti: amministratori, dipendenti e addetti	& Capitolato - Verbale 2021-01-04	\\
					\hline
R1F7&L'amministratore deve poter esplorare la storia delle postazioni occupate da uno specifico utente	& Capitolato - UC7	\\
						\hline
			R1F7.1&L'amministratore deve poter visualizzare in forma tabellare le occupazioni di postazioni da parte di un utente specifico, con date e orari di inizio e fine delle occupazioni e codici delle postazioni.	&Capitolato - UC7.1 	\\
					\hline
			R1F7.2&L'amministratore deve poter effettuare una ricerca tra le postazioni occupate da un utente. I valori che si possono specificare per la ricerca sono: codice della postazione e periodo all'interno del quale l'utente occupava la postazione	& 	Capitolato - UC7.2\\
					\hline
R1F8&L'amministratore deve poter ottenere una tabella delle sanificazioni di tutte le postazioni, con data e ora in cui sono avvenute.	&Capitolato - UC8 	\\
					\hline
R1F13&L'addetto alle pulizie riceve l'elenco delle stanze e delle postazioni che necessitano di igienizzazione	& Interno - UC13	\\
						\hline
			R1F13.1&	L'addetto alle pulizie riceve l'elenco delle stanze che necessitano di igienizzazione& 	Capitolato - UC13.1\\
					\hline
			R1F13.2&L'addetto alle pulizie riceve l'elenco delle postazioni che necessitano di igienizzazione	& Interno - UC13.2	\\
					\hline
R1F14&L'addeto alle pulizie deve registrare sulla blockchain con l'utilizzo di ethereum quando igienizza una stanza o una postazione	& Interno - UC14	\\
					\hline
R1F14.1&L'addetto alle pulizie deve poter marcare, nella sezione apposita, una stanza come igienizzata	& Capitoato - UC14.1	\\
						\hline
			R1F14.2&L'addetto alle pulizie deve poter marcare, nella sezione apposita, una postazione come igienizzata	&Interno - UC14.2 	\\
					\hline
			R1F14.3&L'addetto alle pulizie non può pulire postazioni o stanze occupate	& Verbale 2021-01-04	\\
					\hline
						
	\end{longtable}
\end{center}
\subsection{Requisiti di qualità}
\begin{center}
	\rowcolors{2}{lightest-grayest}{white}
	\begin{longtable}{|c|p{10cm}|p{4cm}|}
		\hline
		\rowcolor{lighter-grayer}
		\textbf{Requisito} & \textbf{Descrizione} & \textbf{Fonti}  \\
		\hline
		\endfirsthead
		
		% ----- Modificare da qui -----
		 R1Q1 &La progettazione e la codifica devono rispettare le norme e le metriche definite nel documento Piano di qualifica v1.0.0 & Interno\\
		\hline
		
	\end{longtable}
\end{center}
\subsection{Requisiti di vincolo}
\begin{center}
	\rowcolors{2}{lightest-grayest}{white}
	\begin{longtable}{|c|p{10cm}|p{4cm}|}
		\hline
		\rowcolor{lighter-grayer}
		\textbf{Requisito} & \textbf{Descrizione} & \textbf{Fonti}  \\
		\hline
		\endfirsthead
		
		% ----- Modificare da qui -----
		 R1V1 &avere il server che esponga, in aggiunta a eventuali altri protocolli per l’interazione con il servizio specifico, delle API Rest attraverso le quali sia possibile utilizzare l'applicativo &Capitolato \\
		\hline
		R3V1.1&è possibile utilizzare gRPC come soluzione alternativa al Rest	& Capitolato	\\
		\hline
R1V2&scansione dei codici nel tempo sufficiente a certificare la presenza  della persona in postazione.	& Capitolato	\\
		\hline
R1V2.1&L’utilizzo del lettore RFID riduce in modo rilevante l’autonomia dei cellulari, l’applicazione è da sviluppare in maniera tale da bilanciare nel miglior modo possibile batteria e scansioni. È richiesto un resoconto delle scelte fatte e dei test effettuati per garantire il miglior rapporto raggiunto.	& Capitolato	\\
		\hline
		R1V3&Avere le componenti applicative correlate da test unitari e d’integrazione.	& Capitolato	\\
		\hline
R1V4&è richiesto che il sistema venga testato nella sua interezza tramite test end-to-end	& Capitolato	\\
		\hline
R2V5&Le comunicazioni tra app e server devono essere cifrate	& Capitolato	\\
		\hline
		R2V6&	Deve essere fornita un'analisi del numero di utenti da supportare e del servizio cloud più adatto per adempiervi, analizzando prezzo, stabilità del servizio ed assistenza (supponendo di disporre di massimo 2 CPU e 1Gi2 per istanza del server).& Capitolato	\\
		\hline
R1V7&Deve essere sviluppata un'applicazione per Android o iOS	& Capitolato	\\
		\hline
R1V8&Devono essere scritti dei test che abbiamo una copertura di almeno l'80 per cento del codice e deve essere fornito il report della loro esecuzione	& 	\\
		\hline

R1V9&Devono essere eseguiti dei test che mettano in realzione la precisione del sistema e il consumo della batteria dei cellulari	& Capitolato - Verbale esterno 2020-12-18	\\
		\hline
R1V10&Deve essere fornita una documentazione delle scelte implementative e progettuali accompagnate dalle loro motivazioni	& Capitolato	\\
		\hline
	R1V11	&Deve essere fornita una documentazione dei problemi che rimangono aperti con eventuali proposte di soluzioni	& Capitolato	\\
		\hline
	\end{longtable}
\end{center}
\subsection{Requisiti prestazionali}

\subsection{Tracciamento}
\begin{center}
	\rowcolors{2}{lightest-grayest}{white}
	\begin{longtable}{|c|c|}
		\hline
		\rowcolor{lighter-grayer}
		\textbf{Requisito} &  \textbf{Fonti}  \\
		\hline
		\endfirsthead
		
		% ----- Modificare da qui -----
		 R & \\
		\hline
		
	\end{longtable}
\end{center}
\subsection{Considerazioni}