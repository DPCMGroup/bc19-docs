\section{Requisiti}
Per la classificazione dei requisiti si fa riferimento alla sezione Verifica\footnote{Sezione §3.4} del documento \dext{Norme di Progetto v. 2.0.0}.
Per ogni requisito delle tabelle sottostanti è stata decisa la seguente struttura: 
\begin{itemize}
	\item\textbf{Requisito:} R[Priorità][Tipologia][Identificativo];
	\item\textbf{Descrizione:} descrizione breve ma completa del requisito, meno ambigua possibile;
	\item\textbf{Fonti:} ogni requisito può derivare dalle seguenti fonti:
	\begin{itemize}
		\item Capitolato: si tratta di un requisito individuato dalla lettura del C1;
		\item Interna: si tratta di un requisito individuato nella fase di analisi;
		\item Caso d'uso: si tratta di un requisito estrapolato dai casi d'uso individuati;
		\item Verbale: si tratta di un requisito individuato nel verbale in seguito ai chiarimenti con il proponente.
	\end{itemize}
\end{itemize}
I verbali esterni che costituiscono fonte di requisiti sono stati approvati da Lorenzo Patera, referente di Imola Informatica.


\subsection{Requisiti funzionali}
\begin{center}
	\rowcolors{2}{lightest-grayest}{white}
	\begin{longtable}{|c|p{10cm}|p{4cm}|}
		\hline
		\rowcolor{lighter-grayer}
		\textbf{Requisito} & \textbf{Descrizione} & \textbf{Fonti}  \\
		\hline
		\endhead
		
		% ----- Modificare da qui -----
		
		%inizio requisiti funzionali%
		R2F1 & L'utente può leggere una breve guida per il login e il funzionamento. & Interna - UC1 \\
		\hline
		R1F2	&	L'utente non autenticato deve potersi autenticare nel webserver inserendo come credenziali: nome utente e password. & Capitolato - UC2	\\
		\hline
		R1F3	&	L'utente non autenticato deve ricevere un messaggio di errore nel caso abbia sbagliato a inserire le credenziali nel web server.& Interna - UC3	\\
		\hline
		R1F4	&	L'utente non autenticato che stia provando ad accedere al web server deve poter ricevere un messaggio di errore nel caso il suo account sia disabilitato.& Interna - UC4	\\
		\hline
		R1F5	&	L'utente non autenticato deve potersi autenticare nell'applicazione mobile inserendo come credenziali nome utente e password. & Capitolato - UC5	\\
		\hline
		R1F6	&	L'utente non autenticato deve ricevere un messaggio di errore nel caso abbia sbagliato a inserire le credenziali nell'applicazione mobile.& Interna - UC6	\\
		\hline
		R1F7	&	L'utente non autenticato che stia provando ad accedere all'applicazione mobile deve poter ricevere un messaggio di errore nel caso il suo account sia disabilitato.& Interna - UC7	\\
		\hline
		R1F8	&	L'utente autenticato come amministratore deve potersi deautenticare dal web server.& Interna - UC8	\\
		\hline
		R1F9	&	L'utente autenticato come dipendente deve potersi deautenticare dall'applicazione mobile.& Interna - UC9	\\
		\hline
		R1F10	&	L'utente autenticato come addetto alle pulizie deve potersi deautenticare dall'applicazione mobile.& Interna - UC10	\\
		\hline
		R1F11	&	L'utente autenticato come amministratore deve poter ottenere la guida completa delle varie funzionalità& Interna - UC11	\\
		\hline
		R1F12	&	L'utente autenticato come dipendente deve poter ottenere la guida completa delle varie funzionalità& Interna - UC12	\\
		\hline
		R1F13	&	L'utente autenticato come addetto alle pulizie deve poter ottenere la guida completa delle varie funzionalità& Interna - UC13	\\
		\hline
		R1F14&L'amministratore deve poter visualizzare le stanze e le postazioni salvate nel sistema.& Capitolato - UC14	\\
		\hline
		R1F14.1&L'amministratore deve poter visualizzare le stanze e le postazioni salvate nel sistema, in modo schematico.& Capitolato - UC14	\\
		\hline
		R1F14.2&L'amministratore deve poter visualizzare le stanze e le postazioni salvate nel sistema colorate in base al loro stato attuale. Gli stati possibili sono:
		\begin{itemize}
			\item libera e igienizzata;
			\item libera e non igienizzata;
			\item occupata;
			\item prenotata e igienizzata;
			\item prenotata e non igienizzata;
			\item guasta e igienizzata;
			\item guasta e non igienizzata.
		\end{itemize}& Capitolato - UC14	\\
		\hline
		R1F14.3&Per ogni stanza deve essere indicato il numero di occupanti attuali.	& Capitolato - UC14	\\
		\hline
		R1F15&L'amministratore deve poter visualizzare un calendario delle prenotazioni delle postazioni.	& Capitolato - UC15	\\
		\hline
		R1F16&L'amministratore deve poter aggiungere una stanza al sistema e assegnarle un nome.	& Capitolato - UC16	\\
		\hline
		R1F17 & Se l'amministratore tenta di assegnare a una stanza un nome già in utilizzo per un'altra stanza deve essergli impedito e deve essere avvisato dell'errore. & Interna - UC17 \\
		\hline
		R1F18&L'amministratore deve poter eliminare una stanza dal sistema.	& Capitolato - UC18	\\
		\hline
		R1F19&L'amministratore deve poter modificare il nome di una stanza.	& Capitolato - UC19	\\
		\hline
		R1F20&L'amministratore deve poter impostare una stanza come inaccessibile per alcuni giorni, a partire anche da un giorno diverso da quello odierno.	& Capitolato - UC20	\\
		\hline
		R1F21&L'amministratore deve poter aggiungere una postazione in una stanza, specificando:
		\begin{itemize}
			\item codice della postazione;
			\item codice del tag NFC che la identifica;
			\item posizione all'interno di una stanza.
		\end{itemize} & Capitolato - UC21	\\
		\hline
		R1F22&	Se si tenta di creare o modificare una postazione assegnandole un codice della postazione già in utilizzo, l'azione deve fallire e deve comparire un messaggio d'errore.& VE\_2020\_12\_18v.1.0.0 - UC22 	\\
		\hline
		R1F23&Se si tenta di creare o modificare una postazione assegnandole un codice del tag già in utilizzo, l'azione deve fallire e deve comparire un messaggio d'errore.	&VE\_2020\_12\_18v.1.0.0 - UC23 	\\
		\hline
		R1F24&Se si tenta di creare o modificare una postazione assegnandole una posizione già in utilizzo, l'azione deve fallire e deve comparire un messaggio d'errore.	&VE\_2020\_12\_18v.1.0.0 - UC24 	\\
		\hline
		R1F25&L'amministratore deve poter eliminare una postazione.	& Capitolato - UC25	\\
		\hline
		R1F26&L'amministratore deve poter modificare i dati e la posizione di una postazione.	& Capitolato - UC26	\\
		\hline
		R1F27&L'amministratore deve poter visualizzare una lista delle credenziali di tutti gli utente del sistema.	& Capitolato - UC27	\\
		\hline
		R1F27.1&Il sistema deve poter memorizzare più credenziali per ogni categoria di utenti: amministratori, dipendenti e addetti.	& Capitolato - VE\_2021\_01\_01v.1.0.0	\\
		\hline
		R1F28&L'amministratore deve poter creare nuove credenziali per accedere al sistema. Le informazioni contenute nel profilo di un utente sono: Nome, Cognome, Nome utente, Password e Email.	& Capitolato - VE\_2021\_01\_01v.1.0.0 - UC28	\\
		\hline
		R1F29&L'amministratore può modificare le credenziali degli utenti del sistema.	&Capitolato - UC29	\\
		\hline
		R1F30&L'amministratore deve poter eliminare credenziali degli utenti del sistema.	& Capitolato - UC30	\\
		\hline	
		R1F31&L'amministratore deve poter esplorare la storia delle occupazioni delle postazioni.	& Capitolato - UC31 - VE\_2021\_01\_04v.1.0.0	\\
		\hline
		R1F32&L'amministratore deve poter effettuare una ricerca tra le postazioni occupate da un utente in un dato periodo, specificando il periodo all'interno del quale l'utente occupava la postazione e ottenere un report. & 	Capitolato - UC32 - VE\_2021\_01\_04v.1.0.0\\
		\hline
		R2F33&L'amministratore deve poter ottenere un report delle occupazioni di postazioni da parte di un utente specifico, con date e orari di inizio e fine delle occupazioni,  codici delle postazioni e, opzionalmente, ore trascorse alla postazione. & 	Capitolato - UC33 - VE\_2021\_01\_04v.1.0.0\\
		\hline
		R1F34&L'amministratore deve poter ottenere un report delle occupazioni di una postazione specifica, con date e orari di inizio e fine delle occupazioni e nomi e cognomi dei dipendenti.	&Capitolato - UC34 - VE\_2021\_01\_04v.1.0.0	\\
		\hline
		R1F35&L'amministratore deve poter ottenere un report delle occupazioni e delle igienizzazioni in base alla sua ricerca.	&Capitolato - UC35 - VE\_2021\_01\_04v.1.0.0	\\
		\hline
		R2F36 & L'amministratore deve poter ottenere un report degli utenti che hanno condiviso la stanza con un determinato utente. & Capitolato - UC36 - VE\_2021\_01\_04v.1.0.0\\
		\hline
		R2F37&L'amministratore deve poter ottenere una tabella delle igienizzazioni di tutte le postazioni, con data e ora in cui sono avvenute, nome e cognome di chi le ha eseguite e il suo ruolo.	&Capitolato - UC37 - VE\_2021\_01\_04v.1.0.0	\\
		\hline
		R1F38 & L'amministratore deve poter scaricare il report che sta visualizzando in un formato leggibile. & Capitolato - UC38 - VE\_2021\_01\_04v.1.0.0\\
		\hline	
		R2F39 & L'amministratore deve poter scaricare il report che sta visualizzando nel formato PDF. & Capitolato - UC39 - VE\_2021\_01\_04v.1.0.0\\
		\hline		 
		R1F40 & Il dipendente deve poter effettuare una scansione del tag NFC presente su una postazione. & Capitolato - UC40\\
		\hline
		R1F41&Il dipendente deve poter visualizzare lo stato di una postazione, tramite tag NFC.	&Capitolato - UC41 	\\
		\hline
		R1F42 & Il dipendente deve poter segnalare la propria presenza su una postazione in tempo reale. & Interna - UC42 \\
		\hline
		R1F42.1&Il dipendente deve poter segnalare la propria presenza su una postazione in tempo reale, tramite tag NFC.	&Capitolato - UC42.1 	\\
		\hline
		R1F42.2&Il dipendente deve poter registrare i momenti di inizio e di fine della sua occupazione di una postazione.	&Interna - UC42.2 	\\
		\hline
		R1F43&Se l'utente sposta il proprio smartphone dal tag NFC per un periodo di tempo maggiore o uguale a trenta minuti avviene la disdetta automatica della prenotazione per il tempo restante.	&Interna - UC43 	\\
		\hline
		R1F44&L’utente viene avvisato della disdetta automatica di una sua prenotazione se dopo 30 minuti dal suo inizio non ha ancora appoggiato il telefono sul tag.	&Interna - UC44 	\\
		\hline
		R1F45&Il dipendente visualizza un messaggio di errore se prova ad occupare una postazione non igienizzata. &Interna - UC45 	\\
		\hline	
		R1F46&Il dipendente visualizza un messaggio di inizio della prenotazione, 30 minuti prima che essa avvenga.	&Interna - UC46 	\\
		\hline
		R1F47&Il dipendente visualizza un messaggio di fine della prenotazione, cinque minuti prima che essa avvenga.	&Interna - UC47 	\\
		\hline
		R1F48&Il dipendente deve poter registrare l'avvenuta igienizzazione autonoma della postazione.	&Capitolato - UC48	\\
		\hline	
		R1F49&Il dipendente deve poter ricercare una postazione in base alla data, all'orario e all'identificativo della stanza.	&Interna - UC49	\\
		\hline
		R1F49.1&Il dipendente deve compilare il campo della data per cercare una postazione.	&Interna - UC49.1	\\
		\hline
		R1F50&Il dipendente visualizza un messaggio di errore se il campo data ha un valore non valido.	&Interna - UC50	\\
		\hline
		R1F49.2&Il dipendente deve compilare il campo dell'ora per cercare una postazione.	&Interna - UC49.2	\\
		\hline
		R1F51&Il dipendente visualizza un messaggio di errore se il campo ora ha un valore non valido.	&Interna - UC51	\\
		\hline
		R1F49.3&Il dipendente deve compilare il campo della stanza per cercare una postazione.		&Interna - UC49.3	\\
		\hline
		R1F52&Il dipendente visualizza un messaggio di errore se il campo stanza ha un valore non valido.	&Interna - UC52	\\
		\hline
		R1F49.4&Il dipendente deve compilare il campo nome dipendente per cercare una postazione.	&Interna - UC49.4	\\
		\hline
		R1F53&Il dipendente visualizza un messaggio di errore se il campo nome dipendente ha un valore non valido.	&Interna - UC53	\\
		\hline
		R1F54&Il dipendente deve poter visualizzare le postazioni di una stanza.	&Capitolato - UC54	\\
		\hline
		R1F55&Il dipendente visualizza un messaggio di errore se non ci sono postazioni prenotabili.	&Interna - UC55	\\
		\hline
		R1F56&Il dipendente deve poter visualizzare l'identificativo della prima postazione igienizzata e libera in una stanza.	&Capitolato - UC56	\\
		\hline
		R1F57&Il dipendente deve poter visualizzare l'identificativo di una postazione igienizzata e libera in una stanza scelta con criterio random.	&Capitolato - UC57	\\
		\hline
		R2F58&Il dipendente deve poter visualizzare l'identificativo di una postazione igienizzata e libera in una stanza scelta con criterio di massima distanza dalle altre postazioni occupate.	&Capitolato - UC58	\\
		\hline
		R1F59&Il dipendente deve poter prenotare una postazione selezionata.	&Capitolato - UC59	\\
		\hline
		R1F60&Il dipendente visualizza un messaggio di errore se la postazione non è prenotabile.	&Capitolato - UC60	\\
		\hline
		R1F61&Se il dipendente scansiona con il proprio smartphone il tag NFC per un tempo maggiore o uguale ad un minuto, prenota in modo automatico la postazione per l’intera giornata lavorativa. &Interna - UC61	\\
		\hline
		R1F62&Se la postazione è occupata, viene emesso un segnale sonoro per avvisare il dipendente.	& Interna - UC62	\\
		\hline
		R1F63&Se il dipendente scansiona con il proprio smartphone il tag NFC per un tempo maggiore o uguale ad un minuto, e questa risulta prenotata in un momento successivo della giornata, la postazione viene prenotata fino all'inizio della prenotazione successiva.	&Interna - UC63	\\
		\hline	
		R1F64&Il dipendente deve poter visualizzare un elenco delle prenotazioni che ha effettuato. & Capitolato - UC64	\\
		\hline
		R1F65&Il dipendente, dopo aver selezionato una prenotazione, deve poterla disdire.	&Capitolato - UC65 	\\
		\hline
		R1F66&Gli utenti che hanno prenotato una postazione devono essere avvisati se la postazione stessa o la stanza in cui si trovano diventa non disponibile.	& VE\_2020\_12\_18v.1.0.0 - UC66	\\
		\hline
		
		R1F67&	L'addetto alle pulizie riceve l'elenco delle stanze che necessitano di igienizzazione.& 	Capitolato - UC67\\
		\hline
		R1F68&L'addetto alle pulizie riceve l'elenco delle postazioni che necessitano di igienizzazione.	& Interna - UC68	\\
		\hline
		R1F69&L'addetto alle pulizie deve poter marcare, nella sezione apposita, una stanza come igienizzata.	& Capitolato - UC69	\\
		\hline
		R1F70&L'addetto alle pulizie deve poter marcare, nella sezione apposita, una postazione come igienizzata.	&Interna - UC70 	\\
		\hline
		R1F70.1&L'addetto alle pulizie non può pulire postazioni o stanze occupate.	& VE\_2021\_01\_01v.1.0.0	\\
		\hline
		R1F71&Dopo 24 ore dalla generazione dell'ultimo certificato su Ethereum deve avvenire un'altra transazione e l'amministratore deve ricevere una notifica se è avvenuta con successo.	& Capitolato - UC71	\\
		\hline
		R1F72&L'amministratore deve ricevere una notifica con un messaggio di errore nel caso in cui la transazione su Ethereum sia fallita.	& Capitolato - UC72	\\
		\hline
		R1F73&Tramite Ethereum vanno salvate tutte le operazioni che riguardano l'occupazione, l'igienizzazione e le varie modifiche alle stanze e alle postazioni.	& Capitolato	\\
		\hline
		R1F74 &Deve essere realizzato e consegnato un Docker file contenente la componente applicativa. & Capitolato\\
		\hline
		%fine requisiti funzionali%
		
	\end{longtable}
\end{center}


\subsection{Requisiti di qualità}
\begin{center}
	\rowcolors{2}{lightest-grayest}{white}
	\begin{longtable}{|c|p{10cm}|p{4cm}|}
		\hline
		\rowcolor{lighter-grayer}
		\textbf{Requisito} & \textbf{Descrizione} & \textbf{Fonti}  \\
		\hline
		\endhead
		
		% ----- Modificare da qui -----
		
		%inizio requisiti qualita%
		R1Q1 &La progettazione e la codifica devono rispettare le norme e le metriche definite nel documento. \dext{Piano di qualifica v2.0.0}. & Interna\\
		\hline	
		R1Q2 &Deve essere prodotto un manuale sviluppatore. & Capitolato\\
		\hline
		R1Q3 &Le componenti applicative devono essere correlate da test unitari e d’integrazione. & Capitolato\\
		\hline
		R1Q4 &Il sistema deve essere testato nella sua interezza tramite test end-to-end. & Capitolato\\
		\hline
		R1Q5 &La copertura dei test deve essere maggiore o uguale all'ottanta per cento e correlata di report. & Capitolato\\
		\hline
		R1Q6 &Devono essere riportati dei report sui test effettuati relativamente all’ottimizzazione dell’applicazione rispetto al consumo della batteria dei cellulari. & Capitolato\\
		\hline
		R2Q7&	Deve essere fornita un'analisi del numero di utenti da supportare e del servizio cloud più adatto per adempiervi, analizzando prezzo, stabilità del servizio ed assistenza (supponendo di disporre di massimo 2 CPU e 1Gi2 per istanza del server).& Capitolato	\\
		\hline
		R1Q8&Deve essere fornita una documentazione delle scelte implementative e progettuali accompagnate dalle loro motivazioni.	& Capitolato	\\
		\hline
		R1Q9	&Deve essere fornita una documentazione dei problemi che rimangono aperti con eventuali proposte di soluzioni.	& Capitolato	\\
		\hline
		R1Q10&Devono essere scritti dei test che abbiamo una copertura di almeno l'80 per cento del codice e deve essere fornito il report della loro esecuzione.	& 	Capitolato \\
		\hline
		
		R1Q11&Devono essere eseguiti dei test che mettano in relazione la precisione del sistema e il consumo della batteria dei cellulari.	& Capitolato - VE\_2020\_12\_18v.1.0.0	\\
		\hline
		R1Q12&L’utilizzo del lettore NFC riduce in modo rilevante l’autonomia dei cellulari, l’applicazione è da sviluppare in maniera tale da bilanciare nel miglior modo possibile batteria e scansioni. È richiesto un resoconto delle scelte fatte e dei test effettuati per garantire il miglior rapporto raggiunto.	& Capitolato	\\
		\hline
		R1Q13& Deve essere prodotto un manuale utente.	& Capitolato	\\
		\hline
		%fine requisiti qualita%
		
	\end{longtable}
\end{center}
\subsection{Requisiti di vincolo}
\begin{center}
	\rowcolors{2}{lightest-grayest}{white}
	\begin{longtable}{|c|p{10cm}|p{4cm}|}
		\hline
		\rowcolor{lighter-grayer}
		\textbf{Requisito} & \textbf{Descrizione} & \textbf{Fonti}  \\
		\hline
		\endhead
		
		% ----- Modificare da qui -----
		
		%inizio requisiti vincolo%
		R1V1 &Avere il server che esponga, in aggiunta a eventuali altri protocolli per l’interazione con il servizio specifico, delle API Rest, o in altrernativa gRPC, attraverso le quali sia possibile utilizzare l’applicativo. & Capitolato \\
		\hline
		
		R2V2&Le comunicazioni tra app e server devono essere cifrate.	& Capitolato	\\
		
		\hline
		R1V3&Deve essere sviluppata un'applicazione per Android o \glock{iOS}. Come versioni minime sono state stabilite Android 6.0 Marshmallow e iOS9.	& Capitolato	\\
		\hline
		%inizio requisiti vincolo%
		
		
	\end{longtable}
\end{center}

\subsection{Requisiti prestazionali}
Non sono stati individuati requisiti prestazionali. Uno dei motivi sta nel fatto che la rete Ethereum, su cui si appoggerà il nostro sistema, ha tempi di esecuzione delle azioni dipendenti dal carico di utenti.
%\begin{center}
%	\rowcolors{2}{lightest-grayest}{white}
%	\begin{longtable}{|c|c|c|}
%		\hline
%		\rowcolor{lighter-grayer}
%		\textbf{Requisito} & \textbf{Descrizione} & \textbf{Fonti}  \\
%		\hline
%		\endfirsthead
%		
%		% ----- Modificare da qui -----
%		 R & & \\
%		\hline
%		
%	\end{longtable}
%\end{center}

\subsection{Tracciamento}
\subsubsection{Fonte - Requisiti}
\begin{center}
	\rowcolors{2}{lightest-grayest}{white}
	\begin{longtable}{|p{44mm}|p{22mm}|}
		\hline
		\rowcolor{lighter-grayer}
		\textbf{Fonte} &  \textbf{Requisiti}  \\
		\hline
		\endhead
		
		% ----- Modificare da qui -----
		
		%inizio tracciamento fonte-requisiti%
Capitolato &
R1F2 \newline
R1F5 \newline
R1F14 \newline
R1F14.1 \newline
R1F14.2 \newline
R1F14.3 \newline
R1F15 \newline
R1F16 \newline
R1F18 \newline
R1F19 \newline
R1F20 \newline
R1F21 \newline
R1F25 \newline
R1F26 \newline
R1F27 \newline
R1F27.1 \newline
R1F28 \newline
R1F29 \newline
R1F30 \newline
R1F31 \newline
R1F32 \newline
R2F33 \newline
R1F34 \newline
R1F35 \newline
R2F36 \newline
R2F37 \newline
R1F38 \newline
R2F39 \newline
R1F40 \newline
R1F41 \newline
R1F42.1 \newline
R1F48 \newline
R1F54 \newline
R1F56 \newline
R1F57 \newline
R2F58 \newline
R1F59 \newline
R1F60 \newline
R1F64 \newline
R1F65 \newline
R1F67 \newline
R1F69 \newline
R1F71 \newline
R1F72 \newline
R1F73 \newline
R1F74 \newline
R1Q2 \newline
R1Q3
\\
\hline
Capitolato &R1Q4 \newline
R1Q5 \newline
R1Q6 \newline
R2Q7 \newline
R1Q8 \newline
R1Q9 \newline
R1Q10 \newline
R1Q11 \newline
R1Q12 \newline
R1Q13 \newline
R1V1 \newline
R2V2 \newline
R1V3 
\\
\hline
Interna &
R2F1 \newline
R1F3 \newline
R1F4 \newline
R1F6 \newline
R1F7 \newline
R1F8 \newline
R1F9 \newline
R1F10 \newline
R1F11 \newline
R1F12 \newline
R1F13 \newline
R1F17 \newline
R1F42 \newline
R1F42.2 \newline
R1F43 \newline
R1F44 \newline
R1F45 \newline
R1F46 \newline
R1F47 \newline
R1F49 \newline
R1F49.1 \newline
R1F50 \newline
R1F49.2 \newline
R1F51 \newline
R1F49.3 \newline
R1F52 \newline
R1F49.4 \newline
R1F53 \newline
R1F55 \newline
R1F61 \newline
R1F62 \newline
R1F63 \newline
R1F68 \newline
R1F70 \newline
R1Q1 
\\
\hline
VE\_2020\_12\_18v.1.0.0 &
R1F22 \newline
R1F23 \newline
R1F24 \newline
R1F66 \newline
R1Q11 
\\
\hline
VE\_2021\_01\_01v.1.0.0 &
R1F27.1 \newline
R1F28 \newline
R1F70.1 
\\
\hline
VE\_2021\_01\_04v.1.0.0 &
R1F31 \newline
R1F32 \newline
R2F33 \newline
R1F34 \newline
R1F35 \newline
R2F36 \newline
R2F37 \newline
R1F38 \newline
R2F39 
\\
\hline
UC1 &
R2F1 
\\
\hline
UC2 &
R1F2 
\\
\hline
UC3 &
R1F3 
\\
\hline
UC4 &
R1F4 
\\
\hline
UC5 &
R1F5 
\\
\hline
UC6 &
R1F6 
\\
\hline
UC7 &
R1F7 
\\
\hline
UC8 &
R1F8 
\\
\hline
UC9 &
R1F9 
\\
\hline
UC10 &
R1F10 
\\
\hline
UC11 &
R1F11 
\\
\hline
UC12 &
R1F12 
\\
\hline
UC13 &
R1F13 
\\
\hline
UC14 &
R1F14 \newline
R1F14.1 \newline
R1F14.2 \newline
R1F14.3 
\\
\hline
UC15 &
R1F15 
\\
\hline
UC16 &
R1F16 
\\
\hline
UC17 &
R1F17 
\\
\hline
UC18 &
R1F18 
\\
\hline
UC19 &
R1F19 
\\
\hline
UC20 &
R1F20 
\\
\hline
UC21 &
R1F21 
\\
\hline
UC22 &
R1F22 
\\
\hline
UC23 &
R1F23 
\\
\hline
UC24 &
R1F24 
\\
\hline
UC25 &
R1F25 
\\
\hline
UC26 &
R1F26 
\\
\hline
UC27 &
R1F27 
\\
\hline
UC28 &
R1F28 
\\
\hline
UC29 &
R1F29 
\\
\hline
UC30 &
R1F30 
\\
\hline
UC31 &
R1F31 
\\
\hline
UC32 &
R1F32 
\\
\hline
UC33 &
R2F33 
\\
\hline
UC34 &
R1F34 
\\
\hline
UC35 &
R1F35 
\\
\hline
UC36 &
R2F36 
\\
\hline
UC37 &
R2F37 
\\
\hline
UC38 &
R1F38 
\\
\hline
UC39 &
R2F39 
\\
\hline
UC40 &
R1F40 
\\
\hline
UC41 &
R1F41 
\\
\hline
UC42 &
R1F42 
\\
\hline
UC42.1 &
R1F42.1 
\\
\hline
UC42.2 &
R1F42.2 
\\
\hline
UC43 &
R1F43 
\\
\hline
UC44 &
R1F44 
\\
\hline
UC45 &
R1F45 
\\
\hline
UC46 &
R1F46 
\\
\hline
UC47 &
R1F47 
\\
\hline
UC48 &
R1F48 
\\
\hline
UC49 &
R1F49 
\\
\hline
UC49.1 &
R1F49.1 
\\
\hline
UC50 &
R1F50 
\\
\hline
UC49.2 &
R1F49.2 
\\
\hline
UC51 &
R1F51 
\\
\hline
UC49.3 &
R1F49.3 
\\
\hline
UC52 &
R1F52 
\\
\hline
UC49.4 &
R1F49.4 
\\
\hline
UC53 &
R1F53 
\\
\hline
UC54 &
R1F54 
\\
\hline
UC55 &
R1F55 
\\
\hline
UC56 &
R1F56 
\\
\hline
UC57 &
R1F57 
\\
\hline
UC58 &
R2F58 
\\
\hline
UC59 &
R1F59 
\\
\hline
UC60 &
R1F60 
\\
\hline
UC61 &
R1F61 
\\
\hline
UC62 &
R1F62 
\\
\hline
UC63 &
R1F63 
\\
\hline
UC64 &
R1F64 
\\
\hline
UC65 &
R1F65 
\\
\hline
UC66 &
R1F66 
\\
\hline
UC67 &
R1F67 
\\
\hline
UC68 &
R1F68 
\\
\hline
UC69 &
R1F69 
\\
\hline
UC70 &
R1F70 
\\
\hline
UC71 &
R1F71 
\\
\hline
UC72 &
R1F72 
\\
\hline%fine tracciamento fonte-requisiti%
		
	\end{longtable}
\end{center}

\subsubsection{Requisito - Fonti}
\begin{center}
	\rowcolors{2}{lightest-grayest}{white}
	\begin{longtable}{|p{22mm}|p{44mm}|}
		\hline
		\rowcolor{lighter-grayer}
		\textbf{Requisito} &  \textbf{Fonti}  \\
		\hline
		\endhead
		
		% ----- Modificare da qui -----
		
		%inizio tracciamento requisito-fonti%

R2F1 &
Interna \newline
UC1 
\\
\hline

R1F2 &
Capitolato \newline
UC2 
\\
\hline

R1F3 &
Interna \newline
UC3 
\\
\hline

R1F4 &
Interna \newline
UC4 
\\
\hline

R1F5 &
Capitolato \newline
UC5 
\\
\hline

R1F6 &
Interna \newline
UC6 
\\
\hline

R1F7 &
Interna \newline
UC7 
\\
\hline

R1F8 &
Interna \newline
UC8 
\\
\hline

R1F9 &
Interna \newline
UC9 
\\
\hline

R1F10 &
Interna \newline
UC10 
\\
\hline

R1F11 &
Interna \newline
UC11 
\\
\hline

R1F12 &
Interna \newline
UC12 
\\
\hline

R1F13 &
Interna \newline
UC13 
\\
\hline

R1F14 &
Capitolato \newline
UC14 
\\
\hline

R1F14.1 &
Capitolato \newline
UC14 
\\
\hline

R1F14.2 &
Capitolato \newline
UC14 
\\
\hline

R1F14.3 &
Capitolato \newline
UC14 
\\
\hline

R1F15 &
Capitolato \newline
UC15 
\\
\hline

R1F16 &
Capitolato \newline
UC16 
\\
\hline

R1F17 &
Interna \newline
UC17 
\\
\hline

R1F18 &
Capitolato \newline
UC18 
\\
\hline

R1F19 &
Capitolato \newline
UC19 
\\
\hline

R1F20 &
Capitolato \newline
UC20 
\\
\hline

R1F21 &
Capitolato \newline
UC21 
\\
\hline

R1F22 &
VE\_2020\_12\_18v.1.0.0 \newline
UC22 
\\
\hline

R1F23 &
VE\_2020\_12\_18v.1.0.0 \newline
UC23 
\\
\hline

R1F24 &
VE\_2020\_12\_18v.1.0.0 \newline
UC24 
\\
\hline

R1F25 &
Capitolato \newline
UC25 
\\
\hline

R1F26 &
Capitolato \newline
UC26 
\\
\hline

R1F27 &
Capitolato \newline
UC27 
\\
\hline

R1F27.1 &
Capitolato \newline
VE\_2021\_01\_01v.1.0.0 
\\
\hline

R1F28 &
Capitolato \newline
VE\_2021\_01\_01v.1.0.0 \newline
UC28 
\\
\hline

R1F29 &
Capitolato \newline
UC29 
\\
\hline

R1F30 &
Capitolato \newline
UC30 
\\
\hline

R1F31 &
Capitolato \newline
UC31 \newline
VE\_2021\_01\_04v.1.0.0 
\\
\hline

R1F32 &
Capitolato \newline
UC32 \newline
VE\_2021\_01\_04v.1.0.0 
\\
\hline

R2F33 &
Capitolato \newline
UC33 \newline
VE\_2021\_01\_04v.1.0.0 
\\
\hline

R1F34 &
Capitolato \newline
UC34 \newline
VE\_2021\_01\_04v.1.0.0 
\\
\hline

R1F35 &
Capitolato \newline
UC35 \newline
VE\_2021\_01\_04v.1.0.0 
\\
\hline

R2F36 &
Capitolato \newline
UC36 \newline
VE\_2021\_01\_04v.1.0.0 
\\
\hline

R2F37 &
Capitolato \newline
UC37 \newline
VE\_2021\_01\_04v.1.0.0 
\\
\hline

R1F38 &
Capitolato \newline
UC38 \newline
VE\_2021\_01\_04v.1.0.0 
\\
\hline

R2F39 &
Capitolato \newline
UC39 \newline
VE\_2021\_01\_04v.1.0.0 
\\
\hline

 
R1F40 &
Capitolato \newline
UC40 
\\
\hline

R1F41 &
Capitolato \newline
UC41 
\\
\hline

R1F42 &
Interna \newline
UC42 
\\
\hline

R1F42.1 &
Capitolato \newline
UC42.1 
\\
\hline

R1F42.2 &
Interna \newline
UC42.2 
\\
\hline

R1F43 &
Interna \newline
UC43 
\\
\hline

R1F44 &
Interna \newline
UC44 
\\
\hline

R1F45 &
Interna \newline
UC45 
\\
\hline

R1F46 &
Interna \newline
UC46 
\\
\hline

R1F47 &
Interna \newline
UC47 
\\
\hline

R1F48 &
Capitolato \newline
UC48 
\\
\hline

R1F49 &
Interna \newline
UC49 
\\
\hline

R1F49.1 &
Interna \newline
UC49.1 
\\
\hline

R1F50 &
Interna \newline
UC50 
\\
\hline

R1F49.2 &
Interna \newline
UC49.2 
\\
\hline

R1F51 &
Interna \newline
UC51 
\\
\hline

R1F49.3 &
Interna \newline
UC49.3 
\\
\hline

R1F52 &
Interna \newline
UC52 
\\
\hline

R1F49.4 &
Interna \newline
UC49.4 
\\
\hline

R1F53 &
Interna \newline
UC53 
\\
\hline

R1F54 &
Capitolato \newline
UC54 
\\
\hline

R1F55 &
Interna \newline
UC55 
\\
\hline

R1F56 &
Capitolato \newline
UC56 
\\
\hline

R1F57 &
Capitolato \newline
UC57 
\\
\hline

R2F58 &
Capitolato \newline
UC58 
\\
\hline

R1F59 &
Capitolato \newline
UC59 
\\
\hline

R1F60 &
Capitolato \newline
UC60 
\\
\hline

R1F61 &
Interna \newline
UC61 
\\
\hline

R1F62 &
Interna \newline
UC62 
\\
\hline

R1F63 &
Interna \newline
UC63 
\\
\hline

R1F64 &
Capitolato \newline
UC64 
\\
\hline

R1F65 &
Capitolato \newline
UC65 
\\
\hline

R1F66 &
VE\_2020\_12\_18v.1.0.0 \newline
UC66 
\\
\hline

R1F67 &
Capitolato \newline
UC67 
\\
\hline

R1F68 &
Interna \newline
UC68 
\\
\hline

R1F69 &
Capitolato \newline
UC69 
\\
\hline

R1F70 &
Interna \newline
UC70 
\\
\hline

R1F70.1 &
VE\_2021\_01\_01v.1.0.0 
\\
\hline

R1F71 &
Capitolato \newline
UC71 
\\
\hline

R1F72 &
Capitolato \newline
UC72 
\\
\hline

R1F73 &
Capitolato 
\\
\hline

R1F74 &
Capitolato 
\\
\hline

R1Q1 &
Interna 
\\
\hline

R1Q2 &
Capitolato 
\\
\hline

R1Q3 &
Capitolato 
\\
\hline

R1Q4 &
Capitolato 
\\
\hline

R1Q5 &
Capitolato 
\\
\hline

R1Q6 &
Capitolato 
\\
\hline

R2Q7 &
Capitolato 
\\
\hline

R1Q8 &
Capitolato 
\\
\hline

R1Q9 &
Capitolato 
\\
\hline

R1Q10 &
Capitolato 
\\
\hline

R1Q11 &
Capitolato \newline
VE\_2020\_12\_18v.1.0.0 
\\
\hline

R1Q12 &
Capitolato 
\\
\hline

R1Q13 &
Capitolato 
\\
\hline

R1V1 &
Capitolato 
\\
\hline

R2V2 &
Capitolato 
\\
\hline

R1V3 &
Capitolato 
\\
\hline%fine tracciamento requisito-fonti%
		
	\end{longtable}
\end{center}

\subsection{Considerazioni}
Durante l'avanzamento del progetto i requisiti individuati potranno essere modificati e ampliati, e potranno esserne aggiunti di nuovi.