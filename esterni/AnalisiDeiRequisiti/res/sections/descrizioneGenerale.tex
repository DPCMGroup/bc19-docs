\section{Descrizione generale}

\subsection{Obiettivi del prodotto}
L'obiettivo finale del prodotto è di sviluppare un’applicazione in grado di segnalare ad un server dedicato la presenza di un utente su una determinata postazione appartenente ad una stanza e di gestire la pulizia delle varie postazioni.
La maggior parte della logica applicativa dovrà essere consegnata nella \glock{blockchain} \glock{Ethereum}.

\subsection{Funzioni del prodotto}
Le funzioni principali del prodotto devono essere sviluppate per due macro-tipologie di soggetti: amministratore e utente.
Deve essere possibile gestire più stanze per:
\begin{itemize}
	\item sapere in ogni momento se la postazione è occupata, prenotata oppure da pulire; \\
	\item controllare quali postazioni sono prenotate, da chi e bloccare le prenotazioni per una determinata stanza; \\
	\item prevedere una tracciatura autenticata e tutti i cambiamenti di stato relativi alla pulizia della postazione, nonché le informazioni su chi ha igienizzato la postazione, devono essere salvate su memoria immutabile e certificata; \\
	\item prenotare una postazione con granularità di 1 ora; \\
\end{itemize}
Dopo la registrazione dell’applicazione un amministratore del sistema può creare le utenze ai dipendenti e agli addetti delle pulizie. Definisce le postazioni e le stanze, inserisce i tag RFID e li associa alle rispettive postazioni.
Permetterà di monitorare le postazioni occupate, prenotate, da pulire e pulite. Inoltre sarà possibile esportare un report delle pulizie per ogni singola postazione o per stanza.
L’applicazione cellulare permette operazioni come:
\begin{itemize}
	\item recupero lista delle postazioni libere; \\
	\item prenotazione di una postazione; \\
	\item tracciamento in tempo reale tramite tag RFID; \\
	\item pulizia di una postazione; \\
	\item storico delle postazioni occupate; \\
	\item storico delle postazioni igienizzate; \\ 
\end{itemize}
Le comunicazioni tra applicazione e server avvengono nel momento in cui lo smartphone viene a contatto con il tag RFID. Grazie al tag viene registrata la presenza di una persona in una determinata postazione, che viene segnalata come occupata e quindi da pulire. Il dipendente può inoltre pulire in autonomia la postazione, tramite il kit di pulizia e segnalare questa attività sull’applicazione in una sezione dedicata.
L’addetto alle pulizie può segnalare la pulizia di una postazione per volta, oppure segnalare la pulizia dell’intera stanza igienizzando tutte le postazioni.

\subsection{Caratteristiche degli utenti}
Le categorie di utenti individuate nel capitolato sono tre:
\begin{itemize}
	\item amministratore di un azienda;
	\item dipendente di un'azienda o studente di un'università;
    \item addetto alle pulizie.
\end{itemize}
per tutte e tre le utenze non sono richieste conoscenze tecniche specifiche per l'utilizzo rispettivamente dell'applicazione web e di quella mobile.
\subsection{Piattaforma d'esecuzione}
Il backend avrà il compito di interagire con ethereum, in modo da creare una sorta di database certificato, dove registrare le varie azioni. Il frontend sarà costituito da una user interface sotto forma di applicazione web e app mobile. 
\subsection{Vincoli generali}
Ad un amministratore, per usufruire del servizio, è sufficiente un browser su di un computer con connessione internet; per gli utenti dell'applicazione un dispositivo con SO Android, una connessione a internet e un tag RFID (in particolare NFC) o in modo opzionale della tecnologia bluetooth low energy.

