\newpage \section{A}
\subsection{Android} Sistema operativo per dispositivi mobili.
\subsection{AWS} Amazon Web Services (AWS) è una piattaforma cloud che fornisce servizi da data center a livello globale, secondo la logica cloud computing.
\subsection{AWS Amplify} AWS Amplify è un insieme di prodotti e strumenti che consente agli sviluppatori di applicazioni mobili e web front-end di creare e distribuire applicazioni stack complete, sicure e scalabili, basate su AWS.
\subsection{AWS Appsync} AWS AppSync è un servizio completamente gestito che facilita lo sviluppo di API GraphQL gestendo le attività impegnative derivanti dalla connessione sicura a fonti di dati come AWS DynamoDB, Lambda e altro.
\subsection{AWS Gamelift} Amazon GameLift è un servizio che distribuisce, gestisce e dimensiona i server cloud per giochi multigiocatore.
\subsection{AWS Lambda} AWS Lambda è un servizio di calcolo che esegue codice in risposta ad eventi e automaticamente gestisce le risorse richieste dal codice di programmazione.
\subsection{Angular} Framework per lo sviluppo e il design di applicazioni single-page.
\subsection{Apache Kafka} Apache Kafka è una piattaforma che consente di:
\begin{itemize}
	\item scrivere e leggere flussi di eventi / dati in tempo reale;
	\item immagazzinare flussi di eventi / dati;
	\item analizzare flussi di eventi / dati in tempo reale o retrospettivamente.
\end{itemize}
\subsection{API} Acronimo per Application Programming Interface (API). È un'interfaccia con ruolo di intermediazione fra software diversi.
\subsection{Applicazione} Programma per calcolatore predisposto per eseguire una categoria di funzioni specifiche.
\subsection{Apprendimento automatico} L’apprendimento automatico (noto anche come machine learning) è una branca dell'intelligenza artificiale che raccoglie un insieme di metodi statistici per migliorare progressivamente la performance di un algoritmo nell'identificare pattern nei dati.
\subsection{Architettura software} L’architettura software ci dice come un sistema è organizzato, ossia quali sono i ruoli delle componenti del sistema e quali interazioni esistono fra di esse. Essa definisce, inoltre, le interfacce necessarie all’interazione tra componenti o con l’ambiente. Il tutto non viene fatto in modo estemporaneo, ma seguendo dei paradigmi di composizione precisi.
\subsection{Asincrono} Dispositivo che opera senza un riferimento temporale di sincronizzazione rispetto a un altro dispositivo. \\
Protocollo asincrono: modalità di trasmissione dati che non dipende dal compiersi di altri processi.

\newpage \section{B}
\subsection{Brainstormig} Attività di gruppo per risolvere problemi. Essa è paritaria (non vi sono gerarchie nelle opinioni: ogni opinione può essere valida), disciplinata (tutti devono parlare e uno per volta), focalizzata (su un problema e non troppo lunga) e che lascia traccia di sè (la discussione deve portare a una decisione che deve essere registrata).
\subsection{Back-end} Back-end è un termine largamente utilizzato per caratterizzare le interfacce che hanno come destinatario un programma. Una applicazione back-end è un programma con il quale l'utente interagisce indirettamente, in generale attraverso l'utilizzo di una applicazione front-end.
\subsection{Blockchain} Una blockchain è una struttura dati condivisa e immutabile. È definita come un registro
digitale le cui voci sono raggruppate in blocchi, concatenati in ordine cronologico, e la cui integrità è garantita dall'uso della crittografia.
\subsection{Bluetooth} Tecnologia per la comunicazione senza fili a distanza di poche decine di metri.
\newpage \section{C}
\subsection{Capitolato} Documento in cui il proponente del capitolato fissa le sue aspettative o illustra i desiderata dell'utilizzatore.
\subsection{Client} Programma o calcolatore che richiede di scambiare dati con un server.
\subsection{Cloud} Con il termine inglese cloud (computing) si indica un paradigma di erogazione di servizi offerti on demand da un fornitore ad un cliente finale attraverso la rete Internet.
\subsection{Cloud computing} vedi \glock{Cloud}.
\subsection{Covid} È una malattia infettiva respiratoria causata dal virus denominato SARS-CoV-2 appartenente alla famiglia dei coronavirus.
\subsection{CSS} CSS (Cascading Style Sheets), è un linguaggio usato per definire la formattazione di documenti HTML, XHTML e XML, ad esempio i siti web e le relative pagine.
\newpage \section{D}
\subsection{dApp} dApp sta per Applicazioni distribuite, ossia applicazioni che operano su un sistema informatico distribuito.
\subsection{D3.js} D3.js è una libreria javascript per la visualizzazione dei dati con HTML, SVG e CSS.
\subsection{Diagrammi di Gantt} Diagrammi per la rappresentazione di attività organizzate cronologicamente.
\subsection{Discord} Discord è un’applicazione VoIP progettata per le comunità di videogiocatori. In questa applicazione è possibile creare dei canali testuali e vocali suddividendoli per tematica.
\subsection{Docker} Il software Docker è una tecnologia di containerizzazione che consente la creazione e l’utilizzo dei container Linux.
\subsection{Dropbox} Servizio di file hosting gestito dalla società Dropbox Inc., che offre cloud storage, sincronizzazione automatica dei file, cloud personale e software client.
\newpage \section{E}
\subsection{E-commerce} Commercio elettronico che consiste nell'acquisto e nella vendita di beni o servizi tramite transazioni di denaro e dati attraverso Internet.
\subsection{Ethereum} Blockchain per la creazione e pubblicazione di \glock{smart contract}.
\newpage \section{F}
\subsection{Framework} È una infrastruttura intesa come "struttura o complesso di elementi che costituiscono la base di sostegno o comunque la parte sottostante di altre strutture". Si intende la piattaforma che funge da strato intermedio tra un sistema operativo e il software che lo utilizza.
\subsection{Front-end} Con il termine front end si indica generalmente l’interfaccia utente di una applicazione o sito web.
\newpage \section{G}
\subsection{Gulpease} Valore tra 0 e 100 che indica la leggibilità di un documento, dove il valore 100 indica la leggibilità più alta e 0 la leggibilità più bassa. 
\subsection{GitHub} GitHub è un servizio di hosting per progetti software.
\subsection{GitHub Actions} GitHub Actions è uno strumento fornito da GitHub che permette l’automazione di compiti di varia natura.
\subsection{Google Drive} Google Drive è un servizio web di memorizzazione e sincronizzazione.
\subsection{GraphQL} GraphQL è un linguaggio di query per API e un runtime lato server per interagire con il database. Non è vincolato ad alcun tipo di database e può essere utilizzato nel codice esistente.
\subsection{Gruppo} Gruppi di progetto costituiti il 22 ottobre 2020.
\subsection{GUI} Acronimo per il termine inglese Graphical User Interface. È l'interfaccia grafica.
\newpage \section{H}
\subsection{HTML} È un linguaggio di markup per la strutturazione delle pagine web.
\newpage \section{I}
\subsection{IaaS} Acronimo di Infrastructure as a Service, infrastruttura come servizio. È un modello di \glock{cloud computing} nel quale il venditore fornisce ai clienti esclusivamente le risorse di calcolo, memorizzazione e connessione di rete. Su di esse i clienti potranno basare gli applicativi di loro interesse.
\subsection{Imola Informatica} Azienda proponente del capitolato.
\subsection{Indice di Gulpease} Vedi \textit{Gulpease}.
\subsection{Inspection} Tecnica di analisi statica che consiste nell’analizzare il prodotto solo nelle parti in cui si prevede che ci possano essere dei difetti.
\subsection{Interfaccia} Servizi offerti da una entità a un'altra entità.
\subsection{IOS} Sistema operativo dei dispositivi mobili di Apple Inc..
\newpage \section{J}
\subsection{Javascript} JavaScript è un linguaggio di scripting orientato agli oggetti e agli eventi, comunemente utilizzato nella programmazione Web.
\subsection{Java} Java è un linguaggio di programmazione ad alto livello, orientato agli oggetti e a tipizzazione statica. Si appoggia sulla Java Virtual Machine per l'esecuzione e questo lo rende indipendente dalla piattaforma hardware sottostante.
\newpage \section{K}
\subsection{Keras} È un API per il deep learning scritta in Python ed eseguita sulla piattaforma TensorFlow.
\subsection{Kotlin} Kotlin è un linguaggio di programmazione versatile, multi-paradigma e open-source creato da JetBrains.
\subsection{Kubernetes} Kubernetes è una piattaforma portatile, estensibile e open-source per la gestione di carichi di lavoro e servizi containerizzati, in grado di facilitare sia la configurazione dichiarativa che l'automazione.
\newpage \section{L}
\subsection{Leaflet} Leaflet è una libreria javascipt per la visualizzazione delle mappe su dispositivi mobili.
\subsection{Linux} È una famiglia di sistemi operativi open-source di tipo Unix-like.
\subsection{\LaTeX} Linguaggio di markup per la preparazione di testi, basato sul programma di composizione tipografica TEX.
\newpage \section{M}
\subsection{MetaMask} MetaMask è un'estensione browser che permette di eseguire dApps, senza far parte di Ethereum, ossia essendo un nodo di Ethereum. MetaMask permette la connessione con un altro nodo Ethereum, chiamato INFURA ed esegue smart contract (cioè un programma che viene messo in esecuzione sui nodi validatori di una blockchain) su quel nodo. Metamask gestice il portafoglio Ethereum, che raccoglie gli Ethers (o il denaro) e permette di inviare e ricevere Ethers tramite una dApp di interesse.
\subsection{MacOS} Sistema operativo Unix-like dei personal computer di Apple Inc..
\subsection{Modello a V} Modello di sviluppo software in cui ad ogni fase di sviluppo corrisponde una fase di testing per controllare che la fase di sviluppo corrispondente sia corretta.
\subsection{Milestone} È un termine inglese che letteralmente significa pietra miliare. In ingegneria del software indica importanti traguardi intermedi nello sviluppo del progetto.
\subsection{MQTT} Protocollo client-server di tipo publish/subscribe per il trasporto di messaggi. È utilizzato nella comunicazione Machine to Machine (M2M) e nell'Internet of Things (IoT).
\subsection{MVC} Acronimo di Model-View-Controller (MVC, talvolta tradotto in italiano con la dicitura modello-vista-controllo). È un “pattern architetturale”  soprattutto utilizzato per la programmazione orientata agli oggetti, in grado di separare la rappresentazione interna dei dati da ciò che viene presentato e accettato dall'utente.
\newpage \section{N}
\subsection{Nectcloud} Piattaforma collaborativa, open-source e on-premises.
\subsection{NFC} Acronimo di Near Field Communication (Comunicazione a Corto Raggio). Indica una tecnologia di trasmissione di dati senza fili a distanze tipicamente inferiori ai 10 cm.
\subsection{Node.js} Ambiente open-source per l'esecuzione di codice Javascript a run-time, al di fuori dei browser. Ciò permette l'esecuzione di codice Javascript server-side.
\subsection{NoSQL} Questi tipi di database sono ottimizzati specificatamente per applicazioni che necessitano di grandi volumi di dati, latenza bassa e modelli di dati flessibili, ottenuti snellendo alcuni dei criteri di coerenza dei dati degli altri database.
\subsection{NumPy} NumPy è una libreria Python per il calcolo scientifico. Fornisce array multidimensionali, matrici, ndarray, operazioni sugli array di carattere logico, matematico, di ordinamento, operazioni algebriche, statistiche, simulazioni aleatorie e altro.
\newpage \section{O}
\subsection{Open Exchange} Suite di prodotti web per la comunicazione, la collaborazione e la produttività.
\subsection{OpenShift} Red Hat OpenShift è una piattaforma container per le imprese basata su Kubernetes, che offre operazioni automatizzate in tutto lo stack per gestire deployment di cloud ibridi e multicloud. Red Hat OpenShift è ottimizzato per incrementare la produttività degli sviluppatori e promuovere l'innovazione. 
\subsection{Organizzazione} Soggetto obbligato a tracciare le presenze delle persone nelle postazioni di lavoro, in maniera autenticata ed in tempo reale tramite tag RFID. 
\newpage \section{P}
\subsection{PaaS} Platform as a service (PaaS) è un'attività economica che consiste nel servizio di messa a disposizione di piattaforme di elaborazione (Computing platform) e di solution stack. Gli elementi del PaaS permettono di sviluppare, sottoporre a test, implementare e gestire le applicazioni aziendali senza i costi e la complessità associati all'acquisto, alla configurazione, all'ottimizzazione e alla gestione dell'hardware e del software di base.
\subsection{PDF} Acronimo di Portable Document Format (Formato per Documenti Portatili), è un formato per la codifica dei documenti.
\subsection{PdfLatex} Compilatore di documenti \LaTeX.
\subsection{Pandas} Libreria di Python per la manipolazione e l'analisi dei dati. In particolare, offre strutture dati e operazioni per manipolare dati tabellari.
\subsection{Piattaforma} Ambiente nel quale un software è eseguito.
\subsection{Postazione} Spazio fisico identificato da un tag RFID univoco dove l’utilizzatore appoggia il cellulare mentre sta svolgendo il suo lavoro. Ciascuna postazione di lavoro è inserita in una stanza
dell'organizzazione (laboratorio, ufficio, biblioteca, etc\dots).
\subsection{Power-up} Nel campo dei videogiochi, è un oggetto mostrato a video che conferisce una particolare abilità temporanea al giocatore o ne incrementa le statistiche quando raccolto.
\subsection{Processo} Insieme di attività correlate e coese che in base a dei bisogni offrono dei prodotti seguendo un piano che assicuri quantificabilità, efficacia ed efficienza. Questo avviene attraverso misurazioni retrospettive al termine delle attività ma anche durante le attività aiutando a dare indicazioni correttive, quando si è ancora in tempo utile per correggere.
\subsection{Product baseline} Documento che illustra l'architettura essenziale del prodotto in via di sviluppo. \\
Riporta i design pattern adottati contestualizzati rispetto al prodotto e i diagrammi delle classi e di sequenza.
\subsection{Proof of Concept} Prototipo software che presenta le caratteristiche essenziali del prodotto che si vuole sviluppare. Serve a dimostrare la fattibilità del prodotto, la fondatezza dei principi alla sua base e la conformità dello sviluppo con le attese del proponente.
\subsection{Proponente} Ente o azienda che propone il capitolato d’appalto per un progetto.
\subsection{Protocollo} Insieme di regole e convenzioni necessarie a stabilire un dialogo fra due o più parti.
\subsection{Publish/Subscribe} Protocollo di messaggistica in cui i mittenti di messaggi, chiamati publisher, non programmano di inviare il messaggio direttamente ai destinatari, chiamati subscriber, ma raccolgono i messaggi in classi senza sapere a chi siano destinati, eventualmente anche a nessuno. Similmente i subscriber possono esprimere interesse per una o più classi di messaggi senza sapere l'autore del messaggio.
\subsection{Pull} Comando git per l'aggiornamento del repository locale dal repository remoto.
\subsection{Pull request} Metodo per contribuire a un progetto, tipicamente open-source. Una pull request si verifica quando uno sviluppatore chiede che le modifiche apportate a un repository vengano prese in considerazione. Segue un processo di revisione, tramite discussioni, richieste di ulteriori modifiche e relative correzioni. Quando le modifiche apportate raggiungono uno stato accettabile si può procedere alla loro integrazione nel repository.
\subsection{Push} Comando git per l'aggiornamento del repository remoto dal repository locale.
\subsection{Python} Python è un linguaggio di programmazione di più "alto livello" rispetto alla maggior parte degli altri linguaggi, orientato a oggetti, adatto, tra gli altri usi, a sviluppare applicazioni distribuite, scripting, computazione numerica e system testing.
\subsection{PyTorch} Libreria open-source ideata per il machine learning.
\newpage \section{Q}
\subsection{QR Code} Metodo di codifica dei dati che produce un codice a barre bidimensionale facilmente leggibile da un dispositivo compatibile. La maggior parte degli smartphone oggi in commercio permette la scansione di questi codici.
\subsection{Qt} Qt è una libreria multipiattaforma per lo sviluppo di programmi con interfaccia grafica tramite l'uso di widget (congegni o elementi grafici).
\newpage \section{R}
\subsection{Rancher} Rancher è una piattaforma di orchestrazione multi-cluster open-source, consente ai team operativi di implementare e gestire Kubernetes.
\subsection{Real-time} Con il termine real-time o tempo reale viene inteso che i dati memorizzati hanno storicità recente, se non istantanea.
\subsection{Repository} Spazio di archiviazione remoto. È utilizzato per la condivisione di software e documenti.
\subsection{RFID} Radio-frequency identification, in telecomunicazioni ed elettronica, si intende una tecnologia per l'identificazione e/o memorizzazione automatica di informazioni inerenti a oggetti, animali o persone basata sulla capacità di memorizzazione di dati da parte di particolari etichette elettroniche, chiamate tag, e sulla capacità di queste di rispondere all'interrogazione a distanza da parte di appositi apparati fissi o portatili, chiamati reader.
\newpage \section{S}
\subsection{Sanificazione} Viene inteso l'atto di pulizia delle postazioni o stanze.
\subsection{Scalabilità} La scalabilità denota in genere la capacità di un sistema di aumentare o diminuire di scala in funzione delle necessità e disponibilità. Un sistema che gode di questa proprietà viene detto scalabile.
\subsection{SceneKit} Interfaccia per la programmazione di applicazioni in grafica 3D per dispositivi Apple.
\subsection{Scikit-learn} Scikit-learn è una libreria open-source di apprendimento automatico per Python.
\subsection{Server} Calcolatore che svolge funzioni di servizio per tutti i calcolatori collegati oppure programma, generalmente sempre attivo, che esegue determinate funzioni quando queste sono richieste da altri programmi.
\subsection{Serverless} Vengono intese tutte quelle applicazioni che non si appoggiano a server specifici, ma bensì a servizi esterni quali AWS.
\subsection{Software} Con software viene identificato la componente applicativa di un pc o cellulare.
\subsection{SpriteKit} Libreria per iOS per la creazione di giochi 2D.
\subsection{Swift} È un linguaggio di programmazione object-oriented per sistemi macOS, iOS, watchOS, tvOS e Linux.
\subsection{SwiftUI} Libreria grafica di Swift.
\subsection{SQL} È un linguaggio standardizzato per database basati sul modello relazionale.
\subsection{Smart contract} Protocolli per facilitare, attuare e verificare la negoziazione di un contratto in versione digitale.
Permettono di ottenere lo stesso valore di un contratto reale senza l'ausilio di un garante esterno. Le transazioni che avvengono con questo protocollo sono tracciabili e irreversibili. Uno smart contract rappresenta del codice che può essere eseguito.
\newpage \section{T}
\subsection{Tag NFC} Sono dei transponder RFID, ovvero dei minuscoli chip collegati a un'antenna. Il chip ha un codice univoco e una parte di memoria riscrivibile. L'antenna permette al chip di interagire con un lettore NFC, come uno smartphone NFC.
\subsection{Technology baseline} Documento che indica e motiva le tecnologie, le librerie e
i framework che verranno utilizzati durante lo sviluppo del prodotto, e dimostra la fattibilità e l'adeguatezza di quest'ultimo tramite un Proof of concept.
\subsection{TensofFlow} Software end-to-end open-source per il machine learning.
\subsection{Test} Esperimento variamente espletato allo scopo di saggiare, mediante determinate reazioni, l'entità o la consistenza di un'attitudine o di una capacità individuale.
\subsection{Telegram} Servizio e applicazione di messaggistica istantanea.
\subsection{Theano} Libreria per Python e compilatore con ottimizzatore per manipolare e valutare espressioni matematiche, specialmente matrici.
\subsection{Tomcat} Apache Tomcat software è una implementazione open-source di Java Servlet, JavaServer Pages, Java Expression Language e Java WebSocket technologies.
\subsection{Tracciatura} Rilevamento in tempo reale della presenza ad una specifica
postazione. Il dato deve essere salvato in modo immutabile e automaticamente certificato dal
sistema informatico per poter essere opponibile a terzi (nel caso di controversie legali per l’azienda).
\subsection{Typescript} TypeScript è un linguaggio di programmazione open-source sviluppato da Microsoft.
\newpage \section{U}
\subsection{Uber} Uber è un'azienda con sede a San Francisco che fornisce un servizio di trasporto automobilistico privato attraverso un'applicazione mobile che mette in collegamento diretto passeggeri e autisti.
\newpage \section{V}
\subsection{Versionamento} È la gestione di versioni multiple di un insieme di informazioni: gli strumenti software per il controllo versione sono ritenuti molto spesso necessari per la maggior parte dei progetti di sviluppo software o documentali gestiti da un team collaborativo di sviluppo o redazione.
\newpage \section{W}
\subsection{WebSocket} Protocollo che garantisce canali di comunicazione full-duplex su una singola connessione TCP.
\subsection{Windows} Sistema operativo creato da Microsoft.
\subsection{Workflow} È l’automazione totale o parziale di un processo aziendale, in cui documenti, informazioni o compiti passano da un partecipante a un altro per svolgere attività, secondo un insieme di regole definite.
\subsection{Walkthrough} Tecnica di analisi statica che consiste nell’analizzare il prodotto per intero in ogni sua parte.
\newpage \section{Z}
\subsection{Zextras Drive} È una piattaforma per la condivisione file simile a Google Drive e Dropbox.
\subsection{Zimbra} Open-source server e client software per lo scambio di email.
