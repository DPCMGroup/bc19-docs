\section{Web app}

\subsection{Introduzione e scopo del prodotto}
L'applicazione web è sviluppata per il solo utente amministratore.

In generale offre all'utente le seguenti funzionalità:
\begin{itemize}
	\item \textbf{Login:} L'utente ha la possibilità di autenticarsi inserendo il proprio username e password; \\
	\item \textbf{Logout:} L'utente ha la possibilità di deautenticarsi premendo sul bottone "Logout" del menù principale in alto a destra. \\
\end{itemize}
Per quanto riguarda le stanze, permette all'amministratore di:
\begin{itemize}
	\item \textbf{Aggiungere una stanza}: l'amministratore ha la possibilità di aggiungere una nuova stanza premendo sul bottone "Add Room"; \\
	\item \textbf{Ricercare una determinata postazione}: l'amministratore ha la possibilità di ricerca una postazione inserendo determinati parametri premendo sul bottone "Search Workstation"; \\
	\item \textbf{Modificare una stanza}: l'amministratore ha la possibilità di modificare le caratteristiche di una stanza premendo sul bottone "Edit Room"; \\
	\item \textbf{Rimuovere una stanza}: l'amministratore ha la possibilità di rimuovere una stanza premendo sul bottone "Remove Room"; \\
	\item \textbf{Aggiungere una postazione all'interno di una stanza}: l'amministratore ha la possibilità di aggiungere una postazione all'interno di una stanza premendo sul bottone "Add Workstation"; \\
	\item \textbf{Rimuovere una postazione all'interno di una stanza}: l'amministratore ha la possibilità di eliminare una postazione all'interno di una stanza premendo sul bottone dove è raffigurato un cestino; \\
	\item \textbf{Modificare una postazione all'interno di una stanza}:  l'amministratore ha la possibilità di modificare le caratteristiche di una postazione all'interno di una stanza premendo sul bottone dove è raffigurata una "E". \\
\end{itemize}
Per quanto riguarda le credenziali, permette all'amministratore di:
\begin{itemize}
	\item \textbf{Aggiungere una credenziale}: l'amministratore ha la possibilità di aggiungere una credenziale premendo sul bottone "Add Credential"; \\
	\item \textbf{Eliminare una credenziale}: l'amministratore ha la possibilità di eliminare una credenziale premendo sul bottone dove è raffigurato un cestino; \\
	\item \textbf{Modificare una credenziale}:  l'amministratore ha la possibilità di modificare le caratteristiche di una credenziale premendo sul bottone dove è raffigurata una "E". \\
\end{itemize}

\subsection{Requisiti e installazione}
La web app viene eseguita sul browser, non necessita di installazione e ha come unico requisito una connessione ad internet.

\subsection{Utilizzo}
\subsubsection{Login}
\begin{figure}[H]
	\centering
	\includegraphics[width=15cm]{res/images/login.jpg}
	\caption{Login}
\end{figure}
L'amministratore può autenticarsi inserendo il proprio username e la password. L’username è da inserire nel primo campo di testo (1), la password nel secondo (2). Dopo l’inserimento dei dati può cliccare sul pulsante 'Sign in' per effettuare l'autenticazione.

\subsubsection{Visualizzazione errore di autenticazione}
\begin{figure}[H]
	\centering
	\includegraphics[width=15cm]{res/images/error.png}
	\caption{Visualizzazione errore}
\end{figure}
Se le credenziali inserite risultano errate, l’applicazione mostra un errore avvisando la persona che o non è un amministratore o ancora non è registrata.

\subsubsection{Logout}
\begin{figure}[H]
	\centering
	\includegraphics[width=15cm]{res/images/logout.jpg}
	\caption{Logout}
\end{figure}
L'amministratore, una volta effettuata l'autenticazione, può eseguire il logout da qualsiasi pagina della web-app. Sarà sufficiente cliccare sul pulsante in alto a destra denominato 'Logout'.

\subsubsection{Visualizzazione di una stanza}
\begin{figure}[H]
	\centering
	\includegraphics[width=15cm]{res/images/visStanza.png}
	\caption{Visualizzazione stanze}
\end{figure}
Per poter visualizzare le stanze, l'amministratore deve premere sul bottone 'Rooms'.
Dopo aver premuto sul bottone, per ogni stanza l'amministratore visualizza:
\begin{enumerate}
	\item identificativo della stanza;
	\item nome della stanza;
	\item dimensioni della stanza.
\end{enumerate}
Per ogni postazione all'interno di una stanza l'amministratore visualizza:
\begin{enumerate}
	\item identificativo della postazione;
	\item identificativo del tag;
	\item il nome della postazione;
	\item l'ascissa della postazione nella stanza;
	\item l'ordinata della postazione nella stanza;
	\item lo stato della postazione;
	\item lo stato dell' igienizzazione;
	\item opzione per eliminare la postazione;
	\item opzione per modificare la postazione.
	
\end{enumerate}

\subsubsection{Aggiunta di una stanza}
Per poter aggiungere una stanza, l'amministratore deve premere sul bottone 'Add Room'.
\begin{figure}[H]
	\centering
	\includegraphics[width=15cm]{res/images/bottoneAddRoom.png}
	\caption{Visualizzazione bottoni}
\end{figure}
Una volta premuto sul bottone, comparirà un popup che l'amministratore dovrà compilare inserendo:
\begin{enumerate}
	\item il nome della stanza;
	\item la dimensione X;
	\item la dimensione Y.
\end{enumerate}
\begin{figure}[H]
	\centering
	\includegraphics[width=15cm]{res/images/aggiungiStanza1.png}
	\caption{Visualizzazione popup aggiunta stanza}
\end{figure}

\subsubsection{Ricerca di una postazione}
Per poter ricercare una determinata postazione, l'amministratore deve premere sul bottone 'Search Workstation'.
\begin{figure}[H]
	\centering
	\includegraphics[width=15cm]{res/images/bottoneSearchWorkstation.png}
	\caption{Visualizzazione bottoni}
\end{figure}
Una volta premuto sul bottone, comparirà un popup che l'amministratore dovrà compilare inserendo:
\begin{enumerate}
	\item l'identificativo della postazione;
	Oppure
	\item l'username di chi la occupa.
\end{enumerate}
\begin{figure}[H]
	\centering
	\includegraphics[width=15cm]{res/images/ricercaPostazione.png}
	\caption{Visualizzazione popup ricerca postazione}
\end{figure}

\subsubsection{Modifica di una stanza}
Per poter modificare le caratteristiche di una stanza, l'amministratore deve premere sul bottone 'Edit Room'.
\begin{figure}[H]
	\centering
	\includegraphics[width=15cm]{res/images/bottoneEditRoom.png}
	\caption{Visualizzazione bottoni}
\end{figure}
Una volta premuto sul bottone, comparirà un popup che l'amministratore dovrà compilare inserendo:
\begin{enumerate}
	\item il nome della stanza;
	\item la dimensione X;
	\item la dimensione Y.
\end{enumerate}
\begin{figure}[H]
	\centering
	\includegraphics[width=15cm]{res/images/modificaStanza.png}
	\caption{Visualizzazione popup modifica stanza}
\end{figure}

\subsubsection{Rimozione di una stanza}
Per poter rimuovere una stanza, l'amministratore deve premere sul bottone 'Remove Room'.
\begin{figure}[H]
	\centering
	\includegraphics[width=15cm]{res/images/bottoneRemoveRoom.png}
	\caption{Visualizzazione bottoni}
\end{figure}
Una volta premuto sul bottone, comparirà un popup che chiederà all' amministratore una conferma per eliminare definitivamente la stanza.

\subsubsection{Aggiunta di una postazione all'interno di una stanza}
Per poter inserire una nuova postazione all'interno di una stanza, l'amministratore deve premere sul bottone 'Add Workstation'.
\begin{figure}[H]
	\centering
	\includegraphics[width=15cm]{res/images/bottoneAddWorkstation.png}
	\caption{Visualizzazione bottoni}
\end{figure}
Una volta premuto sul bottone, comparirà un popup che l'amministratore dovrà compilare inserendo:
\begin{enumerate}
	\item il tag della postazione;
	\item il nome della postazione;
	\item la dimensione X della postazione nella stanza;
	\item la dimensione Y della postazione nella stanza.
\end{enumerate}
\begin{figure}[H]
	\centering
	\includegraphics[width=15cm]{res/images/addWorkstation.png}
	\caption{Visualizzazione popup inserimento postazione dentro a una stanza}
\end{figure}

\subsubsection{Eliminazione di una postazione all'interno di una stanza}
Per poter eliminare una postazione all'interno di una stanza, l'amministratore deve premere sul bottone dove è raffigurato un cestino.
\begin{figure}[H]
	\centering
	\includegraphics[width=15cm]{res/images/bottoneCestinoWorkstation.png}
	\caption{Visualizzazione bottoni}
\end{figure}
Una volta premuto sul bottone, comparirà un popup che chiederà all' amministratore una conferma per eliminare definitivamente la postazione dentro alla stanza.

\subsubsection{Modifica di una postazione all'interno di una stanza}
Per poter modificare le caratteristiche di una postazione all'interno di una stanza, l'amministratore deve premere sul bottone 'E' di colore blu.
\begin{figure}[H]
	\centering
	\includegraphics[width=15cm]{res/images/bottoneEditWorkstation.png}
	\caption{Visualizzazione bottoni}
\end{figure}
Una volta premuto sul bottone, comparirà un popup che l'amministratore dovrà compilare inserendo:
\begin{enumerate}
	\item il tag della postazione;
	\item il nome della postazione;
	\item la dimensione X della postazione nella stanza;
	\item la dimensione Y della postazione nella stanza.
\end{enumerate}
\begin{figure}[H]
	\centering
	\includegraphics[width=15cm]{res/images/editWorkstation.png}
	\caption{Visualizzazione popup modifica postazione all'interno di una stanza}
\end{figure}

\subsubsection{Visualizzazione delle credenziali}
\begin{figure}[H]
	\centering
	\includegraphics[width=15cm]{res/images/credential.jpg}
	\caption{Visualizzazione credenziali}
\end{figure}
L’amministratore può visualizzare la lista delle credenziali di ogni utente. Questo sarà permesso da ogni pagina cliccando sul pulsante Credentials.
La lista di credenziali è composta dai seguenti campi:
\begin{enumerate}
	\item Id;
	\item Username;
	\item Name;
	\item Surname;
	\item Email;
	\item Type;
	\item Option.
\end{enumerate}
Type indica il tipo di utente, Option presenta due pulsanti per eliminare o modificare le credenziali.

\subsubsection{Aggiunta di nuove credenziali}
Per poter aggiungere nuove credenziali, l'amministratore deve premere sul bottone 'Add Credential' di colore blu.
\begin{figure}[H]
	\centering
	\includegraphics[width=15cm]{res/images/addCredential.jpg}
	\caption{Bottone di aggiunta credenziali}
\end{figure}
Una volta premuto sul bottone, comparirà un popup che l'amministratore dovrà compilare inserendo:
\begin{enumerate}
	\item il nome dell'utente (1);
	\item il cognome dell'utente (2);
	\item l'username(3);
	\item l'email (4);
	\item il tipo di utente (5);
	\item la password (6).
\end{enumerate}
\begin{figure}[H]
	\centering
	\includegraphics[width=15cm]{res/images/addc.jpg}
	\caption{Visualizzazione popup di inserimento credenziali}
\end{figure}


\subsubsection{Eliminazione delle credenziali}
Per poter eliminare le credenziali di un utente, l'amministratore deve premere sul bottone che raffigura il cestino.
\begin{figure}[H]
	\centering
	\includegraphics[width=15cm]{res/images/optionDelete.jpg}
	\caption{Bottone di cancellazione credenziali}
\end{figure}

\subsubsection{Modifica delle credenziali}
Per poter modificare le credenziali di un utente, l'amministratore deve premere sul bottone 'E' di colore blu.
\begin{figure}[H]
	\centering
	\includegraphics[width=15cm]{res/images/optionEdit.jpg}
	\caption{Bottone di modifica credenziali}
\end{figure}
Una volta premuto sul bottone, comparirà un popup che l'amministratore dovrà compilare inserendo:
\begin{enumerate}
	\item il nome dell'utente (1);
	\item il cognome dell'utente (2);
	\item l'username(3);
	\item l'email (4);
	\item il tipo di utente (5);
	\item la password (6).
\end{enumerate}
\begin{figure}[H]
	\centering
	\includegraphics[width=15cm]{res/images/editc.jpg}
	\caption{Visualizzazione popup modifica credenziali}
\end{figure}
