\section{Glossario}

 %\addtocontents{toc}{\setcounter{tocdepth}{5}}

\subsection*{A}
\subsubsection*{Android} Sistema operativo per dispositivi mobili.

\subsection*{N}
\subsubsection*{NFC} Acronimo di Near Field Communication (Comunicazione a Corto Raggio). Indica una tecnologia di trasmissione di dati senza fili a distanze tipicamente inferiori ai 10 cm.

\subsection*{O}
\subsubsection*{Organizzazione} Soggetto obbligato a tracciare le presenze delle persone nelle postazioni di lavoro, in maniera autenticata ed in tempo reale tramite tag RFID. 
\subsection*{P}
\subsubsection*{Postazione} Spazio fisico identificato da un tag RFID univoco dove l’utilizzatore appoggia il cellulare mentre sta svolgendo il suo lavoro. Ciascuna postazione di lavoro è inserita in una stanza
dell'organizzazione (laboratorio, ufficio, biblioteca, etc\dots).
\subsection*{T}
\subsubsection*{Tag NFC} Sono dei transponder RFID, ovvero dei minuscoli chip collegati a un'antenna. Il chip ha un codice univoco e una parte di memoria riscrivibile. L'antenna permette al chip di interagire con un lettore NFC, come uno smartphone NFC.
\subsubsection*{Tempo reale} Con il termine real-time o tempo reale viene inteso che i dati memorizzati hanno storicità recente, se non istantanea.

%\addtocontents{toc}{\setcounter{tocdepth}{4}}
