\section{Introduzione}

\subsection{Scopo del documento}
Lo scopo di questo documento è illustrare tutte le funzionalità dell'applicazione mobile e della web app sviluppate per il progetto BlockCOVID. L'utente finale in questo modo avrà a disposizione tutte le indicazioni per il corretto uso del software.
\subsection{Scopo del prodotto}
\subsubsection{App mobile}
L'applicazione mobile è sviluppata per due diverse tipologie di utente ovvero il dipendente e l'addetto alle pulizie.

In generale offre all'utente le seguenti funzionalità:
\begin{itemize}
	\item \textbf{Login:} L'utente ha la possibilità di autenticarsi inserendo il proprio username e password; \\
	\item \textbf{Logout:} L'utente ha la possibilità di deautenticarsi premendo sull'elemento della lista "Logout" del menù principale in alto a destra. \\
\end{itemize}

Per quanto riguarda il dipendente, l'applicazione offre i seguenti servizi:
\begin{itemize}
	\item \textbf{Scansione:} Il dipendente ha la possibilità di scansionare una postazione per poter visualizzare lo stato di essa e altre informazioni come le prenotazioni associate ad essa; \\
	\item \textbf{Occupazione:} Il dipendente dopo aver scansionato una postazione può occuparla se questa è prenotata da lui o è libera e igienizzata; \\
	\item \textbf{Igienizzare:} Il dipendente dopo aver scansionato una postazione la può igienizzare se questa risulta non igienizzata; \\
	\item \textbf{Lista prenotazioni:} Il dipendente può visualizzare le prenotazioni effettuate premendo sull'elemento della lista "Visualizza prenotazioni" del menù principale in alto a destra; \\
	\item \textbf{Disdire prenotazione:} Il dipendente può disdire una prenotazione dopo che è entrato nella pagina in cui visualizza tutte le prenotazioni effettuate; \\
	\item \textbf{Guida:} Il dipendente può visualizzare la guida premendo sull'elemento della lista "Guida" del menù principale in alto a destra; \\
	\item \textbf{Prenota postazione:} Il dipendente può prenotare una postazione premendo sull'elemento della lista "Prenota postazione" del menù principale in alto a destra.
	Dopo aver premuto dovrà inserire la data, l'ora di inizio, l'ora di fine e la stanza obbligatoriamente e in modo facoltativo anche il nome del collega.
	Una volta premuto sul bottone "Cerca", se è stato inserito il nome del collega, visualizzerà tutte le sue prenotazioni effettuate nella stanza, nel range orario e nella data inseriti e scorrendo sotto visualizzerà tutte le postazioni di quella stanza con il loro stato e potrà decidere quale prenotare se disponibile. \\	
\end{itemize}

\subsubsection{Web app}
L'applicazione web è sviluppata per il solo utente amministratore.

In generale offre all'utente le seguenti funzionalità:
\begin{itemize}
	\item \textbf{Login:} L'utente ha la possibilità di autenticarsi inserendo il proprio username e password; \\
	\item \textbf{Logout:} L'utente ha la possibilità di autenticarsi premendo sul bottone "Logout" del menù principale in alto a destra. \\
\end{itemize}
Per quanto riguarda le stanze, permette all'amministratore di:
\begin{itemize}
\item \textbf{Aggiungere una stanza}: l'amministratore ha la possibilità di aggiungere una nuova stanza premendo sul bottone "Add Room"; \\
\item \textbf{Ricercare una determinata postazione}: l'amministratore ha la possibilità di ricerca una postazione inserendo determinati parametri premendo sul bottone "Search Workstation"; \\
\item \textbf{Modificare una stanza}: l'amministratore ha la possibilità di modificare le caratteristiche di una stanza premendo sul bottone "Edit Room"; \\
\item \textbf{Rimuovere una stanza}: l'amministratore ha la possibilità di rimuovere una stanza premendo sul bottone "Remove Room"; \\
\item \textbf{Aggiungere una postazione all'interno di una stanza}: l'amministratore ha la possibilità di aggiungere una postazione all'interno di una stanza premendo sul bottone "Add Workstation"; \\
\item \textbf{Rimuovere una postazione all'interno di una stanza}: l'amministratore ha la possibilità di eliminare una postazione all'interno di una stanza premendo sul bottone dove è raffigurato un cestino; \\
\item \textbf{Modificare una postazione all'interno di una stanza}:  l'amministratore ha la possibilità di modificare le caratteristiche di una postazione all'interno di una stanza premendo sul bottone dove è raffigurata una "E". \\
\end{itemize}
Per quanto riguarda le credenziali, permette all'amministratore di:
\begin{itemize}
\item \textbf{Aggiungere una credenziale}: l'amministratore ha la possibilità di aggiungere una credenziale premendo sul bottone "Add Credential"; \\
\item \textbf{Eliminare una credenziale}: l'amministratore ha la possibilità di eliminare una credenziale premendo sul bottone dove è raffigurato un cestino; \\
\item \textbf{Modificare una credenziale}:  l'amministratore ha la possibilità di modificare le caratteristiche di una credenziale premendo sul bottone dove è raffigurata una "E". \\
\end{itemize}

\subsection{Glossario}
All’interno del documento sono presenti termini che assumono significati diversi a seconda del contesto. Per evitare ambiguità, è stata posta alla fine del documento una sezione di nome Glossario che conterrà tali termini con il loro significato specifico. Per segnalare che un termine del testo è presente all’interno del Glossario, verrà aggiunta una G pedice posta a fianco del termine ambiguo.



