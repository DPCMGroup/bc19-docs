\section{Consuntivo}
Il bilancio potrà essere:
\begin{itemize}
	\item Positivo: se sono state utilizzate meno ore di quelle preventivate;
	\item Pari: se sono state utilizzate tutte e sole le ore preventivate
	\item Negativo: se sono state utilizzate più ore di quelle preventivate
\end{itemize}
\subsection{Fase di Analisi}
Le ore impiegate per la fase di analisi sono considerate come ore di investimento personale per cui non sono rendicontate.
\begin{table}[H]
	\rowcolors{2}{lightest-grayest}{white}
	\centering
	\renewcommand{\arraystretch}{1.5}
	\begin{tabular}{|c|c|c|}
		\hline
		\rowcolor{lighter-grayer}
		Ruolo & Ore & Costo \\
		\hline
		Responsabile &  &  \\
		\hline
		Amministratore &  &  \\
		\hline
		Analista &  &  \\
		\hline
		Progettista&  &  \\
		\hline
		Programmatore &  &  \\
		\hline
		Verificatore &  & 50\euro \\
		\hline
		Totale Preventivo &  &  2000,00\euro \\
		\hline
		Totale Consuntivo &  &  2400,00\euro \\
		\hline
		Scostamento &  &  400,00\euro \\
		\hline
	\end{tabular}
	\caption*{\textbf{Tabella 17}: Consuntivo riguardante la fase di Analisi\\}
\end{table}
\subsubsection{Conclusioni}
\subsection{Preventivo a finire}