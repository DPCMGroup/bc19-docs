\section{Consuntivo di periodo}
Di seguito si analizza lo scostamento tra le spese effettive per ogni ruolo e il preventivo. Il bilancio potrà essere:
\begin{itemize}
	\item Positivo: se sono state utilizzate meno ore di quelle preventivate;
	\item Pari: se sono state utilizzate tutte e sole le ore preventivate;
	\item Negativo: se sono state utilizzate più ore di quelle preventivate.
\end{itemize}
\subsection{Periodo di Analisi}
Le ore impiegate per il periodo di analisi sono considerate come ore di investimento personale per cui non sono rendicontate.
\begin{table}[H]
	\rowcolors{2}{lightest-grayest}{white}
	\centering
	\renewcommand{\arraystretch}{1.5}
	\begin{tabular}{|c|c|c|}
		\hline
		\rowcolor{lighter-grayer}
Ruolo & Ore & Costo \\ \hline
Responsabile & 34(+0) & 1020(+0\euro) \\ \hline
Amministratore & 64(+3) & 1408(+66\euro) \\ \hline
Analista & 79(+6) & 1975(+150\euro) \\ \hline
Progettista & 0(+0) & 0(+0\euro) \\ \hline
Programmatore & 0(+0) & 0(+0\euro) \\ \hline
Verificatore & 82(+3) & 1312(+48\euro) \\ \hline
Totale preventivo & 259 & 5715\euro \\ \hline
Totale consuntivo & 271 & 5979\euro \\ \hline
Totale scostamento & 12 & 264\euro \\ \hline
	\end{tabular}
	\caption*{\textbf{Tabella 13}: Consuntivo riguardante il periodo di Analisi\\}
\end{table}
\subsubsection{Conclusioni}
Come emerge dalla tabella precedente è stato necessario investire più tempo di quanto preventivato
nei ruoli di Amministratore, Analista e Verificatore.
Il bilancio risultante è negativo e di seguito sono esposte le cause dei ritardi:
\begin{itemize}
	\item Amministratore: è stato necessario più tempo del previsto per individuare uno scheletro per la stesura del Piano di Qualifica. In particolare,
	nella ricerca della modalità per parallelizzare i compiti il più possibile, senza perdere in termini di uniformità del documento;
	\item Analista: alcuni requisiti si sono rivelati di non facile comprensione, quindi sono state necessarie 
	più ore di lavoro del previsto;
	\item Verificatori: alcuni requisiti sono stati individuati tardivamente e di conseguenza per
	l’aggiunta di contenuti nel documento di Analisi dei Requisiti è stato necessario impiegare
	qualche ora in più per controllare nuovamente il documento.
\end{itemize}
\subsection{Preventivo a finire}
Il preventivo a finire presenta un surplus di 264\euro, poiché in questo periodo di analisi sono state necessarie più ore del previsto.
Ciò non è ritenuto un problema in quanto le ore lavorative e i costi sostenuti in questo periodo non verranno rendicontati.