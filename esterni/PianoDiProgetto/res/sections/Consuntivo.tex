\section{Consuntivo di periodo}
Di seguito si analizza lo scostamento tra le spese effettive per ogni ruolo e il preventivo. Il bilancio potrà essere:
\begin{itemize}
	\item Positivo: se sono state utilizzate meno ore di quelle preventivate;
	\item Pari: se sono state utilizzate tutte e sole le ore preventivate;
	\item Negativo: se sono state utilizzate più ore di quelle preventivate.
\end{itemize}
\subsection{Periodo di Analisi}
Le ore impiegate per il periodo di analisi sono considerate come ore di investimento personale per cui non sono rendicontate.
\begin{table}[H]
	\rowcolors{2}{lightest-grayest}{white}
	\centering
	\renewcommand{\arraystretch}{1.5}
	\begin{tabular}{|c|c|c|}
		\hline
		\rowcolor{lighter-grayer}
Ruolo & Ore & Costo \\ \hline
Responsabile & 34(+0) & 1020(+0\euro) \\ \hline
Amministratore & 64(+3) & 1408(+66\euro) \\ \hline
Analista & 79(+6) & 1975(+150\euro) \\ \hline
Progettista & 0(+0) & 0(+0\euro) \\ \hline
Programmatore & 0(+0) & 0(+0\euro) \\ \hline
Verificatore & 82(+3) & 1312(+48\euro) \\ \hline
Totale preventivo & 259 & 5715\euro \\ \hline
Totale consuntivo & 271 & 5979\euro \\ \hline
Totale scostamento & 12 & 264\euro \\ \hline
	\end{tabular}
	\caption*{\textbf{Tabella 13}: Consuntivo riguardante il periodo di Analisi\\}
\end{table}
\subsubsection{Conclusioni}
Come emerge dalla tabella precedente è stato necessario investire più tempo di quanto preventivato
nei ruoli di Amministratore, Analista e Verificatore.
Il bilancio risultante è negativo e di seguito sono esposte le cause dei ritardi:
\begin{itemize}
	\item Amministratore: è stato necessario più tempo del previsto per individuare uno scheletro per la stesura del Piano di Qualifica. In particolare,
	nella ricerca della modalità per parallelizzare i compiti il più possibile, senza perdere in termini di uniformità del documento;
	\item Analista: alcuni requisiti si sono rivelati di non facile comprensione, quindi sono state necessarie 
	più ore di lavoro del previsto;
	\item Verificatori: alcuni requisiti sono stati individuati tardivamente e di conseguenza per
	l’aggiunta di contenuti nel documento di Analisi dei Requisiti è stato necessario impiegare
	qualche ora in più per controllare nuovamente il documento.
\end{itemize}
\subsection{Preventivo a finire}
Il preventivo a finire presenta un surplus di 264\euro, poiché in questo periodo di analisi sono state necessarie più ore del previsto.
Ciò non è ritenuto un problema in quanto le ore lavorative e i costi sostenuti in questo periodo non verranno rendicontati.
\subsection{Preventivo a finire}

\begin{table}[H]
	\rowcolors{2}{lightest-grayest}{white}
	\centering
	\renewcommand{\arraystretch}{1.5}
	\begin{tabular}{|c|p{10mm}|p{10mm}|p{10mm}|p{10mm}|p{10mm}|p{10mm}|}
		\hline
		\rowcolor{lighter-grayer}
		Attività & Re & Am & An & Pt & Pm & Ve \\ \hline
		Aggiornamento PdQ & \cellcolor{white}  &  &  &  &  & 20 \\
		Aggiornamento Adr & \cellcolor{white} &     & 40 &  &  & 2 \\
		Aggiornamento NdP & \multirow{-3}*{\cellcolor{white}2} & 6 &  &  &  & 2 \\ \hline
		Aggiornamento PdP     & 10	&   &    &    &  & 2  \\ \hline
		Technology baseline   & 1	&   &    & 20 &  &    \\ \hline
		Proof of concept      & 1	&   &    & 40 &  &    \\ \hline
		Attività accessorie   & 2& 5 & 1  & 2  &  & 4  \\ \hline
		Comunicazione con docenti               & 2&   &    &    &  &    \\ \hline
		Comunicazione con proponente            & 2&   &    &    &  &    \\ \hline
		Presentazione         & 1& 1 & 1  & 1  &  & 1 \\
		\hline
	\end{tabular}
	\caption*{\textbf{Tabella n}: Pianificazione riguardante il periodo di Progettazione architetturale\\}
\end{table}
I motivi dello sforamento rispetto alla pianificazione settimanale sono riportati di seguito:
\begin{itemize}
	\item sono stati riscontrati dei dubbi imprevisti da chiarire con il proponente e con i docenti, in particolare 
per quanto riguarda l'analisi dei requisiti, si è deciso di sfruttare appieno l'interazione con la tecnologia NFC e
quindi modificare i relativi casi d'uso.
\item sono stati implementati degli strumenti gestionali per la definizione del nostro cruscotto, che ci hanno aiutato nella pianificazione 
e verifica di quest'ultima. 
\item si è riscontrata una certa difficoltà a chiudere alcune issues in breve tempo. 
A tal proposito si è deciso di scrivere issues con compiti più definiti e
a grana ancora più fine. In questo modo 
chi le dovrà svolgere sarà meno disorientato, il progresso del lavoro risulterà più semplice da calcolare
e la verifica più puntuale. Ciò comporta una rivalutazione al rialzo delle ore da assegnare al responsabile, in quanto
presente in tutte le attività per coordinare al meglio il gruppo e per determinare le issue stesse.

\item si è impiegato più tempo del previsto nell'integrazione fra le parti del Poc, in particolare back-end e Ethereum, attraverso la libreria web3.py. Si è provveduto a fornire una guida per la configurazione di queste tecnologie, in modo da evitare incongruenze e/o perdite di tempo all'interno del gruppo.
\end{itemize}


Il preventivo a finire presenta un risparmio di 820\euro,
poiché in questo periodo di progettazione architetturale
sono state necessarie meno ore del previsto.
I motivi sono i seguenti:
\begin{itemize}
	\item sono state dedicate più ore alla formazione personale dei membri del gruppo riguardo alle tecnologie individuate,
	rispetto all'effettivo sviluppo del Poc. In questo caso, solo la seconda è un'attiva rendicontabile, mentre la prima no.
	\item sono state preventivate più ore del necessario per l'aggiornamento della documentazione.
\end{itemize}
Si è deciso di reinvestire le ore risparmiate, nel lavoro di codifica
e allo sviluppo di funzionalità opzionali nel periodo successivo di progettazione di dettaglio e codifica.

Progettazione di dettaglio e codifica
Attività da svolgere e distribuzione delle ore prevista
\begin{table}[H]
	\rowcolors{2}{lightest-grayest}{white}
	\centering
	\renewcommand{\arraystretch}{1.5}
	\begin{tabular}{|c|p{10mm}|p{10mm}|p{10mm}|p{10mm}|p{10mm}|p{10mm}|}
		\hline
		\rowcolor{lighter-grayer}
		Attività & Re & Am & An & Pt & Pm & Ve \\ \hline
		Aggiornamento PdQ& 1 &    &    &    &    & 20 \\ \hline
		Aggiornamento AdR& 1 &    & 20 &    &    & 2  \\ \hline
		Aggiornamento NdP& 1 & 10 &    &    &    & 2  \\ \hline
		Aggiornamento PdP& 10&    &    &    &    & 2  \\ \hline
		Product baseline & 2 &    &    & 18 &    & 2  \\ \hline
		Codifica della struttura delle componenti     & 2 &    &    &    & 35 & 15 \\ \hline
		\begin{tabular}[x]{@{}c@{}}Codifica delle interazioni \\ tra le componenti interne\end{tabular}      & 2 &    &    &    & 35 & 15 \\ \hline
		Codifica delle interazioni con gli attori esterni  & 2 &    &    &    & 35 & 15 \\ \hline
		\begin{tabular}[x]{@{}c@{}}Codifica delle funzionalità secondarie \\ non ancora implementate\end{tabular} & 1 &    &    &    & 32 & 15 \\ \hline
		Stesura manuale utente& \cellcolor{white}  & 12 &    &    &    & 4  \\
		Stesura manuale manutentore & \multirow{-2}*{\cellcolor{white}1}  & 12 &    &    &    & 4  \\ \hline
		Attività accessorie    & 2 & 5  & 1  & 1  & 2  & 3  \\ \hline
		Comunicazione con docenti   & 2 &    &    &    &    &    \\ \hline
		Comunicazione con proponente& 2 &    &    &    &    &    \\ \hline
		Presentazione    & 1 & 1  & 1  & 1  & 1  & 1 \\
		\hline
	\end{tabular}
	\caption*{\textbf{Tabella n1}: Pianificazione riguardante il periodo di Progettazione di dettaglio e codifica\\}
\end{table}
