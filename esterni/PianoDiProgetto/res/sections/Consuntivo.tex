\section{Consuntivo}
Di seguito si analizza lo scostamento tra le spese effettive per ogni ruolo e il preventivo. Il bilancio potrà essere:
\begin{itemize}
	\item Positivo: se sono state utilizzate meno ore di quelle preventivate;
	\item Pari: se sono state utilizzate tutte e sole le ore preventivate
	\item Negativo: se sono state utilizzate più ore di quelle preventivate
\end{itemize}
\subsection{Fase di Analisi}
Le ore impiegate per la fase di analisi sono considerate come ore di investimento personale per cui non sono rendicontate.
\begin{table}[H]
	\rowcolors{2}{lightest-grayest}{white}
	\centering
	\renewcommand{\arraystretch}{1.5}
	\begin{tabular}{|c|c|c|}
		\hline
		\rowcolor{lighter-grayer}
		Ruolo & Ore & Costo \\
		\hline
		Responsabile & 30(+0) & 900(+0\euro) \\
		\hline
		Amministratore & 60(+3) & 1200(+63\euro)  \\
		\hline
		Analista &  80(+6) & 2000(+150\euro)  \\
		\hline
		Progettista& - & - \\
		\hline
		Programmatore & - & - \\
		\hline
		Verificatore & 82(+3) & 1230(+45\euro) \\
		\hline
		Totale Preventivo & 252 & 5330\euro \\
		\hline
		Totale Consuntivo & 264 & 5588\euro \\
		\hline
		Scostamento & 12 & 258\euro \\
		\hline
	\end{tabular}
	\caption*{\textbf{Tabella 15}: Consuntivo riguardante la fase di Analisi\\}
\end{table}
\subsubsection{Conclusioni}
\subsection{Preventivo a finire}