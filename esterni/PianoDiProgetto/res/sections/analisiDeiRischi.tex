\section{Analisi dei rischi}
L'analisi è un'attività fondamentale nello sviluppo di un progetto complesso in quanto permette di individuare problemi che possono essere evitati.
A questo scopo è stata effettuata un'attività di analisi dei rischi riguardante tutti i rischi che possono insorgere durante il progetto.
Per ogni voce presente nella tabella sottostante è stata utilizzata la seguente procedura:
\begin{itemize}
	\item \textbf{Individuazione dei rischi:} Attività di studio e riflessione dove sono stati indiviuduati tutti i fattori che possono portare a rallentare il regolare svolgimento del progetto o che possono creare problemi nel normale proseguimento del progetto; \\
	\item \textbf{Analisi dei rischi:} Attività di studio e riflessione sui rischi precedentemente individuati. Durante questa fase ci si è concentrati sulla pericolosità e frequenza del rischio e a questo scopo sono stati determinati un indice di probabilità e un indice di gravità con entrambi gli indici che possono ottenere valori pari a: Alto, Medio e Basso; \\
	\item \textbf{Pianificazione di controllo:} Attività che permette di pianificare una metodologia al fine di cercare di evitare il rischio e nel caso questo non sia possibile permette di proseguire il progetto nel migliore modo rispettando tutte le regole imposte; \\
	\item \textbf{Monitoraggio dei rischi :} Attività che viene svolta in modo continuo al fine di evitare l'insorgere dei rischi. Nel caso non sia possibile evitarli permette di agire tempestivamente; \\
\end{itemize}
Sono stati definit i seguenti codici per definire i fattori di rischi presenti nella tabella sottostante:
\begin{itemize}
	\item \textbf{RT:} Rischi riguardanti la tecnologia;
	\item \textbf{RO:} Rischi riguardanti l'organizzazione;
	\item \textbf{RR:} Rischi riguardanti i requisiti;
	\item \textbf{RP:} Rischi riguardanti le persone;
\end{itemize}
\begin{center}
	\rowcolors{2}{lightest-grayest}{white}
		\begin{longtable}{|p{0.15\columnwidth}|p{0.30\columnwidth}|p{0.30\columnwidth}|p{0.15\columnwidth}|}
			\hline
			\rowcolor{lighter-grayer}
			\centering\textbf{Nome \\ Codice} & \centering\textbf{Descrizione} & \centering\textbf{Rilevamento} & \textbf{Grado di rischio} \\
			\hline
			\endfirsthead
		
		% ----- Modificare da qui -----
		% Manca titolo tabella , collegamento a indice tabella, bordo sinistro riga Piano di contingenza, centrare ultima colonna,aggiustare centrature titoli, aggiustare tabella a pagina 6;
		\hline
		\centering Inesperienza Tecnologica RT1& Per la maggior parte dei membri del gruppo le tecnologie da usare non sono familiari e questo potrebbe portare problemi al normale proseguimento del progetto.  & Il responsabile dovrà richiedere a ogni membro del gruppo il livello di familiarità e di conoscenza delle tecnologie. Inoltre ogni membro del gruppo dovrà comunicare le difficoltà riscontrate. & Occorrenza: \textbf{Alta} Pericolosità: \textbf{Alta} \\
		\hline
		\centering Piano di contingenza & \multicolumn{3}{p{0.84\textwidth}}{I compiti che richiedono maggiori conoscenze tecnologiche verrano asseganti a più persone e, se è possibile, alle persone più compotenti con quelle tecnologie favorendo l'aiuto reciproco e la risoluzione del problema.} \\
		\hline
		\centering Tempistiche RO1& Nella realizzazione del progetto, i membri del gruppo dovranno imparare in modo autonomo delle nuove tecnologie e questo potrebbe portare a imprecisioni nel calcolo delle tempistiche specialmente per le tecnologie sconosciute. & Durante lo sviluppo del progetto verranno definite delle scadenze per portare a termine ciascun compito. In caso di problemi bisognerà sempre comunicarlo. & Occorrenza: \textbf{Alta} Pericolosità: \textbf{Alta} \\
		\hline
		\centering Piano di contingenza & \multicolumn{3}{p{0.84\textwidth}}{In caso di si riscontri il problema , il responsabile potrà indicare le persone più esperte con quella tecnologia di fornire assistenza nel problema e in certi casi potrà spostare le scadenze temporali.} \\
		\hline
		\centering Calcolo costi RO2& La pianificazione prevede dei costi per le attività. Considerando l'inesperienza dei membri del gruppo questo potrà portare a errori o imprecisioni nelle valutazioni. & Ogni membro del gruppo si impegnerà a rispettare le scadenze imposte e le decisioni prese, comunicando al responsabile eventuali problemi. Inoltre sarà compito del responsabile verificare che tutto si svolga in modo corretto monitorando le ore di lavoro dei membri del gruppo. & Occorrenza: \textbf{Media} Pericolosità: \textbf{Alta}\\
		\hline
		\centering Piano di contingenza & \multicolumn{3}{p{0.84\textwidth}}{In caso si verifichino delle variazioni il responsabile cercherà di ridistribuire le attività per rispettare le scadenze  in caso di forti mutamenti verrà comunicato tempestivamente al committente.  } \\
		\hline
		\centering Impegni universitari RO3& Durante lo sviluppo del progetto è molto probabile che molti membri del gruppo sia indisponibili a causa di impegni accademici.  &Al fine di evitare ritardi, ogni membro del gruppo comunicherà eventuali impegni accademici. & Occorrenza: \textbf{Alta} Pericolosità: \textbf{Bassa} \\
		\hline 
		\centering Piano di contingenza & \multicolumn{3}{p{0.84\textwidth}}{Al fine di evitare ritardi, gli incarchi saranno assegnati in base alle comunicazioni effettuate dai membri del gruppo.} \\
		\hline
		\centering Scadenze RO4& Vari problemi come quelli sopracitati possono portare a ritardi e di conseguenza a non rispettare le scadenze fissate.  & Ogni membro del gruppo dovrà segnlare l'impossibiltà di rispettare le scadenze. & Occorrenza: \textbf{Media} Pericolosità: \textbf{Media} \\
		\hline
		\centering Piano di contingenza & \multicolumn{3}{p{0.84\textwidth}}{Il responsabile, se necessario, riassegnerà i compiti e le risorse al fine di garantire il corretto avanzamento del progetto.} \\
		\hline
		\centering Impegni personali RO5& Ogni membro del gruppo con molta probabilità nel corso del progetto avrà degli impegni personali che non li consentiranno di svolegere il proprio compito in determinati periodi. & Ciascun membro comunicherà il prima possibile i propri impegni personali. Eventuali imprevisti saranno segnalati tempestivamente al responsabile. & Occorrenza: \textbf{Bassa} Pericolosità: \textbf{Bassa} \\
		\hline
		\centering Piano di contingenza & \multicolumn{3}{p{0.84\textwidth}}{Come per gli impegni universitari(RO3) si cercherà di rispettare gli eventuali impegni personali dei membri del gruppo al fine di evitare ritardi.} \\
		\hline
		\centering Modifica dei requisiti RR1& Nonostante i requisiti siano molto chiari, può accadere che subiscano delle variazioni.  & Ogni incontro con il proponente sarà verbalizzato per tenere traccia dei cambiamenti. Inoltre ci saranno continui contatti con il proponente al fine di rimanere costantemente aggiornati. & Occorrenza: \textbf{Bassa} Pericolosità: \textbf{Media} \\
		\hline
		\centering Piano di contingenza & \multicolumn{3}{p{0.84\textwidth}}{Il gruppo discuterà i cambiamenti con il proponente in modo da trovare un accordo e riorganizzarsi in modo efficiente.} \\
		\hline
		\centering Scarsa esperienza RP1& Nessun membro del gruppo ha mai lavorato a un progetto così lungo e complesso e questo potrebbe portare a ritardi vista la scarsa esperienza del gruppo. & Ogni membro del gruppo comunicherà al responsabile eventuali difficoltà riscontrate. & Occorrenza: \textbf{Alta} Pericolosità: \textbf{Alta} \\
		\hline
		\centering Piano di contingenza & \multicolumn{3}{p{0.84\textwidth}}{I compiti più difficili saranno assegnati a più persone e se possibile a quelle con maggiore esperienza.} \\
		\hline
		\centering Comunicazione nel gruppo RP2& Nel corso del progetto è possibile che ci siano delle incompresioni all'interno del gruppo oppure possono emergere contrasti di idee tra i vari componenti. &Ciascun membro del gruppo si impegnerà a limitare determinati comportamenti al fine di favorire il corretto svolgimento del progetto. & Occorrenza: \textbf{Bassa} Pericolosità: \textbf{Media} \\
		\hline
		\centering Piano di contingenza & \multicolumn{3}{p{0.84\textwidth}}{Il responsabile avrà il compito di fare da mediatore in caso di controversie.} \\
		\hline
		\centering Comunicazione con il proponente RP3& Durante il progetto ci saranno continui contatti con il proponente del progetto e in alcuni casi sfortunati può portare a incomprensioni.& Il gruppo si impegna a contattare regolarmente il proponente su telegram o inviando una mail per eventuali chiarimenti. & Occorrenza: \textbf{Media} Pericolosità: \textbf{Media} \\
		\hline
		\centering Piano di contingenza & \multicolumn{3}{p{0.84\textwidth}}{Il responsabile, se necessario, si impegna a richiedere incontri con il proponente.} \\
		\hline			
	\end{longtable}
\end{center}
