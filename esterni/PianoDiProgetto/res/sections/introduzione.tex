\section{Introduzione}

\subsection{Scopo del documento}
Lo scopo del documento è illustrare la pianificazione e le modalità attraverso le quali il gruppo DPCM2077 svilupperà il progetto BlockCOVID.
Il documento presenta i seguenti punti:
\begin{itemize}
	\item analisi dei rischi;
	\item modello di sviluppo;
	\item pianificazione dei ruoli e delle attività;
	\item stima dei costi per la realizzazione del progetto.
\end{itemize}

\subsection{Scopo del prodotto}
Il prodotto da sviluppare ha lo scopo di monitorare e regolare l'utilizzo e l'igienizzazione delle postazioni all'interno di uno spazio condiviso, al fine di ridurre il rischio di trasmissione di un'infezione e di adempiere alle norme di legge. 
\subsection{Glossario e documenti} 
All'interno del  documento sono presenti termini che assumono significati diversi a seconda del contesto.
Per evitare ambiguità, è stato creato un  documento di nome Glossario che  conterrà tali termini con il loro significato specifico. Per segnalare che un termine del testo è presente all'interno del Glossario, verrà aggiunta una G pedice posta a fianco del termine ambiguo.
Quando si fa riferimento a un altro documento riguardante questo progetto vi si pone a pedice una D.

\subsection{Riferimenti}
\subsubsection{Riferimenti Normativi}
\begin{itemize}
	\item \dext{Norme di Progetto v. 2.0.0}.
    \item{Organigramma del gruppo e offerta tecnico-economica} \\
 \url{https://www.math.unipd.it/~tullio/IS-1/2020/Progetto/RO.html#Org}
\end{itemize}
\subsubsection{Riferimenti Informativi}
\begin{itemize}
	\item {L06 - Gestione di Progetto} \\
 \url{https://www.math.unipd.it/~tullio/IS-1/2020/Dispense/L06.pdf}
\end{itemize}

\subsection{Scadenze}
Il gruppo DPCM2077 si impegna a rispettare le seguenti scadenze per lo sviluppo del progetto BlockCOVID:
\begin{itemize}
	\item \textbf{Revisione dei Requisiti:} 2021-01-18; \\
	\item \textbf{Revisione di Progettazione:} 2021-03-08; \\
	\item \textbf{Revisione di Qualifica:} 2021-04-09; \\
	\item \textbf{Revisione di Accettazione:} 2021-05-10. \\
\end{itemize}




