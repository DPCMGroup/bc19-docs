\section{Introduzione}

\subsection{Scopo del documento}
Lo scopo del documento è quello di illustrare la pianificazione e le modalità attraverso le quali il gruppo DPCM2077 svilupperà il progetto blockCOVID.
Il documento presenta i seguenti punti:
\begin{itemize}
	\item analisi dei rischi;
	\item modello di sviluppo;
	\item pianificazione dei ruoli e delle attività;
	\item stima dei costi per la realizzazione del progetto.
\end{itemize}

\subsection{Scopo del prodotto}
Il capitolato C1 - blockCOVID si pone come obiettivo quello di creare un'applicazione web e una mobile,
con cui interagire tramite un’interfaccia utente.
Il progetto nasce dalla necessità di adempiere alle normative vigenti in tema di salute e sicurezza sul lavoro.
Le applicazioni dovranno permettere rispettivamente le seguenti azioni: monitorare le presenze di un utente nelle $postazioni_G$ di un $organizzazione_G$,
$tracciandone_G$ l'igienizzazione, in modo tale da evitare contagi da $Covid_G$ e realizzare una reportistica certificata riguardo 
le presenze e le igienizzazioni; prenotare una postazione da remoto e consultare la lista degli ambienti utilizzati 
dall'ultima igienizzazione, per evitare di sanificare ambienti non utilizzati.

\subsection{Glossario}
All'interno del  documento sono presenti termini che presentano significati ambigui a seconda del contesto.
Per evitare questa ambiguità è stato creato un  documento di nome Glossario che  conterrà tali termini con il loro significato specifico. Per segnalare che un termine del testo è presente all'interno del Glossario  
verrà aggiunta una G pedice posta a fianco del termine ambiguo. 

\subsection{Riferimenti}
\subsubsection{Riferimenti Normativi}
\begin{itemize}
	\item Norme di Progetto: Norme di Progetto v1.0.0.
    \item \href{https://www.math.unipd.it/~tullio/IS-1/2020/Progetto/RO.html#Org}{Organigramma del gruppo e offerta tecnico-economica.}
\end{itemize}
\subsubsection{Riferimenti Informativi}
\begin{itemize}
	\item \href{https://www.math.unipd.it/~tullio/IS-1/2020/Dispense/L06.pdf}{L06 - Gestione di Progetto.}
\end{itemize}

\subsection{Scadenze}
Il gruppo DPCM2077 si impegna a rispettare le seguenti scadenze per lo sviluppo del progetto BlockCOVID:
\begin{itemize}
	\item \textbf{Revisione dei Requisiti:} 2021-01-18; \\
	\item \textbf{Revisione di Progettazione:} 2021-03-08; \\
	\item \textbf{Revisione di Qualifica:} 2021-04-09; \\
	\item \textbf{Revisione di Accettazione:} 2021-05-10; \\
\end{itemize}




