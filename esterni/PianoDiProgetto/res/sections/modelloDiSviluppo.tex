\section{Modello di sviluppo}
Un modello di sviluppo per il ciclo di vita del software e per la pianificazione di progetto è fondamentale e a questo scopo è stato scelto il \textbf{modello incrementale}.

\subsection{Modello incrementale}
Il modello incrementale prevede rilasci multipli e successivi ciascuno dei quali realizza un incremento di funzionalità.
L'aggiunta, la cancellazione e la modifica dei requisiti non sono possibili durante lo sviluppo dell'incremento però sono comunque consentiti negli altri casi dopo l'approvazione del proponente.
Questo modello può essere usato molto bene con il versionamento del sistema in quanto permette di tracciare tutte le modifiche avvenute.\\
I principali vantaggi di questo modello sono:
\begin{itemize}
	\item producono valore a ogni incremento, così ogni funzionalità diventa subito disponibile e può essere valutata dal proponente;\\
	\item ogni incremento riduce il rischio di fallimento, perché limitate al singolo incremento;\\
	\item le funzionalità principali, che vanno sviluppate prima, diventano sempre più stabili;\\
	\item l'analisi dei requisiti e la progettazione architetturale vengono svolte una sola volta così da stabilizzare i requisiti principali e l'architettura del sistema;\\
	\item le individuazioni e le correzioni degli errori sono più economiche; \\
	\item i test e le verifiche sono più mirati. \\
\end{itemize}

