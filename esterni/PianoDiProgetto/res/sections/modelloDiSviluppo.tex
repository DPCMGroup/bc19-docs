\section{Modello di sviluppo}


\subsection{Inizio del progetto}
All'inizio del progetto abbiamo individuato le macroattività che dovranno essere eseguite per portarlo a termine.
Abbiamo suddiviso le macroattività nei i 4 macroperiodi che separano una revisione dalla successiva.
All'inizio del progetto è stato steso anche un preventivo orario e di costo.
\subsection{Macroperiodo}
All'inizio di ogni macroperiodo vengono assegnate ad ogni macroattività di quel periodo delle ore.
\subsection{Periodo}
All'inizio di ogni periodo vengono scelte le macroattività da cominciare o continuare ad affrontare.
Ognuna viene suddivisa in attività e a queste attività viene assegnato un monte ore, suddiviso tra i componenti del gruppo che le dovranno eseguire.
All'interno del periodo non viene deciso quando l'attività debba essere svolta, le persone a cui è assegnata si organizzato liberamente.
Ogni volta che una persona si dedica ad una attività, deve indicare il tempo che ci ha impiegato.
Quando l'attività è completa va comunicato.
Quando tutte le attività di una macroattività sono complete, la macroattività viene considerata completata.

\subsection{Monitoraggio}
Se ci sono ritardi o anticipi lo si può notare dallo scostamento orario, del quale si fa un resoconto alla fine di ogni periodo. Nel periodo successivo si potrà utilizzare questa informazione per pianificare in modo migliore.
Gli strumenti di monitoraggio dell'attività sono visibili a tutti i componenti del gruppo e ognuno può utilizzarli. In questo modo l'amministratore ha un cruscotto di monitoraggio sempre aggiornato che gli mostra:
\begin{itemize}
	\item quanto siamo in ritardo o anticipo rispetto al preventivo
	\item quanto siamo in ritardo o anticipo rispetto alla pianificazione
	\item quante cose sono state fatte e quante ancora da fare
\end{itemize}


