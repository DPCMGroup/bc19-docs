\section{Modello di sviluppo}
Abbiamo adottato un modello di sviluppo che suddivide il tempo in macroperiodi e questi ultimi in microperiodi. \newline
I macroperiodi sono scanditi dalle revisioni di progetto. \newline
I microperiodi all'incirca di una settimana.

\subsection{Inizio del progetto}
All'inizio del progetto abbiamo individuato tutte le attività che dovranno essere eseguite per portarlo a termine.
Abbiamo suddiviso le attività nei 4 macroperiodi.
\subsection{Macroperiodo}
All'inizio di ogni macroperiodo vengono assegnate ad ogni macroattività di quel periodo delle ore.
All fine del macroperiodo viene steso un consuntivo delle ore utilizzate realmente, e in base a questo viene redatto un preventivo a finire dei macroperiodi successivi. Questo PaF potrà essere preso, eventualmente con qualche modifica, come pianificazione all'inizio del macroperiodo successivo.
\subsection{Microperiodo}
All'inizio di ogni periodo vengono scelte le attività da cominciare o proseguire, vi vengono assegnate delle persone, e per ogni persona delle ore, prendendo come riferimento la distribuzione eseguita all'inizio del macroperiodo. \newline
All'interno del periodo non viene deciso quando l'attività debba essere svolta, le persone a cui è assegnata si organizzato liberamente. \newline
Alla fine del microperiodo si esegue un consuntivo della settimana e si verifica quali attività sono state completate. Nel periodo successivo si potrà utilizzare questa informazione per pianificare in modo migliore.

\subsection{Monitoraggio}
Gli strumenti di monitoraggio sono raccolti in una cartella su Google Drive accessibile a tutti i componenti del gruppo sono i seguenti:
\begin{enumerate}
	\item Pianificazione di revisione: tabella rappresentante la suddivisione delle ore all'interno del macroperiodo tra le varie attività e i vari ruoli.
	\item Consuntivo di revisione: tabella rappresentante le ore effettivamente svolte da ogni ruolo per le diverse attività all'interno del macroperiodo.
	\item Pianificazione settimanale: tabella rappresentante la suddivisione delle ore all'interno del microperiodo tra le varie attività, i vari ruoli e le varie persone.
	\item Consuntivo settimanale: tabella rappresentante le ore effettivamente svolte da ogni ruolo per le diverse attività all'interno del microperiodo. La somma dei consuntivi settimanali fornisce il consuntivo di revisione.
	\item Grafici di scostamento settimanale: rappresentano per ogni settimana lo scostamento orario rispetto a quanto pianificato, suddiviso per ruolo e totale.
	\item Istogramma dell'esperienza per ruolo: mostra l'esperienza avuta da ogni componente per ogni ruolo. Viene usato dal responsabile per suddividere i ruoli nelle prossime attività.
	\item Grafico a torta del progresso dell'attività: mostra la percentuale di attività completate rispetto a quelle previste per il macroperiodo.
	\item Istogrammi del confronto orario e di costo: rappresentano il consuntivo orario e di costo per il macroperiodo presente e lo mettono a confronto con la pianificazione di macroperiodo.
\end{enumerate}

\noindent Gli strumenti di monitoraggio dell'attività sono visibili a tutti i componenti del gruppo e ognuno può utilizzarli. In questo modo l'amministratore ha un cruscotto di monitoraggio aggiornato alla fine di ogni settimana che gli dice:
\begin{itemize}
	\item qual è la previsione delle ore che spenderemo (strumenti 1 e 3);
	\item quanto siamo in ritardo o anticipo rispetto alla pianificazione (strumenti 2, 4, 5 e 8);
	\item come le ore sono state distribuite tra i componenti del gruppo (strumento 6);
	\item quante cose sono state fatte e quante ancora da fare (strumento 7).
\end{itemize}


