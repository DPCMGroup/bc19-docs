\section{Preventivo}
La suddivisione oraria viene fatta tenendo conto del fatto che nel corso del progetto ogni  membro del gruppo deve ricoprire ogni ruolo almeno una volta.
Per facilitare la lettura delle successive tabelle sono state utilizzate le seguenti abbreviazioni per identificare i diversi ruoli:
\begin{itemize}
	\item \textbf{Re}: \textit{Responsabile};
	\item \textbf{Am}: \textit{Amminstratore};
	\item \textbf{An}: \textit{Analista};
	\item \textbf{Pt}: \textit{Progettista};
	\item \textbf{Pm}: \textit{Programmatore};
	\item \textbf{Ve}: \textit{Verificatore};
\end{itemize}
Inoltre per determinare le ore nulle nelle successive tabelle verrà usato il simbolo -, che ne indica l'assenza.

\subsection{Fase di Analisi}
\subsubsection{Prospetto orario}
Nella fase di Analisi la distribuzione oraria è la seguente:
\begin{center}
	\rowcolors{2}{lightest-grayest}{white}
	\begin{longtable}{|p{0.25\columnwidth}|p{0.03\columnwidth}|p{0.03\columnwidth}|p{0.03\columnwidth}|p{0.03\columnwidth}|p{0.03\columnwidth}|p{0.03\columnwidth}|p{0.03\columnwidth}|}
		\hline
		\rowcolor{lighter-grayer}
		\centering\textbf{Nome} & \centering\textbf{Re} & \centering\textbf{Am} & \centering\textbf{An} &  \centering\textbf{Pt}&  \centering\textbf{Pm}&  \centering\textbf{Ve} & \textbf{Tot}\\
		\hline
		\endfirsthead
		
		% ----- Modificare da qui -----
		% Manca centrare ultima colonna, titolo tabella, collegamento tabella a indice tabelle, dati;
		\hline
		\centering Badan Antonio & \centering & \centering & \centering & \centering & \centering - & \centering & 30 \\
		\hline
		\centering Bertoldo Damiano & \centering & \centering & \centering & \centering & \centering - & \centering & 30 \\
		\hline
		\centering Budai Matteo & \centering & \centering & \centering & \centering & \centering - & \centering & 30 \\
		\hline
		\centering De Grandi Samuele & \centering & \centering & \centering & \centering & \centering - & \centering & 30 \\
		 \hline
		\centering Piacere Ivan & \centering & \centering & \centering & \centering & \centering - & \centering & 30 \\
		 \hline
		\centering Privitera Sara & \centering & \centering & \centering & \centering & \centering - & \centering & 30 \\
		 \hline
		\centering Spigolon Daniele & \centering & \centering & \centering & \centering & \centering - & \centering & 30 \\
		 \hline
		\centering\textbf{Ore totali}  & \centering & \centering & \centering & \centering & \centering - & \centering & 210 \\
		\hline
		
	\end{longtable}
\end{center}

I dati ottenuti possono essere riassunti nel seguente istogramma:
% fare istogramma (come?), dare tittolo e collegare all'indice delle figure;
\\

\subsubsection{Prospetto economico}
In questa fase il costo per ogni ruolo è il seguente:

\begin{center}
	\rowcolors{2}{lightest-grayest}{white}
	\begin{longtable}{|p{0.20\columnwidth}|p{0.04\columnwidth}|p{0.10\columnwidth}|}
		\hline
		\rowcolor{lighter-grayer}
		\centering\textbf{Ruolo} & \centering\textbf{Ore} & \textbf{Costo} \\
		\hline
		\endfirsthead
		
		% ----- Modificare da qui -----
		% Manca centrare ultima colonna, titolo tabella, collegamento tabella a indice tabelle, dati;
		\centering Responsabile & \centering & \euro\\
		\hline
		\centering Amministatore & \centering & \euro\\
		\hline
		\centering Analista & \centering & \euro\\
		\hline
		\centering Progettista & \centering & \euro\\
		\hline
		\centering Programmatore & \centering - & - \\
		\hline
		\centering Verificatore & \centering & \euro\\
		\hline
		\centering\textbf{Totale} & \centering & \euro\\
		\hline
	\end{longtable}
\end{center}
I dati ottenuti possono essere riassunti nel seguente areogramma:
% Dare titolo e collegare all'indice delle figure;
\\

\begin{tikzpicture}
	\pie{5/Responsabile, 10/Amministartore, 25/Analista, 5/Progettista, 55/Verificatore}
\end{tikzpicture}

	
\subsection{Fase di Consolidamento dei requisiti}
\subsubsection{Prospetto orario}
Nella fase di Consolidamento dei requisiti la distribuzione oraria è la seguente:
\begin{center}
	\rowcolors{2}{lightest-grayest}{white}
	\begin{longtable}{|p{0.25\columnwidth}|p{0.03\columnwidth}|p{0.03\columnwidth}|p{0.03\columnwidth}|p{0.03\columnwidth}|p{0.03\columnwidth}|p{0.03\columnwidth}|p{0.03\columnwidth}|}
		\hline
		\rowcolor{lighter-grayer}
		\centering\textbf{Nome} & \centering\textbf{Re} & \centering\textbf{Am} & \centering\textbf{An} &  \centering\textbf{Pt}&  \centering\textbf{Pm}&  \centering\textbf{Ve} & \textbf{Tot}\\
		\hline
		\endfirsthead
		
		% ----- Modificare da qui -----
		% Manca centrare ultima colonna, titolo tabella, collegamento tabella a indice tabelle, dati;
		\hline
		\centering Badan Antonio & \centering & \centering & \centering & \centering & \centering - & \centering & 6\\
		\hline
		\centering Bertoldo Damiano & \centering & \centering & \centering & \centering & \centering - & \centering & 6\\
		\hline
		\centering Budai Matteo & \centering & \centering & \centering & \centering & \centering - & \centering & 6\\
		\hline
		\centering De Grandi Samuele & \centering & \centering & \centering & \centering & \centering - & \centering & 6\\
		\hline
		\centering Piacere Ivan & \centering & \centering & \centering & \centering & \centering - & \centering & 6\\
		\hline
		\centering Privitera Sara & \centering & \centering & \centering & \centering & \centering - & \centering & 6\\
		\hline
		\centering Spigolon Daniele & \centering & \centering & \centering & \centering & \centering - & \centering & 6\\
		\hline
		\centering\textbf{Ore totali}  & \centering & \centering & \centering & \centering & \centering - & \centering & 42\\
		\hline
		
	\end{longtable}
\end{center}
I dati ottenuti possono essere riassunti nel seguente istogramma:
% fare istogramma (come?), dare titolo e collegare all'indice delle figure;
\\

\subsubsection{Prospetto economico}
In questa fase il costo per ogni ruolo è il seguente:

\begin{center}
	\rowcolors{2}{lightest-grayest}{white}
	\begin{longtable}{|p{0.20\columnwidth}|p{0.04\columnwidth}|p{0.10\columnwidth}|}
		\hline
		\rowcolor{lighter-grayer}
		\centering\textbf{Ruolo} & \centering\textbf{Ore} & \textbf{Costo} \\
		\hline
		\endfirsthead
		
		% ----- Modificare da qui -----
		% Manca centrare ultima colonna, titolo tabella, collegamento tabella a indice tabelle, dati;
		\centering Responsabile & \centering & \euro\\
		\hline
		\centering Amministatore & \centering & \euro\\
		\hline
		\centering Analista & \centering & \euro\\
		\hline
		\centering Progettista & \centering & \euro\\
		\hline
		\centering Programmatore & \centering - & - \\
		\hline
		\centering Verificatore & \centering & \euro\\
		\hline
		\centering\textbf{Totale} & \centering & \euro\\
		\hline
	\end{longtable}
\end{center}

I dati ottenuti possono essere riassunti nel seguente areogramma:
%Dare titolo e collegare all'indice delle figure;
\\
\begin{tikzpicture}
	\pie{5/Responsabile, 10/Amministartore, 30/Analista, 55/Verificatore}
\end{tikzpicture}



\subsection{Fase di Progettazione architetturale}
\subsubsection{Prospetto orario}

\subsubsection{Prospetto economico}

\subsection{Fase di Progettazione di dettaglio e codifica}
\subsubsection{Prospetto orario}

\subsubsection{Prospetto economico}

\subsection{Fase di Validazione e collaudo}
\subsubsection{Prospetto orario}

\subsubsection{Prospetto economico}

\subsection{Riepilogo}