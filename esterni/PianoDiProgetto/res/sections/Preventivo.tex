\section{Preventivo}
La suddivisione oraria viene fatta tenendo conto del fatto che nel corso del progetto ogni  membro del gruppo deve ricoprire ogni ruolo almeno una volta.
Per facilitare la lettura delle successive tabelle sono state utilizzate le seguenti abbreviazioni per identificare i diversi ruoli:
\begin{itemize}
	\item \textbf{Re}: \textit{Responsabile};
	\item \textbf{Am}: \textit{Amministratore};
	\item \textbf{An}: \textit{Analista};
	\item \textbf{Pt}: \textit{Progettista};
	\item \textbf{Pm}: \textit{Programmatore};
	\item \textbf{Ve}: \textit{Verificatore}.
\end{itemize}
Inoltre per determinare le ore nulle nelle successive tabelle verrà usato il simbolo -, che ne indica l'assenza.

Si può notare che tutti i componenti hanno, per ogni fase, almeno un'ora indicata nei ruoli di Responsabile, Amministratore e Verificatore. Ciò deriva dalle attività di produzione dei verbali delle riunioni. Ogni riunione infatti implica la produzione di un verbale, che coinvolge tre componenti nei tre ruoli appena citati. Per garantire una distribuzione equa del lavoro, questo compito è assegnato con la seguente rotazione alfabetica: se un verbale è redatto da Badan Antonio, verificato da Bertoldo Damiano e approvato da Budai Matteo, il successivo verbale sarà redatto da Bertoldo Damiano, verificato da Budai Matteo e approvato da De Grandi Samuele e così via.

Ogni documento diverso dai verbali è prodotto invece da chi sta effettivamente ricoprendo il ruolo preposto per quel documento. Per esempio il Piano di progetto è stato redatto dai responsabili, che nella fase di Analisi sono Budai Matteo, De Grandi Samuele e Piacere Ivan, come si può notare dalla tabella sottostante. 

La produzione dei documenti diversi dai verbali sono assegnate in modo da evitare il conflitto di interessi. Questo significa che dalla stesura all'approvazione di un documento nella versione $X.Y.Z$ con $X$ fissato (vedi \dext{Norme di Progetto v.1.0.0} - sezione 3.2.2.1), una persona si può occupare o della stesura o della verifica o dell'approvazione del documento. È ammesso che nella versione $X.Y.Z$ con $X$ fissato, un redattore possa essere anche verificatore purché in parti diverse del documento: ossia stesura della sezione $a$ e verifica della sezione $b$ con $a \neq b$. 

Nei registri delle modifiche dei documenti, chi verifica è sempre indicato nel ruolo di Verificatore e chi approva, sempre nel ruolo di Responsabile. 


\subsection{Fase di Analisi}
\subsubsection{Prospetto orario}
Nella fase di Analisi la distribuzione oraria è la seguente:
\begin{table}[H]
	\rowcolors{2}{lightest-grayest}{white}
	\centering
	\renewcommand{\arraystretch}{1.5}
	\begin{tabular}{|c|c|c|c|c|c|c|c|c|c|}
		\hline
		\rowcolor{lighter-grayer}
Nome & Re & Am & An & Pt & Pm & Ve & Totale \\ \hline
Badan Antonio & 1 & 1 & 19 & - & - & 16 & 37 \\ \hline
Bertoldo Damiano & 1 & 20 & - & - & - & 16 & 37 \\ \hline
Budai Matteo & 10 & 1 & 20 & - & - & 6 & 37 \\ \hline
De Grandi Samuele & 10 & 1 & 20 & - & - & 6 & 37 \\ \hline
Piacere Ivan & 10 & 1 & 20 & - & - & 6 & 37 \\ \hline
Privitera Sara & 1 & 20 & - & - & - & 16 & 37 \\ \hline
Spigolon Daniele & 1 & 20 & - & - & - & 16 & 37 \\ \hline
\textbf{Totale} & 34 & 64 & 79 & - & - & 82 & 259 \\ \hline
		
\end{tabular}
\caption*{\textbf{Tabella 2}: Distribuzione delle ore nella Fase di Analisi\\}
\end{table}	

I dati ottenuti possono essere riassunti nel seguente istogramma:
% fare istogramma (con dati poi), collegare all'indice delle figure;
\begin{figure}[H]
	\centering
	\includegraphics[width=0.7\linewidth]{res/images/IstogrammaFase1.png}
	\caption*{\textbf{Figura 6}: Istogramma della suddivisione delle ore durante la Fase di Analisi}
	\label{fig:Figura2}
\end{figure}

\subsubsection{Prospetto economico}
In questa fase il costo per ogni ruolo è il seguente:

\begin{table}[H]
	\rowcolors{2}{lightest-grayest}{white}
	\centering
	\renewcommand{\arraystretch}{1.5}
	\begin{tabular}{|c|c|c|}
		\hline
		\rowcolor{lighter-grayer}
Ruolo & Ore & Costo \\ \hline
Responsabile & 34 & 1020\euro \\ \hline
Amministratore & 64 & 1408\euro \\ \hline
Analista & 79 & 1975\euro \\ \hline
Progettista & - & 0\euro \\ \hline
Programmatore & - & 0\euro \\ \hline
Verificatore & 82 & 1312\euro \\ \hline
\textbf{Totale} & 259 & 5715\euro \\ \hline
\end{tabular}
	\caption*{\textbf{Tabella 3}: Prospetto dei costi per ruolo nella Fase di Analisi\\}
\end{table}
I dati ottenuti possono essere riassunti nel seguente areogramma:
%collegare all'indice delle figure;
\begin{figure}[H]
	\centering
	\begin{tikzpicture}
		\pie{13.1/Responsabile, 24.7/Amministratore, 30.5/Analista, 31.7/Verificatore}
	\end{tikzpicture}
	\caption*{\textbf{Figura 7}:  Areogramma della ripartizione di ore per ruolo nella Fase di Analisi}
	\label{fig:Figura3}
\end{figure}	



\subsection{Fase di Progettazione architetturale}
\subsubsection{Prospetto orario}
Nella fase di Progettazione architetturale la distribuzione oraria è la seguente:

\begin{table}[H]
	\rowcolors{2}{lightest-grayest}{white}
	\centering
	\renewcommand{\arraystretch}{1.5}
	\begin{tabular}{|c|c|c|c|c|c|c|c|}
		\hline
		\rowcolor{lighter-grayer}
Nome & Re & Am & An & Pt & Pm & Ve & Totale \\ \hline
Badan Antonio & 3 & 1 & 6 & 9 & - & 5 & 24 \\ \hline
Bertoldo Damiano & 3 & 1 & 6 & 9 & - & 5 & 24 \\ \hline
Budai Matteo & 1 & 1 & 6 & 10 & - & 6 & 24 \\ \hline
De Grandi Samuele & 1 & 1 & 6 & 10 & - & 6 & 24 \\ \hline
Piacere Ivan & 1 & 2 & 6 & 9 & - & 6 & 24 \\ \hline
Privitera Sara & 3 & 2 & 6 & 7 & - & 6 & 24 \\ \hline
Spigolon Daniele & 3 & 2 & 6 & 7 & - & 6 & 24 \\ \hline
\textbf{Totale} & 15 & 10 & 42 & 61 & - & 40 & 168 \\ \hline
	\end{tabular}
	\caption*{\textbf{Tabella 4}: Distribuzione delle ore nel periodo di Progettazione architetturale\\}
\end{table}	
I dati ottenuti possono essere riassunti nel seguente istogramma:

\begin{figure}[H]
	\centering
	\includegraphics[width=0.7\linewidth]{res/images/IstogrammaFase2.png}
	\caption*{\textbf{Figura 8}: Istogramma della suddivisione delle ore durante il periodo di Progettazione architetturale}
	\label{fig:Figura10}
\end{figure}


\subsubsection{Prospetto economico}
In questa fase il costo per ogni ruolo è il seguente:

\begin{table}[H]
	\rowcolors{2}{lightest-grayest}{white}
	\centering
	\renewcommand{\arraystretch}{1.5}
	\begin{tabular}{|c|c|c|}
		\hline
		\rowcolor{lighter-grayer}
Ruolo & Ore & Costo \\ \hline
Responsabile & 15 & 450\euro \\ \hline
Amministratore & 10 & 220\euro \\ \hline
Analista & 42 & 1050\euro \\ \hline
Progettista & 61 & 1464\euro \\ \hline
Programmatore & - & 0\euro \\ \hline
Verificatore & 40 & 640\euro \\ \hline
\textbf{Totale} & 168 & 3824\euro \\ \hline
	\end{tabular}
	\caption*{\textbf{Tabella 5}: Prospetto dei costi per ruolo nel periodo di Progettazione architetturale\\}
\end{table}

I dati ottenuti possono essere riassunti nel seguente areogramma:


\begin{figure}[H]
	\centering
	\begin{tikzpicture}
		\pie{8.9/Responsabile, 6/Amministratore, 25/Analista, 36.3/Progettista, 23.8/Verificatore}
	\end{tikzpicture}
	\caption*{\textbf{Figura 9}: Areogramma della ripartizione di ore per ruolo in Progettazione architetturale}
	\label{fig:Figura10}
\end{figure}



\subsection{Fase di Progettazione di dettaglio e codifica}
\subsubsection{Prospetto orario}
Nella fase di Progettazione di dettaglio e codifica la distribuzione oraria è la seguente:

\begin{table}[H]
	\rowcolors{2}{lightest-grayest}{white}
	\centering
	\renewcommand{\arraystretch}{1.5}
	\begin{tabular}{|c|c|c|c|c|c|c|c|}
		\hline
		\rowcolor{lighter-grayer}
Nome & Re & Am & An & Pt & Pm & Ve & Totale \\ \hline
Badan Antonio & 4 & 5 & 4 & 3 & 20 & 14 & 50 \\ \hline
Bertoldo Damiano & 4 & 5 & 4 & - & 20 & 17 & 50 \\ \hline
Budai Matteo & 5 & 6 & 4 & 5 & 20 & 10 & 50 \\ \hline
De Grandi Samuele & 4 & 6 & - & 5 & 20 & 15 & 50 \\ \hline
Piacere Ivan & 5 & 6 & - & 5 & 20 & 14 & 50 \\ \hline
Privitera Sara & 4 & 6 & 3 & - & 19 & 18 & 50 \\ \hline
Spigolon Daniele & 4 & 6 & 5 & 2 & 21 & 12 & 50 \\ \hline
\textbf{Totale} & 30 & 40 & 20 & 20 & 140 & 100 & 350 \\ \hline
	\end{tabular}
	\caption*{\textbf{Tabella 6}: Distribuzione delle ore nel periodo di Progettazione di dettaglio e codifica\\}
\end{table}	
I dati ottenuti possono essere riassunti nel seguente istogramma:

\begin{figure}[H]
	\centering
	\includegraphics[width=0.7\linewidth]{res/images/IstogrammaFase3.png}
	\caption*{\textbf{Figura 10}: Istogramma della suddivisione delle ore durante il periodo di Progettazione di dettaglio e codifica}
	\label{fig:Figura10}
\end{figure}


\subsubsection{Prospetto economico}
In questa fase il costo per ogni ruolo è il seguente:

\begin{table}[H]
	\rowcolors{2}{lightest-grayest}{white}
	\centering
	\renewcommand{\arraystretch}{1.5}
	\begin{tabular}{|c|c|c|}
		\hline
		\rowcolor{lighter-grayer}
Ruolo & Ore & Costo \\ \hline
Responsabile & 30 & 900\euro \\ \hline
Amministratore & 40 & 880\euro \\ \hline
Analista & 20 & 500\euro \\ \hline
Progettista & 20 & 480\euro \\ \hline
Programmatore & 140 & 2240\euro \\ \hline
Verificatore & 100 & 1600\euro \\ \hline
\textbf{Totale} & 350 & 6600\euro \\ \hline
	\end{tabular}
	\caption*{\textbf{Tabella 7}: Prospetto dei costi per ruolo nel periodo di Progettazione di dettaglio e codifica\\}
\end{table}

I dati ottenuti possono essere riassunti nel seguente areogramma:


\begin{figure}[H]
	\centering
	\begin{tikzpicture}
		\pie{8.6/Responsabile, 11.4/Amministratore, 5.7/Analista, 5.7/Progettista, 40/Programmatore, 28.6/Verificatore}
	\end{tikzpicture}
	\caption*{\textbf{Figura 11}: Areogramma della ripartizione di ore per ruolo in Progettazione di dettaglio e codifica}
	\label{fig:Figura10}
\end{figure}

\subsection{Fase di Validazione e collaudo}
\subsubsection{Prospetto orario}
Nella fase di Validazione e collaudo la distribuzione oraria è la seguente:

\begin{table}[H]
	\rowcolors{2}{lightest-grayest}{white}
	\centering
	\renewcommand{\arraystretch}{1.5}
	\begin{tabular}{|c|c|c|c|c|c|c|c|}
		\hline
		\rowcolor{lighter-grayer}
Nome & Re & Am & An & Pt & Pm & Ve & Totale \\ \hline
Badan Antonio & 4 & 1 & - & 10 & 4 & 7 & 26 \\ \hline
Bertoldo Damiano & 6 & 1 & - & - & 10 & 9 & 26 \\ \hline
Budai Matteo & 1 & 9 & - & - & 5 & 11 & 26 \\ \hline
De Grandi Samuele & 1 & 7 & - & 5 & 4 & 9 & 26 \\ \hline
Piacere Ivan & 1 & 8 & - & - & 4 & 13 & 26 \\ \hline
Privitera Sara & 6 & 1 & - & 10 & 6 & 3 & 26 \\ \hline
Spigolon Daniele & 4 & 1 & - & - & 10 & 11 & 26 \\ \hline
\textbf{Totale} & 23 & 28 & - & 25 & 43 & 63 & 182 \\ \hline
	\end{tabular}
	\caption*{\textbf{Tabella 8}: Distribuzione delle ore nel periodo di Validazione e collaudo\\}
\end{table}	
	I dati ottenuti possono essere riassunti nel seguente istogramma:

\begin{figure}[H]
	\centering
	\includegraphics[width=0.7\linewidth]{res/images/IstogrammaFase4.png}
	\caption*{\textbf{Figura 12}: Istogramma della suddivisione delle ore durante il periodo di Validazione e collaudo}
	\label{fig:Figura10}
\end{figure}
	
	
\subsubsection{Prospetto economico}
In questa fase il costo per ogni ruolo è il seguente:

\begin{table}[H]
	\rowcolors{2}{lightest-grayest}{white}
	\centering
	\renewcommand{\arraystretch}{1.5}
	\begin{tabular}{|c|c|c|}
		\hline
		\rowcolor{lighter-grayer}
Ruolo & Ore & Costo \\ \hline
Responsabile & 23 & 690\euro \\ \hline
Amministratore & 28 & 616\euro \\ \hline
Analista & - & 0\euro \\ \hline
Progettista & 25 & 600\euro \\ \hline
Programmatore & 43 & 688\euro \\ \hline
Verificatore & 63 & 1008\euro \\ \hline
\textbf{Totale} & 182 & 3602\euro \\ \hline
	\end{tabular}
\caption*{\textbf{Tabella 9}: Prospetto dei costi per ruolo nel periodo di Validazione e collaudo\\}
\end{table}

I dati ottenuti possono essere riassunti nel seguente areogramma:


\begin{figure}[H]
	\centering
	\begin{tikzpicture}
		\pie{12.6/Responsabile, 15.4/Amministratore, 13.7/Progettista, 23.6/Programmatore, 34.6/Verificatore}
	\end{tikzpicture}
	\caption*{\textbf{Figura 13}: Areogramma della ripartizione di ore per ruolo in Validazione e collaudo}
	\label{fig:Figura10}
\end{figure}

\subsection{Riepilogo}
\subsubsection{Ore totali}
\paragraph{Prospetto orario totale}
Nella seguente tabella viene riportata la distribuzione oraria di tutte le fasi:

\begin{table}[H]
	\rowcolors{2}{lightest-grayest}{white}
	\centering
	\renewcommand{\arraystretch}{1.5}
	\begin{tabular}{|c|c|c|c|c|c|c|c|}
		\hline
		\rowcolor{lighter-grayer}
Nome & Re & Am & An & Pt & Pm & Ve & Totale \\ \hline
Badan Antonio & 12 & 8 & 29 & 22 & 24 & 42 & 137 \\ \hline
Bertoldo Damiano & 14 & 27 & 10 & 9 & 30 & 47 & 137 \\ \hline
Budai Matteo & 17 & 17 & 30 & 15 & 25 & 33 & 137 \\ \hline
De Grandi Samuele & 16 & 15 & 26 & 20 & 24 & 36 & 137 \\ \hline
Piacere Ivan & 17 & 17 & 26 & 14 & 24 & 39 & 137 \\ \hline
Privitera Sara & 14 & 29 & 9 & 17 & 25 & 43 & 137 \\ \hline
Spigolon Daniele & 12 & 29 & 11 & 9 & 31 & 45 & 137 \\ \hline
\textbf{Totale} & 102 & 142 & 141 & 106 & 183 & 285 & 959 \\ \hline
	\end{tabular}
	\caption*{\textbf{Tabella 10}: Prospetto orario che comprende tutte le fasi trattate in precedenza\\}
\end{table}	
I dati ottenuti possono essere riassunti nel seguente istogramma:

\begin{figure}[H]
	\centering
	\includegraphics[width=0.7\linewidth]{res/images/IstogrammaTotale.png}
	\caption*{\textbf{Figura14}: Istogramma della suddivisione delle ore di tutte le fasi trattate in precedenza}
	\label{fig:Figura10}
\end{figure}

\paragraph{Prospetto economico totale}
Nella seguente tabella vengono mostrati i costi complessivi per ogni ruolo:

\begin{table}[H]
	\rowcolors{2}{lightest-grayest}{white}
	\centering
	\renewcommand{\arraystretch}{1.5}
	\begin{tabular}{|c|c|c|}
		\hline
		\rowcolor{lighter-grayer}
Ruolo & Ore & Costo \\ \hline
Responsabile & 102 & 3060\euro \\ \hline
Amministratore & 142 & 3124\euro \\ \hline
Analista & 141 & 3525\euro \\ \hline
Progettista & 106 & 2544\euro \\ \hline
Programmatore & 183 & 2928\euro \\ \hline
Verificatore & 285 & 4560\euro \\ \hline
\textbf{Totale} & 959 & 19741\euro \\ \hline
	\end{tabular}
	\caption*{\textbf{Tabella 11}: Prospetto dei costi totali per ciascun ruolo \\}
\end{table}

I dati ottenuti possono essere riassunti nel seguente areogramma:


\begin{figure}[H]
	\centering
	\begin{tikzpicture}
		\pie{10.6/Responsabile, 14.8/Amministratore, 14.7/Analista, 11.1/Progettista, 19.1/Programmatore, 29.7/Verificatore}
	\end{tikzpicture}
	\caption*{\textbf{Figura 15}: Areogramma dei costi totali delle ore di investimento e rendicontate}
    \label{fig:Figura10}
\end{figure}

\subsubsection{Ore totali rendicontate}
\paragraph{Prospetto orario totale rendicontato}
Nella seguente tabella viene riportata la distribuzione oraria di tutte le fasi a carico del committente, ad esclusione quindi, dell'Analisi e del Consolidamento dei requisiti:

\begin{table}[H]
	\rowcolors{2}{lightest-grayest}{white}
	\centering
	\renewcommand{\arraystretch}{1.5}
	\begin{tabular}{|c|c|c|c|c|c|c|c|}
		\hline
		\rowcolor{lighter-grayer}
Nome & Re & Am & An & Pt & Pm & Ve & Totale \\ \hline
Badan Antonio & 11 & 7 & 10 & 22 & 24 & 26 & 100 \\ \hline
Bertoldo Damiano & 13 & 7 & 10 & 9 & 30 & 31 & 100 \\ \hline
Budai Matteo & 7 & 16 & 10 & 15 & 25 & 27 & 100 \\ \hline
De Grandi Samuele & 6 & 14 & 6 & 20 & 24 & 30 & 100 \\ \hline
Piacere Ivan & 7 & 16 & 6 & 14 & 24 & 33 & 100 \\ \hline
Privitera Sara & 13 & 9 & 9 & 17 & 25 & 27 & 100 \\ \hline
Spigolon Daniele & 11 & 9 & 11 & 9 & 31 & 29 & 100 \\ \hline
\textbf{Totale} & 68 & 78 & 62 & 106 & 183 & 203 & 700 \\ \hline
	\end{tabular}
	\caption*{\textbf{Tabella 12}: Prospetto orario che comprende tutte le ore rendicontate\\}
\end{table}	
I dati ottenuti possono essere riassunti nel seguente istogramma:

\begin{figure}[H]
	\centering
	\includegraphics[width=0.7\linewidth]{res/images/IstogrammaTotaleRendicontato.png}
	\caption*{\textbf{Figura 16}: Istogramma della suddivisione delle ore rendicontate}
	\label{fig:Figura10}
\end{figure}

\paragraph{Prospetto economico totale rendicontato}
Nella seguente tabella vengono mostrati i costi complessivi rendicontati per ogni ruolo:

\begin{table}[H]
	\rowcolors{2}{lightest-grayest}{white}
	\centering
	\renewcommand{\arraystretch}{1.5}
	\begin{tabular}{|c|c|c|}
		\hline
		\rowcolor{lighter-grayer}
Ruolo & Ore & Costo \\ \hline
Responsabile & 68 & 2040\euro \\ \hline
Amministratore & 78 & 1716\euro \\ \hline
Analista & 62 & 1550\euro \\ \hline
Progettista & 106 & 2544\euro \\ \hline
Programmatore & 183 & 2928\euro \\ \hline
Verificatore & 203 & 3248\euro \\ \hline
\textbf{Totale} & 700 & 14026\euro \\ \hline
	\end{tabular}
	\caption*{\textbf{Tabella 13}: Prospetto dei costi totali rendicontati \\}
\end{table}

I dati ottenuti possono essere riassunti nel seguente areogramma:


\begin{figure}[H]
	\centering
	\begin{tikzpicture}
		\pie{9.7/Responsabile, 11.1/Amministratore, 8.9/Analista, 15.1/Progettista, 26.1/Programmatore, 29/Verificatore}
	\end{tikzpicture}
	\caption*{\textbf{Figura 17}: Areogramma delle ore rendicontate per ruolo}
    \label{fig:Figura10}
\end{figure}

\subsubsection{Conclusioni}
Il costo del progetto considerando le ore rendicontate è: 14026\euro.