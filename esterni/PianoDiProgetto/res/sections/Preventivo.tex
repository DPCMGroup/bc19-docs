\section{Preventivo}
La suddivisione oraria viene fatta tenendo conto che ogni  membro del gruppo deve ricoprire ogni ruolo almeno una volta.
Per facilitare la lettura delle successive tabelle sono state utilizzate le seguenti abbreviazioni per identificare i diversi ruoli:
\begin{itemize}
	\item \textbf{Re}: \textit{Responsabile};
	\item \textbf{Am}: \textit{Amminstratore};
	\item \textbf{An}: \textit{Analisti};
	\item \textbf{Pt}: \textit{Progettista};
	\item \textbf{Pm}: \textit{Programmatore};
	\item \textbf{Ve}: \textit{Verificatore};
\end{itemize}
Inoltre per determinare le ore nulle nelle successive tabelle verrà usato il simbolo -, che ne indica l'assenza.

\subsection{Fase di Analisi}
\subsubsection{Prospetto orario}

\subsubsection{Prospetto economico}

\subsection{Fase di Consolidamento dei requisiti}
\subsubsection{Prospetto orario}

\subsubsection{Prospetto economico}

\subsection{Fase di Progettazione architetturale}
\subsubsection{Prospetto orario}

\subsubsection{Prospetto economico}

\subsection{Fase di Progettazione di dettaglio e codifica}
\subsubsection{Prospetto orario}

\subsubsection{Prospetto economico}

\subsection{Fase di Validazione e collaudo}
\subsubsection{Prospetto orario}

\subsubsection{Prospetto economico}

\subsection{Riepilogo}