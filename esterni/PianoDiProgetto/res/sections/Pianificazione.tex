\section{Pianificazione}
Si noterà che la pianificazione di Analisi dei requisiti è molto diversa, anche nella forma, dalle pianificazioni successive. Ciò è dovuto ai cambiamenti apportati al nostro modello di sviluppo e pianificazione. Abbiamo ritenuto comunque di lasciare la pianificazione dell'Analisi dei requisiti, visto che quel periodo è stato effettivamente svolto usandola come riferimento.


Lo sviluppo del progetto è diviso nei seguenti periodi:
\begin{enumerate}
	\item Analisi dei requisiti;
	\item Progettazione architetturale;
	\item Progettazione di dettaglio e codifica;
	\item Validazione e collaudo.
\end{enumerate} 

\subsection{Analisi dei requisiti (dal 2020-10-22 al 2021-01-17)}

\subsubsection{Ruoli attivi}
\begin{itemize}
	\item Responsabile;
	\item Amministratore;
	\item Analista;
	\item Verificatore.
\end{itemize}

\subsubsection{Periodi e attività}

\paragraph{Primo periodo (dal 2020-10-22 al 2020-11-09)}
Questo periodo comincia con la costituzione del gruppo di progetto. Serve dunque a porre le basi per una comunicazione e collaborazione efficace ed efficiente. Vi si svolgono inoltre attività di ricerca superficiale sui capitolati.

\begin{itemize}
	\item Configurazione degli strumenti collaborativi basilari: comprende la scelta e la configurazione dei mezzi per la comunicazione e il lavoro collaborativo;
	\item Analisi superficiale dei capitolati.
	
\end{itemize}

\paragraph{Secondo Periodo (dal 2020-11-10 al 2020-12-13)}
Gli obiettivi raggiunti in questo periodo sono il consolidamento degli strumenti di collaborazione e l'approfondimento dei capitolati.
\begin{itemize}
	\item Configurazione del \glock{repository}: questa attività comprende la creazione di un repository per la condivisione e il versionamento dei prodotti, ma anche la sua configurazione per la verifica automatica della loro validità;
	\item Raccolta di informazioni dai seminari: stesura di appunti sui seminari, da poter facilmente consultare quando necessario;
	\item Studio dei progetti degli anni passati: questa attività ha il fine di individuare gli errori più comuni e le tecniche da prendere come esempio;
	\item Bozza dello studio di fattibilità;
	\item Bozza delle norme di progetto.
\end{itemize}

\paragraph{Terzo Periodo (dal 2020-12-14 al 2021-01-08)}
Questo periodo segue la scelta del capitolato da affrontare. È dedicato alla sua analisi approfondita e alla stesura della documentazione. 
\begin{itemize}
	\item Stesura delle norme di progetto;
	\item Stesura dello studio di fattibilità;
	\item Stesura del piano di progetto;
	\item Analisi dei requisiti e stesura dell'omonimo documento;
	\item Stesura del piano di qualifica;
	\item Stesura del glossario;
	\item Stesura della lettera di presentazione;
	\item Aggiornamento consuntivo;
	\item Verifica della documentazione.
\end{itemize}

\paragraph{Quarto Periodo (dal 2021-01-09 al 2021-01-17)}
È l'ultimo periodo di analisi dei requisiti. Si concentra quindi sulla preparazione dell'esposizione dei prodotti dei tre periodi precedenti.
\begin{itemize}
	\item Preparazione dell'esposizione;
	\item Verifica dell'esposizione.
\end{itemize}


\begin{landscape}
	\begin{figure}[H]
		\centering
		\includegraphics[width=\linewidth]{res/images/ganttFase1.png}
		\caption{ Grafico di Gantt del periodo di analisi dei requisiti}
		\label{fig:Gantt Analisi dei requisiti}
	\end{figure}
\end{landscape}



\noindent Di seguito vengono riportate le attività che si è previsto di svolgere durante i macroperiodi rimanenti. Per ogni attività e per ogni ruolo viene indicata una quantità di ore che si prevede saranno necessarie per svolgerla. Alcune celle ricoprono più righe. In quei casi la somma delle ore richieste dall'insieme di attività per quel ruolo è pari al valore della cella. Si è deciso di accorpare più celle per evitare valori frazionari. \newline
\indent Le ore sono state distribuite prendendo come riferimento il preventivo del punto §4 e apportandovi modifiche basate sul Preventivo a finire del documento Piano di Progetto v. 1.0.0. A partire dalle versioni successive di Piano di Progetto v. 1.0.0 si è deciso di redigere un Preventivo a finire che mostri una pianificazione per i periodi successivi. Così facendo, il contenuto del Preventivo a finire di questa versione potrà essere utilizzato come Pianificazione nella versione successiva. \newline
\indent La distribuzione delle ore riguarda tutti i periodi rimanenti, tuttavia più la previsione è lontana nel futuro, più la sua realizzazione è difficile. Abbiamo deciso di redigere comunque una pianificazione per i periodi più lontani in modo da applicarvi i miglioramenti dedotti, ad ogni versione, dai periodi precedenti. \newline

\subsection{Progettazione architetturale (dal 2021-01-18 al 2021-03-07)}

Questo periodo porta alla milestone della Revisione di Progettazione.
Durante questo periodo si vogliono individuare le tecnologie da utilizzare per lo sviluppo del prodotto e l'architettura su cui basarlo.

\subsubsection{Pianificazione di macroperiodo}
\begin{table}[H]
	\rowcolors{2}{lightest-grayest}{white}
	\centering
	\renewcommand{\arraystretch}{1.5}
	\begin{tabular}{|c|p{10mm}|p{10mm}|p{10mm}|p{10mm}|p{10mm}|p{10mm}|}
		\hline
		\rowcolor{lighter-grayer}
		Attività & Re & Am & An & Pt & Pm & Ve \\ \hline
		Aggiornamento PdQ & \cellcolor{white}  & - & - & - & - & 20 \\
		Aggiornamento Adr & \cellcolor{white} & - & 40 & - & - & 2 \\
		Aggiornamento NdP & \multirow{-3}*{\cellcolor{white}2} & 6 & - & - & - & 2 \\ \hline
		Aggiornamento PdP     & 10	& - & - & - & - & 2  \\ \hline
		Technology baseline   & 1	& -  & - & 20 & - & -   \\ \hline
		Proof of concept      & 1	& - & - & 40 & - & -   \\ \hline
		Attività accessorie   & 2& 5 & 1  & 2  & - & 4  \\ \hline
		Comunicazione con docenti               & 2& - & - & - & - & -   \\ \hline
		Comunicazione con proponente            & 2& - & - & - & - & -   \\ \hline
		Presentazione         & 1& 1 & 1  & 1  & - & 1 \\
		\hline
	\end{tabular}
	\caption{ Pianificazione riguardante il periodo di Progettazione architetturale\\}
\end{table}

\subsubsection{Pianificazione di microperiodo}
\indent La pianificazione settimanale che segue riporta la suddivisione oraria tra i ruoli ma non tra i componenti. Si è deciso di escludere questa informazione dal documento, anche se essa è presente nello strumento utilizzato, per rendere la lettura più semplice.

\paragraph{Microperiodo 1}
\begin{table}[H]
	\rowcolors{2}{lightest-grayest}{white}
	\centering
	\renewcommand{\arraystretch}{1.5}
	\begin{tabular}{|c|p{10mm}|p{10mm}|p{10mm}|p{10mm}|p{10mm}|p{10mm}|}
		\hline
		\rowcolor{lighter-grayer}
		Attività & Re & Am & An & Pt & Pm & Ve \\ \hline
		Aggiornamento PdQ & \cellcolor{white} & \cellcolor{white} & \cellcolor{white} & \cellcolor{white} & \cellcolor{white} & \cellcolor{white} \\
		Aggiornamento AdR & \cellcolor{white} & \cellcolor{white} & \cellcolor{white} & \cellcolor{white} & \cellcolor{white} & \cellcolor{white} \\
		Aggiornamento NdP & \cellcolor{white} & \cellcolor{white} & \cellcolor{white} & \cellcolor{white} & \cellcolor{white} & \cellcolor{white} \\
		Aggiornamento PdP & \multirow{-4}*{\cellcolor{white}2} & \multirow{-4}*{\cellcolor{white}1} & \multirow{-4}*{\cellcolor{white}8} & \multirow{-4}*{\cellcolor{white}-} & \multirow{-4}*{\cellcolor{white}-} & \multirow{-4}*{\cellcolor{white}5} \\
		\hline
	\end{tabular}
	\caption{ Pianificazione riguardante il 1° microperiodo\\}
\end{table}


\paragraph{Microperiodo 2}
\begin{table}[H]
	\rowcolors{2}{lightest-grayest}{white}
	\centering
	\renewcommand{\arraystretch}{1.5}
	\begin{tabular}{|c|p{10mm}|p{10mm}|p{10mm}|p{10mm}|p{10mm}|p{10mm}|}
		\hline
		\rowcolor{lighter-grayer}
		Attività & Re & Am & An & Pt & Pm & Ve \\ \hline
		Aggiornamento PdQ & \cellcolor{white} & \cellcolor{white} & \cellcolor{white} & \cellcolor{white} & \cellcolor{white} & \cellcolor{white} \\
		Aggiornamento AdR & \cellcolor{white} & \cellcolor{white} & \cellcolor{white} & \cellcolor{white} & \cellcolor{white} & \cellcolor{white} \\
		Aggiornamento NdP & \cellcolor{white} & \cellcolor{white} & \cellcolor{white} & \cellcolor{white} & \cellcolor{white} & \cellcolor{white} \\
		Aggiornamento PdP & \multirow{-4}*{\cellcolor{white}2} & \multirow{-4}*{\cellcolor{white}1} & \multirow{-4}*{\cellcolor{white}8} & \multirow{-4}*{\cellcolor{white}-} & \multirow{-4}*{\cellcolor{white}-} & \multirow{-4}*{\cellcolor{white}5} \\
		\hline
		Technology baseline & - & - & - & 20 & - & - \\
		\hline
	\end{tabular}
	\caption*{\textbf{Tabella 3}: Pianificazione riguardante il 2° microperiodo\\}
\end{table}

\paragraph{Microperiodo 3}
\begin{table}[H]
	\rowcolors{2}{lightest-grayest}{white}
	\centering
	\renewcommand{\arraystretch}{1.5}
	\begin{tabular}{|c|p{10mm}|p{10mm}|p{10mm}|p{10mm}|p{10mm}|p{10mm}|}
		\hline
		\rowcolor{lighter-grayer}
		Attività & Re & Am & An & Pt & Pm & Ve \\ \hline
		Aggiornamento PdQ & \cellcolor{white} & \cellcolor{white} & \cellcolor{white} & \cellcolor{white} & \cellcolor{white} & \cellcolor{white} \\
		Aggiornamento AdR & \cellcolor{white} & \cellcolor{white} & \cellcolor{white} & \cellcolor{white} & \cellcolor{white} & \cellcolor{white} \\
		Aggiornamento NdP & \cellcolor{white} & \cellcolor{white} & \cellcolor{white} & \cellcolor{white} & \cellcolor{white} & \cellcolor{white} \\
		Aggiornamento PdP & \multirow{-4}*{\cellcolor{white}2} & \multirow{-4}*{\cellcolor{white}1} & \multirow{-4}*{\cellcolor{white}8} & \multirow{-4}*{\cellcolor{white}-} & \multirow{-4}*{\cellcolor{white}-} & \multirow{-4}*{\cellcolor{white}5} \\
		\hline
		Comunicazione con docenti & 1 & - & - & - & - & - \\
		\hline
		Proof of Concept & - & - & - & 20 & - & - \\
		\hline
	\end{tabular}
	\caption*{\textbf{Tabella 4}: Pianificazione riguardante il 3° microperiodo\\}
\end{table}

\paragraph{Microperiodo 4}
\begin{table}[H]
	\rowcolors{2}{lightest-grayest}{white}
	\centering
	\renewcommand{\arraystretch}{1.5}
	\begin{tabular}{|c|p{10mm}|p{10mm}|p{10mm}|p{10mm}|p{10mm}|p{10mm}|}
		\hline
		\rowcolor{lighter-grayer}
		Attività & Re & Am & An & Pt & Pm & Ve \\ \hline
		Aggiornamento PdQ & \cellcolor{white} & \cellcolor{white} & \cellcolor{white} & \cellcolor{white} & \cellcolor{white} & \cellcolor{white} \\
		Aggiornamento AdR & \cellcolor{white} & \cellcolor{white} & \cellcolor{white} & \cellcolor{white} & \cellcolor{white} & \cellcolor{white} \\
		Aggiornamento NdP & \cellcolor{white} & \cellcolor{white} & \cellcolor{white} & \cellcolor{white} & \cellcolor{white} & \cellcolor{white} \\
		Aggiornamento PdP & \multirow{-4}*{\cellcolor{white}3} & \multirow{-4}*{\cellcolor{white}1} & \multirow{-4}*{\cellcolor{white}8} & \multirow{-4}*{\cellcolor{white}-} & \multirow{-4}*{\cellcolor{white}-} & \multirow{-4}*{\cellcolor{white}5} \\
		\hline
		Proof of Concept & - & - & - & 20 & - & - \\
		\hline
	\end{tabular}
	\caption{ Pianificazione riguardante il 4° microperiodo\\}
\end{table}


\paragraph{Microperiodo 5}
\begin{table}[H]
	\rowcolors{2}{lightest-grayest}{white}
	\centering
	\renewcommand{\arraystretch}{1.5}
	\begin{tabular}{|c|p{10mm}|p{10mm}|p{10mm}|p{10mm}|p{10mm}|p{10mm}|}
		\hline
		\rowcolor{lighter-grayer}
		Attività & Re & Am & An & Pt & Pm & Ve \\ \hline
		Aggiornamento PdQ & \cellcolor{white} & \cellcolor{white} & \cellcolor{white} & \cellcolor{white} & \cellcolor{white} & \cellcolor{white} \\
		Aggiornamento AdR & \cellcolor{white} & \cellcolor{white} & \cellcolor{white} & \cellcolor{white} & \cellcolor{white} & \cellcolor{white} \\
		Aggiornamento NdP & \cellcolor{white} & \cellcolor{white} & \cellcolor{white} & \cellcolor{white} & \cellcolor{white} & \cellcolor{white} \\
		Aggiornamento PdP & \multirow{-4}*{\cellcolor{white}3} & \multirow{-4}*{\cellcolor{white}2} & \multirow{-4}*{\cellcolor{white}8} & \multirow{-4}*{\cellcolor{white}-} & \multirow{-4}*{\cellcolor{white}-} & \multirow{-4}*{\cellcolor{white}6} \\
		\hline
		Presentazione & 1 & 1 & 1 & 1 & - & 1 \\
		\hline
		Attività accessorie & 2 & 5 & 1 & 2 & - & 4 \\
		\hline
	\end{tabular}
	\caption{ Pianificazione riguardante il 5° microperiodo\\}
\end{table}


\subsection{Progettazione di dettaglio e codifica (dal 2021-03-08 al 2021-05-09)}

Questo periodo porta alla milestone della Revisione di Qualifica.

\subsubsection{Pianificazione di macroperiodo}
\begin{table}[H]
	\rowcolors{2}{lightest-grayest}{white}
	\centering
	\renewcommand{\arraystretch}{1.5}
	\begin{tabular}{|c|p{8mm}|p{8mm}|p{8mm}|p{8mm}|p{8mm}|p{8mm}|}
		\hline
		\rowcolor{lighter-grayer}
		Attività & Re & Am & An & Pt & Pm & Ve \\ \hline
		Aggiornamento PdQ& 1 & - & - & - & - & 20 \\ \hline
		Aggiornamento AdR& 1 & - & 20 & - & - & 2  \\ \hline
		Aggiornamento NdP& 1 & 10 & - & - & - & 2  \\ \hline
		Aggiornamento PdP& 10& - & - & - & - & 2  \\ \hline
		Product baseline & 2 & - & - & 18 & - & 2  \\ \hline
		Codifica della struttura delle componenti     & 2 & - & - & - & 35 & 15 \\ \hline
		\begin{tabular}[x]{@{}c@{}}Codifica delle interazioni \\ tra le componenti interne\end{tabular}      & 2 & - & - & - & 35 & 15 \\ \hline
		Codifica delle interazioni con gli attori esterni  & 2 & - & - & - & 35 & 15 \\ \hline
		\begin{tabular}[x]{@{}c@{}}Codifica delle funzionalità secondarie \\ non ancora implementate\end{tabular} & 1 & - & - & - & 32 & 15 \\ \hline
		Stesura manuale utente& \cellcolor{white}  & 12 & - & - & - & 4  \\
		Stesura manuale manutentore & \multirow{-2}*{\cellcolor{white}1}  & 12 & - & - & - & 4  \\ \hline
		Attività accessorie    & 2 & 5  & 1  & 1  & 2  & 3  \\ \hline
		Comunicazione con docenti   & 2 & - & - & - & - & -   \\ \hline
		Comunicazione con proponente& 2 & - & - & - & - & -   \\ \hline
		Presentazione    & 1 & 1  & 1  & 1  & 1  & 1 \\
		\hline
	\end{tabular}
	\caption{ Pianificazione riguardante il periodo di Progettazione di dettaglio e codifica\\}
\end{table}

\subsubsection{Pianificazione di microperiodo}
\indent La pianificazione settimanale che segue riporta la suddivisione oraria tra i ruoli ma non tra i componenti. Si è deciso di escludere questa informazione dal documento, anche se essa è presente nello strumento utilizzato, per rendere la lettura più semplice.

\paragraph{Microperiodo 1}
\begin{table}[H]
	\rowcolors{2}{lightest-grayest}{white}
	\centering
	\renewcommand{\arraystretch}{1.5}
	\begin{tabular}{|c|p{10mm}|p{10mm}|p{10mm}|p{10mm}|p{10mm}|p{10mm}|}
		\hline
		\rowcolor{lighter-grayer}
		\textbf{Attività}                         & \textbf{Re} & \textbf{Am} & \textbf{An} & \textbf{Pt} & \textbf{Pm} & \textbf{Ve} \\ \hline
		Comunicazione con proponente              & 1           & -           & -           & -           & -           & -           \\ \hline
		PoC mobile                                & 2           & -           & -           & 16          & 12          & -           \\ \hline
		Configurazione analisi statica, CI, etc.. & 3           & -           & -           & -           & 7           & -           \\ \hline 
	\end{tabular}
	\caption{ Pianificazione riguardante il 1° microperiodo\\}
\end{table}

Obiettivi:
\begin{itemize}
	\item incontro con il proponente per l'illustrazione del PoC e per ricevere chiarimenti su Ethereum;
	\item implementazione funzionalità app mobile: login utente e interazione cellulare con tag NFC;
	\item configurazione repository per back-end, app-mobile e web-app.
\end{itemize}

\paragraph{Microperiodo 2}
\begin{table}[H]
	\rowcolors{2}{lightest-grayest}{white}
	\centering
	\renewcommand{\arraystretch}{1.5}
	\begin{tabular}{|c|p{10mm}|p{10mm}|p{10mm}|p{10mm}|p{10mm}|p{10mm}|}
		\hline
		\rowcolor{lighter-grayer}
		\textbf{Attività}                         & \textbf{Re} & \textbf{Am} & \textbf{An} & \textbf{Pt} & \textbf{Pm} & \textbf{Ve} \\ \hline
		
		Aggiornamento AdR                       & 1                                & -                                & \multicolumn{1}{r|}{12}          & -                                & -                                & 2                                \\ \hline
		Aggiornamento PdQ                       & 1                                & -                                & -                                & -                                & -                                & 7                                \\ \hline
		Aggiornamento PdP                       & 4                                & -                                & -                                & -                                & -                                & 1                                \\ \hline
		Aggiornamento NdP                       & 1                                & \multicolumn{1}{r|}{1}           & -                                & -                                & -                                & 1                                \\ \hline
		Comunicazione con docenti               & 1                                & -                                & -                                & -                                & -                                & \multicolumn{1}{l|}{-}           \\ \hline
		
	\end{tabular}
	\caption{ Pianificazione riguardante il 2° microperiodo\\}
\end{table}

Obiettivi:
\begin{itemize}
	\item correzione della documentazione in seguito alle osservazioni effettuate dai docenti durante la revisione. 
\end{itemize}

\paragraph{Microperiodo 3}
\begin{table}[H]
	\rowcolors{2}{lightest-grayest}{white}
	\centering
	\renewcommand{\arraystretch}{1.5}
	\begin{tabular}{|c|p{10mm}|p{10mm}|p{10mm}|p{10mm}|p{10mm}|p{10mm}|}
		\hline
		\rowcolor{lighter-grayer}
		\textbf{Attività}                         & \textbf{Re} & \textbf{Am} & \textbf{An} & \textbf{Pt} & \textbf{Pm} & \textbf{Ve} \\ \hline
		
		Configurazioni ausiliari & 1           &     -        &       -      &       -      & 5           & 2           \\ \hline
		Comunicazione con docenti  & 1           &     -        &      -       &     -        &      -       &        -     \\ \hline     
		
	\end{tabular}
	\caption{ Pianificazione riguardante il 3° microperiodo\\}
\end{table}

Obiettivi:
\begin{itemize}
	\item configurazione server di Imola Informatica;
	\item configurazione parte Docker;
	\item configurazione database.
\end{itemize}

\paragraph{Microperiodo 4}
\begin{table}[H]
	\rowcolors{2}{lightest-grayest}{white}
	\centering
	\renewcommand{\arraystretch}{1.5}
	\begin{tabular}{|c|p{10mm}|p{10mm}|p{10mm}|p{10mm}|p{10mm}|p{10mm}|}
		\hline
		\rowcolor{lighter-grayer}
		\textbf{Attività}                         & \textbf{Re} & \textbf{Am} & \textbf{An} & \textbf{Pt} & \textbf{Pm} & \textbf{Ve} \\ \hline
		
		Product Baseline                                        & 1           &     -        &    -         & 14          &    -         & 2           \\ \hline
		Codifica della struttura delle componenti & 1           &     -        &    -         &      -       & 15          & 7           \\ \hline
		
		
	\end{tabular}
	\caption{ Pianificazione riguardante il 4° microperiodo\\}
\end{table}

Obiettivi:
\begin{itemize}
	\item predisposizione diagrammi delle classi e di sequenza delle architetture adottate; 
	\item app mobile: implementazione login, visualizzazione prenotazioni, scansione tag NFC;
	\item web app: implementazione login, logout, visualizzazione stanze e postazioni;
	\item backend: implementazione login, gestione dell'inserimento, la modifica, la cancellazione di postazioni, utenti, stanze.
\end{itemize}

\paragraph{Microperiodo 5}
\begin{table}[H]
	\rowcolors{2}{lightest-grayest}{white}
	\centering
	\renewcommand{\arraystretch}{1.5}
	\begin{tabular}{|c|p{10mm}|p{10mm}|p{10mm}|p{10mm}|p{10mm}|p{10mm}|}
		\hline
		\rowcolor{lighter-grayer}
		\textbf{Attività}                         & \textbf{Re} & \textbf{Am} & \textbf{An} & \textbf{Pt} & \textbf{Pm} & \textbf{Ve} \\ \hline
		
		Aggiornamento AdR                         & 1           &     -        & 8           &      -       &      -       & 1           \\ \hline
		PB                                        & 1           &      -       &    -         & 5           &      -       & 2           \\ \hline
		Codifica della struttura delle componenti & 1           &       -      &   -          &      -       & 20          & 8           \\ \hline
		
	\end{tabular}
	\caption{ Pianificazione riguardante il 5° microperiodo\\}
\end{table}

Obiettivi:
\begin{itemize}
	\item correzione dei diagrammi dell'AdR secondo le ultime osservazioni del prof. Cardin;
	\item app mobile: implementazione di igienizzazione della postazione e gestione delle prenotazioni;
	\item web app: implementazione di aggiunta / modifica / eliminazione di stanze e postazioni, visualizzazione / aggiunta / modifica / eliminazione delle credenziali;
	\item backend: reperimento dati relativi alla postazione dalla scansione di un tag, gestione dell'igienizzazione della postazione, reperimento delle prenotazioni di un utente.
\end{itemize}

\paragraph{Microperiodo 6}
\begin{table}[H]
	\rowcolors{2}{lightest-grayest}{white}
	\centering
	\renewcommand{\arraystretch}{1.5}
	\begin{tabular}{|c|p{10mm}|p{8mm}|p{8mm}|p{8mm}|p{8mm}|p{8mm}|}
		\hline
		\rowcolor{lighter-grayer}
		\textbf{Attività}                         & \textbf{Re} & \textbf{Am} & \textbf{An} & \textbf{Pt} & \textbf{Pm} & \textbf{Ve} \\ \hline
		
		Codifica delle interazioni tra le componenti interne & 2                                &       -                           &        -                          &             -                     & 35                               & 15                               \\ \hline
		Codifica delle interazioni con gli attori esterni    & 1                                &      -                            &         -                         &                -                  & 15                               & 6                                \\ \hline
		
	\end{tabular}
	\caption{ Pianificazione riguardante il 6° microperiodo\\}
\end{table}

Obiettivi:
\begin{itemize}
	\item gestione interazione backend - web app;
	\item gestione interazione backend - app mobile;
	\item gestione interazione backend - Ethereum.
\end{itemize}

\paragraph{Microperiodo 7}
\begin{table}[H]
	\rowcolors{2}{lightest-grayest}{white}
	\centering
	\renewcommand{\arraystretch}{1.5}
	\begin{tabular}{|c|p{10mm}|p{10mm}|p{10mm}|p{10mm}|p{10mm}|p{10mm}|}
		\hline
		\rowcolor{lighter-grayer}
		\textbf{Attività}                         & \textbf{Re} & \textbf{Am} & \textbf{An} & \textbf{Pt} & \textbf{Pm} & \textbf{Ve} \\ \hline
		
		Aggiornamento PdQ                                                                                & 1           &      -       &    -         &       -      &     -        & 8           \\ \hline
		Aggiornamento PdP                                                                                & 6           &     -        &      -       &     -        &     -        & 1           \\ \hline
		Codifica delle interazioni con gli attori esterni                                                & 1           &     -        &    -        &     -        & 20          & 8           \\ \hline
		Manuale utente      & 1           & 12          &     -        &       -      &        -     & 4           \\ \hline
		Manuale manutentore &     -        & 12          &    -         &     -        &      -       & 4           \\ \hline

		
	\end{tabular}
	\caption{ Pianificazione riguardante il 7° microperiodo\\}
\end{table}

Obiettivi:
\begin{itemize}
	\item PdP: aggiornamento grafici, consuntivo, pianificazione di macroperiodo;
	\item PdQ: aggiornamento grafici;
	\item manuali: stesura manuale riguardante le parti / funzionalità finora implementate.
\end{itemize}

\paragraph{Altro}
Per evitare di frazionare eccessivamente per singolo microperiodo le seguenti attività sono riportate cumulativamente.

\begin{table}[H]
	\rowcolors{2}{lightest-grayest}{white}
	\centering
	\renewcommand{\arraystretch}{1.5}
	\begin{tabular}{|c|p{10mm}|p{10mm}|p{10mm}|p{10mm}|p{10mm}|p{10mm}|}
		\hline
		\rowcolor{lighter-grayer}
		\textbf{Attività}                         & \textbf{Re} & \textbf{Am} & \textbf{An} & \textbf{Pt} & \textbf{Pm} & \textbf{Ve} \\ \hline
		
		Presentazione       & 1           & 1           & 1           & 1           & 1           & 1           \\ \hline
		Attività accessorie & 2           & 5           & 1           & 1           & 2           & 3           \\ \hline
		
	\end{tabular}
	\caption{ Pianificazione di dettaglio\\}
\end{table}

Obiettivi:
\begin{itemize}
	\item produzione verbali;
	\item partecipazione alle riunioni;
	\item coordinamento delle attività da parte del responsabile;
	\item predisposizione della presentazione per la PB e la RQ.
\end{itemize}

\subsection{Validazione e collaudo (dal 2021-05-17 al 2021-05-31)}

\subsubsection{Pianificazione di macroperiodo}
\begin{table}[H]
	\rowcolors{2}{lightest-grayest}{white}
	\centering
	\renewcommand{\arraystretch}{1.5}
	\begin{tabular}{|c|p{10mm}|p{10mm}|p{10mm}|p{10mm}|p{10mm}|p{10mm}|}
		\hline
		\rowcolor{lighter-grayer}
		Attività & Re & Am & An & Pt & Pm & Ve \\ \hline
		Aggiornamento PdQ                                                           & 1           &    -         &      -       &     -        &     -        & 10          \\ \hline
		Aggiornamento NdP                                                           & 1           & 2           &    -         &     -        &      -       & 1           \\ \hline
		Aggiornamento PdP                                                           & 10          &     -        &    -         &     -        &    -         & 4           \\ \hline
		Validazione                                                                 & 2           &   -          &       -      & 12          & 21          & 20          \\ \hline
		Collaudo                                                                    & 2           &     -        &     -        & 11          & 20          & 20          \\ \hline
		Attività accessorie & 2           & 7           &             & 1           & 1           & 3           \\ \hline
		Comunicazione con docenti                                                   & 2           &     -        &      -       &     -        &      -       &     -        \\ \hline
		Comunicazione con proponente                                                & 2           &    -         &       -      &   -          &     -        &    -         \\ \hline
		Presentazione                                                               & 1           & 1           &    -         & 1           & 1           & 1           \\ \hline
		Codifica funzionalità secondarie                                            & 1           &    -         &      -       &    -        & 26          & 5           \\ \hline
		Manuale utente                                                              & 1           & 9           &     -        &    -         &     -        & 2           \\ \hline
		Manuale manutentore                                                         & 1           & 9           &      -       &      -       &     -        & 2           \\ \hline
	\end{tabular}
	\caption{ Pianificazione riguardante il periodo di Validazione e collaudo\\}
\end{table}

\subsubsection{Pianificazione di microperiodo}
\indent La pianificazione settimanale che segue riporta la suddivisione oraria tra i ruoli ma non tra i componenti. Si è deciso di escludere questa informazione dal documento, anche se essa è presente nello strumento utilizzato, per rendere la lettura più semplice.

\paragraph{Microperiodo 1}
\begin{table}[H]
	\rowcolors{2}{lightest-grayest}{white}
	\centering
	\renewcommand{\arraystretch}{1.5}
	\begin{tabular}{|c|p{10mm}|p{10mm}|p{10mm}|p{10mm}|p{10mm}|p{10mm}|}
		\hline
		\rowcolor{lighter-grayer}
		\textbf{Attività}                         & \textbf{Re} & \textbf{Am} & \textbf{An} & \textbf{Pt} & \textbf{Pm} & \textbf{Ve} \\ \hline
		Verbale interno        & - & 1 & - & - & -  & 1 \\ \hline
		Comunicazione con proponente & -           & -           & -           & -           & 1           & -           \\ \hline
		Sviluppo applicazione  & - & - & - & - & 81 & - \\ \hline
		Coordinamento attività & 1 & - & - & - & -  & - \\ \hline
	\end{tabular}
	\caption{ Pianificazione riguardante il 1° microperiodo\\}
\end{table}

Obiettivi di microperiodo:
\begin{itemize}
	\item Si intende implementare le seguenti funzionalità della web-app:
	\begin{itemize}
		\item la visualizzazione delle postazioni e delle stanze tramite una griglia/scacchiera; 
		\item la colorazione delle postazioni in base allo stato;
		\item mostrare il numero di occupanti per stanza.
	\end{itemize}
\item Relativamente all'app-mobile si intende:
\begin{itemize}
	\item effettuare il refactoring del codice esistente;
	\item garantire la visualizzazione della guida utente.
\end{itemize}
\item Riguardo al backend gli obiettivi sono:
\begin{itemize}
	\item la stesura degli unit test del codice esistente;
	\item la gestione della comunicazione fra backend e blockchain tramite web3.
\end{itemize}
\item creazione di un video-demo per aggiornare Imola Informatica sullo stato di avanzamento dei lavori.
\end{itemize}


\paragraph{Microperiodo 2}
\begin{table}[H]
	\rowcolors{2}{lightest-grayest}{white}
	\centering
	\renewcommand{\arraystretch}{1.5}
	\begin{tabular}{|c|p{10mm}|p{10mm}|p{10mm}|p{10mm}|p{10mm}|p{10mm}|}
		\hline
		\rowcolor{lighter-grayer}
		\textbf{Attività}                         & \textbf{Re} & \textbf{Am} & \textbf{An} & \textbf{Pt} & \textbf{Pm} & \textbf{Ve} \\ \hline
		Verbale interno                                   & - & 1 & - & - & -  & 1 \\ \hline
		Comunicazione con proponente					  & - & - & - & - & 1  & - \\ \hline
		Sviluppo applicazione                             & - & - & - & - & 80 & - \\ \hline
		Coordinamento attività                            & 1 & - & - & - & -  & - \\ \hline
	\end{tabular}
	\caption{ Pianificazione riguardante il 2° microperiodo\\}
\end{table}

Obiettivi di microperiodo:
\begin{itemize}
	\item Si intende raggiungere i seguenti obiettivi per la la web-app:
	\begin{itemize}
		\item impostazione delle postazioni nello stato guasto; 
		\item impostazione inacessibilità delle stanze;
		\item visualizzazione della guida;
		\item disabilitazione utente;
		\item scrittura test.
	\end{itemize}
	\item Relativamente all'app-mobile si intende implementare le seguenti funzionalità:
	\begin{itemize}
		\item visualizzazione e gestione delle prenotazioni;
		\item occupazione delle postazioni;
		\item funzionalità dell'addetto alle pulizie (guida, visualizzazione stanze e postazioni da igienizzare e loro igienizzazione).
	\end{itemize}
	\item Riguardo al backend gli obiettivi sono:
	\begin{itemize}
		\item generazione report, calcolo hash e salvataggio nella blockchain;
		\item implementazione delle chiamate API per consentire le funzionalità sopra elencate per la web-app e l'app-mobile.
	\end{itemize}
\item creazione di un video-demo per aggiornare Imola Informatica sullo stato di avanzamento dei lavori.
\end{itemize}

\paragraph{Microperiodo 3}
\begin{table}[H]
	\rowcolors{2}{lightest-grayest}{white}
	\centering
	\renewcommand{\arraystretch}{1.5}
	\begin{tabular}{|c|p{10mm}|p{10mm}|p{10mm}|p{10mm}|p{10mm}|p{10mm}|}
		\hline
		\rowcolor{lighter-grayer}
		\textbf{Attività}                         & \textbf{Re} & \textbf{Am} & \textbf{An} & \textbf{Pt} & \textbf{Pm} & \textbf{Ve} \\ \hline
		Verbale interno              & - & 1 & - & - & -  & 1  \\ \hline
		Comunicazione con proponente & -           & 1           & -           & -           & -           & 1           \\ \hline
		aggiornamento documentazione & - & 4 & - & - & -  & -  \\ \hline
		Validazione e collaudo       & - & - & - & - & 67 & 14 \\ \hline
		Coordinamento attività       & 1 & - & - & - & -  & -  \\ \hline
	\end{tabular}
	\caption{ Pianificazione riguardante il 3° microperiodo\\}
\end{table}

Obiettivi di microperiodo:

\begin{itemize}
	\item Per la web-app si intende:
	\begin{itemize}
		\item concludere la stesura dei test;
		\item creazione dell'interfaccia per la visualizzazione dei report.
	\end{itemize}
	\item Relativamente all'app-mobile si intende:
	\begin{itemize}
		\item consentire la possibilità di fissare una durata per un'occupazione non prevista;
		\item concludere la stesura dei test.
	\end{itemize}
	\item Riguardo al backend gli obiettivi sono:
	\begin{itemize}
		\item concludere la stesura dei test;
		\item implementazione delle chiamate API per consentire le funzionalità sopra elencate per la web-app e l'app-mobile.
	\end{itemize}
\item aggiornamento della documentazione richiesta per la RA;
\item demo live con Imola Informatica.
\end{itemize}