\section{Pianificazione}
Lo sviluppo del progetto è diviso nelle seguenti fasi consecutive.
\begin{enumerate}
	\item Analisi dei requisiti
\end{enumerate}
Ogni fase è costituita da periodi, nei quali sono raggruppate le attività da svolgere.

\subsection{Analisi dei requisiti (dal 2020-10-22 al 2021-01-10)}

\subsubsection{Ruoli attivi}
\begin{itemize}
	\item Responsabile
	\item Amministratore
	\item Analista
	\item Verificatore
\end{itemize}

\subsubsection{Periodi e attività}

\paragraph{Primo periodo (dal 2020-10-22 al 2020-11-09)}
Questo periodo comincia con la costituzione del gruppo di progetto. Serve dunque a porre le basi per una comunicazione e collaborazione efficace ed efficiente. Vi si svolgono inoltre attività di ricerca superficiale sui capitolati.

\begin{itemize}
	\item Configurazione degli strumenti collaborativi basilari: comprende la scelta e la configurazione dei mezzi per la comunicazione e il lavoro collaborativo
	\item Analisi superficiale dei capitolati
	
\end{itemize}

\paragraph{Secondo Periodo (dal 2020-11-10 al 2020-12-13)}
Gli obiettivi raggiunti in questo periodo sono il consolidamento degli strumenti di collaborazione e l'approfondimento dei capitolati.
\begin{itemize}
	\item Configurazione del \glock{repository}: questa attività comprende la creazione di un repository per la condivisone e il versionamento dei prodotti, ma anche la sua configurazione per la verifica automatica della loro validità.
	\item Raccolta di informazioni dai seminari: stesura di appunti sui seminari, da poter facilmente consultare quando necessario.
	\item Studio dei progetti degli anni passati: questa attività ha il fine di individuare gli errori più comuni e, al contrario, le tecniche da prendere come esempio.
	\item Bozza dello studio di fattibilità: questo studio viene eseguito per tutti i capitolati
	\item Bozza delle norme di progetto
\end{itemize}

\paragraph{Terzo Periodo (dal 2020-12-14 al 2020-01-03)}
Questo periodo segue la scelta del capitolato da affrontare. E' dedicato alla sua analisi approfondita e alla stesura della documentazione. 
\begin{itemize}
	\item Stesura delle norme di progetto
	\item Stesura dello studio di fattibilità
	\item Stesura del piano di progetto
	\item Stesura del glossario
	\item Analisi dei requisiti e stesura dell'omonimo documento
	\item Stesura del piano di qualifica
	\item Verifica della documentazione
	\item Stesura della lettera di presentazione
\end{itemize}

\paragraph{Quarto Periodo (dal 2020-01-04 al 2020-01-10)}
Questo periodo è l'ultimo della fase di analisi dei requisiti. Si concentra quindi sulla preparazione dell'esposizione dei prodotti dei 3 periodi precedenti.
\begin{itemize}
	\item Preparazione dell'esposizione
	\item Verifica dell'esposizione
\end{itemize}


\begin{landscape}
	\begin{figure}[H]
		\centering
		\includegraphics[width=\linewidth]{res/images/ganttFaseAnalisi.png}
		\caption{Grafico di Gantt della fase di analisi dei requisiti}
		\label{fig:Gantt Analisi dei requisiti}
	\end{figure}
\end{landscape}

