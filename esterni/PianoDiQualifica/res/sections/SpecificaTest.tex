
\section{Specifica dei Test}
I test permettono di verificare sia la correttezza delle parti di programma sviluppate, sia che tutti gli aspetti del progetto siano implementati e corretti.
Per assicurare la qualità del software prodotto, il gruppo \textit{DPCM 2077} utilizza come modello di sviluppo il \glock{Modello a V}.
\\
Per la classificazione dei test si fa riferimento alla sezioni Verifica\footnote{Sezione §3.4} e Validazione\footnote{Sezione §3.5} del documento \dext{Norme di Progetto v. 1.0.0}.
	\subsection{Tipi di test}
	Vengono individuate quattro tipologie di test:
	\begin{itemize}
		\item \textbf{Test di Sistema [TS]}
		\item \textbf{Test di Accettazione [TA]}
		\item \textbf{Test di Integrazione [TI]}
		\item \textbf{Test di Unità [TU]}
	\end{itemize}
	\subsection{Test di Sistema}
	\begin{center}
		\rowcolors{2}{white}{lightest-grayest}
		\begin{longtable}{|c|C{10cm}|c|}
			\hline
			\rowcolor{lighter-grayer}
			\textbf{Codice} & \textbf{Descrizione} & \textbf{Stato}  \\ 
						
			\hline
			\endhead
			
			\hline
			% Test di sistema per i requisiti funzionali %
			TS1F1 & desc & NI \\
			\hline
			TS1F1 & desc & NI \\			
			\hline
			% Test di sistema per i requisiti prestazionali %
			% Test di sistema per i requisiti di qualità %
			% Test di sistema per i requisiti di vincolo %		
			\hiderowcolors
			\caption{Tabella dei test di sistema}		
		\end{longtable}	
	\end{center}

	\subsection{Test di Accettazione}
	Il test di accettazione comprende l'esecuzione dei test di sistema con la partecipazione dei committenti. Dunque, è in tale circostanza che si eseguiranno i test di accettazione. 
	\subsection{Test di Integrazione}
	Le specifiche di questi test verranno scritte successivamente utilizzando il Modello a V.
	\subsection{Test di Unità}
	Le specifiche di questi test verranno scritte successivamente utilizzando il Modello a V.