
\section{Specifica dei test}
I test permettono di verificare sia la correttezza delle parti di programma sviluppate, sia che tutti gli aspetti del progetto siano implementati e corretti.
Per assicurare la qualità del software prodotto, il gruppo \textit{DPCM 2077} utilizza come modello di sviluppo il \glock{Modello a V}.
\\
Per la classificazione dei test si fa riferimento alle sezioni Verifica\footnote{Sezione §3.4} e Validazione\footnote{Sezione §3.5} del documento \dext{Norme di progetto v. 2.0.0}.
	\subsection{Tipi di test}
	Vengono individuate tre tipologie di test:
	\begin{itemize}
		\item \textbf{Test di Sistema [TS]};
		\item \textbf{Test di Accettazione [TA]};
		\item \textbf{Test di Unità [TU]}.
	\end{itemize}
	\subsection{Test di sistema}
	\begin{center}
		\rowcolors{2}{white}{lightest-grayest}
		\begin{longtable}{|c|p{10cm}|c|}
			\hline
			\rowcolor{lighter-grayer}
			\textbf{Codice} & \textbf{Descrizione} & \textbf{Stato}  \\ 
						
			\hline
			\endhead
			
			\hline
			% Test di sistema per i requisiti funzionali %
			TS2F1 & L'utente deve poter accedere alla guida per il login e per l'applicazione. & I \\
			\hline
			TS1F2 &  L'utente non ancora autenticato deve poter autenticarsi nel web server inserendo come credenziali nome utente e password.  & I \\			
			\hline
			TS1F3 & L'utente non ancora autenticato nel web server deve ricevere un messaggio di errore nel caso abbia inserito le credenziali errate. & I \\			
			\hline
			TS1F4 & L'utente non ancora autenticato nel web server deve ricevere un messaggio di errore nel caso l'account sia disabilitato. & I \\			
			\hline
			TS1F5 & L'utente non ancora autenticato deve poter autenticarsi nell'applicazione mobile inserendo come credenziali nome utente e password. & I \\			
			\hline
			TS1F6 & L'utente non ancora autenticato nell'applicazione mobile deve ricevere un messaggio di errore nel caso abbia inserito le credenziali errate. & I \\			
			\hline
			TS1F7 & L'utente non ancora autenticato nell'applicazione mobile deve ricevere un messaggio di errore nel caso l'account sia disabilitato. & I \\			
			\hline
			TS1F8 & L'utente autenticato come amministratore deve potersi disconnettere dal web server. & I \\			
			\hline
			TS1F9 & L'utente autenticato come dipendente deve potersi disconnettere dall'applicazione mobile. & I \\			
			\hline			
			TS1F10 & L'utente autenticato come addetto alle pulizie deve potersi disconnettere dall'applicazione mobile. & I \\			
			\hline
			TS1F11 & L'utente autenticato come amministratore deve poter accedere ad una giuda completa delle funzionalità. & I \\		
			\hline
			TS1F12 & L'utente autenticato come dipendente deve poter accedere ad una giuda completa delle funzionalità. & I \\			
			\hline
			TS1F13 & L'utente autenticato come addetto alle pulizie deve poter accedere ad una giuda completa delle funzionalità. & I \\
			\hline
			TS1F14 & L'amministratore può visualizzare le stanze e le postazioni salvate. & I \\			
			\hline
			TS1F14.1 & L'amministratore può visualizzare le stanze e le postazioni in modo schematico. & I \\			
			\hline
			TS1F14.2 & Gli stati delle postazioni possono essere i seguenti:
			 \begin{itemize}
			 	\item libera e igienizzata;
			 	\item libera e non igienizzata;
			 	\item occupata;
			 	\item prenotata e igienizzata;
			 	\item prenotata e non igienizzata;
			 	\item guasta e igienizzata;
			 	\item guasta e non igienizzata.
			 \end{itemize}
			 & I \\			
			\hline
			TS1F14.3 & Per ogni stanza deve essere indicato il numero di occupanti attuali. & I \\			
			\hline
			TS1F15 & L'amministratore può visualizzare il calendario delle prenotazioni delle postazioni. & NI \\			
			\hline
			TS1F16 & L'amministratore può aggiungere una stanza al sistema e assegnarle un nome. & I \\			
			\hline
			TS1F17 & L'amministratore non può creare una stanza con un nome già in utilizzo o con delle dimensioni impossibili e deve essere avvisato con un messaggio di errore. & I \\			
			\hline
			TS1F18 & L'amministratore può eliminare una stanza dal sistema. & I \\			
			\hline
			TS1F19 & L'amministratore può modificare il nome di una stanza. & I \\			
			\hline
			TS1F20 & L'amministratore può impostare una stanza come inaccessibile per un determinato periodo di tempo, con data inizio e fine arbitraria. & I \\			
			\hline
			TS1F21 & L'amministratore può aggiungere una postazione in una stanza, specificando:
			\begin{itemize}
				\item codice della postazione;
				\item codice tag NFC che la identifica;
				\item posizione all'interno della stanza.
			\end{itemize}
			& I \\					
			\hline				
			TS1F22 & L'amministratore non può modificare o creare una postazione utilizzando un "codice postazione" già in utilizzo e deve essere avvisato con un messaggio di errore. & I \\			
			\hline			
			TS1F23 & L'amministratore non può modificare o creare una postazione utilizzando un "codice tag" già in utilizzo e deve essere avvisato con un messaggio di errore. & I \\			
			\hline			
			TS1F24 & L'amministratore non può modificare o creare una postazione con un nome già in utilizzo e deve essere avvisato con un messaggio di errore. & I \\			
			\hline
			TS1F25 & L'amministratore può eliminare una postazione. & I \\	
			\hline	
			TS1F26 & L'amministratore può modificare i dati e la posizione di una postazione. & I \\			
			\hline	
			TS1F27 & L'amministratore può visualizzare una lista delle credenziali di tutti gli utenti. & I \\	
			\hline
			TS1F28 & L'amministratore può creare nuove credenziali. Le informazioni contenute nel profilo di un utente sono:
			\begin{itemize}
				\item nome;
				\item cognome;
				\item nome utente;
				\item password;
				\item email.
			\end{itemize}
			Tra queste le informazioni necessarie per l’accesso sono password e nome utente. & I \\	
			\hline
			TS1F29 & L’amministratore può modificare le credenziali degli utenti del sistema. & I \\	
			\hline
			TS1F30 & L’amministratore può eliminare le credenziali degli utenti del sistema. & I \\	
			\hline
			TS1F31 & L'amministratore può vedere lo storico delle occupazioni delle postazioni. & I \\	
			\hline
			TS1F32 & L'amministratore deve poter effettuare una ricerca delle postazioni occupate da un utente, specificando il periodo all'interno del quale l'utente occupava la postazione e ricevere un report. & NI \\	
			\hline
			TS1F33 & L'amministratore deve poter ricevere un report contenente delle occupazioni di postazioni da parte di un utente specifico, specificando date e orari di inizio e fine delle occupazioni, codici delle postazioni e, opzionalmente, ore trascorse alla postazione. & I \\	
			\hline
			TS1F34 & L'amministratore deve vedere lo storico delle occupazioni di una postazione specifica, con data e orari di inizio e fine delle occupazioni e nomi e cognomi dei dipendenti, ricevendo un report. & I \\			
			\hline
			TS1F35 & L'amministratore può ottenere un report delle
			occupazioni e delle igienizzazioni in base alla sua ricerca. & I \\		
			\hline		
			TS2F36 & L'amministratore può scaricare il report degli utenti che hanno condiviso la stanza con un determinato utente. & NI \\		
			\hline
			TS2F37 & L'amministrazione può ricevere un report con lo storico delle igienizzazioni di tutte le postazioni, con data e ora in cui sono avvenute, nome e cognome di chi le ha eseguite e il loro ruolo. & NI \\	
			\hline
			TS1F38 & L'amministratore può scaricare il report che sta visualizzando in un formato leggibile. & I \\	
			\hline
			TS2F39 & L'amministratore può scaricare il report che sta visualizzando in formato PDF. & NI \\	
			\hline			
			TS1F40 & Il dipendente deve poter scansionare un tag NFC presente in una postazione. & I \\	
			\hline
			TS1F41 & Il dipendente deve poter vedere lo stato di una postazione tramite la scansione del tag NFC. & I \\	
			\hline
			TS1F42 & Il dipendente deve poter segnare la propria presenza su una postazione in tempo reale. & I \\	
			\hline			
			TS1F42.1 & Il dipendente deve poter segnare la propria presenza su una postazione in tempo reale, tramite la scansione del tag NFC. & I \\	
			\hline
			TS1F42.2 & Il dipendente deve poter registrare la propria presenza, di inizio e fine occupazione, su una postazione. & NI \\		
			\hline					
			TS1F43 & Se lo smartphone non rileva il tag NFC per un tempo maggiore o uguale a 30 minuti avviene la disdetta automatica della prenotazione per il tempo restante. & NI \\	
			\hline				
			TS1F44 & L’utente viene avvisato della disdetta automatica di una sua prenotazione se sposta il proprio smartphone dal tag NFC per più di 30 minuti. & NI \\	
			\hline				
			TS1F45 & Il dipendente visualizza un messaggio di errore se prova ad occupare una postazione non igienizzata. & I \\	
			\hline
			TS1F46 & Il dipendente deve poter visualizzare un messaggio di inizio prenotazione, 30 minuti prima che essa avvenga. & NI \\	
			\hline
			TS1F47 & Il dipendente deve poter visualizzare un messaggio di fine prenotazione, 5 minuti prima che essa avvenga. & NI \\	
			\hline
			TS1F48 & Il dipendente deve poter registrare l'avvenuta igienizzazione autonoma della postazione. & I \\	
			\hline
			TS1F49 & Il dipendente può ricercare una postazione in base alla data, all'orario e all'identificativo della stanza. & I \\	
			\hline
			TS1F49.1 & Il dipendente deve compilare il campo della data per cercare una postazione. & I \\	
			\hline
			TS1F50 & Il dipendente riceve un messaggio di errore se il campo data ha un valore non valido. & I \\	
			\hline
			TS1F49.2 & Il dipendente deve compilare il campo dell'ora per cercare una postazione. & I \\	
			\hline
			TS1F51 & Il dipendente riceve un messaggio di errore se il campo ora ha un valore non valido. & I \\	
			\hline			
			TS1F49.3 & Il dipendente deve compilare il campo stanza per cercare una postazione. & I \\	
			\hline
			TS152 & Il dipendente riceve un messaggio di errore se il campo stanza ha un valore non valido. & I \\	
			\hline			
			TS1F49.4 & Il dipendente deve compilare il campo nome dipendente per cercare una postazione. & I \\	
			\hline
			TS1F53 & Il dipendente riceve un messaggio di errore se il campo nome dipendente ha un valore non valido. & I \\	
			\hline
			TS1F54 & Il dipendente può visualizzare le postazioni di una stanza. & I \\	
			\hline
			TS1F55 & Il dipendente riceve un messaggio di errore se non ci sono postazioni prenotabili. & I \\	
			\hline
			TS1F56 & Il dipendete può visualizzare l'identificativo della prima postazione igienizzata e libera in una stanza. & I \\	
			\hline
			TS1F57 & Il dipendete può visualizzare l'identificativo di una postazione igienizzata e libera in una stanza scelta con criterio random. & NI \\	
			\hline
			TS2F58 & Il dipendete può visualizzare l'identificativo di una postazione igienizzata e libera in una stanza scelta con criterio top distanza dalle altre postazioni. & NI \\	
			\hline
			TS1F59 & Il dipendente può selezionare e prenotare una postazione di una stanza. & I \\		
			\hline
			TS1F60 & Il dipendente riceve un messaggio di errore se la postazione non è prenotabile. & I \\	
			\hline
			TS1F61 & Se il dipendente scansiona il tag NFC per più di un minuto prenota la postazione per l'intera giornata. & NI \\	
			\hline
			TS1F62 & Se la postazione è occupata viene emesso un segnale sonoro. & NI \\	
			\hline
			TS1F63 & Se il dipendente scansiona il tag NFC per più di un minuto, e la postazione risulta essere prenotata in un momento successivo della giornata, prenota la postazione fino all'inizio della prenotazione successiva. & NI \\	
			\hline
			TS1F64 & Il dipendente deve poter visualizzare l'elenco delle prenotazioni che ha effettuato. & I \\
			\hline
			TS1F65 & Il dipendente dopo aver selezionato una prenotazione, deve poterla disdire. & I \\	
			\hline
			TS1F66 & Gli utenti che hanno una prenotazione devono essere avvisati se la postazione prenotata, o la stanza in cui si trova la postazione prenotata, diventa non disponibile. & I \\	
			\hline
			TS1F67 & L'addetto delle pulizie riceve l'elenco delle stanze che necessitano di igienizzazione. & I \\	
			\hline
			TS1F68 & L'addetto delle pulizie riceve l'elenco delle postazioni che necessitano di igienizzazione. & I \\	
			\hline
			TS1F69 & L'addetto delle pulizie deve poter marcare una stanza come igienizzata. & I \\	
			\hline
			TS1F70 & L'addetto delle pulizie deve poter marcare una postazione come igienizzata. & I \\	
			\hline
			TS1F70.1 & L'addetto delle pulizie non può pulire stanze o postazioni occupate. & NI \\	
			\hline
			TS1F71 & Ogni 24 ore deve avvenire una transazione in Ethereum per la certificazione dei report e l'amministratore deve essere notificato se l'operazione è avvenuta con successo. & I \\	
			\hline			
			TS1F72 & L’amministratore deve ricevere una notifica con un messaggio di errore nel caso in cui la transazione su Ethereum sia fallita. & NI \\	
			\hline						
			TS1F73& In Ethereum vanno salvate tutte le operazioni che riguardano l'occupazione, l'igienizzazione e le varie modifiche a stanze o postazioni. & I \\	
			\hline
			TS1F74 & Realizzazione file Docker contenente l'applicativo. & I \\	
			\hline
			% Test di sistema per i requisiti di qualità%			
			TS1Q1 & Si verifichi che vengano rispettate le norme e metriche definite nel documento \dext{Piano di qualifica v. 2.0.0}. & I \\	
			\hline			
			TS1Q2 & Deve essere prodotto un manuale sviluppatore. & I \\	
			\hline	
			TS1Q3 & Si verifichi che ogni componente applicativo sia correlato a test unitari e d'integrazione. & NI \\	
			\hline
			TS1Q4 & Il sistema deve essere testato nella sua interezza tramite test end-to-end. & NI \\	
			\hline
			TS1Q5 & La copertura dei test deve essere maggiore o uguale dell'ottanta per cento e correlata da report. & NI \\	
			\hline
			TS1Q6 & Si verifichi l'efficienza dell'applicazione al consumo della batteria del cellulare. & NI \\	
			\hline
			TS2Q7 & Analisi sul sistema cloud più adeguato per supportare e la mole di utenti dell'applicazione. & NI \\	
			\hline
			TS1Q8 & Fornire una documentazione sulle scelte implementative e progettuali. & I \\	
			\hline
			TS1Q9 & Fornire una documentazione sui problemi rimasti aperti, con relative proposte di soluzione. & NI \\	
			\hline
			TS1Q10 & Fornire dei test riguardo il consumo della batteria del telefono da parte dell'applicazione mobile. & NI \\	
			\hline
			TS1Q11 & L’utilizzo del lettore NFC riduce in modo rilevante l’autonomia dei cellulari, l’applicazione è da sviluppare in maniera tale da bilanciare nel miglior modo possibile batteria e scansioni. È richiesto un resoconto delle scelte fatte e dei test effettuati per garantire il miglior rapporto raggiunto. & NI \\	
			\hline
			TS1Q12 & Deve essere fornito un manuale utente. & I \\	
			\hline
			% Test di sistema per i requisiti di vincolo %
			TS1V1 & Il server deve esporre delle API Rest, o in alternativa gRPC, attraverso le quali sia possibile utilizzare l'applicativo. & I \\	
			\hline
			TS3V1.1 & Deve essere possibile utilizzare gRPC in alternativa al Rest. & NI \\	
			\hline
			TS2V2 & Si verifichi che le comunicazione tra app e server siano cifrate. & NI \\	
			\hline
			TS1V3 & Si verifichi che l'applicazione sia sviluppata per Android o iOS. La versione minima supportata deve essere  Android 6.0 "Marshmallow" e iOS9. & I \\	
			\hline
			% Test di sistema per i requisiti prestazionali %	
			\hiderowcolors
			\caption{Tabella dei test di sistema}		
		\end{longtable}	
	\end{center}

	\subsection{Test di accettazione}
	Il test di accettazione comprende l'esecuzione dei test di sistema con la partecipazione dei committenti. Dunque, è in tale circostanza che si eseguiranno i test di accettazione. 
	\subsection{Test di Unità}
	\begin{center}
		\rowcolors{2}{white}{lightest-grayest}
		\begin{longtable}{|c|p{10cm}|c|}
			\hline
			\rowcolor{lighter-grayer}
			\textbf{Codice} & \textbf{Descrizione} & \textbf{Stato}  \\ 
			
			\hline
			\endhead
			
			TU1 & Si verifichi che l'indirizzo per la lista delle stanze sia associato al metodo corretto. & S \\	
			\hline
			TU2 & Si verifichi che l'indirizzo per l'inserimento delle stanze sia associato al metodo corretto. & S \\	
			\hline
			TU3 & Si verifichi che l'indirizzo per la cancellazione delle stanze sia associato al metodo corretto. & S \\	
			\hline
			TU4 & Si verifichi che l'indirizzo per la modifica delle stanze sia associato al metodo corretto. & S \\	
			\hline
			TU5 & Si verifichi che l'indirizzo per la lista degli utenti sia associato al metodo corretto. & S \\	
			\hline
			TU6 & Si verifichi che l'indirizzo per l'inserimento degli utenti sia associato al metodo corretto. & S \\	
			\hline
			TU7 & Si verifichi che l'indirizzo per la cancellazione degli utenti sia associato al metodo corretto. & S \\	
			\hline
			TU8 & Si verifichi che l'indirizzo per la modifica degli utenti sia associato al metodo corretto. & S \\	
			\hline
			TU9 & Si verifichi che l'indirizzo per il login sia associato al metodo corretto. & S \\	
			\hline
			TU10 & Si verifichi che l'indirizzo per le prenotazioni dell'utente sia associato al metodo corretto. & S \\	
			\hline
			TU11 & Si verifichi che l'indirizzo per la lista delle postazioni sia associato al metodo corretto. & S \\	
			\hline
			TU12 & Si verifichi che l'indirizzo per l'inserimento delle postazioni sia associato al metodo corretto. & S \\	
			\hline
			TU13 & Si verifichi che l'indirizzo per la cancellazione delle postazioni sia associato al metodo corretto. & S \\	
			\hline
			TU14 & Si verifichi che l'indirizzo per la modifica delle postazioni sia associato al metodo corretto. & S \\	
			\hline
			TU15 & Si verifichi che l'indirizzo per la visualizzazione dello stato della postazione sia associato al metodo corretto. & S \\	
			\hline
			TU16 & Si verifichi che l'indirizzo per la sanificazione delle postazioni sia associato al metodo corretto. & S \\	
			\hline
			TU17 & Si verifichi che la cancellazione di una stanza con id non esistente dia errore. & S \\	
			\hline
			TU18 & Si verifichi che la cancellazione di una stanza con id corretto vada a buon fine. & S \\	
			\hline
			TU19 & Si verifichi che il metodo per il reperimento della lista delle stanze vada a buon fine. & S \\	
			\hline
			TU20 & Si verifichi che il metodo per l'inserimento della  stanza vada a buon fine specificando dei dati corretti. & S \\	
			\hline
			TU21 & Si verifichi che il metodo per l'inserimento della  stanza vada in errore specificando dei dati non univoci. & S \\	
			\hline
			TU22 & Si verifichi che il metodo per l'inserimento della  stanza vada in errore specificando dei dati sbagliati. & S \\	
			\hline
			TU23 & Si verifichi che il metodo per la modifica della  stanza vada a buon fine specificando dei dati corretti. & S \\	
			\hline
			TU24 & Si verifichi che il metodo per la modifica della  stanza vada in errore specificando dei dati non univoci. & S \\	
			\hline
			TU25 & Si verifichi che il metodo per la modifica della  stanza vada in errore specificando dei dati sbagliati. & S \\	
			\hline
			
			TU26 & Si verifichi che il metodo per il reperimento della lista delle prenotazioni dell'utente vada a buon fine. & S \\	
			\hline
			TU27 & Si verifichi che il metodo per il login dell'utente vada a buon fine specificando dati corretti. & S \\	
			\hline
			TU28 & Si verifichi che il metodo per il login dell'utente vada in errore specificando dati sbagliati. & S \\	
			\hline
			TU29 & Si verifichi che la cancellazione di un utente con id non esistente dia errore. & S \\	
			\hline
			TU30 & Si verifichi che la cancellazione di un utente con id corretto vada a buon fine. & S \\	
			\hline
			TU31 & Si verifichi che il metodo per il reperimento della lista degli utenti vada a buon fine. & S \\	
			\hline
			TU32 & Si verifichi che il metodo per l'inserimento di un utente vada a buon fine specificando dei dati corretti. & S \\	
			\hline
			TU33 & Si verifichi che il metodo per l'inserimento di un utente vada in errore specificando dei dati non univoci. & S \\	
			\hline
			TU34 & Si verifichi che il metodo per l'inserimento di un utente vada in errore specificando dei dati sbagliati. & S \\	
			\hline
			TU35 & Si verifichi che il metodo per la modifica di un utente vada a buon fine specificando dei dati corretti. & S \\	
			\hline
			TU36 & Si verifichi che il metodo per la modifica di un utente vada in errore specificando dei dati non univoci. & S \\	
			\hline
			TU37 & Si verifichi che il metodo per la modifica di un utente vada in errore specificando dei dati sbagliati. & S \\	
			\hline
			
			TU38 & Si verifichi che il metodo per la sanificazione di una postazione vada a buon fine specificando dati corretti. & S \\	
			\hline
			TU39 & Si verifichi che il metodo per la sanificazione di una postazione vada in errore specificando dati sbagliati. & S \\	
			\hline
			TU40 & Si verifichi che il metodo per il reperimento dello stato della postazione vada a buon fine specificando dati corretti. & S \\	
			\hline
			TU41 & Si verifichi che il metodo per il reperimento dello stato della postazione vada in errore specificando dati sbagliati. & S \\	
			\hline
			TU42 & Si verifichi che la cancellazione di una postazione con id non esistente dia errore. & S \\	
			\hline
			TU43 & Si verifichi che la cancellazione di una postazione con id corretto vada a buon fine. & S \\	
			\hline
			TU44 & Si verifichi che il metodo per il reperimento della postazioni vada a buon fine. & S \\	
			\hline
			TU45 & Si verifichi che il metodo per l'inserimento di una postazione vada a buon fine specificando dei dati corretti. & S \\	
			\hline
			TU46 & Si verifichi che il metodo per l'inserimento di una postazione vada in errore specificando dei dati non univoci. & S \\	
			\hline
			TU47 & Si verifichi che il metodo per l'inserimento di una postazione vada in errore specificando dei dati sbagliati. & S \\	
			\hline
			TU48 & Si verifichi che il metodo per la modifica di una postazione vada a buon fine specificando dei dati corretti. & S \\	
			\hline
			TU49 & Si verifichi che il metodo per la modifica di una postazione vada in errore specificando dei dati non univoci. & S \\	
			\hline
			TU50 & Si verifichi che il metodo per la modifica di una postazione vada in errore specificando dei dati sbagliati. & S \\	
			\hline
			TU51 & Si verifichi che vengano settati i campi di ReportData correttamente. & S \\	
			\hline
			TU52 & Si verifichi che venga creata un'istanza di ReportData. & S \\	
			\hline
			TU53 & Si verifichi che ShowReportsComponent possa generare report. & S \\	
			\hline
			TU54 & Si verifichi che venga creata un'istanza di OccupationData. & S \\	
			\hline
			TU55 & Si verifichi che vengano settati i campi di OccupationData correttamente. & S \\	
			\hline
			TU56 & Si verifichi che AddEditRoomComponent possa copiare i valori se settata inaccessibile. & S \\	
			\hline
			TU57 & Si verifichi che AddEditRoomComponent possa copiare i valori se settata inaccessibile. & S \\	
			\hline
			TU58 & Si verifichi che AddEditRoomComponent possa chiamare copyValues se modificata. & S \\	
			\hline
			TU59 & Si verifichi che AddEditRoomComponent possa chiamare addRoom. & S \\	
			\hline
			TU60 & Si verifichi che AddEditRoomComponent possa restituire room dai valori locali. & S \\	
			\hline
			TU61 & Si verifichi che AddEditRoomComponent possa restituire l'inaccessibilità della stanza. & S \\	
			\hline
			TU62 & Si verifichi che AddEditRoomComponent copi correttamente i valori se accessibile. & S \\	
			\hline
			TU63 & Si verifichi che AddEditRoomComponent possa chiamare editRoom. & S \\	
			\hline
			TU64 & Si verifichi che AddEditRoomComponent non possa ottenere l'inacessibilità della stanza se accessibile. & S \\	
			\hline
			TU65 & Si verifichi che vengano settati i campi di RoomData correttamente. & S \\	
			\hline
			TU66 & Si verifichi che vengano settati RoomDataWithDates correttamente. & S \\	
			\hline
			TU66 & Si verifichi che venga creata un'istanza di RoomData. & S \\	
			\hline
			TU67 & Si verifichi che AddEditWorkstationComponent possa restituire la workstation dai dati locali. & S \\	
			\hline
			TU68 & Si verifichi che AddEditWorkstationComponent possa restituire il guasto workstation se la workstation è guasta. & S \\	
			\hline
			TU69 & Si verifichi che AddEditWorkstationComponent non possa restituire il guasto workstation se la workstation non è guasta. & S \\	
			\hline
			TU70 & Si verifichi che AddEditWorkstationComponent possa chiamare addWorkstation. & S \\	
			\hline
			TU71 & Si verifichi che AddEditWorkstationComponent possa chiamare editWorkstation. & S \\	
			\hline
			TU72 & Si verifichi che GrigliaComponent possa restituire array da num. & S \\	
			\hline
			TU73 & Si verifichi che GrigliaComponent possa ricevere item e trasmetterli. & S \\	
			\hline
			TU74 & Si verifichi che GrigliaComponent possa creare una griglia dalle postazioni in modo appropriato. & S \\	
			\hline
			TU75 & Si verifichi che GrigliaComponent possa chiamare init on ngOnChanges. & S \\	
			\hline
			TU76 & Si verifichi che GrigliaComponent possa filtrare le postazioni per posizione. & S \\	
			\hline
			TU77 & Si verifichi che GrigliaComponent possa settare roomArray nella chiamata iniziale. & S \\	
			\hline
			TU78 & Si verifichi che WorkStationData possa creare possa creare un'istanza. & S \\	
			\hline
			TU79 & Si verifichi che UserData possa creare un'istanza con i valori inseriti. & S \\	
			\hline
			TU80 & Si verifichi che UserData possa creare un'istanza correttamente. & S \\	
			\hline
			TU81 & Si verifichi che UserData possa settare i propri campi correttamente. & S \\	
			\hline
			TU82 & Si verifichi che SanitizationData possa creare un'istanza correttamente. & S \\	
			\hline
			TU83 & Si verifichi che AddEditCredentialComponent possa copiare i valori correttamente. & S \\	
			\hline
			TU84 & Si verifichi che AddEditCredentialComponent possa copiare i valori durante le modifiche. & S \\	
			\hline
			TU85 & Si verifichi che AddEditCredentialComponent possa aggiungere correttamente. & S \\	
			\hline
			TU86 & Si verifichi che AddEditCredentialComponent possa modificare correttamente. & S \\	
			\hline
			TU87 & Si verifichi che WorkstationService venga creato. & S \\	
			\hline
			TU88 & Si verifichi che RoomFailuresService venga creato. & S \\	
			\hline
			% Test di sistema per i requisiti prestazionali %	
			\hiderowcolors
			\caption{Tabella dei test di unità}		
		\end{longtable}	
	\end{center}

