
\section{Specifica dei Test}
I test permettono di verificare sia la correttezza delle parti di programma sviluppate, sia che tutti gli aspetti del progetto siano implementati e corretti.
Per assicurare la qualità del software prodotto, il gruppo \textit{DPCM 2077} utilizza come modello di sviluppo il \glock{Modello a V}.
\\
Per la classificazione dei test si fa riferimento alle sezioni Verifica\footnote{Sezione §3.4} e Validazione\footnote{Sezione §3.5} del documento \dext{Norme di Progetto v. 1.0.0}.
	\subsection{Tipi di test}
	Vengono individuate quattro tipologie di test:
	\begin{itemize}
		\item \textbf{Test di Sistema [TS]};
		\item \textbf{Test di Accettazione [TA]};
		\item \textbf{Test di Integrazione [TI]};
		\item \textbf{Test di Unità [TU]}.
	\end{itemize}
	\subsection{Test di Sistema}
	\begin{center}
		\rowcolors{2}{white}{lightest-grayest}
		\begin{longtable}{|c|p{10cm}|c|}
			\hline
			\rowcolor{lighter-grayer}
			\textbf{Codice} & \textbf{Descrizione} & \textbf{Stato}  \\ 
						
			\hline
			\endhead
			
			\hline
			% Test di sistema per i requisiti funzionali %
			TS2F1 & L'utente deve poter accedere alla guida per il login e per l'applicazione. & NI \\
			\hline
			TS1F2 & L'utente non ancora autenticato deve poter autenticarsi nel web server. & NI \\			
			\hline
			TS1F2.1 & L'utente non ancora autenticato nel web server deve ricevere un messaggio di errore nel caso abbia inserito le credenziali errate. & NI \\			
			\hline
			TS1F2.2 & L'utente non ancora autenticato nel web server deve ricevere un messaggio di errore nel caso l'account sia disabilitato. & NI \\			
			\hline
			TS1F3 & L'utente non ancora autenticato deve poter autenticarsi nell'applicazione mobile. & NI \\			
			\hline
			TS1F3.1 & L'utente non ancora autenticato nell'applicazione mobile deve ricevere un messaggio di errore nel caso abbia inserito le credenziali errate. & NI \\			
			\hline
			TS1F3.2 & L'utente non ancora autenticato nell'applicazione mobile deve ricevere un messaggio di errore nel caso l'account sia disabilitato. & NI \\			
			\hline
			TS1F4 & L'utente autenticato come amministratore deve potersi disconnettere dal web server. & NI \\			
			\hline
			TS1F5 & L'utente autenticato come dipendente deve potersi disconnettere dall'applicazione mobile. & NI \\			
			\hline			
			TS1F6 & L'utente autenticato come addetto alle pulizie deve potersi disconnettere dall'applicazione mobile. & NI \\			
			\hline
			TS1F7 & L'utente autenticato come amministratore deve poter accedere ad una giuda completa delle funzionalità. & NI \\		
			\hline
			TS1F8 & L'utente autenticato come dipendente deve poter accedere ad una giuda completa delle funzionalità. & NI \\			
			\hline
			TS1F9 & L'utente autenticato come addetto alle pulizie deve poter accedere ad una giuda completa delle funzionalità. & NI \\
			\hline
			TS1F10 & L'amministratore può visualizzare le stanze e le postazioni salvate. & NI \\			
			\hline
			TS1F10.1 & L'amministratore può visualizzare le stanze e le postazioni in modo schematico. & NI \\			
			\hline
			TS1F10.2 & Gli stati delle postazioni possono essere i seguenti:
			 \begin{itemize}
			 	\item libera e igienizzata;
			 	\item libera e non igienizzata;
			 	\item occupata;
			 	\item prenotata e igienizzata;
			 	\item prenotata e non igienizzata;
			 	\item guasta e igienizzata;
			 	\item guasta e non igienizzata.
			 \end{itemize}
			 & NI \\			
			\hline
			TS1F10.3 & Per ogni stanza deve essere indicato il numero di occupanti attuali. & NI \\			
			\hline
			TS1F11 & L'amministratore può visualizzare il calendario delle prenotazioni delle postazioni. & NI \\			
			\hline
			TS1F12 & L'amministratore può aggiungere una stanza al sistema e assegnarle un nome. & NI \\			
			\hline
			TS1F12.1 & La creazione di una stanza inizia il processo di salvataggio in Ethereum. & NI \\			
			\hline
			TS1F12.2 & L'amministratore non può creare una stanza con un nome già in utilizzo o con delle dimensioni impossibili e deve essere avvisato con un messaggio di errore. & NI \\			
			\hline
			TS1F13 & L'amministratore può eliminare una stanza al sistema. & NI \\			
			\hline
			TS1F13.1 & L'eliminazione di una stanza inizia il processo di salvataggio in Ethereum. & NI \\			
			\hline
			TS1F14 & L'amministratore può modificare il nome di una stanza. & NI \\			
			\hline
			TS1F15 & L'amministratore può impostare una stanza come inaccessibile per un determinato periodo di tempo, con data inizio e fine arbitraria. & NI \\			
			\hline
			TS1F16 & L'amministratore può aggiungere una postazione in una stanza, specificando:
			\begin{itemize}
				\item codice della postazione;
				\item codice tag NFC che la identifica;
				\item posizione all'interno della stanza.
			\end{itemize}
			& NI \\			
			\hline			
			TS1F16.1 & L'aggiunta di una postazione inizia il processo di salvataggio in Ethereum. & NI \\			
			\hline				
			TS1F16.2 & L'amministratore non può modificare il "codice postazione" di una postazione con un "codice postazione" già in utilizzo e deve essere avvisato con un messaggio di errore. & NI \\			
			\hline			
			TS1F16.3 & L'amministratore non può modificare il "codice tag" di una postazione con un "codice tag" già in utilizzo e deve essere avvisato con un messaggio di errore. & NI \\			
			\hline			
			TS1F16.4 & L'amministratore non può modificare la posizione di una postazione con una posizione già in utilizzo e deve essere avvisato con un messaggio di errore. & NI \\			
			\hline
			TS1F17 & L'amministratore può eliminare una postazione. & NI \\			
			\hline		
			TS1F17.1 & L'eliminazione di una postazione inizia il processo di salvataggio in Ethereum. & NI \\			
			\hline	
			TS1F18 & L'amministratore può modificare i dati e la posizione di una postazione. & NI \\			
			\hline	
			TS1F18.1 & La modifica di una postazione inizia il processo di salvataggio in Ethereum. & NI \\			
			\hline	
			TS1F19 & L'amministratore gestisce le credenziali degli utenti. & NI \\	
			\hline
			TS1F19.1 & L'amministratore può visualizzare una lista delle credenziali di tutti gli utenti. & NI \\	
			\hline
			TS1F19.2 & L'amministratore può creare nuove credenziali. Le informazioni contenute nel profilo di un utente sono:
			\begin{itemize}
				\item nome;
				\item cognome;
				\item nome utente;
				\item password;
				\item email.
			\end{itemize}
			Tra queste le informazioni necessarie per l’accesso sono password e nome utente. & NI \\	
			\hline
			TS1F19.3 & L’amministratore può modificare le credenziali degli utenti del sistema. & NI \\	
			\hline
			TS1F19.4 & L’amministratore può eliminare le credenziali degli utenti del sistema. & NI \\	
			\hline
			TS1F19.5 & Il sistema può memorizzare più credenziali per ogni tipologia di utente. & NI \\	
			\hline
			TS1F20 & L'amministratore può vedere lo storico delle occupazioni delle postazioni. & NI \\	
			\hline
			TS1F20.1 & L'amministratore deve vedere lo storico delle postazioni occupate da un utente in modo tabellare, con data e orari di inizio e fine delle occupazioni e codici delle prenotazioni. & NI \\	
			\hline
			TS1F20.1.1 & L'amministratore deve poter effettuare una ricerca delle postazioni occupate da un utente, specificando il periodo all'interno del quale l'utente occupava la postazione. & NI \\	
			\hline
			TS2F20.1.2 & L'amministratore deve vedere lo storico delle postazioni occupate da un utente specifico in modo tabellare, con data e orari di inizio e fine delle occupazioni, codici delle prenotazioni e ore trascorse alla postazione. & NI \\	
			\hline
			TS1F20.2 & L'amministratore deve vedere lo storico delle occupazioni di una postazione specifica in modo tabellare, con data e orari di inizio e fine delle occupazioni e nomi e cognomi dei dipendenti. & NI \\			
			\hline
			TS1F20.3 & L'amministratore può scaricare il report delle occupazioni in un formato leggibile. & NI \\		
			\hline
			TS2F20.3.1 & L'amministratore può scaricare il report delle occupazioni in formato PDF. & NI \\		
			\hline			
			TS2F20.4 & L'amministratore può scaricare il report degli utenti che hanno condiviso la stanza con un determinato utente. & NI \\		
			\hline
			TS2F21 & L'amministrazione può vedere lo storico delle igienizzazioni di tutte le postazioni in modo tabellare, con data e ora in cui sono avvenute, nome e cognome di chi le ha eseguite e il loro ruolo. & NI \\	
			\hline
			TS2F21.1 & L'amministratore può scaricare il report delle igienizzazioni in un formato leggibile. & NI \\	
			\hline
			TS2F21.1.1 & L'amministratore può scaricare il report delle igienizzazioni in formato PDF. & NI \\	
			\hline			
			TS1F22 & Il dipendente deve poter scansionare un tag NFC presente in una postazione. & NI \\	
			\hline
			TS1F23 & Il dipendente deve poter vedere lo stato di una postazione tramite la scansione del tag NFC. & NI \\	
			\hline
			TS1F24 & Il dipendente deve poter segnare la propria presenza su una postazione in tempo reale. & NI \\	
			\hline			
			TS1F24.1 & Il dipendente deve poter segnare la propria presenza su una postazione in tempo reale, tramite la scansione del tag NFC. & NI \\	
			\hline
			TS1F24.2 & Il dipendente deve poter registrare la propria presenza, di inizio e fine occupazione, su una postazione tramite Ethereum. & NI \\	
			\hline			
			TS1F24.2.1 & Il dipendente visualizza un messaggio di inizio della prenotazione, 30 minuti prima che essa avvenga. & NI \\	
			\hline						
			TS1F24.2.2 & Il dipendente visualizza un messaggio di fine della prenotazione, cinque minuti prima che essa avvenga. & NI \\	
			\hline					
			TS1F24.3 & Se lo smartphone non rileva il tag NFC per un tempo maggiore o uguale a 30 min avviene la disdetta automatica della prenotazione per il tempo restante. & NI \\	
			\hline				
			TS1F24.4 & L’utente viene avvisato della disdetta automatica di una sua prenotazione se sposta il proprio smartphone dal tag NFC per più di 30 minuti. & NI \\	
			\hline				
			TS1F24.5 & Il dipendente visualizza un messaggio di errore se prova ad occupare una postazione non igienizzata. & NI \\	
			\hline
			TS1F25 & Il dipendente deve poter registrare l'avvenuta igienizzazione autonoma tramite Ethereum. & NI \\	
			\hline
			TS1F26 & Il dipendente può ricercare una postazione. & NI \\	
			\hline
			TS1F26.1 & Il dipendente può ricercare una postazione in base alla data, all'orario e all'identificativo della stanza. & NI \\	
			\hline
			TS1F26.1.1 & Il dipendente deve compilare il campo della data per cercare una postazione. & NI \\	
			\hline
			TS1F26.1.2 & Il dipendente riceve un messaggio di errore se il campo data ha un valore non valido. & NI \\	
			\hline
			TS1F26.1.3 & Il dipendente deve compilare il campo dell'ora per cercare una postazione. & NI \\	
			\hline
			TS1F26.1.4 & Il dipendente riceve un messaggio di errore se il campo ora ha un valore non valido. & NI \\	
			\hline			
			TS1F26.1.5 & Il dipendente deve compilare il campo stanza per cercare una postazione. & NI \\	
			\hline
			TS1F26.1.6 & Il dipendente riceve un messaggio di errore se il campo stanza ha un valore non valido. & NI \\	
			\hline			
			TS1F26.1.7 & Il dipendente deve compilare il campo nome dipendente per cercare una postazione. & NI \\	
			\hline
			TS1F26.1.8 & Il dipendente riceve un messaggio di errore se il campo nome dipendente ha un valore non valido. & NI \\	
			\hline
			TS1F26.2 & Il dipendente può visualizzare le postazioni di una stanza. & NI \\	
			\hline
			TS1F26.3 & Il dipendente riceve un messaggio di errore se non ci sono postazioni prenotabili. & NI \\	
			\hline
			TS1F26.4 & Il dipendete può visualizzare l'identificativo della prima postazione igienizzata e libera in una stanza. & NI \\	
			\hline
			TS1F26.4.1 & Il dipendete può visualizzare l'identificativo di una postazione igienizzata e libera in una stanza scelta con criterio random. & NI \\	
			\hline
			TS2F26.4.2 & Il dipendete può visualizzare l'identificativo di una postazione igienizzata e libera in una stanza scelta con criterio top distanza dalle altre postazioni. & NI \\	
			\hline
			TS1F26.5 & Il dipendente può selezionare una postazione di una stanza. & NI \\	
			\hline
			TS1F26.6 & Il dipendente può prenotare una postazione selezionata. & NI \\	
			\hline
			TS1F26.7 & Il dipendente riceve un messaggio di errore se la postazione non è prenotabile. & NI \\	
			\hline
			TS1F27 & Se il dipendente scansiona il tag NFC per più di un minuto prenota la postazione per l'intera giornata. & NI \\	
			\hline
			TS1F27.1 & Se la postazione è occupata viene emesso un segnale sonoro. & NI \\	
			\hline
			TS1F27.2 & Se il dipendente scansione il tag NFC per più di un minuto, e la postazione risulta essere prenotata in un momento successivo della giornata, prenota la postazione fino all'inizio della prenotazione successiva. & NI \\	
			\hline
			TS1F28 & Il dipendente deve poter disdire una prenotazione. & NI \\	
			\hline
			TS1F28.1 & Il dipendente deve poter visualizzare l'elenco delle prenotazioni che ha effettuato. & NI \\	
			\hline
			TS1F28.2 & Il dipendente deve poter selezionare una prenotazione che vuole disdire. & NI \\	
			\hline
			TS1F28.3 & Il dipendente dopo aver selezionato una prenotazione, deve poterla disdire. & NI \\	
			\hline
			TS1F29 & Gli utenti che hanno una prenotazione devono essere avvisati se la postazione prenotata, o la stanza in cui si trova la postazione prenotata, diventa non disponibile. & NI \\	
			\hline
			TS1F30 & L'addetto delle pulizie riceve l'elenco delle stanze che necessitano di igienizzazione. & NI \\	
			\hline
			TS1F31 & L'addetto delle pulizie riceve l'elenco delle postazioni che necessitano igienizzazione. & NI \\	
			\hline
			TS1F32 & L'addetto delle pulizie deve poter marcare una stanza come igienizzata, tramite Ethereum. & NI \\	
			\hline
			TS1F33 & L'addetto delle pulizie deve poter marcare una postazione come igienizzata, tramite Ethereum. & NI \\	
			\hline
			TS1F34 & L'addetto delle pulizie non può pulire stanze o postazioni occupate. & NI \\	
			\hline
			TS1F35 & Realizzazione file Docker contenente l'applicativo. & NI \\	
			\hline
			% Test di sistema per i requisiti di qualità%			
			TS1Q1 & Si verifichi che vengano rispettate le norme e metriche definite nel documento \dext{Piano di qualifica v. 2.0.0}. & NI \\	
			\hline			
			TS1Q2 & Deve essere prodotto un manuale sviluppatore. & NI \\	
			\hline	
			TS1Q3 & Si verifichi che ogni componente applicativo sia correlato a test unitari e d'integrazione. & NI \\	
			\hline
			TS1Q4 & Il sistema deve essere testato nella sua interezza tramite test end-to-end. & NI \\	
			\hline
			TS1Q5 & La copertura dei test deve essere maggiore o uguale dell'ottanta per cento e correlata da report. & NI \\	
			\hline
			TS1Q6 & Si verifichi l'efficienza dell'applicazione al consumo della batteria del cellulare. & NI \\	
			\hline
			TS2Q7 & Analisi sul sistema cloud più adeguato per supportare e la mole di utenti dell'applicazione. & NI \\	
			\hline
			TS1Q8 & Fornire una documentazione sulle scelte implementative e progettuali. & NI \\	
			\hline
			TS1Q9 & Fornire una documentazione sui problemi rimasti aperti, con relative proposte di soluzione. & NI \\	
			\hline
			TS1Q10 & Devono essere scritti dei test che abbiano una copertura di almeno l'80 per cento del codice e deve essere fornito il report della loro esecuzione. & NI \\	
			\hline
			TS1Q11 & Fornire dei test riguardo il consumo della batteria del telefono da parte dell'applicazione mobile. & NI \\	
			\hline
			TS1Q12 & L’utilizzo del lettore NFC riduce in modo rilevante l’autonomia dei cellulari, l’applicazione è da sviluppare in maniera tale da bilanciare nel miglior modo possibile batteria e scansioni. È richiesto un resoconto delle scelte fatte e dei test effettuati per garantire il miglior rapporto raggiunto. & NI \\	
			\hline
			TS1Q13 & Deve essere fornito un manuale utente. & NI \\	
			\hline
			% Test di sistema per i requisiti di vincolo %
			TS1V1 & Il server deve esporre delle API Rest, o in alternativa gRPC, attraverso le quali sia possibile utilizzare l'applicativo. & NI \\	
			\hline
			TS3V1.1 & Deve essere possibile utilizzare gRPC in alternativa al Rest. & NI \\	
			\hline
			TS2V2 & Si verifichi che le comunicazione tra app e server siano cifrate. & NI \\	
			\hline
			TS1V3 & Si verifichi che l'applicazione sia sviluppata per Android o iOS. La versione minima supportata deve essere  Android 6.0 "Marshmallow" e iOS9. & NI \\	
			\hline
			% Test di sistema per i requisiti prestazionali %	
			\hiderowcolors
			\caption{Tabella dei test di sistema}		
		\end{longtable}	
	\end{center}

	\subsection{Test di Accettazione}
	Il test di accettazione comprende l'esecuzione dei test di sistema con la partecipazione dei committenti. Dunque, è in tale circostanza che si eseguiranno i test di accettazione. 
	\subsection{Test di Integrazione}
	Le specifiche di questi test verranno scritte successivamente utilizzando il Modello a V.
	\subsection{Test di Unità}
	Le specifiche di questi test verranno scritte successivamente utilizzando il Modello a V.
