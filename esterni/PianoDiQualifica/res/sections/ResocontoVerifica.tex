\section{Resoconto attività di verifica}
In questa sezione vengono descritti e analizzati gli esiti delle attività di verifica svolte su tutti i documenti che vengono consegnati nelle varie revisioni di avanzamento del progetto.
	\subsection{Revisione dei Requisiti}
		\subsubsection{Verifica della documentazione}
		Per la verifica della documentazione viene utilizzata la tecnica del \glock{walkthrough}, che ha portato all'individuazione di errori frequenti, insieme alla tecnica di \glock{inspection}, che mira a cercare gli errori più frequenti, come descritto nel documento \dext{Norme di progetto v. 2.0.0}\footnote{sezione §3.4.4.1}
		\subsubsection{Leggibilità documenti}
		Nel seguente grafico vengono riportati gli indici di \glock{Gulpease} dei documenti finora stesi.
		\begin{figure}[H]
			\centering
			\includegraphics[width=0.7\linewidth]{res/images/IndiceGulpease.png}
			\caption{Andamento dell'indice di Gulpease}
			\label{fig:Figura7}
		\end{figure}
	\subsection{Periodo dalla Revisione dei Requisiti alla Revisione di Progettazione}
		\subsubsection{Accuratezza della pianificazione}
		\paragraph{Accuratezza della pianificazione per singolo ruolo}
		\begin{figure}[H]
	\centering
	\includegraphics[width=10cm]{res/images/responsabile}
	\caption{Scostamento orario settimanale ruolo Responsabile}
	\label{fig:Scostamento orario settimanale ruolo Responsabile}
\end{figure}
\begin{figure}[H]
	\centering
	\includegraphics[width=10cm]{res/images/amministratore}
	\caption{Scostamento orario settimanale ruolo Amministratore}
	\label{fig:Scostamento orario settimanale ruolo Amministratore}
\end{figure}
\begin{figure}[H]
	\centering
	\includegraphics[width=10cm]{res/images/analista}
	\caption{Scostamento orario settimanale ruolo Analista}
	\label{fig:Scostamento orario settimanale ruolo Analista}
\end{figure}
\begin{figure}[H]
	\centering
	\includegraphics[width=10cm]{res/images/progettista}
	\caption{Scostamento orario settimanale ruolo Progettista}
	\label{fig:Scostamento orario settimanale ruolo Progettista}
\end{figure}
\begin{figure}[H]
	\centering
	\includegraphics[width=10cm]{res/images/programmatore}
	\caption{Scostamento orario settimanale ruolo Programmatore}
	\label{fig:Scostamento orario settimanale ruolo Programmatore}
\end{figure}
\begin{figure}[H]
	\centering
	\includegraphics[width=10cm]{res/images/verificatore}
	\caption{Scostamento orario settimanale ruolo Verificatore}
	\label{fig:Scostamento orario settimanale ruolo Verificatore}
\end{figure}
\paragraph{Accuratezza della pianificazione per tutti i ruoli}
\begin{figure}[H]
	\centering
	\includegraphics[width=10cm]{res/images/tutti}
	\caption{Scostamento orario settimanale per tutti i ruoli complessivamente}
	\label{fig:Scostamento orario settimanale per tutti i ruoli complessivamente}
\end{figure}

\subsubsection{Metriche}
	\begin{figure}[H]
	\centering
	\includegraphics[width=0.7\linewidth]{res/images/BAC.png}
	\caption{Grafico del BAC nel corso del periodo}
	\label{fig:Figura1}
\end{figure}

\begin{figure}[H]
	\centering
	\includegraphics[width=0.7\linewidth]{res/images/EV.png}
	\caption{Grafico del EV nel corso del periodo}
	\label{fig:Figura2}
\end{figure}

\begin{figure}[H]
	\centering
	\includegraphics[width=0.7\linewidth]{res/images/PV.png}
	\caption{Grafico del PV nel corso del periodo}
	\label{fig:Figura3}
\end{figure}

\begin{figure}[H]
	\centering
	\includegraphics[width=0.7\linewidth]{res/images/AC.png}
	\caption{Grafico del AC nel corso del periodo}
	\label{fig:Figura4}
\end{figure}

\begin{figure}[H]
	\centering
	\includegraphics[width=0.7\linewidth]{res/images/SV.png}
	\caption{Grafico del SV nel corso del periodo}
	\label{fig:Figura5}
\end{figure}

\begin{figure}[H]
	\centering
	\includegraphics[width=0.7\linewidth]{res/images/CV.png}
	\caption{Grafico del CV nel corso del periodo}
	\label{fig:Figura6}
\end{figure}
