\section{Qualità di Prodotto}

Per garantire una buona qualità del prodotto il gruppo DPCM 2077 ha deciso di fare riferimento allo standard ISO/IEC:9126,
che definisce le caratteristiche di qualità necessarie affinché il prodotto finale sia di buona qualità.
Le caratteristiche sono descritte grazie a dei parametri che valutano il grado di raggiungimento di qualità; in seguito vengono descritti \textbf{solo} gli aspetti che il gruppo ha ritenuto importanti per ottenere gli obiettivi qualitativi.

\subsection{Funzionalità}
Rappresenta la capacità del software di soddisfare tutte le funzionalità che sono state individuate attraverso l'\dext{Analisi dei Requisiti v2.0.0}.
\subsubsection{Obiettivi}
\begin{itemize}
\item {\textbf{Completezza}: le funzionalità sviluppate devono essere almeno tante quante quelle richieste dal committente o individuate dal gruppo;}
\item {\textbf{Accuratezza}: il prodotto deve fornire dei risultati con il livello di precisione richiesto;}
\item {\textbf{Conformità}: il prodotto deve rispettare determinati standard.} 
\end{itemize}

\subsubsection{Metriche}
\paragraph{Completezza dell'implementazione}
La completezza di un prodotto software e il rispetto dei requisiti viene indicato con una percentuale.
\begin{itemize}
\item \textbf{Metodo di misura}: si calcola con la seguente formula:
\[C = (1- \frac{N\_FNI}{N\_FI}) * 100\]
dove $N\_FNI$ indica il numero di funzionalità non implementate e $N\_FI$ indica il numero di funzionalità che si potevano implementare individuate dall'analisi;
\item \textbf{valore preferibile}: 100\%;
\item \textbf{valore accettabile}: 100\%.
\end{itemize}

\subsection{Affidabilità}
Capacità del prodotto di mantenere prestazioni elevate anche in caso di situazioni anomale o critiche.
Possibili limitazioni all’affidabilità sono causate da errori nei requisiti, nella progettazione e nello sviluppo del codice.

\subsubsection{Obiettivi}
\begin{itemize}
\item \textbf{Maturità}: capacità del prodotto software di evitare errori e risultati non corretti, quindi anomalie, durante l’esecuzione;
\item \textbf{Tolleranza agli errori}: capacità del prodotto software di conservare il proprio livello di prestazioni anche in caso di malfunzionamenti o di uso non corretto del prodotto;
\item \textbf{Aderenza all'affidabilità}: grado di adesione del prodotto software a standard, regole e direttive che riguardano l'affidabilità.
\end{itemize}

\subsubsection{Metriche}

\paragraph{Densità errori}
L'abilità di un prodotto software di resistere a malfunzionamenti viene indicata con una percentuale.
\begin{itemize}
\item \textbf{Metodo di misura}: si calcola con la seguente formula:
\[M = (1- \frac{N\_Er}{N\_Te}) * 100\]
dove $N\_Er$ indica il numero di anomalie riscontrate durante la fase di testing e $N\_FI$ indica il numero di test eseguiti;
\item \textbf{valore preferibile}: 0\%;
\item \textbf{valore accettabile}: Inferiore al 5\%.
\end{itemize}
\textit{I valori preferibili e accettabili qui sopra espressi sono da ritenersi ancora immaturi e saranno integrati opportunamente prima della verifica dei prodotti.}

\subsection{Efficienza}
Si definisce efficienza la capacità di un prodotto software di realizzare le funzioni richieste nel minore tempo possibile e riducendo al minimo l'utilizzo delle risorse a disposizione.
Per risorse si intendono altri prodotti software, configurazioni hardware e software del sistema e materiali fisici.
\subsubsection{Obiettivi}
\begin{itemize}
\item \textbf{Comportamento rispetto al tempo}: è la capacità di fornire adeguati tempi di risposta, elaborazione e velocità di attraversamento, in tutte le condizioni.
In generale, per creare un buon progetto bisogna ricreare la situazione d'errore e verificare come il software si comporta in quel caso;
\item \textbf{Utilizzo delle risorse}: è la capacità di utilizzo di quantità e tipo di risorse in maniera corretta.
\end{itemize}

\subsubsection{Metriche}
\textit{Sezione non esaustiva. Sarà opportunamente integrata se il gruppo lo riterrà necessario nel corso dello svolgimento del progetto.}

\paragraph{Tempo di risposta}
Il tempo di risposta del prodotto deve essere relativamente breve, se possibile inferiore a quanto richiesto dal committente. Il tempo di risposta viene indicato in secondi.
\begin{itemize}
\item \textbf{valore preferibile}: <60s;
\item \textbf{valore accettabile}: Tra i 5s e i 10s.
\end{itemize}

\subsection{Usabilità}
Capacità del prodotto di essere di facile comprensione e utilizzo da parte degli utenti.
\subsubsection{Obiettivi}
\begin{itemize}
\item \textbf{Comprensibilità}: esprime la facilità di comprensione dei concetti del prodotto, mettendo in grado l'utente di comprendere se il software è appropriato;
\item \textbf{Apprendibilità}: facilità con cui un utente medio può comprendere il funzionamento del prodotto software ed imparare ad usarlo; Il software deve essere accessibile a tutte le categorie di utenti;
\item \textbf{Operabilità}: è la capacità di mettere in condizione gli utenti di utilizzare il prodotto per i propri scopi e controllarne l'utilizzo.
\end{itemize}

\subsubsection{Metriche}

\paragraph{Indice di Gulpease}
\textit{Vedere sezione 2.2.3.1 per maggiori informazioni.}

\paragraph{Errori ortografici}
Gli errori ortografici, oltre ad essere segnalati dall'editor \LaTeX{} usato per scrivere i documenti, vengono continuamente corretti durante la verifica manuale dei documenti da parte dei verificatori.
\begin{itemize}
\item \textbf{valore preferibile}: 0 errori;
\item \textbf{valore accettabile}: 0 errori.
\end{itemize}

\subsection{Manutenibiltà}
Si definisce manutenibilità la capacità del software di essere modificato mediante correzioni, adattamenti o miglioramenti.

\subsubsection{Obiettivi}
\begin{itemize}
\item \textbf{Analizzabilità}: rappresenta la facilità con la quale è possibile analizzare il prodotto e trovarvi i difetti (bug);
\item \textbf{Modificabilità}: la capacità del prodotto software di permettere l'implementazione di una determinata modifica.
\end{itemize}

\subsubsection{Metriche}

\paragraph{Facilità di comprensione}
La facilità con cui l’utente riesce a comprendere cosa fa il codice può essere rappresentata mediante il numero di linee di commento nel codice.
\begin{itemize}
\item \textbf{Metodo di misura}: si calcola con la seguente formula:
\[R = (\frac{N\_Lcom}{N\_Lcod})\]
dove $N\_Lcom$ indica le linee di commento nel codice, $N\_Lcod$ indica le linee effettive di codice;
\item \textbf{valore preferibile}: > 0.20\%;
\item \textbf{valore accettabile}: > 0.15\%.
\end{itemize}

\paragraph{Semplicità dei metodi}
Un metodo è definito complesso quando dispone di molti parametri. Nell'ingegneria del software tale problema è risolto grazie al design pattern creazionale Builder.
\begin{itemize}
\item \textbf{Metodo di misura}: conteggio dei parametri del metodo;
\item \textbf{valore preferibile}: < 3 o 4;
\item \textbf{valore accettabile}: < 5;
\end{itemize}

\textit{I valori preferibili e accettabili qui sopra espressi sono da ritenersi ancora immaturi e saranno integrati opportunamente durante la fase di progettazione e codifica.}

\subsection{Tabella riassuntiva}
		\begin{center}
		\rowcolors{2}{white}{lightest-grayest}
		\begin{longtable}{|c|c|c|}
			\hline
			\rowcolor{lighter-grayer}
			\textbf{Indice} & \textbf{Valore preferibile} & \textbf{Valore accettabile}  \\ 
						
			\hline
			\endhead
			
			\hline
			Completezza dell'implementazione & 100\% & 100\% \\
			\hline
			Densità errori & 0\% & <5\% \\
			\hline
			Tempo di risposta & <60s & $5 < s < 10$ \\
			\hline
			Errori ortografici & 0 errori & 0 errori \\
			\hline
			Facilità di comprensione & > 0.20\% & >0.15\% \\
			\hline
			Semplicità dei metodi & < 3 o 4 & <5 \\
			\hline
				
			\hiderowcolors
			\caption{Tabella riassuntiva delle metriche }		
		\end{longtable}	
	\end{center}
 

