\section{Introduzione}

\subsection{Premessa al documento}
Il Piano di Qualifica è un documento su cui il gruppo intende lavorare per tutta la durata del progetto, dal concepimento al rilascio.
Alcuni contenuti risulteranno essere di natura non stabile. Alcune delle metriche scelte potrebbero non essere applicabili nella prima parte del progetto e quindi non si riesce a valutare la loro utilità subito.
Anche i processi selezionati potrebbero essere non adeguati allo scopo del progetto o non sufficienti. Quindi per le ragioni esposte il documento verrà curato e costruito bene incrementalmente nel corso 
dello svolgimento del progetto. Bisogna anche porre attenzione a non inserire materiale non conforme alle Norme di Progetto\_v1.0.0\ped{D} in quanto la qualità del progetto potrebbe deteriorarsi.

\subsection{Scopo del documento}
Il Piano di Qualifica ha lo scopo di mostrare le strategie di verifica e validazione adottate dal gruppo DPCM 2077 per garantire la qualità del prodotto e del processo. 
Per raggiungere questo obiettivo, occorrerà verificare continuamente le attività svolte in modo da ottenere uno sviluppo del prodotto per costruzione e non per correzione minimizzando l’utilizzo delle risorse.
In questo modo sarà possibile rilevare e correggere instantaneamente eventuali anomalie riscontrate.

\subsection{Scopo del Prodotto}
Lo scopo del prodotto è quello di sviluppare un’applicazione in grado di
segnalare a un \glock{server} dedicato la presenza di un \glock{utente} su una determinata postazione appartenente a
una stanza. 
Nel server deve essere possibile gestire più stanze e postazioni per:
\begin{itemize}
\item{sapere in ogni momento se la postazione è occupata, prenotata oppure da pulire;}
\item{controllare quali postazioni sono prenotate e bloccare le prenotazioni per una determinata
stanza (e.g. all'esaurimento delle postazioni disponibili);}
\item{garantire una \glock{tracciatura} autenticata delle presenze e la registrazione di tutti i cambiamenti di stato relativi alla pulizia della
postazione, nonché le informazioni su chi ha igienizzato la postazione;}
\end{itemize}
L'applicazione dovrà essere in grado di svolgere i seguenti compiti:
\begin{itemize}
\item{recupero lista delle postazioni libere;}
\item{prenotazione di una postazione;}
\item{garantire una tracciatura autenticata delle postazioni in tempo reale tramite \glock{tag } \glock{RFID};}
\item{pulizia di una postazione;}
\item{generare uno storico delle postazioni occupate e igienizzate.}
\end{itemize}
\glock{Client} e server comunicano tra loro nel momento in cui lo smartphone viene  a contatto con il tag RFID.

\subsection{Glossario}
All’interno del  documento sono presenti termini che presentano significati ambigui a seconda del contesto.
Per evitare questa ambiguità è stato creato un  documento di nome Glossario che  conterrà tali termini con il loro significato specifico. Per segnalare che un termine del testo è presente all’ interno del Glossario  
verrà aggiunta una G pedice posta a fianco del termine ambiguo. 
Se viene indicata una D pedice significa che si sta facendo riferimento a un documento.

\subsection{Riferimenti}

\subsubsection{Riferimenti normativi}
\begin{itemize}
\item Capitolato d'appalto C1 - BlockCOVID: \\
\url{https://www.math.unipd.it/~tullio/IS-1/2020/Progetto/C1.pdf}
\end{itemize}
\subsubsection{Riferimenti informativi}
\begin{itemize}
\item ISO/IEC 9126: \\
\url{https://en.wikipedia.org/wiki/ISO/IEC_9126}
\item ISO/IEC 12207:1995 e sue evoluzioni \\
\url{https://www.math.unipd.it/~tullio/IS-1/2009/Approfondimenti/ISO_12207-1995.pdf}
\end{itemize}


