\section{Introduzione}

\subsection{Premessa al documento}
Il Piano di qualifica è un documento su cui il gruppo intende lavorare per tutta la durata del progetto, dal concepimento al rilascio.
Alcuni contenuti risulteranno essere di natura non stabile. Alcune delle metriche scelte potrebbero non essere applicabili nella prima parte del progetto e quindi non si riesce a valutare la loro utilità subito.
Anche i processi selezionati potrebbero essere non adeguati allo scopo del progetto o non sufficienti. Quindi per le ragioni esposte il documento verrà curato e costruito bene incrementalmente nel corso 
dello svolgimento del progetto. Bisogna anche porre attenzione a non inserire materiale non conforme alle \dext{Norme di progetto\_v1.0.0} in quanto la qualità del progetto potrebbe deteriorarsi.

\subsection{Scopo del documento}
Il Piano di qualifica ha lo scopo di mostrare le strategie di verifica e validazione adottate dal gruppo DPCM 2077 per garantire la qualità del prodotto e del processo. 
Per raggiungere questo obiettivo, occorrerà verificare continuamente le attività svolte in modo da ottenere uno sviluppo del prodotto per costruzione e non per correzione minimizzando l’utilizzo delle risorse.
In questo modo sarà possibile rilevare e correggere instantaneamente eventuali anomalie riscontrate.

\subsection{Scopo del prodotto}
Il prodotto da sviluppare ha lo scopo di monitorare e regolare l'utilizzo e l'igienizzazione delle postazioni all'interno di uno spazio condiviso, al fine di ridurre il rischio di trasmissione di un'infezione e di adempiere alle norme di legge. 
\subsection{Glossario e documenti} 
All'interno del  documento sono presenti termini che assumono significati diversi a seconda del contesto.
Per evitare ambiguità, è stato creato un  documento di nome Glossario che  conterrà tali termini con il loro significato specifico. Per segnalare che un termine del testo è presente all'interno del Glossario, verrà aggiunta una G pedice posta a fianco del termine ambiguo.
Quando si fa riferimento a un altro documento riguardante questo progetto vi si pone a pedice una D.

\subsection{Riferimenti}

\subsubsection{Riferimenti normativi}
\begin{itemize}
\item C1: \\
\url{https://www.math.unipd.it/~tullio/IS-1/2020/Progetto/C1.pdf}
\item \dext{VE\_2020\_12\_18 v. 1.0.0}
\item \dext{Norme di progetto v. 2.0.0}
\item \dext{Analisi dei requisiti v. 2.0.0}
\end{itemize}
\subsubsection{Riferimenti informativi}
\begin{itemize}
\item ISO/IEC 9126: \\
\url{https://en.wikipedia.org/wiki/ISO/IEC_9126}
\item ISO/IEC 12207:1995 e sue evoluzioni \\
\url{https://www.math.unipd.it/~tullio/IS-1/2009/Approfondimenti/ISO_12207-1995.pdf}
\end{itemize}


