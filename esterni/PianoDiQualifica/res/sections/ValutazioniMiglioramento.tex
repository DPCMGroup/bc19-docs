\section{Valutazioni per il miglioramento}
In questa sezione verranno tracciati i problemi riguardanti i seguenti ambiti:
\begin{itemize}
	\item \textbf{Organizzazione}: verranno analizzati i problemi riguardanti l'organizzazione e la comunicazione;
	\item \textbf{Ruoli}: verranno analizzati i problemi riguardanti la scelta e del corretto svolgimento dei ruoli;
	\item \textbf{Strumenti}: verranno analizzati i problemi riguardanti l'uso degli strumenti di lavoro scelti.
\end{itemize}
Essendo il primo progetto realistico affrontato, ogni problema viene sollevato sulla base dell'autovalutazione dai membri del gruppo, non essendo presente una persona esterna la quale può dare un giudizio oggettivo.\\
Questa sezione verrà aggiornata con l'avanzamento del lavoro, riportando le nuove problematiche incontrate.
	\subsection{Valutazioni sull'organizzazione}
	\begin{center}
		\rowcolors{2}{white}{lightest-grayest}
		\begin{longtable}{|C{2cm}|p{6cm}|p{6cm}|}
			\hline
			\rowcolor{lighter-grayer}
			\textbf{Problema} & \textbf{Problema} & \textbf{Soluzione}  \\ 			
			\hline
			\endhead
			
			\hline
			Incontro con il gruppo & Dato il periodo che stiamo affrontando con la pandemia, non è possibile fare riunioni fisicamente. & Si è deciso di fare una riunione ogni settimana tramite l'ausilio di \glock{zoom}, Google \glock{meet} e \glock{discord}. È stato anche creato un gruppo nell'app di messaggistica \glock{Telegam}.\\
			\hline			
			\hiderowcolors
			\caption{Problematiche riguardanti l'organizzazione}		
		\end{longtable}	
	\end{center}
	\subsection{Valutazioni sui ruoli}
	\begin{center}
		\rowcolors{2}{white}{lightest-grayest}
		\begin{longtable}{|C{2cm}|p{6cm}|p{6cm}|}
			\hline
			\rowcolor{lighter-grayer}
			\textbf{Problema} & \textbf{Descrizione} & \textbf{Soluzione}  \\ 			
			\hline
			\endhead
			
			\hline
			Ruolo di responsabile & Il gruppo, nel momento dell'assegnazione dei ruoli, ha incontrato dei problemi per la scelta del responsabile, data l'inesperienza di tutti i membri. & È stato decisa la rotazione del ruolo di responsabile nei verbali. Successivamente è stato scelto il responsabile per i documenti ufficiali in base all'esperienza maturata durante le stesure dei verbali. Il tutto a votazione dei membri del gruppo. \\
			\hline
			
			\hiderowcolors
			\caption{Problematiche riguardanti i ruoli}		
		\end{longtable}	
	\end{center}
	\subsection{Valutazione sugli strumenti di lavoro}
	\begin{center}
		\rowcolors{2}{white}{lightest-grayest}
		\begin{longtable}{|C{2cm}|p{6cm}|p{6cm}|}
			\hline
			\rowcolor{lighter-grayer}
			\textbf{Strumento} & \textbf{Descrizione} & \textbf{Soluzione}  \\ 			
			\hline
			\endhead
			
			\hline
			\glock{GitHub} & Il gruppo non conosceva bene il funzionamento di GitHub, inoltre si sono riscontrati conflitti in più occasioni. & Il gruppo è riuscito ad utilizzare a pieno GitHub, grazie all'insegnamento offerto dal corso di \textit{Tecnologie Open Source}. È stato messo a disposizione un documento comune in cui sono specificate le soluzioni da adottare ogni qualvolta si incontra un dato problema. Inoltre è stato dedicato un canale apposito all'interno di discord per eventuali suggerimenti, soluzioni o discussioni di vario genere riguardanti GitHub.\\
			\hline	
			\glock{\LaTeX} & Questo strumento è stato una novità per quasi tutti i membri del gruppo, inoltre si sono riscontrati più volte errori nella compilazione dei documenti. & È stato scelto come compilatore unico \glock{pdfLatex}. Per far sì che le modifiche eseguite non creassero problemi agli altri membri del gruppo, è stata creata una \glock{GitHub Actions} che ne verifica la compilazione. Inoltre è stato dedicato un canale apposito all'interno di discord per eventuali suggerimenti, soluzioni o discussioni di vario genere riguardanti \LaTeX. \\
			\hline	
			
			\hiderowcolors
			\caption{Problematiche riguardante gli strumenti}		
		\end{longtable}	
	\end{center}