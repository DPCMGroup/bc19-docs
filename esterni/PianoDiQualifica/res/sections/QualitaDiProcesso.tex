\section{Qualità di Processo}
Il gruppo \textit{DPCM 2077} ha deciso di seguire lo standard ISO/IEC/IEEE 12207:1995 per garantire la qualità del nostro progetto. Questo standard propone diversi processi, i quali sono stati scelti e adattati secondo le nostre esigenze. 
Il risultato sono i processi presentati di seguito.
	\subsection{Processi di sviluppo}
		\subsubsection{Sviluppo - Metriche}
		\paragraph{PROS: Percentuale di requisiti obbligatori soddisfatti}
		Indica la percentuale di requisiti obbligatori soddisfatti.
		\begin{itemize}
		\item espresso in percentuale: $PROS = \frac{requisiti \ obbligatori \ soddisfatti}{requisiti \ obbligatori \ totali}$;
		\item valore preferibile: 100\%;
		\item valore accettabile: 100\%.
		\end{itemize}
		
		\paragraph{CBO: Accoppiamento tra le classi di oggetti}
		Una classe è acccoppiata alla seconda se usa metodi o variabili definiti nella seconda.
		\begin{itemize}
		\item espresso con un valore intero: CBO;
		\item valore preferibile: $0 \leq CBO \leq 1$;
		\item valore accettabile: $0 \leq CBO \leq 6$.
		\end{itemize}
		
		\paragraph{Profondità della gerarchia}
		Misura la profondità del sito (numero di click necessari per arrivare all'informazione di interesse) in quanto un sito per essere di facile utilizzo non deve essere troppo profondo.
		\begin{itemize}
		\item misura: livello di profondità delle pagine;
		\item valore preferibile: $\leq 4$;
		\item valore accettabile: $ \leq 7$.
		\end{itemize}
		
		\paragraph{Numero di parametri per metodo}
		\begin{itemize}
		\item numero intero;
		\item valore preferibile: $\leq 3$;
		\item valore accettabile: $ \leq 4$.
		\end{itemize}
		
	\subsection{Processi organizzativi}
		\subsubsection{Gestione organizzativa - Metriche}
		\paragraph{BAC: Budget at Completion}
		Indica il budget totale allocato per il progetto.
		\begin{itemize}
		\item numero intero;
		\item valore preferibile: pari al preventivo;
		\item valore accettabile: il valore del preventivo con un errore massimo del $5\%$.
		\end{itemize}
		
		\paragraph{EV: Earned Value}
		Metrica di utilità per il calcolo di SV e CV. Costituisce il costo attribuibile al cliente se il progetto venisse interrotto alla data della misurazione.
		\begin{itemize}
		\item calcolo: $BAC \cdot \% \  di \ lavoro \ completato$;
		\item valore preferibile: $\geq 0$;
		\item valore accettabile: $\geq 0$.
		\end{itemize}
		
		\paragraph{PV: Planned Value}
		Metrica di utilità per il calcolo di SV e CV. Si tratta del valore del lavoro pianificato al momento del calcolo: denaro che si dovrebbe aver guadagnato in quel momento.
		\begin{itemize}
		\item calcolo: $BAC \cdot \% \ di \ lavoro \ pianificato$;
		\item valore preferibile: $\geq 0$;
		\item valore accettabile: $\geq 0$.
		\end{itemize}
		
		\paragraph{AC: Actual Cost}
		Il denaro speso fino al momento del calcolo.
		\begin{itemize}
		\item numero intero;
		\item valore preferibile: $0 \leq AC \le PV$;
		\item valore accettabile: $0 \leq AC \leq budget \ totale$.
		\end{itemize}
		
		\paragraph{SV: Schedule Variance}
		Esprime lo stato di anticipo o ritardo nello svolgimento del progetto rispetto alla pianificazione.
		\begin{itemize}
		\item calcolo: $SV = EV - PV$;
		\item valore preferibile: $\ge 0$;
		\item valore accettabile: $= 0$.
		\end{itemize}
		
		\paragraph{CV: Cost Variance}
		Differenza tra il costo del lavoro effettivamente completato e quello pianificato. Una CV positiva indica che si sta rispettando il budget.
		\begin{itemize}
		\item calcolo: $CV = EV - AC$
		\item valore preferibile: $\ge 0$;
		\item valore accettabile: $\geq 0$.
		\end{itemize}
		
	\subsection{Processi di supporto}
		\subsubsection{Verifica - Metriche}
		\paragraph{CC: Code Coverage}
		Indica il numero di righe di codice percorse dai test durante la loro esecuzione. Per linee di codice totali si intende tutte quelle appartenenti all'unità in fase di test.
		\begin{itemize}
		\item valore percentuale: $CC = \frac{linee \ di \ codice \ percorse}{linee \ di \ codice \ totali}$;
		\item valore preferibile: 100\%;
		\item valore accettabile: 80\%.
		\end{itemize}
		
		\subsubsection{Documentazione - Metriche}
		\paragraph{Indice di Gulpease}
		Indice della leggibilità del testo. Valuta la lunghezza delle parole e delle frasi rispetto al numero totale di lettere.
		\begin{itemize}
		\item valore intero da 0 a 100: $I = 89 + \frac{(300 \cdot numero \ di \ frasi - 10 \cdot numero \ di \ lettere)}{numero \ di \ parole}$;
		\item valore preferibile: $60 < I < 100$;
		\item valore accettabile: $40 < I < 100$.
		\end{itemize}
