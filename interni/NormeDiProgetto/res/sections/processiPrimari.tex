\section{Processi Primari}
	\subsection{Fornitura}
		\subsubsection{Scopo}
		Lo scopo del processo di fornitura è di determinare le procedure e le risorse necessarie allo
		svolgimento del progetto. Il \glock{fornitore} esegue un'attività di analisi e successiva redazione dello \textit{Studio di Fattibilità}: un documento che comprende le richieste del \glock{proponente}, individua i potenziali rischi e criticità del \glock{capitolato}.
		È inoltre necessario stipulare e concordare con il \glock{proponente} un contratto per la consegna e manuntenzione del prodotto. 
		Si passa dunque alla determinazione delle procedure e risorse necessarie, che andranno a confluire nel \textit{Piano di Progetto} che porrà le basi per la consegna e realizzazione del prodotto.
		\\
		Il processo di fornitura è composto dalle seguenti fasi:
		\begin{itemize}
			\item avvio;
			\item approfondimento di risposte alle richieste;
			\item contrattazione;
			\item pianificazione;
			\item esecuzione e controllo;
			\item revisione e valutazione;
			\item consegna e completamento.
		\end{itemize}
		\subsubsection{Aspettative}
		Durante lo svolgimento dell'intero progetto, \textit{DPCM 2077} intende mantenere un costante dialogo con il \glock{proponente}, per avere un riscontro efficace sul lavoro svolto.
		\\
		Tale rapporto permette di arrivare ad una collaborazione per quanto riguarda:
		\begin{itemize}
			\item determinare gli aspetti chiave che il prodotto dovrà soddisfare;
			\item stilare requisiti e vincoli sui processi;
			\item stimare le tempistiche di lavoro;
			\item chiarire eventuali dubbi emersi;
			\item stimare i costi;
			\item acordarsi sulla qualifica del prodotto.
		\end{itemize}
		\subsubsection{Descrizione}
		Questo documento tratta le norme a cui membri del gruppo \textit{DPCM 2077} devono attenersi per diventare fornitori di \textit{Imola Informatica S.p.A} e dei committenti \textit{Prof. Tullio Vardanega} e \textit{Prof. Riccardo Cardin}. Ogni membro è tenuto ad attenersi a quanto esposto per tutte le fasi di progettazione, sviluppo e consegna del prodotto.
		\subsubsection{Attività}
			\paragraph{Studio di Fattibilità}
			In seguito alla presentazione ufficiale dei \glock{capitolati d'appalto}, il \textit{Responsabile di Progetto} organizza riunioni con i membri del gruppo \textit{DPCM 2077} al fine di permettere lo scambio di opinioni sui capitolati proposti. Lo \textit{Studio di Fattibilità}, redatto da ogni \textit{Analista}, indica per ogni \glock{capitolato}:
			\begin{itemize}
				\item \textbf{Informazioni generali}: vengono elencate le informazioni generali del \glock{capitolato}, quali nome del progetto, proponente e committente;
				\item \textbf{Descrizione del capitolato}: è una sintesi su cosa viene richiesto per lo sviluppo del \glock{capitolato};
				\item \textbf{Studio del dominio}: elenco di quali tecnologie, o applicazioni, sono richieste, o consigliate, per lo sviluppo;
				\item \textbf{Pro e Contro}: considerazioni del gruppo su fattori positivi o criticità del \glock{capitolato};
				\item \textbf{Conclusioni}: vengono esposte le ragione per il quale in gruppo ha accettato o rifiutato il \glock{capitolato};
			\end{itemize}
			\paragraph{Piano di Progetto}
			Il \glock{responsabile}, che ne ha in capo la redazione, redige un \textit{Piano di Progetto} da seguire durante	il corso del progetto.
			\\
			Il documento è composto in questo modo: 
			\begin{itemize}
				\item 
			\end{itemize}
			\paragraph{Piano di Qualifica}
			I \glock{verificatori} dovranno occuparsi del documento \textit{Piano di Qualifica}, che contiene le linee guida da adottare per garantire la qualità del materiale prodotto dal gruppo.
			\\
			Il piano è così suddiviso:			 
			\begin{itemize}
				\item 
			\end{itemize}
	\subsection{Sviluppo}
		\subsubsection{Scopo}
		Il processo di sviluppo contiene le attività e i compiti da svolgere, relative al prodotto da sviluppare.
		\subsubsection{Aspettative}
		Le aspettative, riguardo il processo in questione, sono le seguenti:
		\begin{itemize}
			\item stabilire gli obiettivi di sviluppo;
			\item stabilire i vincoli tecnologici;
			\item stabilire i vincoli di design;
			\item realizzare un prodotto finale che superi i test, che soddisfi i requisiti e le richieste del \glock{proponente}.
		\end{itemize}
		\subsubsection{Descrizione}
		Il processo di sviluppo, secondo lo standard \glock{ISO/IEC 12207:1995}, si articola in:
		\begin{itemize}
			\item \textit{Analisi dei requisiti};
			\item Progettazione;
			\item Codifica.
		\end{itemize}
		\subsubsection{Attività}
		Di seguito verranno analizzate le varie sezioni elencate precedentemente in modo dettagliato.
			\paragraph{Analisi dei requisiti}
				\subparagraph{Scopo}
				Gli analisti hanno in capo la stesura delle \textit{Analisi dei Requisiti}, dove il suo scopo è definire ed elencare tutti i requisiti del \glock{capitolato}.
				\\ 
				Il documento conterrà:
				\begin{itemize}
					\item descrizione generale del prodotto;
					\item fornire ai \glock{proggettisti} riferimenti precisi ed affidabili;
					\item funzionalità e requisiti concordati con il cliente;
					\item casi d'uso rappresentati tramite diagrammi \glock{UML};	
					\item tracciamento di riferimenti sulla mole di lavoro per poter eseguire una stima dei costi.			
				\end{itemize}
				\subparagraph{Aspettative}
				L'obiettivo dell'attività di analisi consiste nella creazione della documentazione formale contenente tutti i requisiti richiesti dal proponente.
				\subparagraph{Requisiti}
				Al fine di facilitarne la lettura e comprensione, ogni requisito viene classificato univocamente come segue:				
				\begin{center}
					\textbf{R[Priorità][Tipologia][Codice]}
				\end{center}
				Dove:
				\begin{itemize}
					\item \textbf{R}: abbreviazione di requisito;
					\item \textbf{Priorità}: ogni requisito assumerà uno dei seguenti valori:
					\begin{itemize}
						\item \textbf{\textit{1}}: requisito obbligatorio, ovvero irrinunciabile;
						\item \textbf{\textit{2}}: requisito desiderabile, non strettamente necessario ma a valore aggiunto;
						\item \textbf{\textit{3}}: requisito opzionale, ovvero relativamente utile o contrattabile in corso d'opera.
					\end{itemize}
					\item \textbf{Tipologia} : ogni requisito assumerà uno dei seguenti valori:
					\begin{itemize}
						\item \textbf{\textit{F}}: funzionale;
						\item \textbf{\textit{P}}: prestazionale;
						\item \textbf{\textit{Q}}: qualitativo;
						\item \textbf{\textit{V}}: vincolo.
					\end{itemize}
					\item \textbf{Codice}: codice identificativo univoco in forma gerarchica padre/figlio rappresentato come segue:
					\begin{center}
						\textbf{[padre].[figlio]}
					\end{center}
				\end{itemize}				
				\subparagraph{Casi d'uso}
				La convenzione scelta per la rappresentazione univoca dei casi d'uso è la seguente:
				\begin{center}
					\textbf{UC[codicePadre].[codiceFiglio]}
				\end{center}					
				Dove:
				\begin{itemize}
					\item \textbf{UC}: acronimo utilizzato per \textit{Use Case};
					\item \textbf{codicePadre}: identificativo univoco del caso d'uso;
					\item \textbf{codiceFiglio}: identificativo dei sottocasi, può contenere altri sottolivelli.
				\end{itemize}
				Ogni caso d'uso deve contenere, integralmente o parzialmente, i seguenti campi:
				\begin{itemize}
					\item diagramma \glock{UML};
					\item attori primari;
					\item attori secondari;
					\item scenario principale;
					\item scopo e descrizione;
					\item estensioni (se presenti);
					\item inclusioni (se presenti);
					\item precondizione;
					\item postcondizione.
				\end{itemize}
				\subparagraph{UML}
				I diagrammi \glock{UML} dovranno essere realizzati con la versione 2.0 del lingiaggio.
			\paragraph{Progettazione}	
				\subparagraph{Scopo}
				L'attività di progettazione individua, una volta concluso il documento \textit{Analisi dei Requisiti}, le caratteristiche del prodotto richiesto e la soluzione migliore adottabile che soddisfi gli \glock{stakeholders}. 
				\subparagraph{Aspettative}
				Realizzazione dell'architettura del sistema.
				\subparagraph{Requisiti}
				Si divide nelle seguenti fasi:
				\begin{itemize}
					\item \textbf{Tecnology baseline}: contiene le specifiche della progettazione ad alto livello del prodotto e delle sue componenti, l'elenco dei diagrammi \glock{UML} che saranno utilizzati per la realizzazione dell'architettura e i test di verifica;
					\item \textbf{Product baseline}: specifica ulteriormente l'attività di progettazione, integrando ciò che è riportato nella \textit{Tecnology baseline}. Definisce inoltre i test necessari alla verifica;
					\item \textbf{Diagrammi \glock{UML}}:
					\begin{itemize}
						\item diagrammi delle classi;
						\item diagrammi dei package;
						\item diagrammi delle attività;
						\item diagrammi di sequenza.
					\end{itemize}
					\item \textbf{Tecnologie utilizzate}: descrivono le tecnologie utilizzate specificandone il loro utilizzo all'interno del progetto.
				\end{itemize}			
			\paragraph{Codifica}	
				\subparagraph{Scopo}
				La codifica, svolta dal \glock{programmatore}, ha come scopo quello di normare l'effettiva realizzazione del prodotto software richiesto. I \glock{programmatori} dovranno attenersi a queste norme durante le fasi di programmazione ed implementazione.
				\subparagraph{Aspettative}
				L'obiettivo dell'attività è la creazione di un prodotto software che rispetti le richieste stabilite con il proponente. L'uso di norme e convenzioni in questa fase è fondamentale per permettere la generazione di codice leggibile ed uniforme, agevolare le fasi di manutenzione e verificare, validare e migliorare la qualità del prodotto.
				\subparagraph{Descrizione}
				La scrittura del codice deve rispettare quanto stabilito nella documentazione di prodotto. Dovrà perseguire gli obiettivi descritti all'interno del documento \dext{Piano di Qualifica v1.0.0} al fine di garantire la scrittura di codice di qualità.
				\subparagraph{Stile di codifica}
				Al fine di garantire uniformità nella generazione del codice del progetto, ciascun memnro del gruppo è tenuto a rispettare le seguenti norme:
				\begin{itemize}
					\item \textbf{Identazione}: i blocchi innestati devono seguire una corretta identazione, usando per ciascun livello di identazione quattro (4) spazi (fatta eccezione per i commenti). Al fine di assicurare il rispetto di questa regola si consiglia di configuare adeguatamente il proprio \glock{IDE};
					\item \textbf{Parentesi}: inserimento delle parentesi di delimitazione dei costrutti in linea e non al di sotto di essi; 
					\item \textbf{Struttura dei metodi}: è desiderabile, ove possibile, avere i metodi strutturati in questo modo:
					\begin{itemize}
						\item il nome deve iniziare con una lettera minuscola. Se il metodo è composto da più parole, quelle successive devono iniziare con la lettera maiuscola;
						\item lunghezza del metodo breve;
						\item la parentesi graffa di apertura deve essere nella stessa riga del metodo, separata da una singola spaziatura dopo il nome del metodo.
					\end{itemize}
					\item \textbf{Univocità dei nomi}: classi, metodi, variabili devono avere un nome univoco ed esplicativo al fine di evitare ambiguità ed incomprensioni;
					\item \textbf{Classi}: i nomi delle classi devono iniziare con la lettera maiuscola. Se sono composte da più parole, le successive avranno la lettera maiuscola;
					\item \textbf{Costanti}: i nomi delle costati devono essere scritte totalmente in maiuscolo. Se sono composte da più parole, saranno separate dal carattere \textit{underscore (\_)};
					\item \textbf{Lingua}: i nomi delle variabili, metodi, classi, costruttori e commenti saranno scritte in inglese.
				\end{itemize}
				\subparagraph{Ricorsione}
				L'uso della ricorsione va evitato quanto più possibile in quanto potrebbe indurre ad una maggiore occupazione di memoria rispetto a soluzioni iterative.
		 		
