\section{Processi Primari}
	\subsection{Fornitura}
		\subsubsection{Scopo}
		Lo scopo del processo di fornitura è di determinare le procedure e le risorse necessarie allo
		svolgimento del \glock{progetto}. È necessario che il \glock{fornitore} esegua un'attività di analisi e successiva redazione dello Studio di Fattibilità: un documento che comprende le richieste del \glock{proponente}, individua i potenziali rischi e criticità del C1.
		È inoltre necessario stipulare e concordare con il proponente un contratto per la consegna e manutenzione del prodotto. 
		Si passa dunque alla determinazione delle procedure e risorse necessarie, che andranno a confluire nel Piano di Progetto che porrà le basi per la consegna e realizzazione del prodotto.
		\\
		Il \glock{processo} di fornitura è composto dalle seguenti attività:
		\begin{itemize}
			\item avvio;
			\item approfondimento di risposte alle richieste;
			\item contrattazione;
			\item pianificazione;
			\item esecuzione e controllo;
			\item revisione e valutazione;
			\item consegna e completamento.
		\end{itemize}
		\subsubsection{Aspettative}
		Durante lo svolgimento dell'intero progetto, DPCM 2077 intende mantenere un costante dialogo con il proponente, per avere un riscontro efficace sul lavoro svolto.
		\\
		Tale dialogo permette di arrivare a una collaborazione per quanto riguarda:
		\begin{itemize}
			\item determinare gli aspetti chiave che il prodotto dovrà soddisfare;
			\item stilare requisiti e vincoli sui processi;
			\item stimare le tempistiche di lavoro;
			\item chiarire eventuali dubbi emersi;
			\item stimare i costi;
			\item accordarsi sulla qualifica del prodotto.
		\end{itemize}
		\subsubsection{Descrizione}
		Questo documento tratta le norme a cui i membri del gruppo DPCM 2077 devono attenersi per diventare fornitori di Imola Informatica S.p.A e dei committenti Prof.~Tullio Vardanega e Prof.~Riccardo Cardin. Ogni membro è tenuto ad attenersi a quanto esposto durante la progettazione, lo sviluppo e la consegna del prodotto.
		\subsubsection{Attività}
			\paragraph{Studio di Fattibilità}
			In seguito alla presentazione ufficiale dei \glock{capitolati d'appalto}, il Responsabile di Progetto organizza riunioni con i membri del gruppo DPCM 2077 al fine di permettere lo scambio di opinioni sui capitolati proposti. Lo Studio di Fattibilità, redatto dall'Analista, indica per ogni capitolato:
			\begin{itemize}
				\item \textbf{Informazioni generali}: vengono elencate le informazioni generali del capitolato, quali nome del progetto, proponente e committente;
				\item \textbf{Descrizione del capitolato}: è una sintesi su cosa viene richiesto per lo sviluppo del capitolato;
				\item \textbf{Studio del dominio}: elenco di quali tecnologie, o applicazioni, sono richieste, o consigliate, per lo sviluppo;
				\item \textbf{Pro e Contro}: considerazioni del gruppo su fattori positivi o criticità del capitolato;
				\item \textbf{Conclusioni}: vengono esposte le ragione per il quale il gruppo ha accettato o rifiutato il capitolato.
			\end{itemize}
			\paragraph{Piano di Progetto}
			Il \glock{responsabile}, che ne ha in capo la redazione, redige un Piano di Progetto da seguire durante	il corso del progetto.
			\\
			Il documento è composto in questo modo: 
			\begin{itemize}
				\item \textbf{Analisi dei rischi}: vengono analizzati i rischi che potranno presentarsi. Viene anche analizzato la frequenza con cui potrà presentarsi il rischio con la relativa gravità;
				\item \textbf{Modello di sviluppo}: viene descritto il modello di sviluppo;
				\item \textbf{Pianificazione}: vengono pianificate le attività da eseguire durante il progetto e le rispettive scadenze;
				\item \textbf{Preventivo}: viene data una stima del lavoro necessario per ciascuna fase proponendo un preventivo per il costo totale;
				\item \textbf{Consuntivo}: viene tracciato un consuntivo di periodo relativo all'andamento rispetto a ciò che è stato preventivato.
			\end{itemize}
			\paragraph{Piano di Qualifica}
			I \glock{verificatori} dovranno occuparsi del documento Piano di Qualifica, che contiene le linee guida da adottare per garantire la qualità del materiale prodotto dal gruppo.
			\\
			Il piano è così suddiviso:			 
			\begin{itemize}
				\item \textbf{Qualità di processo}: vengono identificati gli standard dei processi, stabiliti degli obiettivi, escogitate delle strategie per attuarli e individuate le metriche per misurarli e controllarli;
				\item \textbf{Qualità di prodotto}: vengono identificati gli attributi più rilevanti per il prodotto, definiti degli obiettivi per raggiungerli e delle metriche per misurarli;
				\item \textbf{Specifiche dei test}: definiscono una serie di test attraverso i quali il prodotto passa per garantire che soddisfi i requisiti;
				\item \textbf{Standard di qualità}: vengono esposti gli standard di qualità scelti;
				\item \textbf{Valutazioni per il miglioramento}: vengono riportati i problemi e le relative soluzioni nel ricoprire un determinato ruolo e nell'uso degli strumenti scelti;
				\item \textbf{Resoconto delle attività di verifica}: per ogni attività si riportano i risultati delle metriche calcolate in forma di resoconto.
			\end{itemize}
		\subsubsection{Strumenti}
			\paragraph{Fogli Google}
			Software per la creazione di fogli elettronici utilizzato all'interno del Drive del gruppo. Usato per fare calcoli, produrre diagrammi, istogrammi e creare tabelle.
			\paragraph{Gannt Project}
			Utilizzato dai responsabili di progetto nella pianificazione, nell'assegnazione delle risorse, nella verifica del rispetto dei tempi attraverso la creazione di diagrammi di Gantt.
	\subsection{Sviluppo}
		\subsubsection{Scopo}
		Il processo di sviluppo contiene le attività e i compiti da svolgere, relative al prodotto da sviluppare.
		\subsubsection{Aspettative}
		Le aspettative, riguardo il processo in questione, sono le seguenti:
		\begin{itemize}
			\item stabilire gli obiettivi di sviluppo;
			\item stabilire i vincoli tecnologici;
			\item stabilire i vincoli di design;
			\item realizzare un prodotto finale che superi i test, che soddisfi i requisiti e le richieste del proponente.
		\end{itemize}
		\subsubsection{Descrizione}
		Il processo di sviluppo, secondo lo standard ISO/IEC 12207:1995, si articola in:
		\begin{itemize}
			\item Analisi dei requisiti;
			\item Progettazione;
			\item Codifica.
		\end{itemize}
		\subsubsection{Attività}
		Di seguito verranno analizzate le varie sezioni elencate precedentemente in modo dettagliato.
			\paragraph{Analisi dei requisiti}
				\subparagraph{Scopo}
				Gli analisti hanno in capo la stesura dell'Analisi dei Requisiti, il cui scopo è elencare e definire tutti i requisiti del C1.
				\\ 
				Il documento conterrà:
				\begin{itemize}
					\item descrizione generale del prodotto;
					\item fornire ai \glock{progettisti} riferimenti precisi ed affidabili;
					\item funzionalità e requisiti concordati con il cliente;
					\item casi d'uso rappresentati tramite diagrammi \glock{UML};	
					\item calcolo della mole di lavoro per poter eseguire una stima dei costi.			
				\end{itemize}
				\subparagraph{Aspettative}
				L'obiettivo dell'attività di analisi consiste nella creazione della documentazione formale contenente tutti i requisiti richiesti dal proponente.
				\subparagraph{Requisiti}
				Al fine di facilitarne la lettura e comprensione, ogni requisito viene classificato univocamente come segue:				
				\begin{center}
					\textbf{R[Priorità][Tipologia][Codice]}
				\end{center}
				Dove:
				\begin{itemize}
					\item \textbf{R}: abbreviazione di requisito;
					\item \textbf{Priorità}: ogni requisito assumerà uno dei seguenti valori:
					\begin{itemize}
						\item \textbf{\textit{1}}: requisito obbligatorio, ossia irrinunciabile;
						\item \textbf{\textit{2}}: requisito desiderabile, non strettamente necessario ma a valore aggiunto;
						\item \textbf{\textit{3}}: requisito opzionale, ossia relativamente utile o contrattabile in corso d'opera.
					\end{itemize}
					\item \textbf{Tipologia} : ogni requisito assumerà uno dei seguenti valori:
					\begin{itemize}
						\item \textbf{\textit{F}}: funzionale;
						\item \textbf{\textit{P}}: prestazionale;
						\item \textbf{\textit{Q}}: qualitativo;
						\item \textbf{\textit{V}}: vincolo.
					\end{itemize}
					\item \textbf{Codice}: codice identificativo univoco in forma gerarchica padre/figlio rappresentato come segue:
					\begin{center}
						\textbf{[padre].[figlio]}
					\end{center}
				\end{itemize}				
				\subparagraph{Casi d'uso}
				La convenzione scelta per la rappresentazione univoca dei casi d'uso è la seguente:
				\begin{center}
					\textbf{UC[codicePadre].[codiceFiglio]}
				\end{center}					
				Dove:
				\begin{itemize}
					\item \textbf{UC}: acronimo utilizzato per \textit{Use Case};
					\item \textbf{codicePadre}: identificativo univoco del caso d'uso;
					\item \textbf{codiceFiglio}: identificativo dei sottocasi, può contenere altri sottolivelli.
				\end{itemize}
				Ogni caso d'uso deve contenere, integralmente o parzialmente, i seguenti campi:
				\begin{itemize}
					\item diagramma UML;
					\item attori primari;
					\item attori secondari;
					\item scenario principale;
					\item scopo e descrizione;
					\item estensioni (se presenti);
					\item inclusioni (se presenti);
					\item precondizione;
					\item postcondizione.
				\end{itemize}
				\subparagraph{UML}
				I diagrammi UML dovranno essere realizzati con la versione 2.0 del linguaggio.
			\paragraph{Progettazione}	
				\subparagraph{Scopo}
				L'attività di progettazione individua, una volta concluso il documento Analisi dei Requisiti, le caratteristiche del prodotto richiesto e la soluzione migliore adottabile che soddisfi gli \glock{stakeholders}. 
				\subparagraph{Aspettative}
				Realizzazione dell'architettura del sistema.
				\subparagraph{Requisiti}
				Si divide in:
				\begin{itemize}
					\item \textbf{Technology baseline}: contiene le specifiche della progettazione ad alto livello del prodotto e delle sue componenti, l'elenco dei diagrammi UML che saranno utilizzati per la realizzazione dell'architettura e i test di verifica;
					\item \textbf{Product baseline}: specifica ulteriormente l'attività di progettazione, integrando ciò che è riportato nella \textit{Technology baseline}. Definisce inoltre i test necessari alla verifica;
					\item \textbf{Diagrammi UML}:
					\begin{itemize}
						\item diagrammi delle classi;
						\item diagrammi dei package;
						\item diagrammi delle attività;
						\item diagrammi di sequenza.
					\end{itemize}
					\item \textbf{Tecnologie utilizzate}: descrivono le tecnologie utilizzate specificandone il loro utilizzo all'interno del progetto.
				\end{itemize}			
			\paragraph{Codifica}	
				\subparagraph{Scopo}
				La codifica, svolta dal \glock{programmatore}, ha come scopo quello di normare l'effettiva realizzazione del prodotto software richiesto. I \glock{programmatori} dovranno attenersi a queste norme durante la programmazione e l'implementazione.
				\subparagraph{Aspettative}
				L'obiettivo dell'attività è la creazione di un prodotto software che rispetti le richieste stabilite con il proponente. L'uso di norme e convenzioni in questa fase è fondamentale per permettere la generazione di codice leggibile ed uniforme, agevolare la manutenzione e verificare, validare e migliorare la qualità del prodotto.
				\subparagraph{Descrizione}
				La scrittura del codice deve rispettare quanto stabilito nella documentazione di prodotto. Dovrà perseguire gli obiettivi descritti all'interno del documento \dext{Piano di Qualifica v2.0.0} al fine di garantire la scrittura di codice di qualità.
				\subparagraph{Stile di codifica}
				Al fine di garantire uniformità nella generazione del codice del progetto, ciascun membro del gruppo è tenuto a rispettare le seguenti norme:
				\begin{itemize}
					\item \textbf{Indentazione}: i blocchi innestati devono seguire una corretta indentazione, usando per ciascun livello di indentazione quattro (4) spazi (fatta eccezione per i commenti). Al fine di assicurare il rispetto di questa regola si consiglia di configurare adeguatamente il proprio \glock{IDE};
					\item \textbf{Parentesi}: inserimento delle parentesi di delimitazione dei costrutti in linea e non al di sotto di essi; 
					\item \textbf{Struttura dei metodi}: è desiderabile, ove possibile, avere i metodi strutturati in questo modo:
					\begin{itemize}
						\item il nome deve iniziare con una lettera minuscola. Se il metodo è composto da più parole, quelle successive devono iniziare con la lettera maiuscola;
						\item lunghezza del metodo breve;
						\item la parentesi graffa di apertura deve essere nella stessa riga del metodo, separata da una singola spaziatura dopo il nome del metodo.
					\end{itemize}
					\item \textbf{Univocità dei nomi}: classi, metodi, variabili devono avere un nome univoco ed esplicativo al fine di evitare ambiguità ed incomprensioni;
					\item \textbf{Classi}: i nomi delle classi devono iniziare con la lettera maiuscola. Se sono composte da più parole, le successive avranno la lettera maiuscola;
					\item \textbf{Costanti}: i nomi delle costanti devono essere scritte totalmente in maiuscolo. Se sono composte da più parole, saranno separate dal carattere \textit{underscore (\_)};
					\item \textbf{Lingua}: i nomi delle variabili, metodi, classi, costruttori e commenti saranno scritte in inglese.
				\end{itemize}
				\subparagraph{Ricorsione}
				L'uso della ricorsione va evitato quanto più possibile in quanto potrebbe indurre a una maggiore occupazione di memoria rispetto a soluzioni iterative.
		 		
