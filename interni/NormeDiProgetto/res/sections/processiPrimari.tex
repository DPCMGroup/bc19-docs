\section{Processi Primari}
	\subsection{Fornitura}
		\subsubsection{Scopo}
		Lo scopo del processo di fornitura è di determinare le procedure e le risorse necessarie allo
		svolgimento del progetto. Il \glock{fornitore} esegue un'attività di analisi e successiva redazione dello \textit{Studio di Fattibilità}: un documento che comprende le richieste del \glock{proponente}, individua i potenziali rischi e criticità del \glock{capitolato}.
		È inoltre necessario stipulare e concordare con il \glock{proponente} un contratto per la consegna e manuntenzione del prodotto. 
		Si passa dunque alla determinazione delle procedure e risorse necessarie, che andranno a confluire nel \textit{Piano di Progetto} che porrà le basi per la consegna e realizzazione del prodotto.
		\\
		Il processo di fornitura è composto dalle seguenti fasi:
		\begin{itemize}
			\item avvio;
			\item approfondimento di risposte alle richieste;
			\item contrattazione;
			\item pianificazione;
			\item esecuzione e controllo;
			\item revisione e valutazione;
			\item consegna e completamento.
		\end{itemize}
		\subsubsection{Aspettative}
		Durante lo svolgimento dell'intero progetto, \textit{DPCM 2077} intende mantenere un costante dialogo con il \glock{proponente}, per avere un riscontro efficace sul lavoro svolto.
		\\
		Tale rapporto permette di arrivare ad una collaborazione per quanto riguarda:
		\begin{itemize}
			\item determinare gli aspetti chiave che il prodotto dovrà soddisfare;
			\item stilare requisiti e vincoli sui processi;
			\item stimare le tempistiche di lavoro;
			\item chiarire eventuali dubbi emersi;
			\item stimare i costi;
			\item acordarsi sulla qualifica del prodotto.
		\end{itemize}
		\subsubsection{Descrizione}
		Questo documento tratta le norme a cui membri del gruppo \textit{DPCM 2077} devono attenersi per diventare fornitori di \textit{Imola Informatica S.p.A} e dei committenti \textit{Prof. Tullio Vardanega} e \textit{Prof. Riccardo Cardin}. Ogni membro è tenuto ad attenersi a quanto esposto per tutte le fasi di progettazione, sviluppo e consegna del prodotto.
		\subsubsection{Attività}
			\paragraph{Studio di Fattibilità}
			In seguito alla presentazione ufficiale dei \glock{capitolati d'appalto}, il \textit{Responsabile di Progetto} organizza riunioni con i membri del gruppo \textit{DPCM 2077} al fine di permettere lo scambio di opinioni sui capitolati proposti. Lo \textit{Studio di Fattibilità}, redatto da ogni \textit{Analista}, indica per ogni \glock{capitolato}:
			\begin{itemize}
				\item \textbf{Informazioni generali}: vengono elencate le informazioni generali del \glock{capitolato}, quali nome del progetto, proponente e committente;
				\item \textbf{Descrizione del capitolato}: è una sintesi su cosa viene richiesto per lo sviluppo del \glock{capitolato};
				\item \textbf{Studio del dominio}: elenco di quali tecnologie, o applicazioni, sono richieste, o consigliate, per lo sviluppo;
				\item \textbf{Fattori positivi e criticità}: considerazioni del gruppo su fattori positivi o criticità del \glock{capitolato};
				\item \textbf{Conclusioni}: vengono esposte le ragione per il quale in gruppo ha accettato o rifiutato il \glock{capitolato};
			\end{itemize}
			\paragraph{Piano di Progetto}
			\paragraph{Piano di Qualifica}
		\subsubsection{Strumenti}
			\paragraph{Cosa abbiamo utilizzato}
	\subsection{Sviluppo}
		\subsubsection{Scopo}
		\subsubsection{Aspettative}
		\subsubsection{Descrizione}
		\subsubsection{Attività}
			\paragraph{Analisi dei requisiti}
			\paragraph{Progettazione}
			\paragraph{Codifica}
		\subsubsection{Strumenti}
			\paragraph{Cosa abbiamo utilizzato}
		 