\section{Introduzione}

\subsection{Scopo del Documento}
Lo scopo di questo  documento è fissare  tutte  le  regole  e  le  procedure fondamentali per assicurare al gruppo un modo di lavorare comune in modo da   
garantire una collaborazione efficiente tra tutti i diversi membri  del  gruppo; In quest’ottica, questo documento deve essere visionato da tutti i componenti del gruppo, che 
sono obbligati ad applicare quanto scritto al fine di mantenere omogeneità e coesione in ogni aspetto del progetto. Saranno inoltre elencati e discussi tutti gli   
strumenti software scelti internamente al gruppo per rispettare le regole e attuare le procedure.

\subsection{Scopo del Prodotto}
Lo scopo del prodotto è quello di sviluppare un’applicazione in grado di
segnalare ad un \glock{server} dedicato la presenza di un \glock{utente} su una determinata postazione appartenente ad
una stanza. 
Nel server deve essere possibile gestire più stanze e postazioni per:
\begin{itemize}
\item{sapere in ogni momento se la postazione è occupata, prenotata oppure da pulire;}
\item{controllare quali postazioni sono prenotate e bloccare le prenotazioni per una determinata
stanza (e.g. all'esaurimento dei posti disponibili);}
\item{prevedere una tracciatura autenticata e tutti i cambiamenti di stato relativi alla pulizia della
postazione, nonché le informazioni su chi ha igienizzato la postazione;}
\end{itemize}
L'applicazione dovrà essere in grado di svolgere i seguenti compiti:
\begin{itemize}
\item{recupero lista delle postazioni libere;}
\item{prenotazione di una postazione;}
\item{tracciamento in tempo reale tramite \glock{tag } \glock{RFID};}
\item{pulizia di una postazione;}
\item{storico delle postazioni occupate e igienizzate;}
\end{itemize}
\glock{Client} e Server comunicano tra loro nel momento in cui lo smartphone viene  a contatto con il tag RFID

\subsection{Glossario}
All’interno del  documento sono presenti termini che presentano significati ambigui a seconda del contesto.
Per evitare questa ambiguità è stato creato un  documento di nome Glossario che  conterrà tali termini con il loro significato specifico. Per segnalare che un termine del testo è presente all’ interno del Glossario  
verrà aggiunta una G pedice posta a fianco del termine ambiguo. 

\subsection{Riferimenti}
\subsubsection{Riferimenti Normativi}
\begin{itemize}
\item{Standard ISO/IEC 12207:1995: \\
\url{https://www.math.unipd.it/~tullio/IS-1/2009/Approfondimenti/ISO_12207-1995.pdf}}
\item{Capitolato \\
\url{https://www.math.unipd.it/~tullio/IS-1/2020/Progetto/C1.pdf}}
\item{Verbale di incontro esterno con il proponente \glock{Imola Informatica} del 20-12-18}
\end{itemize}

\subsubsection{Riferimenti Informativi}
\begin{itemize}
\item{Sito ufficiale di \glock{git} \\
\url{https://git-scm.com/}}
\item{Documentazione ufficiale di \glock{git} \\
\url{https://git-scm.com/doc}}
\item{L’Arte di Scrivere con \LaTeX \\
\url{http://www.lorenzopantieri.net/LaTeX_files/ArteLaTeX.pdf}}
\item{Libro di Ingegneria del Software: Software Engineering - Ian Sommerville}
\end{itemize}
