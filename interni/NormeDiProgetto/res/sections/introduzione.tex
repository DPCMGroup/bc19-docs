\section{Introduzione}

\subsection{Scopo del Documento}
Lo scopo di questo documento è fissare tutte le  regole e le \glock{procedure} fondamentali per assicurare al \glock{gruppo} DPCM 2077 un modo di lavorare comune che garantisca quantificabilità, efficacia ed efficienza.
In quest’ottica, questo documento deve essere visionato da tutti i componenti di DPCM 2077 a ogni sua approvazione. 
I membri di DPCM 2077 sono obbligati ad applicare quanto scritto al fine di mantenere omogeneità e coesione in ogni aspetto del progetto. Saranno inoltre elencati e discussi tutti gli   
strumenti software scelti internamente al gruppo per rispettare le regole e attuare le procedure.

\subsection{Composizione del Documento}
La struttura del documento si ispira alla suddivisione dei \glock{processi} in tre tipologie desunta dallo standard \dext{ISO/IEC 12207:1995}:
\begin{itemize}
\item{\textbf{Primari}: acquisizione, fornitura, sviluppo, gestione operativa e manutenzione;\footnote{nel presente documento vengono trattati solamente \textit{fornitura} e \textit{sviluppo}}}
\item{\textbf{Supporto}: documentazione, gestione della configurazione, accertamento della qualità, verifica, validazione, revisioni congiunte con il cliente, verifiche ispettive interne e risoluzione dei problemi;}
\item{\textbf{Organizzativi}: gestione dei processi, gestione delle infrastrutture tecniche, miglioramento del processo e formazione del personale.}
\end{itemize}
\subsection{Scopo del Prodotto}
Lo scopo del prodotto è di sviluppare un’\glock{applicazione} in grado di
segnalare a un \glock{server} dedicato la presenza di un \glock{utente} su una determinata \glock{postazione} appartenente a
una stanza. 
Nel server deve essere possibile gestire più stanze e postazioni per:
\begin{itemize}
\item{sapere in ogni momento se la postazione è occupata, prenotata oppure da pulire;}
\item{controllare quali postazioni sono prenotate e bloccare le prenotazioni per una determinata
stanza (e.g. all'esaurimento delle postazioni disponibili);}
\item{garantire una \glock{tracciatura} autenticata delle presenze e la registrazione di tutti i cambiamenti di stato relativi alla pulizia della
postazione, nonché le informazioni su chi ha igienizzato la postazione.}
\end{itemize}
L'applicazione dovrà essere in grado di svolgere i seguenti compiti:
\begin{itemize}
\item{recupero lista delle postazioni libere;}
\item{prenotazione di una postazione;}
\item{tracciatura delle postazioni in tempo reale tramite \glock{tag } \glock{RFID};}
\item{pulizia di una postazione (per pulizia si intende segnare come igienizzata una postazione)};
\item{generare uno storico delle postazioni occupate e igienizzate.}
\end{itemize}
\glock{Client} e server comunicano tra loro nel momento in cui lo smartphone viene  a contatto con il tag RFID.

\subsection{Glossario}
All’interno del  documento sono presenti termini che presentano significati ambigui a seconda del contesto.
Per evitare questa ambiguità è stato creato un  documento di nome Glossario che  conterrà tali termini con il loro significato specifico. Per segnalare che un termine del testo è presente all’ interno del Glossario  
verrà aggiunta una G pedice posta a fianco del termine ambiguo. 
Se viene indicata una D pedice significa che si sta facendo riferimento a un documento.

\subsection{Riferimenti}
\subsubsection{Riferimenti Normativi}
\begin{itemize}
\item{Capitolato \\
\url{https://www.math.unipd.it/~tullio/IS-1/2020/Progetto/C1.pdf}}
\item{VE\_2020\_12\_18}
\end{itemize}

\subsubsection{Riferimenti Informativi}
\begin{itemize}
\item{Standard ISO/IEC 12207:1995 \\
\url{https://www.math.unipd.it/~tullio/IS-1/2009/Approfondimenti/ISO_12207-1995.pdf}}
\item{Sito ufficiale di \glock{git} \\
\url{https://git-scm.com/}}
\item{Documentazione ufficiale di Git} \\
\url{https://git-scm.com/doc}
\item{L’Arte di Scrivere con \LaTeX \\
\url{http://www.lorenzopantieri.net/LaTeX_files/ArteLaTeX.pdf}}
\item{Libro di Ingegneria del Software: Software Engineering - Ian Sommerville}
\item{Sito ufficiale di Java} \\
\url{https://www.java.com/it/}
\item{Sito ufficiale di Python} \\
\url{https://www.python.org/}
\item{Sito ufficiale di NodeJs} \\
\url{https://nodejs.org/it/}
\end{itemize}
