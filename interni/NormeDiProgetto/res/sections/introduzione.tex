\section{Introduzione}

\subsection{Scopo del Documento}
Lo scopo di questo documento è fissare tutte le  regole e le procedure fondamentali per assicurare al \glock{gruppo} DPCM 2077 un modo di lavorare comune che garantisca quantificabilità, efficacia ed efficienza.
In quest’ottica, questo documento deve essere visionato da tutti i componenti di DPCM 2077 a ogni sua approvazione. 
I membri di DPCM 2077 sono obbligati ad applicare quanto scritto al fine di mantenere omogeneità e coesione in ogni aspetto del progetto. Saranno inoltre elencati e discussi tutti gli   
strumenti software scelti internamente al gruppo per rispettare le regole e attuare le procedure.

\subsection{Composizione del Documento}
La struttura del documento si ispira alla suddivisione dei \glock{processi} in tre tipologie desunta dallo standard ISO/IEC 12207:1995:
\begin{itemize}
\item{\textbf{Primari}: acquisizione, fornitura, sviluppo, gestione operativa e manutenzione;\footnote{nel presente documento vengono trattati solamente \textit{fornitura} e \textit{sviluppo}}}
\item{\textbf{Supporto}: documentazione, gestione della configurazione, accertamento della qualità, verifica, validazione, revisioni congiunte con il cliente, verifiche ispettive interne e risoluzione dei problemi;}
\item{\textbf{Organizzativi}: gestione dei processi, gestione delle infrastrutture tecniche, miglioramento del processo e formazione del personale.}
\end{itemize}

\subsection{Scopo del prodotto}
Il prodotto da sviluppare ha lo scopo di monitorare e regolare l'utilizzo e l'igienizzazione delle \glock{postazioni} all'interno di uno spazio condiviso, al fine di ridurre il rischio di trasmissione di un'infezione e di adempiere alle norme di legge. 
\subsection{Glossario e documenti} 
All'interno del  documento sono presenti termini che assumono significati diversi a seconda del contesto.
Per evitare ambiguità, è stato creato un  documento di nome Glossario che  conterrà tali termini con il loro significato specifico. Per segnalare che un termine del testo è presente all'interno del Glossario, verrà aggiunta una G pedice posta a fianco del termine ambiguo.
Quando si fa riferimento a un altro documento riguardante questo progetto vi si pone a pedice una D.

\subsection{Riferimenti}
\subsubsection{Riferimenti Normativi}
\begin{itemize}
\item{C1 \\
\url{https://www.math.unipd.it/~tullio/IS-1/2020/Progetto/C1.pdf}}
\item \dext{VE\_2021\_01\_22 v. 1.0.0}
\end{itemize}

\subsubsection{Riferimenti Informativi}
\begin{itemize}
\item \dext{Piano di Qualifica v. 2.0.0}
\item{Standard ISO/IEC 12207:1995 \\
\url{https://www.math.unipd.it/~tullio/IS-1/2009/Approfondimenti/ISO_12207-1995.pdf}}
\item{Standard ISO 8601 \\
	\url{https://www.iso.org/iso-8601-date-and-time-format.html}}
\item{Sito ufficiale di \glock{git} \\
\url{https://git-scm.com/}}
\item{Documentazione ufficiale di Git} \\
\url{https://git-scm.com/doc}
\item{L’Arte di Scrivere con \LaTeX \\
\url{http://www.lorenzopantieri.net/LaTeX_files/ArteLaTeX.pdf}}
\item{Libro di Ingegneria del Software: Software Engineering - Ian Sommerville - 10th edition}
\item{Sito ufficiale di Java} \\
\url{https://www.java.com/it/}
\item{Sito ufficiale di Python} \\
\url{https://www.python.org/}
\item{Sito ufficiale di NodeJs} \\
\url{https://nodejs.org/it/}
\end{itemize}
