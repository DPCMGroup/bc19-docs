\section{Processi di supporto}
	\subsection{Documentazione}
		\subsubsection{Scopo}
			Lo scopo di questa sezione è definire gli standard e le regole che riguardano la stesura dei documenti e l'approvazione di questi ultimi.
			I documenti sono consultabili nelle apposite sezioni della repository \url{https://github.com/DPCMGroup/dpcm2077-docs}.
		\subsubsection{Aspettative}
			Si vuole fornire uno strumento unico per tutto il gruppo per la stesura dei documenti,in modo da avere una documentazione uniforme e aderente agli standard e regole sotto riportate.
		\subsubsection{Descrizione}
			Questo capitolo fornisce i dettagli su come deve essere redatta, verificata, e approvata la documentazione. Tutte le norme descritte devono essere rispettate in pieno da tutti i documenti, sia interni che esterni, rilasciati durante il ciclo di vita del software.
		\subsubsection{Attività}
			\paragraph{Implementazione del processo}
				\subparagraph{Ciclo di vita}
					Ogni documento viene creato basandosi sul seguente ciclo di vita:
					\begin{itemize}
					\item \textbf{stesura:} il documento viene creato da uno o più redattori in base alle regole accordate. Una volta terminata la stesura il responsabile autorizzerà l'avanzata del documento all'attività successiva.
					\item \textbf{verifica:} viene effettutato da uno o più verificatori un controllo sul documento sia per quanto riguarda la stesura corretta, sia per un controllo semantico e sintattico seguendo le norme di progetto. Terminato il controllo il verificatore riferirà l'esito al responsabile. Se il documento dovesse essere corretto, il responsabile autorizzerà il passaggio del documento alla fase successiva, in caso contrario tornerà alla prima fase.
					\item \textbf{approvazione:} è l'ultima fase del ciclo di vita del documento. In questa fase l'approvatore conferma l'esito positivo al responsabile il quale rilascerà il documento. 
					\end{itemize}
				\subparagraph{Documenti ufficiali}
					Si dicono ufficiali i documenti che:
					\begin{itemize}
						\item sono stati revisionati, verificati ed approvati;
						\item sono gli unici rilasciabili all'esterno del gruppo di progetto.
					\end{itemize}
					
					I documenti ufficiali che saranno prodotti sono:
					\begin{itemize}
						\item \textbf{Studio di fattibilità:} ha l'obiettivo di descrivere in modo breve ogni capitolato, mostrano quelli che per noi componenti del gruppo possono essere punti di forza, aspetti interessanti, o criticità. I estinatari di questo documento sono i membri del gruppo DPCM 2077.
						\item \textbf{Piano di progetto:} stabilisce e documenta tutti gli aspetti del nostro progetto organizzandoli in modo da raggiungere i risultati nel modo più efficace ed efficiente possibile. Organizza le attività, ne analizza i rischi, e le associa ai membri del gruppo. I destinatari di questo documento sono il committente ed il proponente oltre al gruppo DPCM 2077. 
						\item \textbf{Piano di qualifica:} espone e descrive i criteri di valutazione della qualità adottati dal gruppo. I destinatari di questo documento sono il committente ed il proponente oltre al gruppo DPCM 2077. 
						\item \textbf{Analisi dei requisiti:} Ha lo scopo di definire le funzionalità che il nostro progetto deve offrire, e quindi i requisiti che devono essere soddisfatti. I destinatari di questo documento sono il committente ed il proponente oltre al gruppo DPCM 2077.
						\item \textbf{Verbali:} hanno lol scopo di riassumere in modo conciso i contenuti delle riunioni.
						\item \textbf{Glossario:} il documento è composto da un elenco ordinato di tutti i termini usati nella documentazione che il gruppo ritiene necessitino di una definizione esplicita, al fine di rimuovere ogni ambiguità; I destinatari di questo documento sono il committente ed il proponente oltre al gruppo DPCM 2077.
					\end{itemize}
					
			\paragraph{Sviluppo e design}
				\subparagraph{Template LaTeX}
					Il gruppo ha deciso di utilizzare un template LaTeX per la stesura di tutti i documenti in modo da garantire uniformità tra questi.	
				\subparagraph{Struttura generale dei documenti}		
					Un file "main.tex" provvederà a raccogliere tutte le sezioni, pacchetti e comandi necessari per la sua compilazione. Tutti i documenti hanno una struttura predefinita e determinata.
				\subparagraph{Frontespizio}	
					Il frontespizio conterrà il logo ed il nome del gruppo, il nome del documento, il nome del progetto al quale si riferisce e tutte le informazioni riguardanti il documento in sè quali:
					\begin{itemize}
						\item \textbf{versione:} versione attuale del documento;
						\item \textbf{uso:} destinazione d'uso del documento, che potrà essere "interno" o "esterno";
						\item \textbf{stato:} attuale stato del documento, che potrà essere "in redazione" o "approvato";
						\item \textbf{destinatari:} destinatari del documento;
						\item \textbf{redattori:} lista dei membri del gruppo che si sono occupati della stesura dello specifico documento;
						\item \textbf{verificatori:} lista dei membri del gruppo che si sono occupati della verifica dello specifico documento;
						\item \textbf{approvazione:} nominativo del membro del gruppo che ha approvato il documento per il rilascio.
					\end{itemize}
					
				\subparagraph{Registro modifiche}
					Il registro delle modifiche occupa la seconda pagina del documento e consiste in una tabella contenente le informazioni riguardanti il ciclo di vita del documento. Più precisamente, la tabella riporta per ogni modifica:
					\begin{itemize}
						\item \textbf{versione:} versione del documento relativa alla modifica effettuata;
						\item \textbf{descrizione:} breve descrizione della modifica effettuata;
						\item \textbf{data:} data in cui la modifica è stata effettuata;
						\item \textbf{autore:} nominativo della persona che ha effettuato la modifica;
						\item \textbf{ruolo:} ruolo della persona che ha effettuato la modifica.
					\end{itemize}
				\subparagraph{Indice}
				L’indice delle sezioni fornisce al fruitore una visione complessiva della struttura del documento, permette di orientarsi tra i contenuti e di individuare la posizione delle varieparti.Ogni documento presenta un indice dei contenuti, subito dopo il registro delle modifiche;
				\subparagraph{Verbali}
				I verbali sono suddivisi nelle seguenti sezioni:
				\begin{itemize}
						\item \textbf{introduzione:} essa contiene:
							\begin{itemize}
								\item \textbf{luogo:} la piattaforma online in cui si svolge l'incontro;
								\item \textbf{data:} data dell'incontro;
								\item \textbf{ora di inizio:} l'ora dell'inizio dell'incontro in formato HH:MM;
								\item \textbf{ora di fine:} l'ora della fine dell'incontro in formato HH:MM;
								\item \textbf{ordine del giorno:} consiste in una lista degli argomenti che il gruppo si è proposto di discutere durante l'incontro;
								\item \textbf{presenze:} contiene il numero totale dei partecipanti, la lista dei presenti e la lista degli assenti con eventuale giustifica;
							\end{itemize}
							\item \textbf{svolgimento:} per ogni punto presente nell'ordine del giorno, viene riportato un riassunto di ciò che è stato trattato durante l'incontro;
							\item \textbf{tracciamento delle decisioni:} è un riepilogo in formato tabellare delle decisioni prese durante l'incontro; esso è composto da:
							\begin{itemize}
								\item \textbf{codice:} del tipo ``AAAA-MM-GG\_ X.Y" dove la prima parte indica la data dell'incontro a cui fa riferimento seguita da un numero che indica il numero del verbale(X) ed un secondo numero che indica il punto all'ordine del giorno a cui si riferisce(Y);
								\item \textbf{descrizione:} breve descrizione riassuntiva della decisione presa riguardante il punto dell'ordine del giorno.
							\end{itemize}
					\end{itemize}
				\subparagraph{Glossario}
				Il glossario segue la struttura principale di tutti gli altri documenti. Il contenuto è composto da una lista, che segue l'ordine lessicografico, di termini seguiti dalla loro spiegazione. I termini che si trovano nel glossario sono presenti all'interno di almeno uno dei documenti e ne viene data la spiegazione in quanto trattano argomenti ambigui e/o di poca comprensione, oppure hanno natura specifica o sono acronimi. 
				\subparagraph{Lettere}
				La lettera di presentazione dovrà seguire il classico layout per lettere, il che implica la presenza dei mittenti e destinatari, il logo del gruppo e la lista di tutti i documenti rilasciati, nonché il preventivo per il progetto.
			\paragraph{Norme tipografiche}
				
				\subparagraph{Formati comuni}
					\begin{itemize}
						\item \textbf{data:} viene utilizzato lo standard ISO 8601, esempio 2020-01-22, per la rappresentazione di tutte le date presenti della documentazione;
						\item \textbf{ora:} viene utilizzato il formato HH:MM;
					\end{itemize}
				\subparagraph{Sigle}
				Il progetto prevede la redazione di un insieme di documenti, suddivisi in documenti interni e documenti esterni. Essi sono elencati di seguito con le rispettive sigle.\\
				I documenti esterni sono:
				\begin{itemize}
				\item \textbf{Analisi dei requisiti - Adr};
				\item \textbf{Piano di Progetto - PdP};
				\item \textbf{Piano di Qualifica - PdQ};
				\item \textbf{Manuale Utente - MU};
				\item \textbf{Manuale Sviluppatore - MS};
				\end{itemize}
				I documenti interni sono:
				\begin{itemize}
				\item \textbf{Glossario: G};
				\item \textbf{Norme di Progetto: NdP};
				\item \textbf{Studio di Fattibilità - StF};
				\end{itemize}
				Per quanto riguarda il \textbf{Verbale - V} questo può essere sia interno che esterno.
				Le diverse fasi del progetto sono: 
				\begin{itemize}
				\item \textbf{Revisione dei Requisiti - RR}: studio iniziale del capitolato;
				\item \textbf{Revisione di Progettazione - RP}: riguarda la definizione dell'architettura del software e della sua fattibilità;
				\item \textbf{Revisione di Qualifica -RQ}: riguarda la definizione dettagliata e la codifica del prodotto;
				\item \textbf{Revisione di Accettazione - RA}: se il prodotto soddisfa i requisiti del proponente, viene accettato e rilasciato.
	\subsubsection{Strumenti}
			\paragraph{LaTeX}
			Lo strumento scelto per la scrittura di documenti è \LaTeX{}, un linguaggio che permette di scrivere documenti in modo ordinato, modulare e collaborativo.
			\paragraph{Google Drive}
			Si utilizza \glock{Google Drive} per salvare file utili di qualunque genere.
			\paragraph{Diagrammi}
				I software utilizzati per la realizzazione dei diagrammi sono i seguenti:
				\begin{itemize}
					\item \textbf{StarUML:} per la realizzazione di diagrammi ER e schemi semplici;
					\item \textbf{Gantt Project:} applicazione open source per la realizzazione di diagrammi di Gantt che mostrano lo svolgimento delle attività.
				\end{itemize}
			
	\subsection{Gestione della Configurazione}
		\paragraph{Scopo}
			La gestione della configurazione definisce i principi normativi utili a predisporre il \glock{workspace} per tutto il gruppo, semplificando e automatizzando la conservazione dei documenti e delle componenti software, che andranno a formare il prodotto finale.
		\paragraph{Versionamento}
			\subparagraph{Codice di versione del documento}
			Ogni documento deve avere una storia, ricostruibile tramite le sue versioni. Ogni numero di versione è così composto:\\
			\centerline{X.Y.Z}
			\begin{itemize}
			\item \textbf{X}: rappresenta una versione stabile del documento, resa tale dopo l'approvazione dal responsabile di progetto:
				\begin{itemize}
		   			\item inizia da 0;
		   			\item viene incrementato dal responsabile di progetto all'approvazione del documento.
		  		\end{itemize}
		  	\item \textbf{Y}: rappresenta una versione parzialmente stabile del documento che è stata soggetta a verifica da parte di un verificatore:
		  		\begin{itemize}
		   			\item inizia da 0;
		   			\item  viene incrementato dal verificatore ad ogni verifica; 
		   			\item quando viene incrementato X, viene riportato a 0.
		   		\end{itemize}
		   	\item \textbf{Z}: rappresenta una versione instabile del documento in fase di lavorazione da parte dei redattori:
		   		\begin{itemize}
		   			\item inizia da 0;
		   			\item  viene incrementato dal redattore ad ogni modifica; 
		   			\item quando viene incrementato Y, viene riportato a 0.
		   		\end{itemize}
			\end{itemize}
			
			\subparagraph{Versionamento con GitHub}
	Il repository fa uso di un Version Control System (VCS) di tipo distribuito sotto il motore \textit{Git}, che permette la condivisione dal locale al remoto del proprio spazio di lavoro su un luogo comune.\\
		Attraverso l'utilizzo di un web browser, è possibile collegarsi a \textit{GitHub} e controllare i file contenuti in un repository. 
			\subparagraph{Repository}
			Ci sono due tipi di repository la cui struttura è identica e raccolgono i file sorgenti per la compilazione dei documenti, suddivisi tra esterni ed interni:
			\begin{itemize}
		   			\item \textbf{locale}: ogni membro del gruppo lavora sui file clonati dal repository remoto nel proprio computer;
		   			\item \textbf{remoto}: pubblicato su GitHub, contiene il lavoro svolto da ogni componente e che viene condiviso con il team.
		   	\end{itemize}

	
		   	
			
