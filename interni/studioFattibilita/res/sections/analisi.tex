\section{Analisi di fattibilità dei capitolati}

\subsection{Capitolato 1 - BlockCOVID}
\subsubsection{Informazioni generali}
\begin{description}
	\item[Nome:] BlockCOVID: supporto digitale al contrasto della pandemia;
	\item[Proponente:] Imola Informatica;
	\item[Committente:] prof.~Tullio Vardanega e prof.~Riccardo Cardin.
\end{description}

\subsection{Capitolato 2 - EmporioLambda}
\subsubsection{Informazioni generali}
\begin{description}
	\item[Nome:] EmporioLambda: piattaforma di e-commerce in stile Serverless;
	\item[Proponente:] Red Label;
	\item[Committente:] prof.~Tullio Vardanega e prof.~Riccardo Cardin.
\end{description}

\subsection{Capitolato 3 - GDP}
\subsubsection{Informazioni generali}
\begin{description}
	\item[Nome:] GDP: Gathering Detection Platform;
	\item[Proponente:] Sync Lab;
	\item[Committente:] prof.~Tullio Vardanega e prof.~Riccardo Cardin.
\end{description}

\subsection{Capitolato 4 - HD Viz}
\subsubsection{Informazioni generali}
\begin{description}
	\item[Nome:] HD Viz: visualizzazione di dati multidimensionali;
	\item[Proponente:] Zucchetti;
	\item[Committente:] prof.~Tullio Vardanega e prof.~Riccardo Cardin.
\end{description}

\subsection{Capitolato 5 - PORTACS}
\subsubsection{Informazioni generali}
\begin{description}
	\item[Nome:] PORTACS: piattaforma di controllo mobilità autonoma;
	\item[Proponente:] SanMarco Informatica;
	\item[Committente:] prof.~Tullio Vardanega e prof.~Riccardo Cardin.
\end{description}

\subsection{Capitolato 6 - RGP}
\subsubsection{Informazioni generali}
\begin{description}
	\item[Nome:] RGP: Realtime Gaming Platform;
	\item[Proponente:] Zero12;
	\item[Committente:] prof.~Tullio Vardanega e prof.~Riccardo Cardin.
\end{description}

\subsection{Capitolato 7 - SSD}
\subsubsection{Informazioni generali}
\begin{description}
	\item[Nome:] SSD: soluzioni di sincronizzazione desktop;
	\item[Proponente:] Zextras;
	\item[Committente:] prof.~Tullio Vardanega e prof.~Riccardo Cardin.
\end{description}








%
%\subsubsection{Descrizione del capitolato}
%Il capitolato riguarda la creazione di un'applicazione che permetta di contrastare il Covid-19 in ambienti quali laboratori, aule, aziende e , in generale, spazi chiusi tramite il tracciamento della pulizia delle postazioni.
%\subsubsection{}
%\glock{Docker}
%\glock{BlockChain}
%\subsubsection{Obiettivi del progetto}
%\paragraph*{Obiettivi di carattere generale}
%	\begin{itemize}
%		\item monitoraggio delle presenze nelle postazioni in maniera certificata e trasparente;
%		\item realizzazione di una reportistica certificata;
%		\item prenotazione postazione da remoto;
%		\item consultazione lista dei locali utilizzati.	
%	\end{itemize}
%\paragraph*{Obiettivi collegati all'utente}
%	\begin{description}
%		\item [server back-end:] dovrà raccogliere e integrare i dati delle applicazioni;
%		\item [GUI del server:] attraverso la quale un utente amministratore potrà:
%		\begin{itemize}
%				\item configurare le stanze e le postazioni;
%				\item configurare le credenziali degli utenti e del personale di pulizia;
%				\item bloccare la prenotazione delle stanze per un periodo di tempo;
%				\item consultare lo storico degli accessi e delle occupazioni di uno specifico utente;
%				\item conoscere lo stato di occupazione attuale delle postazioni e delle stanze.
%		\end{itemize}
%		\item[Applicazione mobile (ios o android):] dovrà permettere agli utenti di:
%		\begin{itemize}
%			\item segnalare l'occupazione di una postazione;
%			\item conoscere lo stato di una postazione;
%			\item prenotare una postazione;
%			\item segnalare di aver pulito una postazione.
%		\end{itemize}
%		
%		Inoltre dovrà permettere agli operatori di pulizia di:
%		segnalare l'igienizzazione di una stanza
%		conoscere le stanze da igienizzare
%		Test e report
%		Documentazione su: scelte implementative e progettuali effettuate e relative motivazioni. Problemi aperti e eventuali soluzioni proposte da esplorare.
%		
%	\end{description}
%\subsubsection{Pro e Contro del progetto}
%\paragraph{Pro}
%\paragraph{Contro}
%
%\subsubsection{Conclusioni}
%
%
%% EmporioLambda
%\subsection{Capitolato 2 - EmporioLambda}
%\subsubsection{Informazioni generali}
%\begin{description}
%	\item[Nome:] EmporioLambda: piattaforma di e-commerce in stile Serverless;
%	\item[Proponente:] Red Babel;
%	\item[Committente:] prof.~Tullio Vardanega e prof.~Riccardo Cardin.
%\end{description}
%
%\subsubsection{Descrizione del capitolato}
%Il capitolato riguarda la creazione di un sito \glock{e-commerce} \glock{serverless} utilizzando i servizi \glock{AWS Lambda}.
%\subsubsection{Studio del dominio}
%Per affrontare questo capitolato risulta necessario chiarire cosa si intende per e-commerce, definendo caratteristiche, attori coinvolti, dinamiche, evidenziando differenze e affinità rispetto al commercio tradizionale. L'e-commerce prevede lo scambio di servizi e/o prodotti regolato tramite transazioni che avvengono in Internet. Occorre avere padronanza della programmazione con approccio asincrono, nella quale gli eventi non sono costretti a seguire una rigida sequenza ma possono essere svolti indipendentemente e contemporaneamente.
%Infine occorre approfondire il funzionamento delle architetture serverless, che non necessitano la configurazione di server ma lasciano l'onere al cloud provider, nel caso di questo capitolato Amazon. 
%\subsubsection{Obiettivi del progetto}
%Viene richiesto:
%\begin{itemize}
%	\item lo sviluppo tramite la piattaforma \glock{node.js};
%	\item \glock{typescript} come linguaggio principale, sia per front-end sia per back-end;
%	\item la distribuzione deve avvenire su AWS Lambda.
%\end{itemize}
%\subsubsection{Pro e Contro del progetto}
%\paragraph{Pro}
%\begin{itemize}
%	\item AWS è una tecnologia molto utilizzata in ambito enterprise;
%	\item descrizione dei requisiti molto specifica e ben definita;
%	\item affiancamento costante durante il progetto: Red Babel si aspetta che i gruppi partecipanti siano proattivi, ricerchino il dialogo con loro e forniscano dei feedback sul prosieguo dei lavori. Se non vi è una comunicazione con il gruppo per più di 2 settimane significa che molto probabilmente ci sono dei problemi.
%\end{itemize}
%\paragraph{Contro}
%\begin{itemize}
%\item poca flessibilità sui linguaggi di programmazione e tecnologie;
%\item il documento con le specifiche è scritto in inglese (può lasciare a dubbi interpretativi);
%\item l'assenza di seminari di approfondimento;
%\end{itemize}
%\subsubsection{Conclusioni}
%È un capitolato già ben definito, ma con tecnologie richieste su cui nessuno ha alcuna conoscenza. Alto rischio.