\section{Scelta del capitolato}
\begin{center}
	\rowcolors{2}{lightest-grayest}{white}
	\begin{longtable}{|c|c|c|c|c|c|c|c|}
		\hline
		\rowcolor{lighter-grayer}
		\textbf{Parametri} & \textbf{C1} & \textbf{C2} & \textbf{C3} & \textbf{C4} & \textbf{C5}  & \textbf{C6}  & \textbf{C7}  \\
		\hline
		\endfirsthead
		
		% ----- Modificare da qui -----
		
		\hline
		\textbf{P1 - interesse} & 5 & 3 & 4 & 4 & 3 & 5 & 2 \\
		\hline
		\textbf{P2 - competenze} & 3 & 3 & 1 & 4 & 2 & 2 & 4 \\
		\hline
		\textbf{P3 - obiettivi} & 5 & 2 & 3 & 4 & 3 & 5 & 3 \\
		\hline
		
		\hline
		\textbf{Totale} & \textbf{13} & \textbf{8} & \textbf{8} & \textbf{12} & \textbf{8}  & \textbf{12}  & \textbf{9}  \\
		\hline	
	\end{longtable}

\end{center}
La scelta del capitolato è maturata nell'arco di 6 settimane (fine ottobre - inizio dicembre), tempo suddivisibile in 3 periodi:
\begin{enumerate}
\item \textbf{I-II settimana:} in cui vi è stata la conoscenza iniziale dei capitolati tramite lo studio autonomo e la presentazione ufficiale da parte dei proponenti del capitolati. Al termine di questo periodo si è deciso di effettuare una votazione preliminare all'interno di DPCM 2077 per individuare i capitolati preferiti e comunicarla agli altri \glock{gruppi}. Questa votazione non è stata né vincolante né definitiva ma necessaria allo scopo di "marcare il territorio" per evitare future contestazioni da parte di altri gruppi. I capitolati preferiti da DPCM 2077 sono: BlockCOVID, GDP e RGP. Emerge che BlockCOVID risulti essere ambito da molti gruppi.  
\item \textbf{III-V settimana:} in cui vi è stato l'approfondimento dei capitolati tramite la partecipazione a seminari tecnologici, le richieste di chiarimenti rivolti ai proponenti dei capitolati (tramite scambio e-mail) e l'approfondimento autonomo. Si sono maggiormente definiti i parametri di valutazione per la scelta del capitolato. Interessante notare che rispetto alla prima votazione, alcune preferenze del gruppo si sono modificate: HD Viz prende il posto di GDP fra i capitolati preferiti del gruppo. Il seminario di approfondimento tecnologico è stato chiarificatore delle richieste del proponente. Nella tabella sono indicate le valutazioni per capitolato, secondo i parametri definiti nell'introduzione. I tre capitolati preferiti dal gruppo risultano essere: BlockCOVID, RGP e HD Viz.
\item \textbf{VI settimana:} contrattazione con gli altri gruppi partecipanti. Non è stato necessario modificare la nostra prima scelta, in quanto BlockCOVID non presentava più conflitti con altri gruppi rispetto al limite dei posti previsti per l'aggiudicazione.
\end{enumerate}

Il capitolato scelto risulta essere quindi \textbf{BlockCOVID} di Imola Informatica, ovvero il C1.