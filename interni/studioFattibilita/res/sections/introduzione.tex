\section{Introduzione}

\subsection{Scopo del documento}
Lo studio di fattibilità consiste nell'indagine preliminare, che ha come oggetto la totalità dei capitolati e come obiettivo la scelta ragionata di un \glock{capitolato} tramite un uniforme metodo di giudizio. 

È attività \textbf{preliminare} in quanto precedente all'analisi circoscritta a un singolo capitolato. Rispetto a questa si discosta per il focus: non è mirata e quindi volutamente a grana grossa. 

È \textbf{scelta ragionata} perché fondata su motivazioni solide e non effimere.

Utilizza un \textbf{uniforme metodo di giudizio} per essere in grado di valutare nello stesso modo capitolati, che possono spaziare in ambiti diversi e distanti fra loro. È necessario quindi condividere identici parametri di valutazione.

\subsection{Glossario}
Termini che possono dare luogo a incomprensioni o necessitano di definizione sono segnalati con la lettera G in pedice (\glock{ }) e definiti nel glossario. 

\subsection{Riferimenti}
\subsubsection{Riferimenti Normativi}
\begin{itemize}
	\item Norme di Progetto % manca la versione
\end{itemize}

\subsubsection{Riferimenti Informativi}
\begin{itemize}
	\item{Capitolato 1\\
		\url{https://www.math.unipd.it/~tullio/IS-1/2020/Progetto/C1.pdf}};
	\item{Capitolato 2\\
		\url{https://www.math.unipd.it/~tullio/IS-1/2020/Progetto/C2.pdf}};
	\item{Capitolato 3\\
		\url{https://www.math.unipd.it/~tullio/IS-1/2020/Progetto/C3.pdf}};
	\item{Capitolato 4\\
		\url{https://www.math.unipd.it/~tullio/IS-1/2020/Progetto/C4.pdf}};
	\item{Capitolato 5\\
		\url{https://www.math.unipd.it/~tullio/IS-1/2020/Progetto/C5.pdf}};
	\item{Capitolato 6\\
		\url{https://www.math.unipd.it/~tullio/IS-1/2020/Progetto/C6.pdf}};
	\item{Capitolato 7\\
		\url{https://www.math.unipd.it/~tullio/IS-1/2020/Progetto/C7.pdf}};
	\item report seminario capitolato 1 % manca una denominazione dei report
	\item report seminario capitolato 3
	\item report seminario capitolato 4
	\item report seminario capitolato 5
	\item report seminario capitolato 6
\end{itemize}

