\section{Introduzione}

\subsection{Scopo del documento}
Lo studio di fattibilità consiste in un'indagine preliminare che ha come oggetto la totalità dei capitolati e come obiettivo la scelta ragionata di un \glock{capitolato}, tramite un uniforme metodo di giudizio. 
\begin{itemize}
	\item È attività \textbf{preliminare} in quanto precedente all'analisi circoscritta a un singolo capitolato. Rispetto a questa si discosta per il focus: non è mirata e quindi volutamente a grana grossa. 
	\item È \textbf{scelta ragionata} perché fondata su motivazioni solide e non effimere.
	\item Utilizza un \textbf{uniforme metodo di giudizio} per essere in grado di valutare nello stesso modo capitolati, che possono differire sostanzialmente per tematiche, tecnologie adottate, obiettivi. È necessario quindi adottare parametri di valutazione che consentano di confrontare capitolati diversi fra loro in modo che la valutazione conduca alla scelta di un unico capitolato. 
	
	Quali sono questi parametri di valutazione? 
	\begin{description}
		\item[P1 - interesse:] valuta quanto la scelta del capitolato sia condivisa fra i membri del gruppo e non mini alla sua unità;
		\item[P2 - competenze:] valuta il divario fra le competenze presenti all'interno del gruppo e le competenze richieste per affrontare il capitolato;
		\item[P3 - obiettivi:] valuta la chiarezza da parte del gruppo del dominio del capitolato. 
	\end{description}
	Ogni parametro assegna da 1 a 5 punti, indice del raggiungimento dell'obiettivo (1 non raggiungimento, 5 raggiungimento).
\end{itemize}
 
\subsection{Glossario e documenti} 
All'interno del  documento sono presenti termini che assumono significati diversi a seconda del contesto.
Per evitare ambiguità, è stato creato un  documento di nome Glossario che  conterrà tali termini con il loro significato specifico. Per segnalare che un termine del testo è presente all'interno del Glossario, verrà aggiunta una G pedice posta a fianco del termine ambiguo.
Quando si fa riferimento a un altro documento riguardante questo progetto vi si pone a pedice una D.

\subsection{Riferimenti}
\subsubsection{Riferimenti Normativi}
\begin{itemize}
	\item \dext{Norme di Progetto v. 1.0.0} % manca la versione
\end{itemize}

\subsubsection{Riferimenti Informativi}
\begin{itemize}
	\item{Capitolato 1\\
		\url{https://www.math.unipd.it/~tullio/IS-1/2020/Progetto/C1.pdf}}
	\item{Capitolato 2\\
		\url{https://www.math.unipd.it/~tullio/IS-1/2020/Progetto/C2.pdf}}
	\item{Capitolato 3\\
		\url{https://www.math.unipd.it/~tullio/IS-1/2020/Progetto/C3.pdf}}
	\item{Capitolato 4\\
		\url{https://www.math.unipd.it/~tullio/IS-1/2020/Progetto/C4.pdf}}
	\item{Capitolato 5\\
		\url{https://www.math.unipd.it/~tullio/IS-1/2020/Progetto/C5.pdf}}
	\item{Capitolato 6\\
		\url{https://www.math.unipd.it/~tullio/IS-1/2020/Progetto/C6.pdf}}
	\item{Capitolato 7\\
		\url{https://www.math.unipd.it/~tullio/IS-1/2020/Progetto/C7.pdf}}
	
	\item {Seminario tecnologico: strumenti di apprendimento automatico\\
		\url{https://www.math.unipd.it/~tullio/IS-1/2020/Progetto/ST1.pdf}}
	\item {Seminario tecnologico: esempi di visualizzazioni multidimensionali}
	
	\item {Seminario tecnologico: Docker, Blockchain\\
		\url{https://www.math.unipd.it/~tullio/IS-1/2020/Progetto/ST3.pdf}}
	\item {Seminario tecnologico: introduzione al servizio AppSync di AWS}
	
	\item {Seminario tecnologico: uso di Docker in architetture di sistema}
\end{itemize}

