\section*{Introduzione}
\subsection*{Luogo e data dell'incontro}
\begin{itemize}
	\item \textbf{luogo:} videoconferenza su \glock{discord};
	\item \textbf{data:} 2021-02-01;
	\item \textbf{ora di inizio:} 14:00;
	\item \textbf{ora di fine:} 15:30.
\end{itemize}

\subsection*{Ordine del giorno}
\begin{enumerate}
	\item Discussione valutazione RR;
	\item Possibili correzioni;
	\item Varie ed eventuali.
\end{enumerate}

\subsection*{Presenze}
\begin{itemize}
	\item \textbf{assenti:}
	\begin{itemize}
		\item nessun assente da segnalare.
	\end{itemize}
\end{itemize}

\section*{Svolgimento}
\subsection*{Discussione valutazione RR}
Viene discussa la correzione della RR, identificando i punti critici e possibili domande da porre per chiarimenti.
\subsection*{Possibili correzioni}
Vengono discusse delle possibili correzioni dei documenti. Viene identificata la necessità di un colloquio per delucidazioni sui punti critici.
\subsection*{Varie ed eventuali}
Viene discusso in che modo affrontare la prossima revisione, in particolare come impostare la \glock{tecnology baseline} ed il \glock{proof of concept} del prodotto.
