\section*{Introduzione}
\subsection*{Luogo e data dell'incontro}
\begin{itemize}
	\item \textbf{luogo:} \glock{Meet};
	\item \textbf{data:} 2020-11-16;
	\item \textbf{ora di inizio:} 16:00;
	\item \textbf{ora di fine:} 17:00.
\end{itemize}

\subsection*{Ordine del giorno}
\begin{enumerate}
	\item indagine sui progetti degli anni precedenti;
	\item \glock{GitHub};
	\item varie ed eventuali.
\end{enumerate}

\section*{Svolgimento}
\subsection*{Indagine sui progetti degli anni precedenti}
Si è discusso sull'opportunità di effettuare un'indagine sulla documentazione prodotta nei progetti degli anni precedenti. L'obiettivo dell'indagine è comprendere \textbf{criticamente} quali soluzioni sono state adottate dai \glock{gruppi} che ci hanno preceduto (in termini di struttura della documentazione, contenuti, riferimenti bibliografici e standard, etc\dots) e come queste sono state valutate: cercando di capire cosa è bene perseguire e cosa evitare. L'indagine è stata approvata e ogni anno è stato affidato a un componente del gruppo (oggetto di studio: i progetti dal 2010 al 2020).

\subsection*{GitHub}
Si è discusso sull'uso di GitHub come sistema di \glock{versionamento}
e si è chiarito all'interno del gruppo l'uso di alcuni suoi strumenti (wiki, milestone, issue) e il modus operandi che si vorrebbe adottare (git-flow). Per mantenere la repository espressione del lavoro di gruppo e non espressione individuale, è stato stabilito che ogni azione all'interno della \glock{repository} debba essere scaturita da una decisione di DPCM 2077. Ciò significa, concretamente, che ogni azione deve essere per forza collegata a un'issue, la quale è definita da un responsabile del gruppo oppure decisa collettivamente durante una riunione. 
\subsection*{Varie ed eventuali}
Si è deciso di creare un documento dove raccogliere le decisioni prese nel corso della settimana per non perderne traccia. Esse saranno validate alla prima riunione utile. Si è deciso di creare un documento collaborativo dove raccogliere, durante la settimana, gli argomenti che i membri di DPCM 2077 intendono discutere e che saranno gli odg della riunione.
È stato deciso di adottare la denominazione dei verbali in formato AAAA-MM-GG rispetto al numero progressivo finora in uso.