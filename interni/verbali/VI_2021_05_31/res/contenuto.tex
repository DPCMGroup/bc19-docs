\section*{Introduzione}
\subsection*{Luogo e data dell'incontro}
\begin{itemize}
	\item \textbf{luogo:} videoconferenza su \glock{Discord};
	\item \textbf{data:} 2021-05-31;
	\item \textbf{ora di inizio:} 11:00;
	\item \textbf{ora di fine:} 12:30.
\end{itemize}

\subsection*{Ordine del giorno}
\begin{enumerate}
	\item Aggiornamento sulla situazione attuale del prodotto;
	\item Discussione su un possibile incontro con il proponente.
		
\end{enumerate}

\subsection*{Presenze}
\begin{itemize}
	\item \textbf{Presenti: }
	\begin{itemize}
		\item Badan Antonio;
		\item Bertoldo Damiano;
		\item Budai Matteo;
		\item De Grandi Samuele;
		\item Piacere Ivan;
		\item Privitera Sara;
		\item Spigolon Daniele;
	\end{itemize}
\end{itemize}


\section*{Svolgimento}

\subsection*{\hypertarget{link1}{1. Aggiornamento sulla situazione attuale del prodotto}}
Ci siamo aggiornati per quanto riguarda la situazione delle tre componenti principali del prodotto, ovvero:
\begin{itemize}
	\item applicazione web;
	\item applicazione mobile;
	\item backend.
\end{itemize}
Abbiamo identificato, tramite l'utilizzo dell'AdR, i pochi requisiti ancora da implementare per ogni componente del prodotto, dando priorità in termini di importanza ad alcuni di essi.
Il nostro obiettivo è di implementare le funzionalità mancanti entro il 2 Giugno.
\subsection*{\hypertarget{link2}{2. Discussione su un possibile incontro con il proponente}}
Abbiamo pensato di contattare il proponente per fissare un incontro di aggiornamento sul nostro lavoro e per presentargli una demo del nostro prodotto.
Abbiamo inviato un messaggio sul gruppo Telegram, in quanto mezzo di comunicazione più veloce, in cui chiedevamo la sua disponibilità per il giorno 2021-06-01.
Abbiamo subito ricevuto risposta e fissato un incontro per il giorno da noi richiesto alle ore 16:00 su Google Meet.

