\section*{Introduzione}

\subsection*{Luogo e data dell'incontro}
\begin{itemize}
	\item \textbf{luogo:} videoconferenza su \glock{Discord};
	\item \textbf{data:} 2020-12-29;
	\item \textbf{ora di inizio:} 14:05;
	\item \textbf{ora di fine:} 15:25.
\end{itemize}

\subsection*{Ordine del giorno}
	\begin{enumerate}
	\item stesura dell'o.d.g: discussione sul fatto che chi vuole proporre un argomento da aggiungere, deve prima contattare i responsabili di progetto;
	\item discussione se adottare come prassi quella di aggiungere una D a pedice per riferirsi a un documento in particolare;
	\item discussione se pulire il branch master della repository GitHub;
	\item discussione dei ruoli nei documenti;
	\item discussione su quale tecnologia(framework, tool) permette di interfacciarsi con ethereum;
	\item discussione sul ciclo di vita della documentazione: tenere l'attuale o valutare maggior intervento del responsabile fra una fase e l'altra;
	\item discussioni sul nome dei verbali in formato VI\_AAAA\_MM\_GG\_v1.0.0;
	\item decidere nome progetto da mettere nel frontespizio dei documenti: "BlockCOVID" oppure "BlockCOVID: supporto digitale al contrasto della pandemia";
	\item decidere ordine delle modifiche nel registro: la più nuova in basso e la più vecchia in alto o viceversa;
	\item decidere se le sezioni Scopo del prodotto e Glossario tenerle diverse o uguali nei vari documenti

\end{enumerate}

\subsection*{Presenze}
\begin{itemize}
	\item \textbf{totale presenti:} 7 su 7;
	\item \textbf{presenti: }
	\begin{itemize}
		\item Badan Antonio;
		\item Bertoldo Damiano;
		\item Budai Matteo;
		\item De Grandi Samuele;
		\item Piacere Ivan;
		\item Privitera Sara;
		\item Spigolon Daniele.
	\end{itemize}
\end{itemize}

\section*{Svolgimento}


\subsection*{Stesura Ordine del Giorno}
Si è discusso sul fatto che chi vuole proporre un argomento per la riunione successiva, contatti i responsabili, che avranno cura di stendere l'o.d.g e moderare la discussione assegnando la giusta priorità ai vari punti da discutere.
I punti più importanti riguardanti la stesura dei vari documenti per l'ingresso RR verranno discussi sicuramente prima di tutti gli altri punti.

\subsection*{D a pedice per far riferimento a Documenti}
Si è deciso di utlizzare una \ped{D} pedice per fare riferimento a particolari documenti che poi potranno essere consultati per maggiori approfondimenti.

\subsection*{Pulizia branch master repository}
Per il momento si è deciso di lasciare il branch master com'è senza eliminare nulla. Una volta finita la stesura dei documenti per l'ingresso in RR verrà fatta una release nel branch master.

\subsection*{Stati di un documento durante la sua redazione}
Si è deciso che un documento durante la sua redazione avrà solo due stati: in redazione e approvato.

\subsection*{Norme tipografiche / Guide}
Si è deciso di andare a inserire nelle Norme di progetto\ped{D} delle norme tipografiche che i componenti di DPCM 2077 dovranno rispettare al fine di redigere un documento nella maniera
più precisa possibile. Si è deciso inoltre di inserire anche delle guide, per esempio su come utlizzare GitFlow come workflow per la repository su GitHub.
Queste norme tipografiche / guide sono già presenti in Google Drive, dovranno essere riportate nelle Norme di Progetto\ped{D}.

\subsection*{Contatto con il proponente}
Si è deciso che i redattori dell'Analisi dei Requisiti\ped{D} dovranno contattare il Dr. Lorenzo Patera per delle chiarificazioni su Ethereum e intrattenere una videoconferenza per eliminare ogni dubbio
al fine di procedere a una redazione del documento senza anomalie.

\subsection*{Discussione su stesura documenti mancanti}
Si è deciso che il gruppo dovrà terminare la stesura dei documenti il 2021-01-03. Il giorno successivo si terrà una videoconferenza per fare il punto della situazione e procedere con la verifica dei documenti rimanenti.

\subsection*{Nomenclatura dei verbali}
Si è deciso di utilizzare la seguente nomenclatura per i verbali interni: VI\_AAAA\_MM\_GG\_v1.0.0 e quelli esterni:  VE\_AAAA\_MM\_GG\_v1.0.0 dove v1.0.0 è solo un esempio di versione del documento.

\subsection*{Titolo del capitolato nei documenti}
Si è deciso di utilizzare "BlockCOVID" al posto di "BlockCOVID": supporto digitale al contrasto della pandemia come titolo del capitolato da inserire nei vari documenti.

\subsection*{Registro delle modifiche}
Si è deciso che le modifiche più nuove andranno in alto nella tabella e quindi poi quelle più vecchie andranno sempre più in basso.

\subsection*{Scopo del prodotto e Glossario nell'introduzione dei documenti}
Si è deciso che lo Scopo del prodotto e Glossario nell'introduzione dei documenti dovranno essere uguali in ogni documento.

\subsection*{Descrizione nella copertina dei documenti}
Si è deciso che la descrizione nella copertina dei documenti dovrà essere rimossa.








