\section*{Introduzione}
\subsection*{Luogo e data dell'incontro}
\begin{itemize}
	\item \textbf{luogo:} videoconferenza su \glock{Discord};
	\item \textbf{data:}  2021-03-29;
	\item \textbf{ora di inizio:} 14:15;
	\item \textbf{ora di fine:} 16:00.
\end{itemize}

\subsection*{Ordine del giorno}
\begin{enumerate}
	\item PoC mobile;
	\item Creazione ulteriori repository;
	\item Docker.
\end{enumerate}

\subsection*{Presenze}
\begin{itemize}
	\item \textbf{Presenti:}
	\begin{itemize}
		\item Badan Antonio;
		\item Bertoldo Damiano;
		\item Budai Matteo;
		\item De Grandi Samuele;
		\item Piacere Ivan;
		\item Privitera Sara;
		\item Spigolon Daniele.
	\end{itemize}
\end{itemize}

\section*{Svolgimento}
\subsection*{PoC app-mobile}
Come definito nel \dext{PdP v. 2.0.0}, abbiamo deciso di approfondire la parte mobile dell'applicazione tramite un \glock{PoC} dedicato. Le tematiche affrontate dal PoC saranno: autenticazione del dipendente al sistema, scansione del tag nfc e comunicazione mobile - backend. 

\subsection*{Creazione ulteriori repository}
Abbiamo deciso di suddividere il futuro codice in repository differenziate per l'app mobile, la webapp e il backend. Questo per facilitare la suddivisione del lavoro fra i membri del gruppo. Prima dell'implementazione vera e propria dell'applicativo verranno predisposti per ogni repository strumenti di qualità del codice di analisi statica, dinamica e di continuous integration.

\subsection*{Docker}
Abbiamo constatato l'urgenza nell'approfondire la tecnologia Docker. Le tematiche da comprendere entro la prossima riunione sono: sintassi da utilizzare e creazione di Docker file.