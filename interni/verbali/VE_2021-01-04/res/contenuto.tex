\section*{Introduzione}
\subsection*{Dettagli e-mail}
\begin{itemize}
	\item \textbf{Autore dell'e-mail per DPCM 2077:} Ivan Piacere
	\item \textbf{Autore dell'e-mail per Imola Informatica:} Lorenzo Patera
	\item \textbf{periodo:} 2020-01-04 - 2020-01-04
\end{itemize}

\section*{Svolgimento}
Il verbale riporta una selezione delle domande e relative risposte nello scambio di e-mail avvenuto nel periodo: %completare.
\subsection*{Chi può creare e modificare le credenziali degli amministratori? Un amministratore può creare le credenziali anche per gli altri?}

Si, direi che create un amministratore direttamente nel db, e dopo lui può far diventare amministratori altri utenti "normali".

\subsection*{Nel capitolato si dice che l'amministratore deve poter "estrapolare un report, sotto forma tabellare, che evidenzi le ore trascorse alle postazioni di un singolo utente" e "estrapolare un report, sotto forma tabellare, che mostri tutte le sanificazioni effettuate, sia dal personale addetto, che da studenti/dipendenti". Questi report devono essere sia visualizzabili sul sito che scaricabili come file?}

Si, i report devono essere visualizzabili a video e scaricabili. Utilizzate un formato leggibile (se non vi risulta troppo difficile un pdf sarebbe opportuno).

\subsection{Riguardo al report delle sanificazioni, per ognuna di esse deve essere indicata la persona specifica che l'ha effettuata? O almeno deve essere indicato se è stata effettuata da un dipendente o da un addetto alle pulizie?}

Si, è necessario che la sanificazione sia corredata dal nome della persona da cui è stata effettuata e il ruolo (dipendente/addetto pulizie)

\subsection{Riguardo alla ricerca delle postazioni occupate da un utente è sufficiente fornire come campi di ricerca un periodo di tempo e un codice di postazione?}

Vedete voi quali campi risultano utili, direi come minimo dipendente e intervallo di date per le postazioni utilizzate da un dipendente, postazione e intervallo di date per gli utilizzi di una determinata postazione.

\subsection{La chiusura di una stanza da parte dell'amministratore può essere impostata per un qualsiasi periodo oppure deve necessariamente iniziare nella data odierna? In altre parole: supponendo di eseguire oggi l'azione di chiusura di una stanza, essa può essere chiusa solo da oggi a un altro giorno oppure da una qualsiasi data a un'altra qualsiasi data successiva?}

Da qualsiasi data a qualsiasi data (ovviamente successiva alla data odierna)

\subsection{Abbiamo pensato ad una pagina del sito dedicata alla visualizzazione e modifica delle impostazioni del sistema, dove le impostazioni per ora sono: \\
	- intervallo di tempo massimo di presenza del telefono sul tag RFID oltre il quale la postazione risulta essere occupata \\
	- intervallo di tempo massimo di assenza del telefono dal tag RFID oltre il quale la postazione risulta essere libera \\
	Questa pagina sarebbe una cosa apprezzata?}

Mi sembra più un impostazione di sistema da gestire da programmatori. Un utente finale non si mette a "giocare" con i parametri di tempo di rfid. Simulate il sistema e cercate di capire che intervalli sono consoni. Ovviamente fate in modo che siano facilmente modificabili con una ricompilazione (o una scrittura su db, fate voi).


\subsection{Sarebbe ragionevole pensare ad un periodo di tempo che deve trascorrere dopo che una postazione è stata liberata e prima che essa possa essere occupata nuovamente? Questa funzionalità servirebbe per evitare che chi lascia la postazione si incontri con chi la occupa. Nel caso si volesse implementarla bisognerebbe ripensare anche alle pulizie da parte dei dipendenti. Infatti, data una postazione non igienizzata, l'ultimo dipendente che l'ha utilizzata la potrebbe igienizzarla subito mentre un dipendente diverso dovrebbe attendere l'intervallo di sicurezza. }

Non è richiesto. Anche perchè i dipendenti arrivano abbastanza "quando gli pare", quindi questa funzionalità renderebbe l'applicazione slegata dal mondo reale e difficile da utilizzare (mi dimentico perchè dovevo aspettare).

\subsection{Dovremo eseguire un nodo della blockchain nel nostro server, o comunque crearne uno e ospitarlo su un servizio remoto?}

Eseguite un vostro nodo blockchain sulla vostra macchina. Con Ethereum dovrebbero esserci varie installazioni open source e basate su docker per accendi/spegni veloci e senza lasciare traccia sul pc. Qui un esempio di installazione https://medium.com/swlh/how-to-set-up-a-private-ethereum-blockchain-c0e74260492c (come vi ho detto non sono un esperto di blockchain, se doveste avere problemi proviamo a parlarne ed eventualmente vi schedulo una call con un mio collega che ci lavora spesso).

\subsection{La nostra blockchain sarà distribuita tra i vari utenti della rete? E se sì la distribuzione sarà gestita dal servizio che ospita la blockchain (es. Ethereum)?  }

La distribuzione della blockchain è superflua per il vostro applicativo. Vi basta una sola istanza in esecuzione idealmente sullo stesso nodo/cluster dove è presente il server dell'app, essendo in installazione "privata" e ad utilizzo esclusivo.

\subsection{La blockchain ha valore legale anche se non è distribuita?}

Nì. Per il vostro caso d'uso (avendo semplificato il mondo reale) sì. Nelle installazioni reali è necessario inserire nelle transazioni degli hash certificati da terze parti fidate dal sistema legale del paese nel quale si opera (solitamente contenente un hash dei dati ed un timestamp controfirmato). In generale comunque avere un sistema blockchain è già visto come più probatorio rispetto ad un database aperto e modificabile dall'azienda anche a posteriori.