\section*{Introduzione}

\subsection*{Luogo e data dell'incontro}
\begin{itemize}
	\item \textbf{luogo:} videoconferenza su \glock{Meet};
	\item \textbf{data:} 2020-12-14;
	\item \textbf{ora di inizio:} 17:30;
	\item \textbf{ora di fine:} 19:30.
\end{itemize}

\subsection*{Ordine del giorno}
	\begin{enumerate}
	\item discussione sulla definizione dei ruoli e rotazione
	\item selezione editor \glock{\LaTeX}
	\item assegnazione dei compiti per la stesura dei documenti
	\item modalità di stesura del glossario
	\item suddivisione fra documentazione interna e esterna
	\item aggiornamento nomenclatura verbali
	\item chiedere disponibilità e canale informativo a Imola Informatica
	\item creazione milestone interne alla RR
\end{enumerate}

\subsection*{Presenze}
\begin{itemize}
	\item \textbf{totale presenti:} 7 su 7;
	\item \textbf{presenti: }
	\begin{itemize}
		\item Badan Antonio;
		\item Bertoldo Damiano;
		\item Budai Matteo;
		\item De Grandi Samuele;
		\item Piacere Ivan;
		\item Privitera Sara;
		\item Spigolon Daniele;
	\end{itemize}
\end{itemize}

\section*{Svolgimento}
\subsection*{Discussione sulla definizione dei ruoli e rotazione}
			Si è discusso sulla definizione dei ruoli seguendo la pagina
			\href{https://www.math.unipd.it/~tullio/IS-1/2020/Progetto/RO.html#Org}{Organigramma del gruppo}.
			Si è deciso che durante tutta la RR i ruoli resteranno invariati, infatti
			una volta stesi i documenti e compreso meglio il capitolato si capiranno le ore necessarie e solo allora si potrà pensare alla rotazione dei ruoli, in modo che ognuno possa ricoprirli tutti una o più volte. 

\subsection*{Selezione editor \glock{\LaTeX}}
			Si è deciso di utilizzare come editor \glock{\LaTeX} uno tra TexMaker e TexStudio.
			Entrambi gli strumenti sono stati reputati validi.
			


\subsection*{\hypertarget{thesentence}{Assegnazione dei compiti per la stesura dei documenti}}
			Si è deciso di assegnare ad uno o più membri del gruppo la stesura della seguente documentazione:
			\begin{itemize}
			\item Studio di fattibilità: Antonio Badan.
			\item Norme di progetto: Damiano Bertoldo, Sara Privitera, Daniele Spigolon.
			\item Piano di progetto: Matteo Budai, Ivan Piacere, Samuele de Grandi.
			\item Glossario: tutti.
			\end{itemize}
			In base al documento preso in carico si sono stabiliti in modo conseguente i ruoli.

\subsection*{Modalità di stesura del glossario}
		    Si è deciso che ognuno contribuirà al glossario con i termini che utilizza durante la stesura della documentazione affidatagli.

\subsection*{Suddivisione fra documentazione interna e esterna}
			Damiano Bertoldo è stato incaricato per la suddivisione della documentazione tra interna, esterna e per la creazione delle cartelle. 


\subsection*{Aggiornamento nomenclatura verbali}
      		Si è apportata una modifica alla nomenclatura dei verbali per quanto riguarda il rilascio.
      		Una volta approvato un documento la versione sarà x.0.0, azzerando quindi le ultime due cifre rappresentanti la redazione e la verifica.
      		


\subsection*{Chiedere disponibilità e canale informativo a Imola Informatica}
			Si è deciso di spedire una mail a Lorenzo Patera, rappresentante di Imola Informatica per conoscere la sua reperibilità nelle prossime settimane e il canale informativo da lui preferito.


\subsection*{Creazione milestone interne alla RR}
			Si è deciso di creare quattro milestone interne alla RR per gestire meglio il lavoro da fare e le relative scadenze.
			In questo modo risulterà più semplice verificare l'avanzamento della stesura della documentazione.
			Le milestone individuate sono le seguenti:
			\begin{itemize}
			\item 21 dicembre: vedi sezione \hyperlink{thesentence}{Assegnazione dei compiti per la stesura dei documenti.}
			\item 29 dicembre: completamento Analisi dei Requisiti e Piano di Qualifica
			\item 4 gennaio: aver completato(verifica e approvazione) tutti i documenti
			\item 11 gennaio: consegna.
			\end{itemize}





