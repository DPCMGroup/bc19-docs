\section*{Introduzione}

\subsection*{Luogo e data dell'incontro}
\begin{itemize}
	\item \textbf{luogo:} videoconferenza su \glock{Discord};
	\item \textbf{data:} 2020-12-21;
	\item \textbf{ora di inizio:} 17:30;
	\item \textbf{ora di fine:} 19:20.
\end{itemize}

\subsection*{Ordine del giorno}
	\begin{enumerate}
	\item creazione canale Discord;
	\item \glock{Continuous Integration};
	\item decisione strumento per creare \glock{diagrammi di Gantt};
	\item modalità di stesura del glossario;
	\item indicazione parole da riportare sul glossario;
	\item discussione su un limite di pagine per documento;
	\item decisione strumento per creare \glock{diagrammi uml};
	\item discussione su stesura documenti mancanti;
	\item discussione su documenti interni ed esterni.
\end{enumerate}

\subsection*{Presenze}
\begin{itemize}
	\item \textbf{totale presenti:} 7 su 7;
	\item \textbf{presenti: }
	\begin{itemize}
		\item Badan Antonio;
		\item Bertoldo Damiano;
		\item Budai Matteo;
		\item De Grandi Samuele;
		\item Piacere Ivan;
		\item Privitera Sara;
		\item Spigolon Daniele.
	\end{itemize}
\end{itemize}

\section*{Svolgimento}
\subsection*{Creazione canale Discord}
È stato deciso di utilizzare Discord come canale principale per le comunicazioni tra i componenti del gruppo in quanto permette la crezione di più sottocanali suddivisi per argomento oltre che la possibilità di effettuare videoconferenze. È stato creato quindi un canale per ogni documento oltre che canali generali per ogni tipo di comunicazione.

\subsection*{Continuous Integration}
Si è deciso di utilizzare le GitHub Action per la CI.

\subsection*{Decisione strumento per creare diagrammi di Gantt}
Si è deciso di utilizzare il software "GanttProject".
			
\subsection*{Modalità di stesura del glossario}
Si è deciso che ogni componente del gruppo indicherà su un documento denominato "Glossario" sul drive di DPCM 2077 le parole, con spiegazione, che ritiene debbano essere riportate nel documento formale. 
Queste parole saranno poi trascritte nel documento presente sulla repository di GitHub.

\subsection*{Indicazione parole da riportare sul glossario}
Ci siamo accordati sul fatto che saranno marcate con un pedice "G" solo le prime occorrenze, all'interno di un documento, delle parole che saranno poi trascritte sul Glossario.

\subsection*{Discussione su un limite di pagine per documento}
Abbiamo fatto il punto della situazione per quanto riguarda la stesura dei documenti. Non ci siamo dati un limite di pagine ma è stato deciso di continuare il lavoro cercando di restringere il contenuto. Il verificatore dei documenti si occuperà anche di individuare ripetizioni o parti superflue che potrebbero essere tagliate.

\subsection*{Decisione strumento per creare diagrammi uml}
Si è deciso di utilizzare il software "StarUML"

\subsection*{Discussione su stesura documenti mancanti}
È stato discusso il modo secondo il quale assegnare la redazione dei documenti mancanti tra i componenti del gruppo. 
È stato deciso che ogni componente del gruppo finisca il lavoro già assegnato prima di iniziarne altri. Quindi chi ha già terminato inizierà la stesura degli ultimi documenti. 
Inoltre è stato deciso che ogni componente del gruppo dovrà pensare a dei possibili casi d'uso del nostro progetto. Questi saranno presentati in una riunione già fissata in data 2020-12-23 per aiutare la stesura dell'Analisi dei requisiti ai redattori. 
Inoltre per quanto riguarda i documenti da terminare, è stato deciso di aggiungere alle Norme di progetto il fatto che il gruppo non seguirà una struttura fissa per l'invio delle mail al committente/proponente. Queste saranno però inviate dal responsabile avendo cura di includere il logo del gruppo.

\subsection*{Discussione su documenti interni ed esterni}
È stata discussa sopratutto la differenza tra verbali interni ed esterni. Non trovando una risposta sicura è stato deciso di chiedere chiarimenti al professore Tullio Vardanega tramite mail.		


			





