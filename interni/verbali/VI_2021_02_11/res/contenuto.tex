\section*{Introduzione}
\subsection*{Luogo e data dell'incontro}
\begin{itemize}
\item \textbf{luogo:} videoconferenza su \glock{discord};
\item \textbf{data:} 2021-02-11;
\item \textbf{ora di inizio:} 14:00;
\item \textbf{ora di fine:} 16:00.
\end{itemize}

\subsection*{Ordine del giorno}
\begin{enumerate}
	\item Mail;
	\item Discussione relative ai colloqui con i professori;
	\item Suddivisione compiti per correzione documenti;
	\item Idee proof of concept;
	\item Idee sulle tecnologie.
\end{enumerate}

\subsection*{Presenze}
\begin{itemize}
	\item \textbf{assenti:}
	\begin{itemize}
	\item nessun assente da segnalare.
	\end{itemize}
\end{itemize}

\section*{Svolgimento}
\subsection*{Mail}
Viene discussa la mail arrivata dal proponente relativa alla fornitura dei tag NFC. Viene stabilito che saranno consegnati a Matteo Budai.
\subsection*{Discussione relative ai colloqui con i professori}
Vengono discusse le informazioni ottenute dai colloqui con i professori. Ci si fa un'idea più chiara di come correggere i documenti e di come proseguire con il lavoro.
\subsection*{Suddivisione compiti per correzione documenti}
Vengono suddivisi i compiti per correggere i documenti. Vengono create delle issue per monitorare il completamento. Viene deciso che si utilizzeranno le pull requests per controllare al meglio ciò che può entrare o meno nella repo, ed eventualmente segnalare in modo mirato eventuali modifiche da apportare..
\subsection*{Idee proof of concept}
Vengono discusse alcune idee sul proof of concept e viene stabilita un'idea base da cui partire.
In particolare, si è pensato di rappresentare uno o più casi d'uso relativi all'amministratore e la sua interazione con il sistema e la blockchain.
\subsection*{Idee sulle tecnologie}
Vengono discussi i PRO e CONTRO delle tecnologie individuate e successivamente vengono suddivise tra i vari componenti del gruppo per approfondirle.
