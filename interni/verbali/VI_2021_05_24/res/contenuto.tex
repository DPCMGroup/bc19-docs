\section*{Introduzione}
\subsection*{Luogo e data dell'incontro}
\begin{itemize}
	\item \textbf{luogo:} videoconferenza su \glock{Discord};
	\item \textbf{data:} 2021-05-24;
	\item \textbf{ora di inizio:} 14:15;
	\item \textbf{ora di fine:} 15:00.
\end{itemize}

\subsection*{Ordine del giorno}
\begin{enumerate}
	\item Aggiornamento sulla situazione attuale del prodotto;
	\item Pianificazione obiettivi settimanali;
	\item Video dimostrativo.
\end{enumerate}

\subsection*{Presenze}
\begin{itemize}
	\item \textbf{Presenti:}
	\begin{itemize}
		\item Badan Antonio;
		\item Bertoldo Damiano;
		\item Budai Matteo;
		\item De Grandi Samuele;
		\item Piacere Ivan;
		\item Privitera Sara;
		\item Spigolon Daniele.
	\end{itemize}
\end{itemize}

\section*{Svolgimento}
\subsection*{\hypertarget{link1}{1. Aggiornamento sulla situazione attuale del prodotto}}
Tutti gli obiettivi prefissati per la settimana, che si possono trovare nel documento VI\_2021\_05\_17 v1.0.0, sono stati raggiunti.
\subsection*{2. Pianificazione obiettivi settimanali}
Per il periodo che va dal 24 maggio al 30 maggio abbiamo deciso di implementare le seguenti funzionalità:
\begin{itemize}
	\item backend
	\begin{itemize}
		\item generazione report;
		\item salvataggio report su blockchain;
		\item aggiunta di chiamate API;
	\end{itemize}
	\item app mobile
	\begin{itemize}
		\item gestione prenotazione;
		\item funzionalità addetto alle pulizie;
		\item gestione occupazione;
	\end{itemize}
	\item webapp
	\begin{itemize}
		\item impostazione postazioni guaste;
		\item impostazione stanze inaccessibili;
		\item scrittura test;
		\item guida utente.
	\end{itemize}
\end{itemize}
\subsection*{3. Video dimostrativo}
Visto che l'applicazione mobile non ha subito grandi cambiamenti visibili, abbiamo deciso di preparare una demo della sola webapp. Ivan Piacere si occuperà della registrazione e dell'invio a Lorenzo Patera di questo video.
