\section*{Introduzione}
\subsection*{Luogo e data dell'incontro}
\begin{itemize}
	\item \textbf{luogo:} videoconferenza su \glock{Meet};
	\item \textbf{data:} 2020-12-18;
	\item \textbf{ora di inizio:} 15:00;
	\item \textbf{ora di fine:} 15:30.
\end{itemize}

\subsection*{Ordine del giorno}
\begin{enumerate}
	\item porre le seguenti domande sul \glock{capitolato} BlockCOVID proposto dall'azienda \mbox{ImolaInformatica}
	\begin{enumerate}
		\item Come dobbiamo misurare il consumo energetico dello smartphone durante l'utilizzo dell'applicazione?
		
		\item Che documentazione richiedete?
		
		\item Come vanno gestite le \glock{postazioni} eliminate, soprattutto rispetto alla visibilità che ne ha l'amministratore?
		
		\item Chi ha la visibilità delle postazioni in tempo reale?
		
		\item Gli addetti possono igienizzare solo stanze intere o anche singole postazioni?
		
		\item Una postazione può essere igienizzata anche da dipendenti diversi rispetto a chi l'ha utilizzata l'ultima volta?
		
		\item Dobbiamo prevedere un modo per liberare una postazione prenotata se l'utente non si presenta all'orario stabilito?
		
		\item Se una postazione viene prenotata per un determinato orario essa è utilizzabile da altri utenti prima di quell'orario?
		
		\item Nel documento di presentazione del capitolato, nella descrizione del caso d'uso 1, tra le funzionalità fornite all'amministratore è presente quella di "effettuare ricerche sugli accessi e sulle postazioni occupate da uno specifico dipendente". Cosa si intende per accessi?
		
		\item La nostra applicazione deve garantire un distanziamento minimo tra gli utilizzatori delle postazioni?
		
		\item Potrebbe disambiguare tra requisiti opzionali e desiderabili?
		
	\end{enumerate}
\end{enumerate}

\subsection*{Presenze}
\subsubsection*{Interni}
\begin{itemize}
	\item \textbf{totale presenti:} 7 su 7;
	\item \textbf{presenti: }
	\begin{itemize}
		\item Badan Antonio;
		\item Bertoldo Damiano;
		\item Budai Matteo;
		\item De Grandi Samuele;
		\item Piacere Ivan;
		\item Privitera Sara;
		\item Spigolon Daniele;
	\end{itemize}
\end{itemize}
\subsubsection*{Esterni}
\begin{itemize}
	\item \textbf{presenti: }
	\begin{itemize}
		\item Patera Lorenzo, referente di ImolaInformatica;
	\end{itemize}
\end{itemize}

\newpage

\section*{Svolgimento}
La riunione si è svolta come una sequenza di domande da parte di Antonio Badan e risposte da parte di Lorenzo Patera.

\subsection*{Come dobbiamo misurare il consumo energetico dello smartphone durante l'utilizzo dell'applicazione?}
Non è necessario fare un'analisi approfondita. Si può utilizzare lo strumento preinstallato in \glock{Android}, oppure un altro strumento di terze parti.\\
L'obiettivo è di trovare una correlazione tra precisione e consumo energetico.

\subsection*{Che documentazione richiedete?}
L'azienda non richiede documenti ulteriori rispetto a quelli necessari per il corso e a quelli indicati nella presentazione del capitolato.\\
La discussione si è concentrata quindi sui \glock{test} da effettuare.\\
Per ogni macrocomponente dovrà esserci una suite di test che ne verifichi il funzionamento. Inoltre sarebbe apprezzabile anche se non necessario produrre dei test di integrazione almeno parziali. Per esempio va verificato che una postazione appena prenotata non sia disponibile agli altri utenti.\\
Nel report dovremo indicare per ogni test la motivazione, l'esito e gli strumenti utilizzati per eseguirlo.

\subsection*{Come vanno gestite le postazioni eliminate, soprattutto rispetto alla visibilità che ne ha l'amministratore?}
Siete liberi di gestire la problematica come preferite.\\
In ogni caso gli utenti interessati a una postazione eliminata, perché ne scannerizzano il tag \glock{RFID} oppure perché l'hanno prenotata, devono essere avvisati dell'inaccessibilità. Nel caso della prenotazione sarebbe apprezzato fornire all'utente una nuova postazione.\\
Va previsto un avviso anche nel caso una postazione sia guasta.

\subsection*{Chi ha la visibilità delle postazioni in tempo reale?}
Solo l'amministratore può vedere tutte le postazioni in tempo reale.

\subsection*{Gli addetti possono igienizzare solo stanze intere o anche singole postazioni?}
Entrambi.

\subsection*{Una postazione può essere igienizzata anche da dipendenti diversi rispetto a chi l'ha utilizzata l'ultima volta?}
Una postazione che è stata utilizzata può essere igienizzata da qualunque dipendente, oltre che dagli addetti alle pulizie.

\subsection*{Dobbiamo prevedere un modo per liberare una postazione prenotata se l'utente non si presenta all'orario stabilito?}
Se un dipendente non si presenta alla postazione prenotata entro 30 minuti dall'orario stabilito, la postazione viene liberata. È apprezzabile che l'utente che ha eseguito la prenotazione venga avvisato. Non è necessario sollecitare o chiedere conferma della prenotazione all'utente in ritardo.\\
Le postazioni possono essere prenotate solo per periodi compresi nell'orario di lavoro.
Il periodo prenotato dovrà essere di almeno 1 ora e potrà durare al massimo fino alla fine della giornata lavorativa.\\


\subsection*{Se una postazione viene prenotata per un determinato orario essa è utilizzabile da altri utenti prima di quell'orario?}
Sì. Se un utente decide di utilizzarla è utile che gli venga indicato che dovrà lasciarla entro uno specifico orario, di modo che l'utente che ha eseguito la prenotazione possa accedervi.

\subsection*{Nel documento di presentazione del capitolato, nella descrizione del caso d'uso 1, tra le funzionalità fornite all'amministratore è presente quella di "effettuare ricerche sugli accessi e sulle postazioni occupate da uno specifico dipendente". Cosa si intende per accessi?}
In questa frase la parola accessi è un refuso e va esclusa.

\subsection*{La nostra applicazione deve garantire un distanziamento minimo tra gli utilizzatori delle postazioni?}			
Non è richiesto che l'applicazione assicuri una distanza minima tra le postazioni occupate di una stanza.\\
Questo problema sarà in carico all'amministratore del sistema, il quale potrà, per esempio, bloccare l'utilizzo di alcune postazioni.

\subsection*{Potrebbe disambiguare tra requisiti opzionali e desiderabili?}
Non è stata menzionata una differenza tra i due termini.\\
Nel documento di presentazione del capitolato i requisiti sottolineati sono quelli che si possono trascurare nel caso non ci sia tempo per implementarli.

\subsection*{Informazioni generali}
Il sistema sviluppato dovrebbe poter essere utilizzato da qualsiasi \glock{organizzazione}; non dovrebbe essere vincolato all'azienda ImolaInformatica né ai laboratori informatici.\\
Inoltre durante lo sviluppo va tenuto conto che la posizione e il numero delle postazioni non variano frequentemente.
