\section*{Introduzione}
\subsection*{Luogo e data dell'incontro}
\begin{itemize}
	\item \textbf{luogo:} \glock{Meet};
	\item \textbf{data:} 2020-11-03;
	\item \textbf{ora di inizio:} 16:15;
	\item \textbf{ora di fine:} 18:00.
\end{itemize}

\subsection*{Ordine del giorno}
\begin{enumerate}
	\item votazione preliminare dei capitolati;
	\item domande per i proponenti dei capitolati;
	\item nome, logo, e-mail del gruppo.
\end{enumerate}

\section*{Svolgimento}
\subsection*{Votazione preliminare dei capitolati}
Fermo restando quanto deciso in VI\_2020-10-27\_1.2, si è deciso di effettuare una votazione preliminare all'interno del \glock{gruppo} per individuare i tre capitolati preferiti e comunicarla agli altri gruppi. Questa votazione non è né vincolante né definitiva ma necessaria allo scopo di "marcare il territorio" per evitare future contestazioni da parte di altri gruppi. I capitolati preferiti del gruppo risultano essere: BlockCovid, GDP e RGP.
\subsection*{Domande per i proponenti dei capitolati}
Sono state definite le ultime domande da porre ai proponenti dei capitolati nell'incontro di presentazione dei capitolati del 5 novembre.
\subsection*{Nome, logo, e-mail del gruppo}
È stato deciso il nome del gruppo: DPCM2077. Conseguentemente a questo è stato definito il logo e l'e-mail, la quale sarà il canale di comunicazione ufficiale di DPCM2077 verso l'esterno.