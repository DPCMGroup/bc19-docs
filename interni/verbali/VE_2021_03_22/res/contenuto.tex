\section*{Introduzione}
\subsection*{Luogo e data dell'incontro}
\begin{itemize}
	\item \textbf{luogo:} videoconferenza su \glock{Discord};
	\item \textbf{data:} 2021-03-22;
	\item \textbf{ora di inizio:} 14:15;
	\item \textbf{ora di fine:} 16:00.
\end{itemize}

\subsection*{Ordine del giorno}
\begin{enumerate}
	\item discussione valutazione in uscita dalla RP;
	\item discussione per pianificazione futura.
		
\end{enumerate}

\subsection*{Presenze}
\subsubsection*{Interni}
\begin{itemize}
	\item \textbf{totale presenti:} 7 su 7;
	\item \textbf{presenti: }
	\begin{itemize}
		\item Badan Antonio;
		\item Bertoldo Damiano;
		\item Budai Matteo;
		\item De Grandi Samuele;
		\item Piacere Ivan;
		\item Privitera Sara;
		\item Spigolon Daniele;
	\end{itemize}
\end{itemize}


\section*{Svolgimento}

\subsection*{Discussione valutazione in uscita dalla RP}
Si sono discussi gli errori e la valutazione data dai professori al nostro lavoro svolto in sede di RP. Una volta identificati gli errori si sono discusse le possibili soluzioni e i modi per non rifarli in futuro.
\subsection*{Discussione per pianificazione futura}
Si è deciso come suddividere tra i vari componenti del gruppo il lavoro pianificato per questa settimana, sulla base delle correzioni da fare e del lavoro aggiuntivo da presentare per la RQ.
Si è pensato ad un unico posto su cui fare riferimento per inserire il database e la blockchain per il progetto.
La discussione ha riguardato anche la creazione della Product Baseline e dei due manuali da presentare in sede di RQ.