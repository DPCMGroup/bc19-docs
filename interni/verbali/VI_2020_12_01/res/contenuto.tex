\section*{Introduzione}

\subsection*{Luogo e data dell'incontro}
	\begin{itemize}
		\item \textbf{luogo:} videoconferenza su \glock{Meet};
		\item \textbf{data:} 2020-12-01;
		\item \textbf{ora di inizio:} 15:00;
		\item \textbf{ora di fine:} 17:00.
	\end{itemize}

\subsection*{Ordine del giorno}
	\begin{enumerate}
			\item sintesi dei progetti e valutazioni
			\item template documenti vari
			\item documenti ufficiali/non ufficiali e dove salvare le regole
			\item organigramma e suddivisioni delle mansioni
			\item canale comunicativo per lo sviluppo
			\item varie ed eventuali
	\end{enumerate}

\subsection*{Presenze}
	\begin{itemize}
		\item \textbf{totale presenti:} 7 su 7;
		\item \textbf{presenti: }
			\begin{itemize}
				\item Badan Antonio;
				\item Bertoldo Damiano;
				\item Budai Matteo;
				\item De Grandi Samuele;
				\item Piacere Ivan;
				\item Privitera Sara;
				\item Spigolon Daniele;
			\end{itemize}
		\item \textbf{assenti: }
			\begin{itemize}
				\item Alle ore 16.15 Privitera Sara esce dalla riunione per motivi personali.
			\end{itemize}
	\end{itemize}


\newpage
\section*{Svolgimento}

	\subsection*{Sintesi dei progetti e valutazioni}
		Si è discusso riguardo lo studio di fattibilità analizzata dai vari componenti, espondendo pro e contro per i vari capitolati analizzati (Zucchetti, Synclab, Imola Informatica). Sono emersi alcuni dubbi che verranno esposti ai vari referenti dei capitolati.

	\subsection*{Discussione template documenti vari}
		Si sono discussi i punti più criticati dal prof. Vardanega per i vari documenti, cercando di raccogliere i punti critici comuni in modo tale da riuscire a creare un template che rispecchi le specifiche richieste, ed evitare gli errori comuni. Si è scelto l'utilizzo di \glock{\LaTeX}: se è possibile con \glock{Overleaf}, altrimenti con editor \glock{\LaTeX} integrato con \glock{GitHub}.

	\subsection*{Documenti ufficiali/non ufficiali e dove salvare le regole}
		Si è discusso sul posizionamento dei vari documenti: quali inserire in \glock{GitHub}, quali in \glock{Google Drive}.
		
	\subsection*{Organigramma e suddivisioni delle mansioni}
		Si è discusso su come organizzare l'organigramma. Viene scelto di consultare il prof. Vardanega per consigli sulla gestione ed organizzazione dell'organigramma e se segnalare l'assenza di un componente durante una riunione (completa o parziale assenza). Si è deciso i ruoli per le stesure dei verbali e come ruotarli: in ordine alfabetico a rotazione redattore, verificatore, approvatore.
	
	\subsection*{Canale comunicativo per lo sviluppo}
		Viene discusso se utilizzare un canale comunicativo differente da \glock{Telegram} per lo sviluppo del progetto. Si è nominato \glock{Discord} e \glock{Slack}. Si è lasciato in sospeso l'argomento, in quanto può essere necessario che il referente del capitolato sia dentro tale canale e ne prediliga uno in particolare.
		
	\subsection*{Varie ed eventuali}
		Viene deciso, in modo unanime, la necessità di una riunione a settimana.
		Viene sistemato il problema con \glock{GitHub} ed i loro \glock{Branch}.