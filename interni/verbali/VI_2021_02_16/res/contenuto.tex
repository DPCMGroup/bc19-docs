\section*{Introduzione}
\subsection*{Luogo e data dell'incontro}
\begin{itemize}
\item \textbf{luogo:} videoconferenza su \glock{discord};
\item \textbf{data:} 2021-02-16;
\item \textbf{ora di inizio:} 11:00;
\item \textbf{ora di fine:} 13:00.
\end{itemize}

\subsection*{Ordine del giorno}
\begin{enumerate}
	\item Pianificazione fino alla Revisione di Progettazione;
	\item Discussione \glock{Technology Baseline};
	\item Comunicazione con il proponente.
\end{enumerate}

\subsection*{Presenze}
\begin{itemize}
	\item \textbf{Presenti:}
	\begin{itemize}
		\item Badan Antonio;
		\item Bertoldo Damiano;
		\item Budai Matteo;
		\item De Grandi Samuele;
		\item Piacere Ivan;
		\item Privitera Sara;
		\item Spigolon Daniele.
	\end{itemize}
\end{itemize}

\section*{Svolgimento}
\subsection*{Pianificazione}
Viene discussa la pianificazione a grana fine dei prossimi giorni e delle due settimane rimanenti prima della RP. Si è suddiviso, tra i membri del gruppo, il lavoro di ricerca e sviluppo del PoC in tre principali aree tecnologiche: \glock{front-end}, \glock{back-end} dell'applicazione web e Ethereum. 
Nonostante la sessione di esami, si è deciso di mantenere molto attiva la comunicazione su Discord, per gli aggiornamenti di avanzamento sul lavoro svolto.

\subsection*{Discussione Technology Baseline}
Si è discusso in modo approfondito delle tecnologie da utilizzare per la codifica del PoC e condiviso su Drive materiale che potesse risultare utile.

\subsection*{Comunicazione con il proponente}
A seguito dei dubbi riguardanti alcuni casi d'uso del documento dell'Analisi dei requisiti, si è deciso di mettersi in contatto il prima possibile con il proponente, tramite un incontro su Zoom e/o via mail.