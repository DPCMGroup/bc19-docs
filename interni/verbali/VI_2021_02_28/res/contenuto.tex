\section*{Introduzione}
\subsection*{Luogo e data dell'incontro}
\begin{itemize}
	\item \textbf{luogo: videoconferenza su \glock{Discord}} ;
	\item \textbf{data: 2021-02-28} ;
	\item \textbf{ora di inizio: 14:00} ;
	\item \textbf{ora di fine: 16:00} .
\end{itemize}

\subsection*{Ordine del giorno}
\begin{enumerate}
	\item integrazione e test PoC;
	\item situazione documentazione;
	\item scelta delle date di presentazione Technology baseline e consegna documentazione;
	\item suddivisione del lavoro legato alla presentazione.
\end{enumerate}

\subsection*{Presenze}
\begin{itemize}
	\item \textbf{Presenti:}
	\begin{itemize}
		\item Badan Antonio;
		\item Bertoldo Damiano;
		\item Budai Matteo;
		\item De Grandi Samuele;
		\item Piacere Ivan;
		\item Privitera Sara;
		\item Spigolon Daniele.
	\end{itemize}
\end{itemize}

\section*{Svolgimento}
\subsection*{Integrazione e test PoC}
Le varie componenti del PoC sono complete, ma manca la loro integrazione. Si è previsto di completarla entro il 2021-03-01, data in cui verrà eseguita una simulazione completa.

\subsection*{Situazione documentazione}
La documentazione è ancora per la maggior parte da completare. Una volta completato il PoC si dovrà intensificare il lavoro su di essa, in parallelo con la preparazione della presentazione della Technology baseline.

\subsection*{Scelta delle date di presentazione Technology baseline e consegna documentazione}
Vista la situazione del PoC, si prevede di essere pronti per la presentazione entro il 2021-03-04 alle 12:30. Si deve pertanto fissare l'appuntamento col prof. Cardin in tale data e ora.
Riguardo alla documentazione si prevede di concluderla entro il 2021-03-08. Si deve concordare la data con il prof. Vardanega, scrivendogli entro il 2021-03-01.

\subsection*{Suddivisione del lavoro legato alla presentazione}
Tutti i componenti parteciperanno alla preparazione della presentazione secondo la suddivisione delle tecnologie descritta in VI\_2021\_02\_16 v.1.0.0. La preparazione consiste nella stesura della parte grafica e contenutistica delle slide e degli appunti da usare per l'esposizione. L'esposizione effettiva è suddivisa come segue:
\begin{itemize}
	\item Antonio:
	\begin{itemize}
		\item introduzione;
		\item linguaggi parte Ethereum;
		\item tecnologie parte Ethereum;
		\item tecnologie scartate parte Ethereum;
		\item interazione tra le tecnologie Ethereum;
		\item spiegazione codice smart contract.
	\end{itemize}
	\item Damiano:
	\begin{itemize}
		\item linguaggi parte server;
		\item tecnologie parte server;
		\item tecnologie scartate parte server;
		\item interazione tra le tecnologie server;
		\item spiegazione codice interazione client-server;
		\item dimostrazione PoC;
		\item cosa manca da integrare nel PoC.
	\end{itemize}
	\item Ivan:
	\begin{itemize}
		\item spiegazione codice interazione server-blockchain.
	\end{itemize}
\end{itemize}
