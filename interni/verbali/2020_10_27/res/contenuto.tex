\section*{Introduzione}

\subsection*{Luogo e data dell'incontro}
\begin{itemize}
	\item \textbf{luogo:} videoconferenza su \glock{Zoom};
	\item \textbf{data:} 2020-10-27;
	\item \textbf{ora di inizio:} 16:30;
	\item \textbf{ora di fine:} 18:00.
\end{itemize}

\subsection*{Ordine del giorno}
\begin{enumerate}
\item presentazione individuale e conoscenza reciproca;
\item documentazione;
\item discussione libera sui \glock{capitolati};
\item \glock{brainstorming} su nome e logo del gruppo;
\item comunicazioni.
\end{enumerate}

\subsection*{Presenze}
	\begin{itemize}
		\item \textbf{totale presenti:} 7 su 7;
		\item \textbf{presenti: }
			\begin{itemize}
				\item Badan Antonio;
				\item Bertoldo Damiano;
				\item Budai Matteo;
				\item De Grandi Samuele;
				\item Piacere Ivan;
				\item Privitera Sara;
				\item Spigolon Daniele.
			\end{itemize}
	\end{itemize}

\newpage

\section*{Svolgimento}
\subsection*{Presentazione individuale e conoscenza reciproca}
Ogni membro del gruppo si è presentato e ha dato a modo agli altri di conoscerlo meglio.

\subsection*{Documentazione}
Si è discusso sull'importanza di conservare una documentazione del lavoro del gruppo fin da subito, soprattutto in ottica di un futuro tracciamento delle decisioni prese.

\subsection*{Discussione libera sui capitolati}
Si è discusso liberamente sui capitolati mettendo in luce: eventuali criticità, interesse per determinate tematiche, pro e contro di una scelta di un capitolato rispetto a un'altra. Si è stabilito che l'obiettivo della discussione odierna non fosse già quella di scegliere, acriticamente, il capitolato che ci accompagnerà per i prossimi mesi  ma piuttosto l'avvio di un'indagine che permetterà di inquadrare opportunamente ogni capitolato ed effettuare una scelta con cognizione di causa. 
È emersa la volontà di tutto il gruppo di partecipare alle prime scadenze di revisione possibili e concludere il progetto a maggio. Tutti i presenti si sono presi l'impegno di fare in modo che questo avvenga. Sono state predisposte alcune domande da porre ai proponenti in vista dell'incontro di presentazione dei capitolati del 5 novembre 2020.

\subsection*{Brainstorming su nome e logo del gruppo}
Si è iniziato a pensare al nome e al logo del gruppo.

\subsection*{Comunicazioni}
Si è scelto \glock{Telegram} come canale di comunicazione principale per il gruppo e \glock{Google Drive} come archivio della documentazione del gruppo e strumento collaborativo.