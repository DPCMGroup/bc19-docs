\section*{Introduzione}
\subsection*{Luogo e data dell'incontro}
\begin{itemize}
	\item \textbf{luogo:} videoconferenza su \glock{Discord};
	\item \textbf{data:} 2021-05-17;
	\item \textbf{ora di inizio:} 14:15;
	\item \textbf{ora di fine:} 15:30.
\end{itemize}

\subsection*{Ordine del giorno}
\begin{enumerate}
	\item Aggiornamento sulla situazione attuale del prodotto;
	\item Pianificazione obiettivi periodo RQ-RA;
	\item Comunicazione con il proponente.
		
\end{enumerate}

\subsection*{Presenze}
\begin{itemize}
	\item \textbf{Presenti: }
	\begin{itemize}
		\item Badan Antonio;
		\item Bertoldo Damiano;
		\item Budai Matteo;
		\item De Grandi Samuele;
		\item Piacere Ivan;
		\item Privitera Sara;
		\item Spigolon Daniele;
	\end{itemize}
\end{itemize}


\section*{Svolgimento}

\subsection*{\hypertarget{link1}{1. Aggiornamento sulla situazione attuale del prodotto}}
Ci siamo aggiornati per quanto riguarda la situazione delle tre componenti principali del prodotto, ovvero:
\begin{itemize}
	\item applicazione web;
	\item applicazione mobile;
	\item backend.
\end{itemize}
Tutte e tre soddisfano i requisiti obbligatori con una percentuale superiore al 50 \%.
Per la prima si è deciso di implementare la funzionalità della visualizzazione delle postazioni e stanze in modo schematico attraverso una vista a griglia, di colorare le postazioni in base allo stato in cui si trovano e infine mostrare il numero di occupanti per stanza.
Per la seconda si è deciso di continuare con l'azione di refactoring del codice seguendo il pattern MVP e di implementare la funzionalità di visualizzazione della guida utente.
Per la terza si è deciso di ultimare la stesura di unit test, installare la blockchain in un container Docker e completare la comunicazione con quest'ultima tramite web3.
Il nostro obiettivo è di implementare le funzionalità descritte entro il 24 maggio.
\subsection*{2. Obiettivi pianificati periodo RQ-RA}
Si è discusso sulla pianificazione degli obiettivi da raggiungere sia sul lungo periodo, fino al 4 giugno, sia a grana più fine entro il 24 maggio. Questi ultimi sono stati esposti nel paragrafo precedente.
\subsection*{\hypertarget{link3}{3. Comunicazione con il proponente}}
Si è discusso per aumentare la comunicazione con il proponente in questo periodo finale per avere più feedback prima del collaudo. Ogni settimana creeremo dei brevi video demo in cui mostreremo le principali novità che abbiamo introdotto.

